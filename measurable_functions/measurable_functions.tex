
%!name:entire_functions
%!need:complex_analytic_functions
%!refs:yellow/IX/4

\section*{Definition}

An \t{entire function} is a complex function $f: \C  \to \C $ which is analytic for all $z \in \C $.

\blankpage
\sbasic

%%%% MACROS %%%%%%%%%%%%%%%%%%%%%%%%%%%%%%%%%%%%%%%%%%%%%%%

\newcommand{\PM}{\mathbf{P}}

%%%%%%%%%%%%%%%%%%%%%%%%%%%%%%%%%%%%%%%%%%%%%%%%%%%%%%%%%%%


\sstart

\stitle{Measurable Functions}

\ssection{Why}

We define integrals using
an infinite process; in order
for each step of the process
to make sense, need functions
to be measurable.
Maybe: point to simple functions
so that the why is clear.

\ssection{Definition}

A function between
the base sets of two measurable
spaces is
\ct{measurable}{}
with respect to the distinguished
sets of the two spaces if the
inverse image of every
distinguished subset of the
codomain is a distinguished
subset of the domain.

\ssubsection{Notation}

Let $(X, \mathcal{A})$
and $(Y, \mathcal{B})$
be measurable spaces.
Then a function
$f: X \to Y$ is measurable
if $B \in \mathcal{B}$
implies $f^{-1}(B) \in \mathcal{A}$.
We say that $f$ is measurable
with respect to $\mathcal{A}$ and
$\mathcal{B}$.

In this case, we sometimes say
$f$ is a measurable function
from $(X, \mathcal{A})$ to
$(Y, \mathcal{B})$.
We say,
$f: (X, \mathcal{A}) \to (Y, \mathcal{B})$
is measurable, read aloud as
\say{f from X, A to
  Y, B is measurable.}


\strats
