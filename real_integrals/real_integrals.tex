
%!name:entire_functions
%!need:complex_analytic_functions
%!refs:yellow/IX/4

\section*{Definition}

An \t{entire function} is a complex function $f: \C  \to \C $ which is analytic for all $z \in \C $.

\blankpage
\sbasic

\sinput{../sets/macros.tex}
%%%% MACROS %%%%%%%%%%%%%%%%%%%%%%%%%%%%%%%%%%%%%%%%%%%%%%%

\newcommand{\PM}{\mathbf{P}}

%%%%%%%%%%%%%%%%%%%%%%%%%%%%%%%%%%%%%%%%%%%%%%%%%%%%%%%%%%%


\sstart

\stitle{Real Integrals}

\ssection{Why}

We define the area
under an extended
real function.

\ssection{Definition}

The
\ct{posititive part}{positivepart}
of an extended-real-valued function
is the function which is the
pointise maximum of first
function and the function
which is identically zero.
The
\ct{negative part}{negativepart}
of an extended-real-valued function
is the function which is the
pointise additive inverse
of the minimum of the first
function and
the function
which is identically zero.
Both the positive and
negative parts of the function
are non-negative
extended-real-valued
functions.

An extended real function
can be decomposed into
its positive part and
negative part.
The function equals
the positive part
minus the negative part.

Consider a measure space.
An
\ct{integrable}{integrable}
function
is a measurable
extended real function
for which
the non-negative integral
of the posititve part
and the non-negative integral
of the negative part of
the function are finite.

The \ct{integral}{realintegral}
of an integrable function
is the difference
of the non-negative integral
of the posititive part and
and the non-negative integral
of the negative part.

If one but not both
of the parts of the function
are finite, we say that the
integral
\ct{exists}{integralexists}
and again define it as
the difference of the
the non-negative integral
of the positive part and the
non-negative integral of the
negative part; in this
way we avoid arithmetic
between two infinities.

\ssubsection{Notation}

Let $A$ a non-empty set.
Let $g: A \to [-\infty, \infty]$.
We denote the positive part
of $g$ by $g^+$ and the negative
part of $g$ by $g^-$:
\[
  g^+(x) = \max\set{g(x), 0} \quad \text{ and } \quad g^-(x) = -\min\set{g(x), 0}.
\]
We observed that
$g^+(x) \geq 0$ and
$g^-(x) \geq 0$
for all $x \in X$.
Moreover,
$g = g^+ - g^-$.

Let $(X, \mathcal{A}, \mu)$
be a measure space.
Let $f: X \to [-\infty, +\infty]$
measurable and
one of
$\int f^+ d \mu$ or
$\int f^- d \mu$
is finite
(if both are finite,
$f$ is integrable).

We denote the integral
of $f$ with respect to the
measure $\mu$ by
$\int f d \mu$.
We defined:
\[
  \int f d\mu = \int f^+ d\mu - \int f^- d\mu.
\]

\strats
