
%!name:entire_functions
%!need:complex_analytic_functions
%!refs:yellow/IX/4

\section*{Definition}

An \t{entire function} is a complex function $f: \C  \to \C $ which is analytic for all $z \in \C $.

\blankpage
\sbasic

%%%% MACROS %%%%%%%%%%%%%%%%%%%%%%%%%%%%%%%%%%%%%%%%%%%%%%%

\newcommand{\PM}{\mathbf{P}}

%%%%%%%%%%%%%%%%%%%%%%%%%%%%%%%%%%%%%%%%%%%%%%%%%%%%%%%%%%%


\sstart

\stitle{Identity Elements}

\ssection{Why}

We can construct
functions on the
ground set of an algebra
by fixing an element
in the ground set
and defining a function
which maps elements
to the result of the
operation applied to
the fixed element and
the given element.

\ssection{Definition}

Let $(A, +)$ be an algebra.
For each $a \in A$,
denote by $+_a: A \to A$
the function defined
by
\[
  +_a(b) = a + b.
\]
If $+_a$ is the
identity function
on $A$ then we call $a$
a \ct{left identity element}{}
of the algebra.

Similarly, denote
by $+^a: A \to A$ the function
defined by
\[
  +^{a}(b) = b + a.
\]
If $+^a$ is the
identity function
on $A$ then we call $a$
a \ct{right identity element}{}.
of the algebra.

An \ct{identity element}{}
of the algbera is an element
which is both a left and right
identity.

\strats
