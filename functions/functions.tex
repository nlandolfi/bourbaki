\documentclass[12pt]{extarticle}
%\usepackage[margin=1.75in]{geometry}
\usepackage[a5paper,margin=.72in]{geometry}
\usepackage{graphicx}
\renewcommand{\baselinestretch}{1.25}
\setlength{\parskip}{0.5em}

\usepackage{titlesec}
%\titleformat*{\section}{\large\bfseries\sffamily}
%\titleformat*{\subsection}{\normalsize\bfseries\sffamily}
%\titleformat*{\subsubsection}{\large\bfseries}
%\titleformat*{\paragraph}{\large\bfseries}
%\titleformat*{\subparagraph}{\large\bfseries}

\titleformat*{\section}{\bfseries\sffamily}
\titleformat*{\subsection}{\small\bfseries\sffamily}
\titleformat*{\subsubsection}{\large\bfseries}
\titleformat*{\paragraph}{\large\bfseries}
\titleformat*{\subparagraph}{\large\bfseries}
\titlespacing*{\section}{0pt}{0.3cm}{0.1cm}
\titlespacing*{\subsection}{0pt}{0.3cm}{0.1cm}

%\renewcommand{\title}[1]{
%\begin{center}
%  \includegraphics[width=0.05\textwidth]{../../trademark}
%  \\
%  \vspace{0.20cm}
%  {\large \textsf{ #1 }}
%\end{center}
%}

\newcommand{\bpage}{
\vspace*{\fill}
\begin{center}
\includegraphics[width=1cm]{../../trademark}
\end{center}
\vspace{\fill}
}

\newcommand{\gpage}{
\begin{center}
\vspace*{\fill}
\includegraphics{graph}
\vspace{\fill}
\end{center}
}


\newcommand*{\vcenteredhbox}[1]{\begin{tabular}{@{}c@{}}#1\end{tabular}}
\renewcommand{\title}[1]{
\begin{center}
\hspace*{-1cm}
\vcenteredhbox{\vspace*{-0.1cm} \includegraphics[height=0.5cm]{../../trademark}\hspace*{0.1cm}}
\vcenteredhbox{\large \textsf{#1}}
\end{center}
}

\usepackage{hyperref}
\hypersetup{
  colorlinks,
  citecolor=black,
  filecolor=black,
  linkcolor=black,
  urlcolor=black
}


%\usepackage{ccfonts}% http://ctan.org/pkg/{ccfonts}
%\usepackage[cm]{sfmath}
\usepackage[T1]{fontenc}
\DeclareMathAlphabet{\mathbfsf}{\encodingdefault}{\sfdefault}{bx}{n}
\def\mathword#1{\mathop{\textup{#1}}}

\usepackage{caption}
\usepackage{subcaption}

%%%% MACROS %%%%%%%%%%%%%%%%%%%%%%%%%%%%%%%%%%%%%%%%%%%%%%%

\newcommand{\PM}{\mathbf{P}}

%%%%%%%%%%%%%%%%%%%%%%%%%%%%%%%%%%%%%%%%%%%%%%%%%%%%%%%%%%%


%%%% MACROS %%%%%%%%%%%%%%%%%%%%%%%%%%%%%%%%%%%%%%%%%%%%%%%

% use \set{stuff} for { stuff }
% use \set* for autosizing delimiters
\DeclarePairedDelimiter{\set}{\{}{\}}

% use \Set{a}{b} for {a | b}
% use \Set* for autosizing delimiters
\DeclarePairedDelimiterX{\Set}[2]{\{}{\}}{#1 \nonscript\;\delimsize\vert\nonscript\; #2}

% use \powerset{A} for power set of A
\newcommand{\powerset}[1]{2^{#1}}

\renewcommand{\emptyset}{\varnothing}

\newcommand{\SA}{\mathcal{A}}
\newcommand{\SB}{\mathcal{B}}
\newcommand{\SC}{\mathcal{C}}
\newcommand{\SD}{\mathcal{D}}
\newcommand{\SE}{\mathcal{E}}
\newcommand{\SF}{\mathcal{F}}
\newcommand{\SG}{\mathcal{G}}
\newcommand{\SH}{\mathcal{H}}
\newcommand{\SI}{\mathcal{I}}
\newcommand{\SJ}{\mathcal{J}}
\newcommand{\SK}{\mathcal{K}}
\newcommand{\SL}{\mathcal{L}}

%%%%%%%%%%%%%%%%%%%%%%%%%%%%%%%%%%%%%%%%%%%%%%%%%%%%%%%%%%%


%%%% MACROS %%%%%%%%%%%%%%%%%%%%%%%%%%%%%%%%%%%%%%%%%%%%%%%

\newcommand{\PM}{\mathbf{P}}

%%%%%%%%%%%%%%%%%%%%%%%%%%%%%%%%%%%%%%%%%%%%%%%%%%%%%%%%%%%


\begin{document}
\title{Functions}

\section{Why}
We want a notion for a correspondence between two sets.

\section{Definition}

To each \term{sets}{element} of a first \term{sets}{set} we associate a unique \term{sets}{element} of a second \term{sets}{set}.
We call this correspondence a \definition{function}.
We call the first set the \definition{domain} and the second set the \definition{codomain}.
We say that the \term{functions}{function} \definition{maps} elements from the \term{functions}{domain} into \term{functions}{codomain}.

We call the subset of ordered pairs whose first element is in the domain and whose second element is the corresponding element of the codomain the \definition{graph} of the \term{functions}{function}.
The graph is a relation between the domain and codomain.
So a function can be viewed as or specified as a relation between these two sets.

\subsection{Notation}
We commonly denote a \term{functions}{function} by a lower case latin letter, especially $f$, $g$, and $h$.
Of course, $f$ is a mnemonic for \term{functions}{function}.
Let $A$ and $B$ be two non-empty sets.
When we want to be explicit that the domain of a function $f$ is $A$ and its codomain is $B$ we write $f: A \to B$ read \say{f from A to B.}
For each element $a$ in the domain, we denote the unique element that $f$ maps to by $f(a)$, read aloud \say{f of a.}

The set $\Set{(a, f(a))}{a \in A}$ of ordered pairs is the graph of $f$, and we commonly denote it by $\Gamma_f$.
$\Gamma$ is a mnemonic for graph.

\section{Other Sets}

For a function $f: A \to B$ and a set $C \subset A$, we call the set $\Set{f(c)}{c \in C}$ the \definition{image} of $C$.
We call the \term{functions}{image} of the \term{functions}{domain} the \definition{range} of the function.

The range need not be the codomain; though it, like every other image, is a subset of the codomain.
In the case that it the range and codomain are equal, we call the function \definition{onto}.
We say that the function maps the domain onto the range.
This language suggests that every element of the codomain is used by $f$.
It means that for each element $b$ of the codomain, we can find an element $a$ of the domain so that $f(a) = b$.

There may be multiple elements of the domain corresponding to any particular element of the codomain.
By this we mean that the function uses an element of the codomain more than once.
This overlapping of a function is a regular occurence.
More interesting is the property which makes the correspondence unique in the sense that each element of the codomain only has one corresponding element of the domain.
If a function has this property, we call it \definition{one-to-one}.
This language is meant to suggest exactly this notion of a unique correspondence.

\subsection{Notation}

let $f: A \to B$ and $C \subset C$.
We denote the image of $C$ by $f(C)$, read aloud \say{f of C.}
The property that $f$ is onto can be written succintly as $f(A) = B$.

There is some ambiguity here, as we have overloaded notation.
Let $c \in C$.
Notice that we have defined notation for $f(c)$ and $f(C)$.
$f(c)$ is an element of the codomain, indeed the range, of $f$.
$f(C)$ is a subset of the codomain, indeed the range, of $f$.
They are both read \say{f of C} and so we must be careful to have specified what $C$ is.
When we have taken this precaution, there is no trouble.

\end{document}
