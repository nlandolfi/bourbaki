\documentclass[12pt]{extarticle}
%\usepackage[margin=1.75in]{geometry}
\usepackage[a5paper,margin=.72in]{geometry}
\usepackage{graphicx}
\renewcommand{\baselinestretch}{1.25}
\setlength{\parskip}{0.5em}

\usepackage{titlesec}
%\titleformat*{\section}{\large\bfseries\sffamily}
%\titleformat*{\subsection}{\normalsize\bfseries\sffamily}
%\titleformat*{\subsubsection}{\large\bfseries}
%\titleformat*{\paragraph}{\large\bfseries}
%\titleformat*{\subparagraph}{\large\bfseries}

\titleformat*{\section}{\bfseries\sffamily}
\titleformat*{\subsection}{\small\bfseries\sffamily}
\titleformat*{\subsubsection}{\large\bfseries}
\titleformat*{\paragraph}{\large\bfseries}
\titleformat*{\subparagraph}{\large\bfseries}
\titlespacing*{\section}{0pt}{0.3cm}{0.1cm}
\titlespacing*{\subsection}{0pt}{0.3cm}{0.1cm}

%\renewcommand{\title}[1]{
%\begin{center}
%  \includegraphics[width=0.05\textwidth]{../../trademark}
%  \\
%  \vspace{0.20cm}
%  {\large \textsf{ #1 }}
%\end{center}
%}

\newcommand{\bpage}{
\vspace*{\fill}
\begin{center}
\includegraphics[width=1cm]{../../trademark}
\end{center}
\vspace{\fill}
}

\newcommand{\gpage}{
\begin{center}
\vspace*{\fill}
\includegraphics{graph}
\vspace{\fill}
\end{center}
}


\newcommand*{\vcenteredhbox}[1]{\begin{tabular}{@{}c@{}}#1\end{tabular}}
\renewcommand{\title}[1]{
\begin{center}
\hspace*{-1cm}
\vcenteredhbox{\vspace*{-0.1cm} \includegraphics[height=0.5cm]{../../trademark}\hspace*{0.1cm}}
\vcenteredhbox{\large \textsf{#1}}
\end{center}
}

\usepackage{hyperref}
\hypersetup{
  colorlinks,
  citecolor=black,
  filecolor=black,
  linkcolor=black,
  urlcolor=black
}


%\usepackage{ccfonts}% http://ctan.org/pkg/{ccfonts}
%\usepackage[cm]{sfmath}
\usepackage[T1]{fontenc}
\DeclareMathAlphabet{\mathbfsf}{\encodingdefault}{\sfdefault}{bx}{n}
\def\mathword#1{\mathop{\textup{#1}}}

\usepackage{caption}
\usepackage{subcaption}

%%%% MACROS %%%%%%%%%%%%%%%%%%%%%%%%%%%%%%%%%%%%%%%%%%%%%%%

\newcommand{\PM}{\mathbf{P}}

%%%%%%%%%%%%%%%%%%%%%%%%%%%%%%%%%%%%%%%%%%%%%%%%%%%%%%%%%%%


\reference{sets}
\reference{relations}

%%%% MACROS %%%%%%%%%%%%%%%%%%%%%%%%%%%%%%%%%%%%%%%%%%%%%%%

\newcommand{\PM}{\mathbf{P}}

%%%%%%%%%%%%%%%%%%%%%%%%%%%%%%%%%%%%%%%%%%%%%%%%%%%%%%%%%%%


\begin{document}
\title{Functions}

\section{Why}
We want a notion for a correspondence between two sets.

\section{Definition}

To each \term{sets}{element} of a first \term{sets}{set} we associate an \term{sets}{element} of a second \term{sets}{set}.
We call this correspondence a \definition{function}.
We call the first set the \definition{domain} and the second set the \definition{codomain}.
We say that the \term{functions}{function} \definition{maps} elements from the \term{functions}{domain} into the \term{functions}{codomain}.

We call the codomain element associated with the domain element the \definition{result} of \definition{applying} the function to the domain element.
We call the subset of ordered pairs whose first element is in the domain and whose second element is the corresponding result the \definition{graph} of the \term{functions}{function}.
The graph is a relation between the domain and codomain.
So a function can be viewed as or specified as a relation between these two sets.

\subsection{Notation}
We often denote \term{functions}{functions} by lower case latin letters, especially $f$, $g$, and $h$.
Of course, $f$ is a mnemonic for \term{functions}{function}; $g$ and $h$ follow $f$ in the alphabet.
Let $A$ and $B$ be two non-empty sets.
When we want to be explicit that the domain of a function $f$ is $A$ and its codomain is $B$ we write $f: A \to B$, read aloud as \say{f from A to B.}


For each element $a$ in the domain, we denote the result of applying $f$ to $a$ by $f(a)$, read aloud \say{f of a.}
We sometimes drop the parentheses, and write the result as $f_a$, read aloud as \say{f sub a.}

Let $g: A \times B \to C$.
We often write $g(a,b)$ or $g_{ab}$ instead of $g((a,b))$. We read $g(a, b)$ aloud as \say{g of a and b}.
We read $g_{ab}$ aloud as \say{g sub a b.}

For  $f: A \to B$, the set $\Set{(a, f(a)) \in A \cross B}{a \in A}$ of ordered pairs is the graph of $f$.
We often denote it by $\Gamma_f$; \say{gamma} is a mnemonic for graph.

\section{Properties}

For a function $f: A \to B$ and a set $C \subset A$, we call the set $\Set{f(c) \in B}{c \in C}$ the \definition{image} of $C$.
We call the \term{functions}{image} of the \term{functions}{domain} the \definition{range} of the function.

The range need not equal the codomain; though it, like every other image, is a subset of the codomain.
If the range and codomain are equal, we call the function \definition{onto}.
We say that the function maps the domain onto the range.
This language suggests that every element of the codomain is used by $f$.
It means that for each element $b$ of the codomain, we can find an element $a$ of the domain so that $f(a) = b$.

An element of the codomain may be the result of several elements of the domain.
This overlapping, using an element of the codomain more than once, is a regular occurrence.
If a function is a unique correspondence in that every domain element has a different result, we call it \definition{one-to-one}.
This language is meant to suggest that each element of the domain corresponds to one and exactly one element of the codomain, and vice versa.

\subsection{Notation}

Let $f: A \to B$ and $C \subset A$.
We denote the image of $C$ by $f(C)$, read aloud as \say{f of C.}
The property that $f$ is onto can be written succintly as $f(A) = B$.

Let $c \in C$.
Notice that we have defined notation for $f(c)$ and $f(C)$.
In overloading notation, we have introduced ambiguity.
$f(c)$ is an element of the codomain.
$f(C)$ is a subset of the codomain.
Read aloud, the two are indistinguishable, must be careful to have specified whether we mean an element $c$ or a set $C$.
There is no trouble when we take this precaution.

\end{document}
