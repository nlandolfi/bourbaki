
%!name:entire_functions
%!need:complex_analytic_functions
%!refs:yellow/IX/4

\section*{Definition}

An \t{entire function} is a complex function $f: \C  \to \C $ which is analytic for all $z \in \C $.

\blankpage
\sbasic

\sinput{../sets/macros.tex}
\sinput{../ordered_pairs/macros.tex}
\sinput{../relations/macros.tex}

%%%% MACROS %%%%%%%%%%%%%%%%%%%%%%%%%%%%%%%%%%%%%%%%%%%%%%%

\newcommand{\PM}{\mathbf{P}}

%%%%%%%%%%%%%%%%%%%%%%%%%%%%%%%%%%%%%%%%%%%%%%%%%%%%%%%%%%%


\sstart

\stitle{Functions}

\ssection{Why}

We want a notion for a correspondence
between two sets.

\ssection{Definition}

A \ct{functional}{functional} relation
on two sets relates each element of the
first set with a unique element of the
second set.
A \ct{function}{function} is a
functional \rt{relation}{relation}.

The \ct{domain}{domain} of the function is
the first set and \ct{codomain}{codomain}
of the function is the second set.
The \rt{function}{function}
\ct{maps}{maps} elements
\ct{from}{functionfrom} the
\rt{domain}{domain} \ct{to}{functionto}
the \rt{codomain}{codomain}.
We call the codomain element associated
with the domain element the
\ct{result}{functionresult} of
\ct{applying}{functionapplication}
the function to the domain element.

\ssubsection{Notation}

Let $A$ and $B$ be sets.
If $A$ is the domain and $B$ the codomain,
we denote the set of functions from $A$ to
$B$ by $A \to B$, read aloud as \say{A to B}.
%A function is an element of the set $A \to B$,

We denote functions by lower case latin letters,
especially $f$, $g$, and $h$.
Of course, $f$ is a mnemonic for
\rt{function}{function};
$g$ and $h$ follow $f$ in the Latin alphabet.
We denote that $f \in A \to B$ by
$f: A \to B$, read aloud as
\say{f from A to B}.

Let $f: A \to B$.
For each element $a \in A$, we denote the
result of applying
$f$ to $a$ by $f(a)$, read aloud
\say{f of a.}
We sometimes drop the parentheses, and write
the result as $f_a$, read aloud as
\say{f sub a.}
%The set $\Set{(a, f(a)) \in A \cross B}{a \in A}$ of ordered pairs is the graph of $f$.
%We often denote it by $\Gamma_f$; \say{gamma} is a mnemonic for graph.

Let $g: A \times B \to C$.
We often write $g(a,b)$ or $g_{ab}$
instead of $g((a,b))$.
We read $g(a, b)$ aloud as
\say{g of a and b}.
We read $g_{ab}$ aloud as
\say{g sub a b.}


\ssection{Properties}

Let $f: A \to B$.
The \definition{image}{image} of a set $C \subset A$
is the set $\Set{f(c) \in B}{c \in C}$.
The \definition{range}{range} of $f$ is the image
of the domain.
The \definition{inverse image}{inverseimage} of a set
$D \subset B$ is the set $\Set{a \in A}{f(a) \in B}$.

The range need not equal the codomain; though it,
like every other image, is a subset of the codomain.
The function \rt{maps}{functionmaps} to domain
\ct{on}{functionon} to the codomain if the
range and codomain are equal;
in this case we call the function \definition{onto}{onto}.
This language suggests that every element of
the codomain is used by $f$.
It means that for each element $b$ of the codomain,
we can find an element $a$ of the domain so that
$f(a) = b$.

An element of the codomain may be the result
of several elements of the domain.
This overlapping, using an element of the
codomain more than once, is a regular occurrence.
If a function is a unique correspondence in that
every domain element has a different result,
we call it \ct{one-to-one}{one-to-one}.
This language is meant to suggest that each
element of the domain corresponds to one and
exactly one element of the codomain, and vice versa.

\ssubsection{Notation}

Let $f: A \to B$.
We denote the image of $C \subset A$ by $f(C)$, read aloud as \say{f of C.}
This notation is overloaded: for $c \in C$, $f(c) \in A$, whereas $f(C) \subset A$.
Read aloud, the two are indistinguishable, so we must be careful to specify whether we mean an element $c$ or a set $C$.
The property that $f$ is onto can be written succintly as $f(A) = B$.
We denote the inverse image of $D \subset B$ by $f^{-1}(D)$, read aloud as \say{f inverse D.}


\strats
