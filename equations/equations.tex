
%!name:entire_functions
%!need:complex_analytic_functions
%!refs:yellow/IX/4

\section*{Definition}

An \t{entire function} is a complex function $f: \C  \to \C $ which is analytic for all $z \in \C $.

\blankpage
\sbasic

%%%% MACROS %%%%%%%%%%%%%%%%%%%%%%%%%%%%%%%%%%%%%%%%%%%%%%%

% use \set{stuff} for { stuff }
% use \set* for autosizing delimiters
\DeclarePairedDelimiter{\set}{\{}{\}}

% use \Set{a}{b} for {a | b}
% use \Set* for autosizing delimiters
\DeclarePairedDelimiterX{\Set}[2]{\{}{\}}{#1 \nonscript\;\delimsize\vert\nonscript\; #2}

% use \powerset{A} for power set of A
\newcommand{\powerset}[1]{2^{#1}}

\renewcommand{\emptyset}{\varnothing}

\newcommand{\SA}{\mathcal{A}}
\newcommand{\SB}{\mathcal{B}}
\newcommand{\SC}{\mathcal{C}}
\newcommand{\SD}{\mathcal{D}}
\newcommand{\SE}{\mathcal{E}}
\newcommand{\SF}{\mathcal{F}}
\newcommand{\SG}{\mathcal{G}}
\newcommand{\SH}{\mathcal{H}}
\newcommand{\SI}{\mathcal{I}}
\newcommand{\SJ}{\mathcal{J}}
\newcommand{\SK}{\mathcal{K}}
\newcommand{\SL}{\mathcal{L}}

%%%%%%%%%%%%%%%%%%%%%%%%%%%%%%%%%%%%%%%%%%%%%%%%%%%%%%%%%%%


\sstart

\stitle{Equations}

\ssection{Why}

If we know the result of two operations

\ssection{Definition}

An \ct{equation}{equation} is a formula
with an equality.

A \ct{set}{set} is a collection of objects.
We use the term \ct{object}{object} in the
usual sense of the English language.
So a set is itself an object, but of the peculiar nature
that it contains other objects.
In thinking of a set, then, we regularly consider the
objects it contains.
We call the objects contained in a set the
\ct{members}{members} or
\ct{elements}{element} of the set.
So we say that an object contained in a set is a
\ct{member of}{memberof} or an
\ct{element of}{elementof} the set.

\ssubsection{Notation}

We denote sets by upper case latin letters: for example, $A$, $B$, and $C$.
We denote elements of sets by lower case latin letters: for example, $a$, $b$, and $c$.
We denote that an object $A$ is an element of a set $A$ by $a \in A$.
We read the notation $a \in A$ aloud as \say{a in A.}
The $\in$ is a stylized $\epsilon$, which possesses the mnemonic for element.
We write $a \not\in A$, read aloud as \say{a not in A,} if $a$ is not an element of $A$.

If we can write down the elements of $A$, we do so using a brace notation.
For example, if the set $A$ is such that it contains only the elements $a, b, c$, we denote $A$ by $\set{a, b, c}$.
If the elements of a set are well-known we introduce the set in English and name it; often we select the name mnemonically.
For example, let $L$ be the set of latin letters.

If the elements of a set are such that they satisfy some common condition, we use the braces and include the condition.
For example, if $V$ is the set of vowels we denote $V$ by $\Set{l \in L}{l \text{ is a vowel}}$.
The $\mid$ is read aloud as \say{such that,} the notation reads aloud as \say{l in L such that l is a vowel.}
We call the notation $\Set{l \in L}{l \text{ is a vowel}}$ \ct{set-builder notation}{setbuildernotation}.
Set-builder notation is indispensable for sets defiined implicitly by some condition.
Here we could have alternatively denoted $V$ by $\set{\say{a},\say{e},\say{i},\say{o},\say{u}}$.
We prefer the former, slighly more concise notation.

\ssection{Two Sets}

If every element of a first set is also an element of second set,
we say that the first set is a \ct{subset}{subset} of or is
\ct{contained in}{containedin} the second set.
Conversely, we say that the second set is a \ct{superset}{superset}
of or \ct{contains}{contains} the first set.
If a first set is a subset of a second set and the second set is a subset of
the first set, we say the two sets are \ct{equal}.

We call the set of subsets of a set $A$ the \ct{powerset}{powerset} of $A$.
We call the set which has no members the \ct{empty set}{emptyset}.
The empty set is contained in every other set.

\ssubsection{Notation}
Let $A$ and $B$ be sets.
We denote that $A$ is a subset of $B$ by $A \subset B$.
We read the notation $A \subset B$ aloud as \say{A subset B}.
We denote that $A$ is equal to $B$ by $A = B$.
We read the notation $A = B$ aloud as \say{A equals B}.
We denote the empty set by $\emptyset$, read aloud as \say{empty.}
We denote the power set of $A$ by $\powerset{A}$, read aloud as \say{two to the A.}

\strats
