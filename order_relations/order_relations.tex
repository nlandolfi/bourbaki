
%!name:entire_functions
%!need:complex_analytic_functions
%!refs:yellow/IX/4

\section*{Definition}

An \t{entire function} is a complex function $f: \C  \to \C $ which is analytic for all $z \in \C $.

\blankpage
\sbasic

\sinput{../sets/macros.tex}
\sinput{../cartesian_product/macros.tex}
\sinput{../relations/macros.tex}

%%%% MACROS %%%%%%%%%%%%%%%%%%%%%%%%%%%%%%%%%%%%%%%%%%%%%%%

\newcommand{\PM}{\mathbf{P}}

%%%%%%%%%%%%%%%%%%%%%%%%%%%%%%%%%%%%%%%%%%%%%%%%%%%%%%%%%%%


\sstart

\stitle{Order Relations}

\ssection{Why}

We want to handle elements of a set in a particular order.

\ssection{Definition}

Let $R$ be a relation on a non-empty set $A$.
$R$ is a \definition{partial order} if it is reflexive, transitive, and anti-symmetric.

A \definition{partially ordered set} is a set and a partial order.
The language partial is meant to suggest that two elements need not be comparable.
For example, suppose $R$ is $\Set{(a,a)}{a \in A}$; we may justifiably call this no order at all and call $A$ totally unordered, but it is a partial order by our definition.

Often we want all elements of the set $A$ to be comparable.
We call $R$ \definition{connexive} if for all $a, b \in A$, $(a, b) \in R$ or $(b, a) \in R$.
If $R$ is a partial order and connexive, we call it a \definition{total order}.


A \definition{totally ordered set} is a set together with a total order.
The language is a faithful guide: we can compare any two elements.
Still, we prefer one word to three, and so we will use the shorter term \definition{chain} for a totally ordered set; other terms include \definition{simply ordered set} and \definition{linearly ordered set}.

\ssubsection{Notation}
We denote total and partial orders on a set $A$ by $\preceq$.
We read $\preceq$ aloud as \say{precedes or equal to} and so read $a\preceq b$ aloud as \say{a precedes or is equal to b.}
If $a \preceq b$ but $a \neq b$, we write $a \prec b$, read aloud as \say{a precedes b.}


\strats
