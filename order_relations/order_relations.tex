\documentclass[12pt]{extarticle}
%\usepackage[margin=1.75in]{geometry}
\usepackage[a5paper,margin=.72in]{geometry}
\usepackage{graphicx}
\renewcommand{\baselinestretch}{1.25}
\setlength{\parskip}{0.5em}

\usepackage{titlesec}
%\titleformat*{\section}{\large\bfseries\sffamily}
%\titleformat*{\subsection}{\normalsize\bfseries\sffamily}
%\titleformat*{\subsubsection}{\large\bfseries}
%\titleformat*{\paragraph}{\large\bfseries}
%\titleformat*{\subparagraph}{\large\bfseries}

\titleformat*{\section}{\bfseries\sffamily}
\titleformat*{\subsection}{\small\bfseries\sffamily}
\titleformat*{\subsubsection}{\large\bfseries}
\titleformat*{\paragraph}{\large\bfseries}
\titleformat*{\subparagraph}{\large\bfseries}
\titlespacing*{\section}{0pt}{0.3cm}{0.1cm}
\titlespacing*{\subsection}{0pt}{0.3cm}{0.1cm}

%\renewcommand{\title}[1]{
%\begin{center}
%  \includegraphics[width=0.05\textwidth]{../../trademark}
%  \\
%  \vspace{0.20cm}
%  {\large \textsf{ #1 }}
%\end{center}
%}

\newcommand{\bpage}{
\vspace*{\fill}
\begin{center}
\includegraphics[width=1cm]{../../trademark}
\end{center}
\vspace{\fill}
}

\newcommand{\gpage}{
\begin{center}
\vspace*{\fill}
\includegraphics{graph}
\vspace{\fill}
\end{center}
}


\newcommand*{\vcenteredhbox}[1]{\begin{tabular}{@{}c@{}}#1\end{tabular}}
\renewcommand{\title}[1]{
\begin{center}
\hspace*{-1cm}
\vcenteredhbox{\vspace*{-0.1cm} \includegraphics[height=0.5cm]{../../trademark}\hspace*{0.1cm}}
\vcenteredhbox{\large \textsf{#1}}
\end{center}
}

\usepackage{hyperref}
\hypersetup{
  colorlinks,
  citecolor=black,
  filecolor=black,
  linkcolor=black,
  urlcolor=black
}


%\usepackage{ccfonts}% http://ctan.org/pkg/{ccfonts}
%\usepackage[cm]{sfmath}
\usepackage[T1]{fontenc}
\DeclareMathAlphabet{\mathbfsf}{\encodingdefault}{\sfdefault}{bx}{n}
\def\mathword#1{\mathop{\textup{#1}}}

\usepackage{caption}
\usepackage{subcaption}

%%%% MACROS %%%%%%%%%%%%%%%%%%%%%%%%%%%%%%%%%%%%%%%%%%%%%%%

\newcommand{\PM}{\mathbf{P}}

%%%%%%%%%%%%%%%%%%%%%%%%%%%%%%%%%%%%%%%%%%%%%%%%%%%%%%%%%%%


\reference{sets}
\reference{relations}

%%%% MACROS %%%%%%%%%%%%%%%%%%%%%%%%%%%%%%%%%%%%%%%%%%%%%%%

\newcommand{\PM}{\mathbf{P}}

%%%%%%%%%%%%%%%%%%%%%%%%%%%%%%%%%%%%%%%%%%%%%%%%%%%%%%%%%%%


\begin{document}
\title{Order Relations}

\section{Why}

\boxed{TODO}

\section{Definition}

Let $R$ a relation on the non-empty set $A$.
We call $R$ \definition{anti-symmetric} if for two nonequal elements $a, b \in A$, $(a, b) \in R \implies (b, a) \not\in R$.
If $R$ is reflexive, transitive, and anti-symmetric then we call $R$ a \definition{partial order} on $A$.

A \definition{partially ordered set} is a set together with a partial order.
The language partial is meant to suggest that two elements need not be comparable.
For example, suppose $R$ is $\Set{(a,a)}{a \in A}$; we may justifiably call this no order at all and call $A$ totally unordered, but it is a partial order by our definition.

Often we want all elements of the set $A$ to be comparable.
We call $R$ \definition{connexive} if for all $a, b \in A$, $(a, b) \in R$ or $(b, a) \in R$.
If $R$ is a partial order and connexive, we call it a \definition{total order}.


A \definition{totally ordered set} is a set together with a total order.
The language is a faithful guide: we can compare any two elements.
Still, we prefer one word to three, and so we will use the shorter term \definition{chain} for a totally ordered set; other terms include \definition{simply ordered set} and \definition{linearly ordered set}.

\subsection{Notation}
We denote total and partial orders on a set $A$ by $\preceq$.
We read $\preceq$ aloud as \say{precedes or equal to} and so read $a\preceq b$ aloud as \say{a precedes or is equal to b.}
If $a \preceq b$ but $a \neq b$, we write $a \prec b$, read aloud as \say{a precedes b.}

\end{document}
