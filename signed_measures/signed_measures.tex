
%!name:entire_functions
%!need:complex_analytic_functions
%!refs:yellow/IX/4

\section*{Definition}

An \t{entire function} is a complex function $f: \C  \to \C $ which is analytic for all $z \in \C $.

\blankpage
\sbasic

\sinput{../sets/macros.tex}
\sinput{../set_operations/macros.tex}
\sinput{../sequences/macros.tex}
\sinput{../extended_real_numbers/macros.tex}
%%%% MACROS %%%%%%%%%%%%%%%%%%%%%%%%%%%%%%%%%%%%%%%%%%%%%%%

\newcommand{\PM}{\mathbf{P}}

%%%%%%%%%%%%%%%%%%%%%%%%%%%%%%%%%%%%%%%%%%%%%%%%%%%%%%%%%%%


\sstart

\stitle{Signed Measures}

\ssection{Why}

We allow measures to take
negative values.
We want a vector space
of finite measures.
TODO

\ssection{Definition}

An extended-real-valued
function on a
sigma algebra is
\ct{countably additive}{countablyadditive}
if the result of the function applied to
the union of a disjoint countable family of
distinguished sets is the limit of the partial
sums of the results of the function applied
to each of the sets individually.
The limit of the partial sums must
exist irregardless of the summand order.

A
\ct{signed measure}{signedmeasure}
is an extended-real-valued
function on a
sigma algebra that is
(1) zero on the empty set and
(2) countably additive.
We call the result of the function
applied to a set in the sigma
algebra the
\ct{signed measure}{signedmeasure}
(or when no ambiguity arises, the
\ct{measure}{})
of the set.

A
\ct{finite}{finitesignedmeasure}
signed measure is one for
which the measure of every set
is finite.
This condition is equivalent
to the base set having finite
measure (see below).

\ssubsection{Notation}

We denote signed measures by $\mu$
a mnemonic for \say{measure.}
Let
$(X, \mathcal{A})$
be a measurable space
and let
$\mu: \mathcal{A} \to \eri$.
Then $\mu$ is a signed measure if
\begin{enumerate}
  \item $\mu(\emptyset) = 0$ and
  \item
  $\mu(\union_{i} A_i) =
    \lim_{n \to \infty}
      \sum_{k = 1}^{n} \mu(A_k)$
  for all disjoint families $\seq{A}$.
\end{enumerate}

\ssection{Infinite Exlcusivity}

\begin{prop}
\label{prop:finitesignedmeasures}
A signed measure is finite
if and only if it is finite
on the base set.

\begin{proof}
Let $(X,\SA)$ be a measurable space.
Let $\mu: \SA \to \eri$ be a signed measure.
($\Rightarrow$) If $\mu$ is finite, then $\mu(X)$
is finite since $X \in \mathcal{A}$.
($\Leftarrow)$
Next, suppose $\mu(X)$ is finite.
Let $A \in \SA$.
Then $X = A \union (X - A)$,
with these sets disjoint,
so by countable additivity
of $\mu$,
$\mu(X) = \mu(A) + \mu(X - A)$.
Since $\mu(X)$ finite, $\mu(A)$
and $\mu(X - A)$ are both finite.
\end{proof}
\end{prop}

\begin{prop}
A signed measure never takes
both positive infinity and
negative infinity.
\begin{proof}
Let $(X,\SA)$ be a measurable space.
Let $\mu: \SA \to \eri$ be a signed measure.
First, suppose $\mu(X)$ is finite,
Then by
Proposition~\ref{prop:finitesignedmeasures}
$\mu$ is finite for each $A \in \SA$.

Suppose $\mu(X) = \infty$.
Let $A \in \SA$.
As before,
$\mu(X) = \mu(A) + \mu(X - A)$.
Since $\mu(X) = +\infty$, then
both of $\mu(A)$ and $\mu(X-A)$
must be either finite or $+\infty$.
Argue similarly for $\mu(X) = -\infty$.
\end{proof}
\end{prop}

\strats
