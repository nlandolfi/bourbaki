\documentclass[12pt]{extarticle}
%\usepackage[margin=1.75in]{geometry}
\usepackage[a5paper,margin=.72in]{geometry}
\usepackage{graphicx}
\renewcommand{\baselinestretch}{1.25}
\setlength{\parskip}{0.5em}

\usepackage{titlesec}
%\titleformat*{\section}{\large\bfseries\sffamily}
%\titleformat*{\subsection}{\normalsize\bfseries\sffamily}
%\titleformat*{\subsubsection}{\large\bfseries}
%\titleformat*{\paragraph}{\large\bfseries}
%\titleformat*{\subparagraph}{\large\bfseries}

\titleformat*{\section}{\bfseries\sffamily}
\titleformat*{\subsection}{\small\bfseries\sffamily}
\titleformat*{\subsubsection}{\large\bfseries}
\titleformat*{\paragraph}{\large\bfseries}
\titleformat*{\subparagraph}{\large\bfseries}
\titlespacing*{\section}{0pt}{0.3cm}{0.1cm}
\titlespacing*{\subsection}{0pt}{0.3cm}{0.1cm}

%\renewcommand{\title}[1]{
%\begin{center}
%  \includegraphics[width=0.05\textwidth]{../../trademark}
%  \\
%  \vspace{0.20cm}
%  {\large \textsf{ #1 }}
%\end{center}
%}

\newcommand{\bpage}{
\vspace*{\fill}
\begin{center}
\includegraphics[width=1cm]{../../trademark}
\end{center}
\vspace{\fill}
}

\newcommand{\gpage}{
\begin{center}
\vspace*{\fill}
\includegraphics{graph}
\vspace{\fill}
\end{center}
}


\newcommand*{\vcenteredhbox}[1]{\begin{tabular}{@{}c@{}}#1\end{tabular}}
\renewcommand{\title}[1]{
\begin{center}
\hspace*{-1cm}
\vcenteredhbox{\vspace*{-0.1cm} \includegraphics[height=0.5cm]{../../trademark}\hspace*{0.1cm}}
\vcenteredhbox{\large \textsf{#1}}
\end{center}
}

\usepackage{hyperref}
\hypersetup{
  colorlinks,
  citecolor=black,
  filecolor=black,
  linkcolor=black,
  urlcolor=black
}


%\usepackage{ccfonts}% http://ctan.org/pkg/{ccfonts}
%\usepackage[cm]{sfmath}
\usepackage[T1]{fontenc}
\DeclareMathAlphabet{\mathbfsf}{\encodingdefault}{\sfdefault}{bx}{n}
\def\mathword#1{\mathop{\textup{#1}}}

\usepackage{caption}
\usepackage{subcaption}

%%%% MACROS %%%%%%%%%%%%%%%%%%%%%%%%%%%%%%%%%%%%%%%%%%%%%%%

\newcommand{\PM}{\mathbf{P}}

%%%%%%%%%%%%%%%%%%%%%%%%%%%%%%%%%%%%%%%%%%%%%%%%%%%%%%%%%%%


%!name:entire_functions
%!need:complex_analytic_functions
%!refs:yellow/IX/4

\section*{Definition}

An \t{entire function} is a complex function $f: \C  \to \C $ which is analytic for all $z \in \C $.

\blankpage
\renewcommand{\sstart}{}
\renewcommand{\strats}{\clearpage}
\renewcommand{\stitle}[1]{
  \begin{center}
    \includegraphics[width=0.1\textwidth]{../trademark}
    \\
    \vspace{0.5cm}
    {\fontfamily{cmss}\selectfont \LARGE
      \section{#1}
    }
  \end{center}
}
\renewcommand{\sbasic}{}
\renewcommand{\sinput}[1]{}
\renewcommand{\ssection}[1]{\subsection{#1}}
\renewcommand{\ssubsection}[1]{\subsubsection{#1}}



\usepackage{tocloft}
\setlength{\cftsecnumwidth}{3em}

\begin{document}

\begin{center}
  \includegraphics[width=0.1\textwidth]{../trademark}
  \\
  \vspace{0.5cm}
  \textsf{\Large The Bourbaki Project}

{\small \textsf{Edited by N. C. Landolfi}}
\end{center}

\vspace{\fill}

\begin{center}
  {\small \textsf{Edition 1 --- Summer 2021}} \\

  {\footnotesize \textsf{Printed in Menlo Park, California}}
\end{center}

\thispagestyle{empty}

\clearpage

\begin{center}
\textbf{Editor's Preface}
\end{center}

This project is one of the more ambitious with which I am affiliated.
Its two-fold goal is to explain mathematics to the novice and provide standardized language for the expert.
The reader should note that I have cut this edition under the pressure of time, in accordance with my annual goals for the project, and not because I felt we had reached a reasonable landmark, or that the content was particularly polished.

So then, what is here?
An attempt to talk about language, symbols, intangible objects and logical reasoning enough to get to a few principles having to do with intangible objects called sets and a few things you can build out of these sets.
The construction of real numbers and their relation to the lines of geometry becomes quite sparse toward the end, but the outline is included.
The $n$-dimensional real space is touched upon, and barely metric spaces, barely topological spaces.

On that last point, I should mention that the original goal for this edition was to reach topological spaces.
We agreed that this topic involved sufficiently abstract concepts which could test the project's assumptions.
We all agree, now, that there was much more to be said (to the novice) about topics much preliminary to topological spaces.
More than we anticipated.
We could, early on, define a topological space in terms of sets.
But we could not say why we cared.
And this idea, that we might say why we wanted a new concept before we introduced it, was an assumption we were testing with this project.

What were the other assumptions to be tested?
%What are these assumptions?
First, that the concepts and discussion could be so ordered that we only use prior concepts and discussion.
Second, that we could structure the book so that topics are treated by short, two-page sheets.
%Third, that a development could be given which justified the introduction of each new concept as we go, even if on superficial grounds (this is why each section has a \say{Why}).
Third, that such a treatment would be useful as a reference.
Fourth, that we could standardize language (perhaps formally) and use it in all theorems, definitions and proofs.

These traits would undoubtedly be useful.
The sheets could serve both as a beginner's guide and a reference.
When reaching for a particular topic, the prerequisites would be clear, fine-grained, and each one only two-pages long.
And a standardized language to facilitate understanding and communication is a centuries-old endeavor.
That no such text exists, to our knowledge, must indicate that its construction is accompanied by great difficulty.
But that is not to say impossible, and computers and screens may facilitate the process.

% The name of our project is a reference to a similarly themed endeavor of the last century.
% We have not space to outline our differences, but our readers will find us doing much more work to justify not only why a particular concept or theory is being defined, but even why anything is being said at all.
%The differences between our approach and theirs will be obvious to the reader.
%I have not space here to outline the differences between our approach and theirs, but the hypotheses of the previous paragraph are very nearly guiding principles for this project.
%As one example, Bourbaki began abstract, and then specialized.
%The treatment is meant to be \say{genetic,} in the words of Otto Toeplitz.
%In \textit{Calculus: A Genetic Approach}, Toeplitz explores the infinite process (taking limits, derivatives, integrals) by reviewing the sorts of things one wants to do (approximate quantities, model trajectory motion, compute areas) and then developing the mathematics.
% How then do we view this edition?
% I am known to many by an upbeat demeanor.
% True to my blood I will here include the formal statement that we have (if only just slightly) achieved our goal.
% Future editions will be the test.
% I would say we have just slightly achieved this goal, but the treatment is weak.
% This then is the influence of one of the assumptions of the project.

The text you hold is the first edition.
And we might call it a first attempt.
It is incomplete and with flaws.
But that is not to say useless.
There is visible in it the form of what is to come, if only you look at it properly.
And, in any case, it is time that we have a first edition.

\begin{flushright}
N.C.L. \\
16 July 2021 \\
Menlo Park, California \\
\end{flushright}

\clearpage

\begin{center}
\vspace*{-1.5cm}
  \sf To the Reader
\end{center}


%!name:introduction

The Bourbaki Project is a collection of documents describing mathematical concepts, terms, results and notation.

%% Each is named and labeled with prerequisites.
%% The prerequisites are the names of those documents which should be first read by the unaquainted reader.

\section*{Sheets}

We call these documents \t{sheets}.
They are only ever two-pages long and sometimes shorter.
They can be printed on a single sheet of paper, hence the name sheet.
In a book, they occupy two facing pages.
The decision to cap at two pages is arbitrary.
But our experience suggests it is convenient.

\section*{Prerequisites}

Each sheet is labeled with the names of those sheets which are its immediate prerequisites, with the names of those sheets for which it is an immediate prerequisite, and a diagram illustrating the dependencies between all its prerequisites.

For example, the sheet \sheetref{relations}{Relations}needs the sheet \sheetref{ordered_pairs}{Ordered Pairs}.
The reason, in this case, is that the concept of a relation is discussed using the concept of an ordered pair of objects.
And since the phrase \say{ordered pair of objects} makes sense only if we know what is meant by object (discussed in the sheet \sheetref{objects}{Objects}), the sheet \sheetref{relations}{Relations}needs the sheet \sheetref{objects}{Objects} also.
The reader unacquainted with ordered pairs and objects must read (at least) these two sheets before the sheet on relations.
In this case (and in every case) the prerequisites are naturally ordered.
 \sheetref{objects}{Objects}ought to be read first, before \sheetref{ordered_pairs}{Ordered Pairs}, before \sheetref{relations}{Relations}.
Such an ordering always exists because we ensure that if a sheet $X$ needs a sheet $Y$, then $Y$ can not need $X$ or any sheet that needs $X$.
A sheet is an immediate prerequisite if it is not prerequisite to any other prerequisite.

%% For convenience, we list only the immediate prerequisitesfor each sheet only those prerequisite sheets which are not a prerequisite to another prerequisite.
%% So if \sheetref{ordered_pairs}{Ordered Pairs} needs \sheetref{objects}{Objects}, then even though \sheetref{relations}{Relations} needs both \sheetref{ordered_pairs}{Ordered Pairs} and \sheetref{objects}{Objects}, we would only list \sheetref{ordered_pairs}{Ordered Pairs}.
%% a page are minimal. That is to say,
%% for sheet $X$ we only list the sheets
%% which are in the needs of $X$
%% and not needed by any other sheet
%% in the needs of $X$.
%% If $X$, $Y$ and $Z$ are as before,
%% then we only list $Y$ as $X$'s needs
%% because $Z$ is implicit (through $Y$).
%% The sheets and their needs are probably
%% best explored by browsing the project;
%% the index is
%% a reasonable starting point.

\section*{Preface}

The project is like a map.
The landmarks are sheets, or really concepts.
Walking is reading.
And you must walk along the trails specified by the prerequisites.

\subsection*{Aims}

Our primary aim is two-fold.
First, to provide useful exposition to teach the concepts to an unacquainted reader (here the prerequisites help).
And second, to serve as a reference for further work.
It is a welcomed concomitant that we better understand and develop the mathematical concepts ourselves.

\subsection*{Caveats}

There are two caveats.
First, we give only one path to concepts.
The point is that our way of structuring the concepts (and hence the prerequisites) is just one way, and there are many ways, since there are equivalent concepts, alternate proofs, and so on.
The second caveat is a wink.
These sheets are fiction.
They contain only ideas.
We have done our best to eliminate all false statements.
The game for the practical cogitator is to fit these puzzle pieces to reality.
%% In this edition, very little is said about fitting these puzzle pieces to reality.
%% But very little is said about fitting these puzzle pieces to reality.
%%\vfill
%%\begin{center}{\small The project is supported by
%%funds from the Department of
%%Defense of the United States of America
%%and Stanford University.}\end{center}


\clearpage

\tableofcontents
% \blankpage

\clearpage



\newpage
\stitle{Objects}
%%%% MACROS %%%%%%%%%%%%%%%%%%%%%%%%%%%%%%%%%%%%%%%%%%%%%%%

\newcommand{\PM}{\mathbf{P}}

%%%%%%%%%%%%%%%%%%%%%%%%%%%%%%%%%%%%%%%%%%%%%%%%%%%%%%%%%%%

%!name:objects

\ssection{Why}

We want to talk about things.

\ssection{Definition}

We use the word \t{object} with its usual sense in the English language.
If we can touch the object, we say that it is \t{tangible}.
% A \t{tangible} object is one that we can touch.
Otherwise, we say that the object is \t{intangible}.

\ssection{Examples}

We pick up a pebble for an example of a tangible object.
The pebble is the object.
We can hold it and we can touch it.
We can toss it back and forth.
And because we can touch it, the pebble is tangible.

We consider the color gray for an example of an intangible object.
The pebble may be gray.
Although we can touch the pebble, we can not touch the color.
At least in the usual English sense of the word \say{touch}.
The color gray is an object.
But since we can not touch it, the color gray is intangible.

The following sheets contain many more examples of such intangible objects.
In fact, they contain little else besides this.

\subsection{Names}

To discuss objects we give them \t{names}.
For example, \say{the pebble} or \say{the color gray}.

\say{Variables}

A single Latin letter regularly suffices.
To aid our memory, we tend to choose the letter mnemonically.

The use of letters to name objects is convenient, since they are short.
But we must take care when speaking of objects by their names that we know which object is referred to.


\ssubsection{Notation}

We use italics when writing the name.
We introduce a name by the word \say{let,} followed by the name in italics and then the word \say{be} followed by a description of the object the name refers to.
For example: let $a$ be an object.
Here the description is \say{an object}.

\blankpage


\newpage
\stitle{Identity}
%%%% MACROS %%%%%%%%%%%%%%%%%%%%%%%%%%%%%%%%%%%%%%%%%%%%%%%

\newcommand{\PM}{\mathbf{P}}

%%%%%%%%%%%%%%%%%%%%%%%%%%%%%%%%%%%%%%%%%%%%%%%%%%%%%%%%%%%

%!name:identity
%!need:objects

\ssection{Why}

We can give the
same object two different
names.

\ssection{Definition}

An object
\ct{is}{} itself.
If the object that two
names refer to is
the same,
then we say
that the first name
\ct{equals}{}
the second name.

\ssubsection{Notation}

We denote that
the object named $a$ and
the object named $b$ refer
to the same object
by $a = b$,
read aloud as:
\say{a is b.}
We denote that
the object $a$ and
$b$ refer to different
objects by $a \neq b$,
read aloud as:
\say{a is not b.}

We may also read the notation
$a = b$ aloud as \say{a
equals b}.
Other English readings
include: \say{a is the
same as b},
\say{a is equivalent
to b},
\say{a refers to
the same object
as b.}


\newpage
\stitle{Sets}
%%%% MACROS %%%%%%%%%%%%%%%%%%%%%%%%%%%%%%%%%%%%%%%%%%%%%%%

% use \set{stuff} for { stuff }
% use \set* for autosizing delimiters
\DeclarePairedDelimiter{\set}{\{}{\}}

\renewcommand{\emptyset}{\varnothing}

\newcommand{\SA}{\mathcal{A}}
\newcommand{\SB}{\mathcal{B}}
\newcommand{\SC}{\mathcal{C}}
\newcommand{\SD}{\mathcal{D}}
\newcommand{\SE}{\mathcal{E}}
\newcommand{\SF}{\mathcal{F}}
\newcommand{\SG}{\mathcal{G}}
\newcommand{\SH}{\mathcal{H}}
\newcommand{\SI}{\mathcal{I}}
\newcommand{\SJ}{\mathcal{J}}
\newcommand{\SK}{\mathcal{K}}
\newcommand{\SL}{\mathcal{L}}
\newcommand{\SM}{\mathcal{M}}
\newcommand{\SN}{\mathcal{N}}
\newcommand{\SO}{\mathcal{O}}
\newcommand{\SP}{\mathcal{P}}
\newcommand{\SQ}{\mathcal{Q}}
\newcommand{\SR}{\mathcal{R}}
%\newcommand{\SS}{\mathcal{S}}
\newcommand{\ST}{\mathcal{T}}
\newcommand{\SU}{\mathcal{U}}
\newcommand{\SV}{\mathcal{V}}
\newcommand{\SW}{\mathcal{W}}
\newcommand{\SX}{\mathcal{X}}
\newcommand{\SY}{\mathcal{Y}}
\newcommand{\SZ}{\mathcal{Z}}

%%%%%%%%%%%%%%%%%%%%%%%%%%%%%%%%%%%%%%%%%%%%%%%%%%%%%%%%%%%

%!name:sets
%!need:names
%!see:https://people.maths.ox.ac.uk/knight/lectures/b1st.html
%!refs:bert_mendelson/introduction_to_topology/theory_of_sets/sets_and_subsets
%!refs:bert_mendelson/introduction_to_topology/theory_of_sets/set_operations

\ssection{Why}

We want to talk about none, one, or several objects considered together, as an aggregate.

\ssection{Definition}

When we think of several objects considered as an intangible whole, or group, we call the intangible object which is the group a \t{set}.
We say that these objects \t{belong} to the set.
They are the set's \t{members} or \t{elements}.
They are \t{in} the set.

% \begin{principle}[Existence of Sets]
% 	Intangible groups exist.
% \end{principle}

A set may have other sets as its members.
This is subtle but becomes familiar.
We call a set which contains no objects \t{empty}.
Otherwise we call a set \t{nonempty}.

\ssection{Denoting a set}

Let $A$ denote a set.
Then $A$ is a name for an object.
That object is a set.
So $A$ is a name for an object which is a grouping of other objects.

\s{Belonging}

Let $a$ denote an object and $A$ denote a set.
So we are using the names $a$ and $A$ as placeholders for some object and some set, we do not particularly know which.
Suppose though, that whatever this object and set are, it is the case that the object belongs to the set.
In other words, the object is a member or an element of the set.
We say \say{The object denoted by $a$ belongs to the set denoted by $A$}.

\ssubsection{Not symmetric}

Notice that belonging is not symmetric.
Saying \say{the object denoted by $a$ belongs to the set denoted by $A$} does not mean the same as \say{the set denoted by $A$ belongs to the object denoted by $a$}
In fact, the latter sentence is nonsensical unless the object denoted by $a$ is also a set.

\ssubsection{Not transitive}

Let $a$ denote an object and let $A$ and $B$ both denote sets.
If the object denoted by $a$ is \say{a part of} the set denoted by $A$, and the set denoted by $A$ is \say{a part of} the set denoted by $B$, then usual English usage would suggest that $a$ is \say{a part of} the set denoted by $B$.
In other words, if a thing is a part of a second thing, and the second thing is part of a third thing, then the first thing is often said to be a part of the third thing.

The relation of belonging does not follow this familiar usage.
In contrast, if an object is an element of a set, that set may be an element of another set, but this does not mean that the the first object is also an element of that other set.
The upshot is that sets are nested: we can have intangible groups of intangible groups, and have them be different than the intangible group of all the members of each group.

% \blankpage

%Besides these two cases, we think of sets as containing two or more objects.

% \s{Equality of Sets}
%
% If two sets are e
%
% We denote the ob

%
%Suppose a set has few elements, and we can list them.
%If we give the objects names, then let us denote the set by listing the names of its elements between braces.
%For example, let $a$, $b$, and $c$ be distinct objects.
%Denote by $\set{a, b, c}$ the set containing these objects and only these objects.
%We can further compress notation, and denote this set of objects by $A$: so, $A = \set{a, b, c}$.
%Then $a \in A$, $b \in A$, and $c \in A$.
%Moreover, if $d$ is an object and $d \in A$, then $d = a$ or $d = b$ or $d = c$.
%
%Let $a$ be an object.
%Note that $a \neq \set{a}$.
%The left hand side, $a$, is the object $a$.
%The right hand side, $\set{a}$, is the set whose element is the object $a$.
%We distinguish the set containing one element from the element itself.

%If the elements of a set are
%so well-known that we can
%avoid ambiguity, then we can
%describe the set in English.
%To aid our memory,
%let us tend to name such sets
%mnemonically.
%For example,
%let $L$ be the set of Latin letters.

%We develop herein a language
%for specifying things by either
%listing them explicitly or
%by listing their defining properties.


\newpage
\stitle{Set Examples}
%%%% MACROS %%%%%%%%%%%%%%%%%%%%%%%%%%%%%%%%%%%%%%%%%%%%%%%

\newcommand{\PM}{\mathbf{P}}

%%%%%%%%%%%%%%%%%%%%%%%%%%%%%%%%%%%%%%%%%%%%%%%%%%%%%%%%%%%


\section*{Why}

By set we mean any aggregate or group, considered as a whole.
Examples are ubiquitous.

\section*{Examples}

A pack of wolves, a bunch of grapes, or a flock of pigeons are all examples of sets of things.
We can readily visualize each member of these sets.

The hairs on your head, the grains of sand on the beaches of Earth, the blades of grass in a field are all examples of sets.
Although we can not readily visualize all the elements at once, we can conceive of them, and visualize the elements one by one.

\blankpage

\newpage
\stitle{Set Equality}
%%%% MACROS %%%%%%%%%%%%%%%%%%%%%%%%%%%%%%%%%%%%%%%%%%%%%%%

\newcommand{\PM}{\mathbf{P}}

%%%%%%%%%%%%%%%%%%%%%%%%%%%%%%%%%%%%%%%%%%%%%%%%%%%%%%%%%%%

%!name:set_equality
%!need:identities
% wierd because statements needs sets
% %!need:sets
%!need:standardized_accounts

\ssection{Why}

When are two sets the same?

\ssection{Definition}

Given sets $A$ and $B$, if $A = B$ then every element of $A$ is an element of $B$ and every element of $B$ is an element of $A$.

\begin{account}[Joint Membership]
  \nameee{$A$}{$B$}{$x$}
  \have{set_equality:joint_membership:equality}{$A = B$}
  \have{set_equality:joint_membership:belonging}{$x \in B$}
  \thus{set_equality:joint_membership:conclusion}{$x \in A$}{\ref{set_equality:joint_membership:equality},\ref{set_equality:joint_membership:belonging}}
\end{account}



What of the converse?
Suppose every element of $A$ is an element of $B$ and every element of $B$ is an element of $A$.
Is $A = B$ true?
We define it to be so.
Two sets are \t{equal} if and only if every element of one is an element of the other.
In other words, two sets are the same if they have the same elements.
This statement is sometimes called the \t{axiom of extension}.
Roughly speaking, if we refer to the elements of a set as its \t{extension}, then we have declared that if we know the extension then we know the set.
A set is determined by its extension.

This definition gives us a way to argue that $A = B$ from the properties of the elements of $A$ and $B$.
It may not be obvious that the sets are the same.
We first argue that each element of $A$ is an element of $B$ and then argue that each element of $B$ is an element of $A$.
With these two implications, we use the axiom of extension to conclude that the sets are the same.


The logical statement is: $(((\forall x)(x \in A \implies x \in B) \land (\forall x)(x \in B \implies x \in A))) \implies (A = B)$
% We declare the affirmative.  Thus we can assert equality of sets.
Here is an example of applying that:

\begin{account}[Extension]
\namee{$A$}{$B$}
\have{set_equality:extension:first}{$(\forall x)((x \in A) \implies (x \in B))$}
\have{set_equality:extension:second}{$(\forall x)((x \in B) \implies (x \in A))$}
\thus{set_equality:extension:conclusion}{$A = B$}{\ref{set_equality:extension:first},\ref{set_equality:extension:second}}
\end{account}


% \ssubsection{Notation}
%
% As with any objects, we denote that $A$ and $B$ are equal
% by $A = B$.
% We denote that they are not equal by $A \neq B$.
% We denote the unique empty set by $\emptyset$.
%
%
% The axiom of extension is
% \[
%   A = B \Leftrightarrow (a \in A \implies a \in B) \land (b \in B \implies b \in A).
% \]
%

\ssection{A Contrast}

We can compare the axiom of extension
for sets and their elements with an
analogous statement
for human beings and their ancestors.

On the one hand, if two human beings are equal then they have the same ancestors.
The ancestors being the person's parents, grandparents, greatgrandparents, and so on.
This direction, same human implies same ancestors, is the analogue of the \say{only if} part of the axiom of extension.
It is true.
On the other hand, if two human beings have the same set of ancestors, they need not be the same human.
This direction, same ancestors implies same human, is the analogue of the \say{if} part of the axiom of extension.
It is false.
For example, siblings have the same ancestors but are different people.

We conclude that the axiom of extension
is more than a statement about equality.
It is also a statement about our notion of
belonging, of what it means
to be an element of a set, and what a set is.

\blankpage


\newpage
\stitle{Set Inclusion}
%%%% MACROS %%%%%%%%%%%%%%%%%%%%%%%%%%%%%%%%%%%%%%%%%%%%%%%

\newcommand{\PM}{\mathbf{P}}

%%%%%%%%%%%%%%%%%%%%%%%%%%%%%%%%%%%%%%%%%%%%%%%%%%%%%%%%%%%

%!name:set_inclusion
%!need:set_equality

\ssection{Why}

We want language for all of the elements of a first set being the elements of a second set.

\ssection{Definition}

Denote a set by $A$ and a set by $B$.
If every element of the set denoted by $A$ is an element of the set denoted by $B$, then we say that the set denoted by $A$ is a \t{subset} of the set denoted by $B$.
We say that the set denoted by $A$ is \t{included} in the set denoted by $B$.
We say that the set denoted by $B$ is a \t{superset} of the set denoted by $A$ or that the set denoted by $B$ \t{includes} the set denoted by $A$.
A set includes and is included in itself.

If the sets denoted by $A$ and $B$ are identical, then each contains the other.
% \begin{account}[]
%   \namee{$A$}{$B$}
%   \have{}{$A = B$}
%   \thus{}{$(\forall x)(x \in A \implies x \in B)$}
%   \thus{}{$(\forall x)(x \in B \implies x \in A)$}
% \end{account}
% are identical
If $A = B$, then $A$ includes $B$ and $B$ includes $A$.
The axiom of extension asserts the converse also holds.
If $A$ includes $B$ and $B$ includes $A$, then $A = B$.
In other words, if $A$ is a subset of $B$ and $B$ a subset of $A$, then $A = B$.

The empty set is a subset of every other set.
Suppose toward contradiction that $A$ were a set which did not include the empty set.
Then there would exist an element in the empty set which is not in $A$.
But then the empty set would not be empty.
We call the empty set and $A$ \t{improper subsets} of $A$.
All other subsets we call \t{proper subsets}.
In other words, $B$ is an improper subset of $A$ if and only if $A$ includes $B$, $B \neq A$ and $B \neq \emptyset$.

\ssubsection{Notation}
Given two sets $A$ and $B$, we denote that $A$ is included in $B$ by $A \subset B$.
We read the notation $A \subset B$ aloud as \say{A is included in B} or \say{A subset B}.
Or we write $B \supset A$, and read it aloud \say{B includes A} or \say{B superset $A$}.

In this notation, we express the axiom of extension
\[
  A = B \Leftrightarrow (A \supset B) \land (A \subset B).
\]
The notation $A \subset B$
is a concise symbolism for
the sentence
\say{every element of $A$ is an element of
$B$.} Or for the alternative notation
$a \in A \implies a \in B$.

\ssubsection{Properties}

Given a set $A$, $A \subset A$.
Like equality, we say that inclusion is \t{reflexive}.
Given sets $A$ and $B$, if $A \subset B$ and $B \subset C$ then $A \subset C$.
Like equality, we say that inclusion is \t{transitive}.
If $A \subset B$ and $B \subset A$, then $A = B$ (by the axiom of extension).
Unlike equality, which is symmetric, we say that inclusion is \t{antisymmetric}.

\ssubsection{Comparison with belonging}

Given a set $A$ inclusion is reflexive.
$A \subset A$ is always true.
Is $A \in A$ ever true?
Also, inclusion is transitive.
Whereas belonging is not.


\newpage
\stitle{Set Specification}
%%%% MACROS %%%%%%%%%%%%%%%%%%%%%%%%%%%%%%%%%%%%%%%%%%%%%%%

% use \Set{a}{b} for {a | b}
% use \Set* for autosizing delimiters
\DeclarePairedDelimiterX{\Set}[2]{\{}{\}}{#1 \nonscript\;\delimsize\vert\nonscript\; #2}

%%%%%%%%%%%%%%%%%%%%%%%%%%%%%%%%%%%%%%%%%%%%%%%%%%%%%%%%%%%
%!name:set_specification
%%!need:set_inclusion
%!need:set_equality

\ssection{Why}

We want to construct new sets out of old ones.
So, can we always construct subsets?

\ssection{Definition}

We will say that we can.
More specifically, if we have a set and some statement which may be true or false for the elements of that set, a set exists containing all and only the elements for which the statement is true.

Roughly speaking, the principle is like this.
We have a set which contains some objects.
Suppose the set of playing cards in a usual deck exists.
We are taking as a principle that the set of all fives exists, so does the set of all fours, as does the set of all hearts, and the set of all face cards.
Roughly, the corresponding statements are \say{it is a five}, \say{it is a four}, \say{it is a heart}, and \say{it is a face card}.
%Of course, the sets in question will not always be so simple.
%If

\begin{principle}[Specification]
	For any statement and any set, there is a subset whose elements satisfy the statement.
\end{principle}

We call this the \t{principle of specification}.
We call the second set (obtained from the first) the set obtained by \t{specifying} elements according to the sentence.
The principle of extension (see \sheetref{set_equality}{Set Equality}) says that this set is unique.
All  basic principles about sets (other than the principle of extension, see \sheetref{set_equality}{Set Equality}) assert that we can construct new sets out of old ones in reasonable ways.

% The basic principles of set theory other than the axiom of extension allow us to construct new sets out of existing ones.
% The fir
% The basis principles of set theory--- the axiom of extension, allow us to
% We assert that to every set and every sentence predicated of elements of the set there exists a second set (a subset of the first) whose elements satisfy the sentence.
% It is an consequence of the axiom of extension that this set is unique.

\ssubsection{Notation}

Let $A$ denote a set.
Let $s$ denote a statement in which the symbol $x$ and $A$ appear unbound.
We assert that there is a set, denote it by $B$, for which belonging is equivalent to membership in $A$ and $s$.
In other words,
\[
	(\forall x)((x \in B) \iff ((x \in A) \land s(x))).
\]
We denote $B$ by $\Set*{x \in A}{s(x)}$.
We read the symbol $\mid$ aloud as \say{such that.}
We read the whole notation aloud as \say{a in A such that...}
We call it \ct{set-builder notation}{setbuildernotation}.


%For example:
%\begin{account}[Example Specification]
%\name{$A$,$y$}
%\thus{set_specification:asdf:asdf}{$(\exists A')((x \in A') \iff (x \ne y)))$}{Axiom:Specification}
%\end{account}


%Consider the statement $x \not\in x$.
%Let $A$ denote a set.
%And define $A'$ by specifying the elements denoted by $x$ which satisfy $x \not\in x$.
%Then $A' \not\in A$.
%Since $x \in A'$ is equivalent to $x \not\in x$
%If $A' \in A$, then $A' \not\in A$.

% Set-builder notation avoids enumerating
% elements.
% This notation is really indispensable for
% sets which have many members, too many
% to reasonably write down.

\ssection{Nothing contains everything}

As an example of the principle of specification and an important consequence, consider the statement $x \not\in x$.
Using this statement and the principle of specification, we can prove that there is not set which contains every thing.

%TODO(next edition): explain this more.
\begin{proposition}
  No set contains all sets.\footnote{We might call such a set, if we admitted its existence, a \t{universe of discourse} or \t{universal set}.
  With the principle of specification, a \say{principle of a universal set} would give a contradiction (called \t{Russell's paradox}).}
\begin{proof}
  Suppose there exists a set, denote it $A$ which contains all sets.
  In other words, suppose $(\exists A)(\forall x)(x \in A)$.
  Use the principle of specification to construct $B = \Set{x \in A}{x \not\in x}$.
  So $(\forall x)(x \in B \iff (x \in A \land x \not\in x))$
  In particular, $(B \in B \iff (B \in A \land B \not\in B))$.
  So $B \not \in A$.
% \begin{blankaccount}
%   \namee{$A$}{$B := \Set{x \in A}{x \not\in x}$}
%   \have{set_specification:noU:membership}{$(B \in B) \iff ((B \in A) \land B \not\in B)$}
%   \thus{set_specification:noU:conclusion}{$B \notin A$}{\ref{set_specification:noU:membership}}
% \end{blankaccount}
\end{proof}
\end{proposition}

% In other words, nothing contains everything.


%TODO(next edition) bring back example? split out universe?
%\ssection{Example}
%
%For example, let $a, b, c, d$
%be distinct objects.
%Let $A = \set{a, b, c, d}$.
%Then
%$\Set*{x \in A}{x \neq a}$
%is the set $\set{b, c, d}$
%
%Now let $B$ be an arbitrary
%set.
%The set $\Set*{b \in B}{b \neq b}$
%specifies the empty set.
%Since the statement $b \neq b$ is
%false for all objects $b$.

%!TODO: russel's paradox? p 6 of halmos



\newpage
\stitle{Set Unions}
\newcommand{\union}{\,\cup\,}
\newcommand{\bunion}{\bigcup}

%!name:set_unions
%%!need:set_specification (via empty_set)
% TODO(next edition): drop empty set
%!need:empty_set
% need pairs for properties, could maybe drop?
%!need:unordered_pairs
%!refs:paul_halmos/naive_set_theory/section_04

\ssection{Why}

Can we combine sets?

\ssection{Definition}

We say yes. 
For example, if we have a first set denoted $A$ and a second set denoted $B$, then we want a third set including all the elements of the set denoted by $A$ and the elements of the set denoted by $B$.
If an object appears in the set denoted by $A$ and in the set denoted by $B$, it appears in the new set.
If an object appears in one set but not the other, it appears in the new set.
Indeed, if we have a set of sets, the same should hold.

\begin{principle}[Union]
	Given a set of sets, there exists a set which contains all elements which belong to any of the sets.	
\end{principle}
We call this the \t{principle of union}.
If we have one set and another, the axiom of unions says that there exists a set which contains all the elements that belong to at least one of the former or the latter.

The set guaranteed by the principle of union may contain more elements than just those which are elements of a member of the the given set of sets.
No matter: apply the axiom of specification (see \sheetref{set_specification}{Set Specification}) to form the set which contains only those elements which are appear in at least one of any of the sets.
The set is unique by the principle of extension.
We call that unique set \t{the union} of the sets.

\ssection{Notation}

Let $\CA$ be a set of sets.
We denote the union of $\CA$ by $\bigcup \CA$.
So 
\[
	(\forall x)((x \in (\bigcup\CA)) \iff (\exists A)((A \in \CA) \land x \in A)).
\]

\ssection{Simple Facts}

It is reasonable for the union of the empty set to be empty and for the union of the singleton of a set to be itself.

\begin{prop}
	$\bigcup \emptyset = \emptyset$
\end{prop}
\begin{proof}
Immediate\footnote{Future editions will include the account.}
%	\begin{caccount}
%		\chave{}{$(\forall x)(x \in (\union \emptyset)) \iff (\exists y)(y \in \emptyset \land x \in y)$};
%		\cthus{set_unions:myref}{$\neg(\exists y)(y \in \emptyset) \implies (\forall x)(x \not\in (\union \emptyset))$}{contrapositive};
%		\cthus{}{$\union\emptyset = \emptyset$}{\lref{set_unions:myref} and Def}.
%	\end{caccount}
\end{proof}

\begin{prop}
	$\bigcup \set{A} = A$
\end{prop}
\begin{proof}
%	\begin{caccount}
%		\chave{}{$(\forall x)(x \in (\union \set{A})) \iff (\exists y)(y \in \set{A} \land x \in y)$};
%		\cthus{set_unions:}{$\neg(\exists y)(y \in \emptyset) \implies (\forall x)(x \not\in (\union \emptyset))$}{contrapositive};
%		\cthus{}{$\union\emptyset = \emptyset$}{\lref{myref}}.
%	\end{caccount}
Immediate\footnote{Future editions will include the account.}
\end{proof}

\newpage
\stitle{Ordered Pairs}
%%%% MACROS %%%%%%%%%%%%%%%%%%%%%%%%%%%%%%%%%%%%%%%%%%%%%%%

\newcommand{\cross}{\times}

\newcommand{\op}[1]{\left(#1\right)}
\newcommand{\tuple}[1]{\left(#1\right)}
\newcommand{\tu}[1]{\left(#1\right)}

%%%%%%%%%%%%%%%%%%%%%%%%%%%%%%%%%%%%%%%%%%%%%%%%%%%%%%%%%%%

%!name:ordered_pairs
%!need:unordered_pairs
%!refs:paul_halmos/naive_set_theory/section_06

\ssection{Why}

We want to order two objects.

\ssection{Definition}

Let $a$ and $b$ denote objects.
The \ct{ordered pair}{pair} of $a$ and $b$ is the set $\set{\set{a}, \set{a, b}}$.
The \t{first coordinate} of $\set{\set{a}, \set{a, b}}$ is the object denoted by $a$ and the \t{second coordinate} is the object denoted by $b$.


\ssubsection{Notation}

We denote the ordered pair $\set{\set{a},\set{a,b}}$ by $(a, b)$.
\ssection{Equality}

To agree with our inuition, an if two ordered pairs are the same, they shoulld have the same objects in the same place.
And conversely, if they have the same objects in the same place, they should be the same.
We can prove that our definition of an ordered pair (as a set, see \sheetref{ordered_pairs}{Ordered Pairs}) agrees with this intuition.
In other words,

\begin{proposition}
  $(a, b) = (c, d)$ if and only if $a = b$ and $c = d$.
\end{proposition}




\newpage
\stitle{Relations}
\newcommand{\dom}{\mathword{dom}}
\newcommand{\ran}{\mathword{range}}
\newcommand{\range}{\ran}
%!name:relations
%!need:cartesian_products
%!refs:paul_halmos/naive_set_theory/section_07
%!refs:bert_mendelson/introduction_to_topology/theory_of_sets/relations

\ssection{Why}

How can we relate the elements of two sets?

\ssection{Definition}

A \t{relation} is a set of ordered pairs (see \sheetref{ordered_pairs}{Ordered Pairs}).
So if an object $z$ is an element of a relation, there exists two other objects $x, y$ so that $z = (x, y)$.

The \t{domain} of a relation is the set of all elements which appear as the first coordinate of some ordered pair of the relation (the projection onto the first coordinate, see \sheetref{ordered_pair_projections}{Ordered Pair Projections})
The \t{range} of a relation is the set of all elements which appear as the second coordinate of some ordered pair of the relation (the projection onto the second coordinate).

When the domain of a relation $R$ is a subset of $X$ and the range is a subset of $Y$, we say $R$ is \t{from $X$ to $Y$} or \t{between} $X$ and $Y$.
If $X = Y$, then $R$ speak of a relation \t{in} or \t{on} $X$.

% Let $A$ and $B$ be two nonempty sets.
% A relation on $A$ and $B$ is a subset of $A \cross B$.
% Let $C$ be a nonempty set.
% A relation on a $C$ is a subset of $C \cross C$.
%
% Let $a \in A$ and $b \in B$.
% The ordered pair $(a, b)$ may or may not be in a relation on $A$ and $B$.
% Also notice that if $A \neq B$, then $(b, a)$ is not a member of the product $A \cross B$, and therefore not in any relation on $A$ and $B$.
% If $A = B$, however, it may be that $(b, a)$ is in the relation.

\ssubsection{Notation}

If $R$ is a relation, we express that $(x, y) \in R$ by writing $x\,R\,y$, which we read as \say{$x$ is in relation $R$ to $y$}.
We denote the domain of $R$ by $\dom R$ and the range of $R$ by $\ran R$.

\ssection{Examples}

For an uninteresting relation, consider the empty set.
In the empty (set) relation, no object is related to any other.
Both the domain and range of $\emptyset$ are $\emptyset$.
For another simple relation, consider the product of any two sets $X$ and $Y$.
In $X \times Y$, all objects are related.
The domain is $X$ and the range is $Y$.

For a more interesting example, consider the set
\[
  R := \Set{(x, y) \in X \times X}{x = y}.
\]
This relation is the relation of equality (see \sheetref{identities}{Identities}) between two objects.
Here $x\,R\,y \iff x = y$.
$\dom R = \ran R = X$.
Another similar example is if we consider the set $X$ and $\powerset{X}$, and the relation
\[
  R := \Set{(x, y) \in X \times \powerset{X}}{x \in y}.
\]
This relation is the relation of belonging (see \sheetref{sets}{Sets}).
Here $x\,R\,y \iff x \in y$.
Here $\dom R = X$ and $\ran R = \powerset{X}$.

% \ssubsection{Notation}
% Let $A$ and $B$ be nonempty sets
% with $a \in A$ and $b \in B$.
% Since relations are sets,
% we can use upper case Latin letters.
% Let $R$ be a relation on $A$ and $B$.
% We denote that $(a, b) \in R$ by
% $a R b$, read aloud as
% \say{a in relation $R$ to b.}
%
% When $A = B$, we tend to use other symbols instead of letters.
% For example,
% $\sim$, $=$, $<$,
% $\leq$, $\prec$, and $\preceq$.

\ssection{Properties}

Often relations are defined over a single set, and there are a few useful properties to distinguish.

A relation is \t{reflexive} if every element is related to itself.
A relation is \t{symmetric} if two objects are related regardless of their order.
% A relation is \t{antisymmetric}{antisymmetric} if two different objects are related only in one order, and never both.
A relation is \t{transitive} if a first element is related to a second element and the second element is related to the third element, then the first and third element are related.
Equality is reflexive, symmetric and transitive whereas belonging is neither.
Exercise: what is inclusion?


% \ssubsection{Notation}
%
% Let $R$ be a relation on
% a non-empty set $A$.
% $R$ is reflexive if
% $$(a, a) \in R$$
% for all $a \in A$.
% $R$ is transitive if
% $$(a, b) \in R \land (b, c) \in R \implies (a, c) \in R$$
% for all $a, b, c \in A$.
% $R$ is symmetric if
% $$(a, b) \in R \implies (b, a) \in R$$
% for all $a, b \in A$.
% $R$ is anti-symmetric if
% $$(a, b) \in R \implies (b, a) \not\in R$$
% for all $a, b \in A$.
%
% \blankpage


\newpage
\stitle{Functions}
%%%% MACROS %%%%%%%%%%%%%%%%%%%%%%%%%%%%%%%%%%%%%%%%%%%%%%%

\newcommand{\id}{\mathword{id}}

%%%%%%%%%%%%%%%%%%%%%%%%%%%%%%%%%%%%%%%%%%%%%%%%%%%%%%%%%%%
%!name:functions
%!need:relations
%!refs:paul_halmos/naive_set_theory/section_08

\ssection{Why}

We want a notion for a correspondence between two sets.

\ssection{Definition}

A \t{function} $f$ \t{from} a set $X$ \t{to} a set $Y$ is a relation (see \sheetref{relations}{Relations}) whose domain is $X$ and whose range is a subset of $Y$ such that for each $x \in X$, there exists a unique $y \in Y$ so that $(x, y) \in f$.

We call the unique $y \in Y$ the \t{result} of the function \t{at} the \t{argument} $x$.
We call $Y$ the \t{codomain}.
If the range is $Y$ we say that $f$ is a function from $X$ \t{onto} $Y$.
If distinct elements of $X$ are mapped to distinct elements of $y$, we say that the function is \t{one-to-one}.

% We know functions by how the associate elements of their codomain with elements of their domain.
We say that the function \t{maps} elements from the domain to the codomain.
Since the word function and the verb \say{maps} connote activity, some authors refer to the concept that we have defined as a function as the \t{graph} of a function---namely, the set of ordered pairs which that function produces---and leave the concept of function undefined.

\ssubsection{Notation}

Let $X$ and $Y$ denote sets.
We denote a function named $f$ whose domain is $X$ and whose codomain is $Y$ by $f: X \to Y$.
We read the notation aloud as \say{$f$ from $X$ to $Y$}.
We denote the set of all functions from $X$ to $Y$ (which is a subset of $\powerset{(X \times Y)}$) by $Y^{X}$.
A less standard but equally good notation is $X \to Y$, read aloud as \say{$A$ to $B$}.
Using the notations introduced so far, we denote that $f \in (A \to B)$ by $f: A \to B$.
We tend to denote function by lower case latin letters, especially $f$, $g$, and $h$.
$f$ is a mnemonic for function and $g$ and $h$ are nearby.

Let $f: A \to B$.
For each element $a \in A$, we denote the result of applying $f$ to $a$ by $f(a)$, read aloud \say{f of a.}
We sometimes drop the parentheses, and write the result as $f_a$, read aloud as \say{f sub a.}
Let $g: A \times B \to C$.
We often write $g(a,b)$ or $g_{ab}$ instead of $g((a,b))$.
We read $g(a, b)$ aloud as \say{g of a and b}.
We read $g_{ab}$ aloud as \say{g sub a b.}

\s{Examples}

If $X \subset Y$, the function $\Set{(x, y) \in X \times Y}{x = y}$ is the \t{inclusion map} of $X$ into $Y$.
We often introduce such a function as \say{the function from $X$ to $Y$ defined by $f(x) = y$}.
We mean by this that $f$ is a function and that we are specifying the appropraite ordered pairs using the statement, called \t{argument-value notation}.
The inclusion map of $X$ into $X$ is called the \t{identity map} of $X$.
If we view the identity map as a relation on $X$, it is the relation of equality on $X$.

The functions $f: (X \times Y) \to X$ defined by $f(x, y) = x$ is the \t{pair projection} of $X \times Y$ ono $X$.
Similarly $g: (X \times Y) \to Y$ defined by $g(x, y) = y$ is the pair projection of $X \times Y$ onto $Y$.
% The range of $f$ is the first coordinate projection of $f$ and the range of $g$ is the second coordinate projection of $f$ (see \sheetref{ordered_pair_projections}{Ordered Pair Projections}).
The identity map is one-to-one and onto, the inclusion maps are one-to-one but not always onto, and the pair projections are usually not one-to-one.


\newpage
\stitle{Operations}
%%%% MACROS %%%%%%%%%%%%%%%%%%%%%%%%%%%%%%%%%%%%%%%%%%%%%%%

\newcommand{\PM}{\mathbf{P}}

%%%%%%%%%%%%%%%%%%%%%%%%%%%%%%%%%%%%%%%%%%%%%%%%%%%%%%%%%%%

%!name:operations
%!need:functions

\ssection{Why}

We want to \say{combine}
elements of a set.

\ssection{Definition}

Let $A$ be a non-empty set.
An \ct{operation}{operation} on $A$ is
a function from ordered pairs of elements
in the set to the same set.
We use operations to combine the elements.
We operate on pairs.
%An \ct{algebra}{algebra} is a set and
%an operation.
%We call the set the
%\ct{ground set}{groundset}.


\ssubsection{Notation}

Let $A$ be a set and
$g: A \times A \to A$.
We tend to forego the notation
$g(a, b)$ and
write $a\,g\,b$ instead.
We call this
\ct{infix notation}{infixnotation}.

Using lower case latin
letters for elements and for operators
confuses, so we tend to use
special symbols for operations.
For example,
$+$, $-$, $\cdot$, $\circ$, and $\star$.

Let $A$ be a non-empty set
and $+: A \times A \to A$ be
an operation on $A$.
According to the above paragraph,
we tend to write
$a+b$ for the result of applying $+$
to $(a,b)$.


\newpage
\stitle{Algebras}
%%%% MACROS %%%%%%%%%%%%%%%%%%%%%%%%%%%%%%%%%%%%%%%%%%%%%%%

\newcommand{\PM}{\mathbf{P}}

%%%%%%%%%%%%%%%%%%%%%%%%%%%%%%%%%%%%%%%%%%%%%%%%%%%%%%%%%%%

%!name:algebras
%!need:operations

\ssection{Why}

We name a set together with an operation.

\ssection{Definition}

An \t{algebra} is an ordered pair whose first element is a non-empty set and whose second element is an operation on that set.
The \t{ground set} of the algebra is the set on which the operation is defined.

\ssubsection{Notation}

Let $A$ be a non-empty set and let $+: A \times A \to A$ be an operation on $A$.
As usual, we denote the ordered pair by $(A, +)$.

\blankpage


\newpage
\stitle{Natural Numbers}
% mnemonic for (s)et(suc)cesor
\newcommand{\ssuc}[1]{{#1}^{+}}
\newcommand{\N}{\mathbfsf{N}}
\newcommand{\upto}[1]{\set{1, 2, \dots, #1}}


% 
%Notes on starting at 1; Spivak does in Calculus

\section*{Why}

What are numbers?
We want to count, forever.
Does a set exist which contains zero, and one, and two, and three, and all the rest?

\section*{Definition}

In \sheetref{successor_sets}{Successor Sets}, we said ``and we continue as usual using the English language...'' in our definition of zero, and one and two and three.
Can this really be carried on and on?
We will say yes.
We will say that there exists a set which contains zero and contains the successor of each of its elements.

\begin{principle}[Natural Numbers]
A set which contains 0 and contains the successor of each of its elements exists.
\end{principle}

This principle is sometimes called the \t{principle of infinity} (or \t{axiom of infinity}).

We want this set to be unique.
The principle says one successor set exists, but not that it is unique.
To see that it is unique, notice that the intersection of a nonempty family of successor sets is a successor set.\footnote{This account will be expanded in future editions.}
Consider the intersection of the family of all successor sets.
The intersection is nonempty by the principle of infinity (see \sheetref{intersection_of_empty_set}{Intersection of Empty Set}for this subtlety).
The principle of extension guarantees that this intersection, which is a successor set contained in every other successor set, is unique.
We summarize:

\begin{proposition}[Minimal Successor Set]

\label{natural_numbers:proposition:omega}There exists a unique smallest successor set.
\end{proposition}

The \t{set of natural numbers} is the minimal successor set.
A \t{natural number} (or \t{number}, \t{natural}) is an element of this minimal successor set.

\subsection*{Notation}


% todo fix 
%    Proposition~\ref{natural_numbers:proposition:omega} both refs and the underscore in  natural_numbers
%    which gets escaped in tex
%    

We denote the unique smallest successor set by $\omega $.\footnote{We use this notation to follow many authorities on the subject, and to meet the exigencies of time in producing this first edition.
Future editions are likely to rework the treatment.}
We denote the set of natural numbers without 0 by $\N  $, a mnemonic for natural.
In other words $\N   = \omega  - \set{0}$.
We often denote elements of $\omega $ or $\N  $ by $n$, a mnemonic for number, or $m$, the letter before $m$ in the conventional ordering of the \sheetref{letters}{Latin alphabet}(see \sheetref{letters}{Letters}).

We denote the natural numbers up to $n$ by $\upto{n}$.
Recall that $n$ \textit{is a set}.
In other words, we have defined $n$ so that $n - \set{0} = \upto{n}$.

%% We are saying, in the language of sets, that the essence
%% of counting is starting with one and adding one repeatedly.
%% \begin{proposition}
%% The intersection of every nonempty family of successor sets is a successor set.
%% \end{proposition}
%%
%% So the intersection of all the successor sets is a successor set.
%% Since the intersection of any two sets is contained in the two sets, the intersection of the family of successor sets is contained in every other succesor set.
%%
%%
%% The
%% successor function
%% is the correspondence between elements
%% of the natural numbers and their successors.
%% Its domain and codomain is the set of natural numbers.
%% It is a one-to-one correspondence.
%% It is not onto, however, since the element one
%% has no successor.
%%%!name:zero
%%%!need:equation_solutions
%%%!need:natural_numbers
%%
%% \ssection{Why}
%%
%% If I am holding
%% three pebbles, and I have three
%% in my left hand, how many
%% might I have in my right hand?
%% None, of course!
%%
%% In the notation we have developed
%% we find solutions of
%% \[
%%   3 + n = 3,
%% \]
%% where $n$ is a natural number.
%% Unfortunately, any natural number
%% added to three is a number different
%% than three.
%% So there is no natural number
%% such that this equation holds.
%%
%% How can we expand our algebra
%% so that we can express this
%% new, but common, situation in
%% our language?
%%
%% \ssection{Definition}
%%
%% We consider a superset
%% of the natural numbers with
%% one additional element.
%% We call this new elemenet
%% \t{zero}
%% and we call this new set
%% the
%% \t{natural numbers with zero}.
%%
%% \ssubsection{Extending Arithmetic}
%%
%% We extend the algebra on
%% the natural numbers to an
%% algebra on the natural numbers
%% with zero.
%%
%% We define an extension of addition,
%% also called
%% \t{addition}.


\newpage
\stitle{Integer Numbers}
%%%% MACROS %%%%%%%%%%%%%%%%%%%%%%%%%%%%%%%%%%%%%%%%%%%%%%%

\newcommand{\PM}{\mathbf{P}}

%%%%%%%%%%%%%%%%%%%%%%%%%%%%%%%%%%%%%%%%%%%%%%%%%%%%%%%%%%%


\section*{Why}

We want to subtract numbers.\footnote{Future editions will change this why. In particular, by referencing \sheetref{inverse_elements}{Inverse Elements}and the lack thereof in $\omega $.}

\section*{Definition}

Consider the set $\omega  \times \omega $.
This set is the set of ordered pairs of $\omega $.
In other words, the ordered pairs of natural numbers.

We call two such pairs $(a, b)$ and $(c, d)$ of $\omega \times  \omega $ \t{integer equivalent} if
\[
a + d = b + c
\]
Briefly, the intuition is that $(a, b)$ represents $a$ less $b$, or in the usual notation ``$a - b$''.\footnote{This account will be expanded in future editions.}
So this equivalence relation says these two are the same if $a - b = c - d$.
Rearranging gives $a + d = b + c$.

\begin{proposition}
Integer equivalence is an equivalence relation.\footnote{The proof is straightforward. It will be included in future editions.}
\end{proposition}

The \t{set of integer numbers} is the set of equivalence classes (see \sheetref{equivalence_relations}{Equivalence Relations}) under integer equivalence on $\omega  \times  \omega $.
We call an element an \t{integer number} (or \t{integer}).

\subsection*{Notation}

We denote the set of integers by $\Z $.
If we denote integer equivalence by $\sim$ then $\Z  = (\omega \times \omega )/\mathord{\sim}$.

\blankpage
%macros.tex
%\newcommand{\Z}{\mathbfsf{Z}}


\newpage
\stitle{Groups}
%%%% MACROS %%%%%%%%%%%%%%%%%%%%%%%%%%%%%%%%%%%%%%%%%%%%%%%

\newcommand{\PM}{\mathbf{P}}

%%%%%%%%%%%%%%%%%%%%%%%%%%%%%%%%%%%%%%%%%%%%%%%%%%%%%%%%%%%


\section*{Why}

% TODO: the genetic approach: polynomial roots and Galois? 

We further drop conditions on the structure of the binary operations, and study only the algebraic structure of addition over the integers.

\section*{Definition}

A \t{group} is an \sheetref{operations}{algebra}$(G, \circ)$ for which $\circ: G \times  G \to G$ is associative, has an identity element in $G$, and has inverse elements.
A group is a \t{commutative group} (or \t{abelian group}) if $\circ$ is commutative.
A group is a \t{finite group} if $G$ is a finite set.

\section*{Additive groups}

Suppose that $(R, +, \cdot )$ is ring.
Then $(R, +)$ is a commutative group.
Conversely, suppose $(G, +)$ is a commutative group.
Define multiplication on $S$ by $a\cdot b = 0$ for all $a, b \in R$.
Then $(S, +, \cdot )$ is a ring, called the \t{zero ring} of $(G, +)$.
For this reason, it is customary to write $+$ for the operation $\circ$ when handling commutative groups.

\section*{Group Operations}

Along with the group operation, we call the function which maps an element to its inverse element the \t{group operations}.

\blankpage
%macros.tex
%%%%% MACROS %%%%%%%%%%%%%%%%%%%%%%%%%%%%%%%%%%%%%%%%%%%%%%%
%%%%%%%%%%%%%%%%%%%%%%%%%%%%%%%%%%%%%%%%%%%%%%%%%%%%%%%%%%%%


\newpage
\stitle{Fields}
%%%% MACROS %%%%%%%%%%%%%%%%%%%%%%%%%%%%%%%%%%%%%%%%%%%%%%%

\newcommand{\PM}{\mathbf{P}}

%%%%%%%%%%%%%%%%%%%%%%%%%%%%%%%%%%%%%%%%%%%%%%%%%%%%%%%%%%%


%!name:fields
%!need:rational_numbers
%!need:rings
%!refs:peter_cameron/introduction_to_algebra

\section*{Why}

We generalize the algebraic structure of addition and multiplication over the rationals.

\section*{Definition}

A \t{field} is a ring $(R, +, \cdot  )$ for which $\cdot  $ is commutative (i.e., $ab = ba$ for all $a, b \in R$) and $\cdot  $ has inverses for all elements except $0$.
In this case, we refer to \t{field addition} and \t{field multiplication}.

We also give names to the objects which have one of these additional properties or the other.
A ring which for which multipliation is commutative is called a \t{commutative ring}.
Note that a ring is \textit{always} commutative with respect to addition, here the term commutative refers to multiplication.
A ring for which there are inverse elements, excepting 0, is called a \t{division ring}).

\subsection*{Notation}

Since our guiding example is the set of rationals $\Q  $ with addition and multiplication defined in the usual manner, and we use a bold font for $\Q  $, we tend to denote an arbitrary field by $\F  $, a mnemonic for ``field.''

\section*{Field operations}

Along with field addition and field multiplication, we call the function which takes an element of a field to its additive inverse and the function which takes an element of a field to its multiplicative inverse the \t{field operations}.
\blankpage


\newpage
\stitle{Real Numbers}
%%%% MACROS %%%%%%%%%%%%%%%%%%%%%%%%%%%%%%%%%%%%%%%%%%%%%%%

\newcommand{\PM}{\mathbf{P}}

%%%%%%%%%%%%%%%%%%%%%%%%%%%%%%%%%%%%%%%%%%%%%%%%%%%%%%%%%%%

%!name:real_numbers

\blankpage


\newpage
\stitle{Absolute Value}
% absolute_value macros
% \newcommand{\abs}[1]{\left|#1\right|}

%!name:absolute_value
%!need:length
%!mcro:length

\ssection{Why}

We want a notion
of distance between
elements of the real
line.

\ssection{Definition}

We define a function
mapping a real number
to its length from zero.




\clearpage

% \stitle{Financial Support}
%
% The Bourbaki project is partially supported by funds from the United States Department of Defense and Stanford University.
%
% \clearpage

\blankpage

\begin{center}
\vspace*{-1.5cm}
  \sf Note on Printing
\end{center}
\noindent\noindent The font is \textit{Computer Modern.}
The document was typeset using \LaTeX.
This pamphlet was printed, folded, and stitched in Menlo Park, California.

\clearpage
% \bpage
% \newcommand{\bpage}{
\vspace*{\fill}
\begin{center}
\includegraphics[width=0.15\textwidth]{../trademark}
\end{center}
\vspace{\fill}
\thispagestyle{empty}
% }
\end{document}

