
%!name:entire_functions
%!need:complex_analytic_functions
%!refs:yellow/IX/4

\section*{Definition}

An \t{entire function} is a complex function $f: \C  \to \C $ which is analytic for all $z \in \C $.

\blankpage
\sbasic

\sinput{../sets/macros.tex}
\sinput{../extended_real_numbers/macros.tex}
%%%% MACROS %%%%%%%%%%%%%%%%%%%%%%%%%%%%%%%%%%%%%%%%%%%%%%%

\newcommand{\PM}{\mathbf{P}}

%%%%%%%%%%%%%%%%%%%%%%%%%%%%%%%%%%%%%%%%%%%%%%%%%%%%%%%%%%%


\sstart

\stitle{Signed Set Decomposition}

\ssection{Why}

Given a signed measure,
can we split
the base set into two sets,
one with positive measure
and one with negative measure?

\ssection{Definition}

By "positive" and "negative"
we mean "non-negative"
and "non-positive."
Let $(X, \SA)$ be
a measurable space.
Let $\mu: \SA \to \eri$
be a signed measure.

A \ct{positive set}{}
is a measurable set
with the property that
each of its subsets have
non-negative measure
under $\mu$.
A \ct{negative set}{}
is a measurable set
with the property that
each of its subsets have
non-positive measure
under $\mu$.
A \ct{signed-set decomposition}{}
of $X$ under $\mu$
is a partition of $X$
into a positive and a negative set.

\ssubsection{Existence}
\begin{prop}
Let $(X, \SA)$ be
a measurable space.
Let $\mu: \SA \to \eri$
be a signed measure.
There exists a signed-set
decomposition of $X$ under $\mu$.
  \begin{proof}
    TODO
  \end{proof}
\end{prop}

\ssubsection{Uniqueness}


\ssubsection{Notation}

We usually denote a positive
set by $P$ and a negative set
by $N$.
When we say \say{let $(P, N)$
be a signed-set decomposition
of $X$ under $\mu$},
we mean that $P$ is
the positive set and $N$
is the negative set.

\strats
