
%!name:entire_functions
%!need:complex_analytic_functions
%!refs:yellow/IX/4

\section*{Definition}

An \t{entire function} is a complex function $f: \C  \to \C $ which is analytic for all $z \in \C $.

\blankpage
\sbasic

%%%% MACROS %%%%%%%%%%%%%%%%%%%%%%%%%%%%%%%%%%%%%%%%%%%%%%%

\newcommand{\PM}{\mathbf{P}}

%%%%%%%%%%%%%%%%%%%%%%%%%%%%%%%%%%%%%%%%%%%%%%%%%%%%%%%%%%%


\sstart

\stitle{Almost Everywhere}

\ssection{Why}

We treat properties failing
on a set of measure zero
as
though they occur everyhwere;
especially in discussions of
convergence.

\ssection{Definition}

A subset of
the base set of a measure
space is
\ct{negligible}{}
if there exists a
measurable set
with measure zero
containing
the subset.
Negligible sets
need not be
measurable.
If the measure space
is complete then every
negligible set is
measurable.

A property holds
\ct{almost everywhere}{almosteverywhere}
with respect to a measure
on a measure space if
the set of elements
of the base set on which
the property does not hold
is negligible.

\ssubsection{Notation}

Let
$(X, \mathcal{A}, \mu)$
be a measure space.
A set
$N \subset X$ is
negligible if there
exists $A \in \mathcal{A}$
with $N \subset A$ and
$\mu(N) = 0$.

We abbreviate almost
everywhere as a.e.
We say that a property
holds a.e., read
\say{almost everywhere}.
If the measure $\mu$ is
not clear from context,
we say that the property
holds almost everywhere
$[\mu]$ or
$\mu$-a.e., read
\say{mu almost everywhere.}

\strats
