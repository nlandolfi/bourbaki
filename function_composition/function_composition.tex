
%!name:entire_functions
%!need:complex_analytic_functions
%!refs:yellow/IX/4

\section*{Definition}

An \t{entire function} is a complex function $f: \C  \to \C $ which is analytic for all $z \in \C $.

\blankpage
\sbasic

\sinput{../sets/macros.tex}
\sinput{../ordered_pairs/macros.tex}
\sinput{../relations/macros.tex}

%%%% MACROS %%%%%%%%%%%%%%%%%%%%%%%%%%%%%%%%%%%%%%%%%%%%%%%

\newcommand{\PM}{\mathbf{P}}

%%%%%%%%%%%%%%%%%%%%%%%%%%%%%%%%%%%%%%%%%%%%%%%%%%%%%%%%%%%


\sstart

\stitle{Function Composition}

\ssection{Why}

We want a notion
for applying
two functions
one after the other.
We apply a first function
then a second function.

\ssection{Definition}

Consider
two functions
for which the codomain
of the first function
is the domain of the
second function.

The
\ct{composition}{composition}
of the second function with
the first function
is the function which
associates each element
in the first's domain
with the element in
the second's codomain
that the second function
associates with the
result of the first function.

The idea is that we take
an element in the first
domain.
We apply the first function
to it.
We obtain an element in the
first's codomain.
This result is an element of
the second's domain.
We apply the second function
to this result.
We obtain an element
in the second's codomain.
The composition of the second
function with the first
is the function so constructed.

\ssubsection{Notation}

Let $A, B, C$ be
non-empty sets.
Let $f: A \to B$
and $g: B \to C$.
We denote the
composition
of $g$ with $f$
by
$g \comp f$
read aloud as
\say{g composed with f.}
To make clear the domain
and comdomain, we denote
the composition
$g \comp f: A \to C$.

In previously introduced
notation,
$g \comp f$
satisfies
\[
  (g \comp f)(a) = g(f(a))
\]
for all $a \in A$.

\ssection{Inverses}

The
\ct{identity function}{identityfunction}
on a set is a function
which associates each
element with itself.
A function

A second function
is an
\ct{inverse function}{inversefunction}
of a first function
if the domain of
the first function
is the codomain
of the second function,
the domain
of the second function
is the codomain of
the first function,
and the composition
of the first function
with the second is
the identity function
on the first domain
and the composition
of the second function
on the first is the
identity function on
the second domain.

\strats
