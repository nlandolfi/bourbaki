
%!name:entire_functions
%!need:complex_analytic_functions
%!refs:yellow/IX/4

\section*{Definition}

An \t{entire function} is a complex function $f: \C  \to \C $ which is analytic for all $z \in \C $.

\blankpage
\sbasic

\sinput{../sets/macros.tex}
\sinput{../cartesian_product/macros.tex}
\sinput{../relations/macros.tex}

%%%% MACROS %%%%%%%%%%%%%%%%%%%%%%%%%%%%%%%%%%%%%%%%%%%%%%%

\newcommand{\PM}{\mathbf{P}}

%%%%%%%%%%%%%%%%%%%%%%%%%%%%%%%%%%%%%%%%%%%%%%%%%%%%%%%%%%%


\sstart

\stitle{Nets}

\ssection{Why}
We want to generalize the notion of sequence.


\ssection{Definition}

Recall that a sequence is a function on the naturals.
The naturals are ordered and have the property that we can always go further out.
If handed two natural numbers $m$ and $n$, we can always find another, for example $\max\set{m,n}+1$, larger than $m$ and $n$.
We might think of larger as being further out from the first natural number, namely 1.
These observations motivate definining a directed set.

\begin{defn}
A \ct{directed set}{directedset} is a set $D$ with a partial order $\preceq$ satisfying one additional property: for all $a, b \in D$, there exists $c \in D$ such that $a \preceq c$ and $b \preceq c$.
\end{defn}

\begin{defn}
A \ct{net}{net} is a function on a directed set.
\end{defn}

A sequence, then, is a net.
The directed set is the set of natural numbers and the partial order is $m \preceq n$ if $m \leq n$.

\ssubsection{Notation}

Directed sets involve a set and a partial order.
We commonly assume the partial order, and just denote the set.
We use the letter $D$ as a mnemonic for directed.

For nets, we use function notation and generalize sequence notation.
We denote the net $x: D \to A$ by $\net{a_{\alpha}}$, emulating notation for sequences.
The use of $\alpha$ rather than $n$ reminds us that $D$ need not be the set of natural numbers.

\strats
