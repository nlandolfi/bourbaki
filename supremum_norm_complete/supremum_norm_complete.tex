
%!name:entire_functions
%!need:complex_analytic_functions
%!refs:yellow/IX/4

\section*{Definition}

An \t{entire function} is a complex function $f: \C  \to \C $ which is analytic for all $z \in \C $.

\blankpage
\sbasic

\sinput{../sets/macros.tex}
\sinput{../absolute_value/macros.tex}
\sinput{../norms/macros.tex}
\sinput{../sequences/macros.tex}
\sinput{../real_limits/macros.tex}
\sinput{../supremum_norm/macros.tex}
%%%% MACROS %%%%%%%%%%%%%%%%%%%%%%%%%%%%%%%%%%%%%%%%%%%%%%%

\newcommand{\PM}{\mathbf{P}}

%%%%%%%%%%%%%%%%%%%%%%%%%%%%%%%%%%%%%%%%%%%%%%%%%%%%%%%%%%%


\sstart

\stitle{Supremum Norm Complete}

\ssection{Why}

We want a complete
norm on the vector
space of continuous
functions.

\ssection{Result}

\begin{prop}
The supremum norm
is complete.

\begin{proof}
Let $R$ denote the real numbers.
Let $\seq{f}$ be an egoprox sequence
in $C[a, b]$.
Then
$\forall \epsilon > 0, \exists N$
so that
\[
  m, n > N \implies \supnorm{f_n - f_m} < \epsilon.
\]
Since $\supnorm{f_n - f_m} < \epsilon \implies
\abs{f_n(x) - f_m(x)} < \epsilon$ for
all $x \in [a, b]$, the
sequence of real numbers $\set{f_n(x)}_n$
is egoprox for each $x \in [a, b]$.
Since the metric space $(R, \abs{\cdot})$
is complete, there is a limit $l_x$
such that $f_n(x) \goesto l_x$ as
$n \goesto \infty$, for each $x \in [a, b]$.
Define $f: [a, b] \to R$
by $f(x) = l_x$ for each $x \in [a, b]$.

First, we argue that $f$ is continuous.
Let $x_0 \in [a, b]$ and let $\epsilon > 0$.
For each $n$, $f_n$ is a continuous function
on a closed interval, and therefore is
uniformly continuous:
$\forall \epsilon > 0, \exists \delta > 0$
so that $\forall x, y \in [a, b]$,
\[
  \abs{x - y} < \delta \implies \abs{f(x) - f(y)} < \epsilon.
\]

For $\epsilon/3 > 0$, there exists an $n_1$
so that
\[
  n > n_1 \implies \abs{f_n(x_0) - f(x_0)} < \epsilon/3.
\]
For $\epsilon/3 > 0$, there exists an $n_2$
so that
\[
  n > n_2 \implies \abs{f_n(x_0) - f(x_0)} < \epsilon/3.
\]
Let $n_0 = \max\set{n_1,n_2}$.
The function $f_{n_0}$ is
continuous, so for $\epsilon/3$,
there is a $\delta > 0$ so that
for all $x \in [a, b]$,
\[
  \abs{x_0 - x} < \delta \implies \abs{f_{n_0}(x_0) - f_{n_0}(x_0)} < \epsilon/3.
\]
By the triangle inequality,
\[
  \begin{aligned}
    \abs{f(x_0) - f(x)} \leq \abs{f(x_0) - f_{n_0}(x_0)} + \abs{f_{n_0}(x_0) - f(x)}
  \end{aligned}
\]
Since $n_0 \geq n_1$,
$\abs{f(x_0) - f_{n_0}(x_0)} < \epsilon/3$.
Using this fact, and the triangle inequality
\[
  \begin{aligned}
    \abs{f(x_0) - f(x)} &< \epsilon/3 + \abs{f_{n_0}(x_0) - f(x)} \\
    &\leq \epsilon/3 + \abs{f_{n_0}(x_0) - f_{n_0}(x)} + \abs{f_{n_0}(x_0) - f(x)} \\
  \end{aligned}
\]
Since $\abs{x_0 - x} < \delta$,
$\abs{f_{n_0}(x_0) - f_{n_0}(x)} < \epsilon/3$.
Since $n_0 > n_2$,
$\abs{f_{n_0}(x_0) - f(x)} < \epsilon/3$.
We conclude
\[
  \begin{aligned}
    \abs{f(x_0) - f(x)} &< \epsilon/3 + \epsilon/3 + \epsilon/3 = \epsilon.
  \end{aligned}
\]
for all $\abs{x_0-x} < \delta$.
Since $\epsilon$ was arbitrary, $f$
is continuous at $x_0$.
Since $x_0$ was arbitrary,
$f$ is continuous everywhere.
\end{proof}

\end{prop}

\strats
