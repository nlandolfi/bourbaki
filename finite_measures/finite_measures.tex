
%!name:entire_functions
%!need:complex_analytic_functions
%!refs:yellow/IX/4

\section*{Definition}

An \t{entire function} is a complex function $f: \C  \to \C $ which is analytic for all $z \in \C $.

\blankpage
\sbasic

%%%% MACROS %%%%%%%%%%%%%%%%%%%%%%%%%%%%%%%%%%%%%%%%%%%%%%%

\newcommand{\PM}{\mathbf{P}}

%%%%%%%%%%%%%%%%%%%%%%%%%%%%%%%%%%%%%%%%%%%%%%%%%%%%%%%%%%%


\sstart

\stitle{Finite Measures}

\ssection{Why}

Sometimes we want finite measures.

\ssection{Definition}

A measurable set is
\ct{finite}{setfinite}
if its measure is a real number.
The measure space itself is
\ct{finite}{spacefinite}
if the base set is finite.

A measurable set is
\ct{sigma-finite}{setsigmafinite}
if there exists a sequence of
finite measurable sets whose
union is the set.
The measure space itself is
\ct{sigma-finite}{spacesigmafinite}
if the base set is sigma finite.

\ssubsection{Notation}

We denote that a measure
space is finite by saying
\say{Let $(A, \mathcal{A}, \mu)$
and $\mu(A) < +\infty$.}

\begin{expl}
Let $(A, \mathcal{A})$ be a measurable space.

The counting measure on $(A, \mathcal{A})$ is
finite if and only if the base set is finite.
It is sigma finite if and only if the base
set is a union of a sequence of finite sets.

If $\mathcal{A} = 2^A$, then the counting
measure is sigma finite if and only if
$A$ is countable.
\end{expl}

\begin{expl}
A point mass measure is finite.
\end{expl}

\begin{expl}
Let $R$ be the set of real numbers.
The Lebesgue measure on
$(R, \mathcal{B}(R))$ is sigma finite.
\end{expl}

\strats
