
%!name:entire_functions
%!need:complex_analytic_functions
%!refs:yellow/IX/4

\section*{Definition}

An \t{entire function} is a complex function $f: \C  \to \C $ which is analytic for all $z \in \C $.

\blankpage
\sbasic

%%%% MACROS %%%%%%%%%%%%%%%%%%%%%%%%%%%%%%%%%%%%%%%%%%%%%%%

\newcommand{\PM}{\mathbf{P}}

%%%%%%%%%%%%%%%%%%%%%%%%%%%%%%%%%%%%%%%%%%%%%%%%%%%%%%%%%%%


\sstart

\stitle{Length Common Notions}

\ssection{Why}

We want to define the length
of a subset of real numbers.

\ssection{Notions}

We take two common notions:

\begin{enumerate}

  \item
  The length of a whole
  is the sum of the lengths
  of its parts;
  the \ct{additivity principle}{}.

  \item
  The length of a whole
  is the at least the length
  of any whole it contains
  the \ct{containment principle}{}.

\end{enumerate}

The task is to make precise
the use of
\say{whole,},
\say{parts,}
and \say{contains.}
We start with intervals.

\ssection{Definition}

By whole we mean set.
By part we mean an element
of a partition.
By contains we mean set
containment.

The
\ct{length}{intervallength}
of an interval is the difference
of its endpoints: the larger minus
the smaller.

Two intervals are
\ct{non-overlapping}{}
if their intersection
is a single point or empty.
The \ct{length}{} of the union of
two non-overlapping intervals is
the sum of their lengths.

A \ct{simple}{} subset
of the real numbers
is a finite union
of non-overlapping intervals.
The length of a simple subset
is the sum of the lengths of
its family.

A \ct{countably simple}{} subset
of the real numbers
is a countable union
of non-overlapping intervals.
The length of a countably simple subset
is the limit of the sum of the lengths
of its family; as we have defined it,
length is positive, so this series is either
bounded and increasing and so converges, or is
infinite, and so converges to $+\infty$.

At this point, we must confront the obvious
question: are all subsets of the real numbers
countably simple?
Answer: no.
So, what can we say?

A \ct{cover}{}
of a set $A$ of real numbers
is a family whose union
is a contains $A$.
Since a cover always contains
the set $A$, it's length, which
we understand, must be larger
(containment principles) than
$A$.
So what if we declare that
the length of an arbitrary set
$A$ be the greatest lower bound
of the lengths of all sequences
of intervals covering $A$.
Will this work?

\ssubsection{Cuts}

If $a, b$ are real numbers and
$a < b$, then we \ct{cut}{} an interval
with $a$ and $b$ as its endpoints
by selecting $c$ such that
$a < c$ and $c < b$.
We obtain two intervals, one with endpoints
$a,c$ and one with endpoints $c, b$;
we call these two the \ct{cut pieces}{}.

Given an interval, the length of
the interval is the sum of any
two cut pieces, because the pieces
are non-overlapping.

\ssection{All sets}



\begin{prop}
  Not all subsets of
  real numbers are simple.

  Exhibit: R is not finite.
\end{prop}


\begin{prop}
  Not all subsets of
  real numbers are countably simple.

  Exhibit: the rationals.
\end{prop}

Here's the great insight:
approximate a set
by a countable family of intervals.

\ssubsection{Notation}

\strats
