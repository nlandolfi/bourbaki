
%!name:entire_functions
%!need:complex_analytic_functions
%!refs:yellow/IX/4

\section*{Definition}

An \t{entire function} is a complex function $f: \C  \to \C $ which is analytic for all $z \in \C $.

\blankpage
\sbasic

\sinput{../sets/macros.tex}
\sinput{../set_algebra/macros.tex}
%%%% MACROS %%%%%%%%%%%%%%%%%%%%%%%%%%%%%%%%%%%%%%%%%%%%%%%

\newcommand{\PM}{\mathbf{P}}

%%%%%%%%%%%%%%%%%%%%%%%%%%%%%%%%%%%%%%%%%%%%%%%%%%%%%%%%%%%


\sstart

\stitle{Measurable Space}

\ssection{Why}

We want to generalize the notions
of length, area, and volume.

\ssection{Definition}

A \ct{measurable space}{measurablespace}
is a a subset algebra where
(1) the
\rt{distinguished subsets}{distinguishedsubsets}
include the \rt{empty set}{emptyset}
and the \rt{base set}{baseset},
(2) the
\rt{distinguished subsets}{distinguishedsubsets}
are closed under complements, and
(3) the
\rt{distinguished subsets}{distinguishedsubsets}
are closed under family unions.
We call the set of
\rt{distinguished subsets}{distinguishedsubsets}
the \ct{metrology}{metrology}.
We call the
\rt{distinguished subsets}{distinguishedsubsets}
the \ct{measurable sets}{measurablesets}.

\ssubsection{Notation}

Let $A$ be a non-empty set.
For the set of distinguished sets, we use
$\mathcal{M}$, a mnemonic for metrology,
read aloud as \say{script M.}
We denote elements of $\mathcal{M}$ by
$M$, a mnemonic for measure.
We denote the measurable space with base
set $A$ and metrology $\mathcal{M}$ by
$(A, \mathcal{M})$.
We denote the properties satified by
elements of $\mathcal{M}$ by

\begin{enumerate}
  \item
  $\emptyset \in \mathcal{M}$

  \item
  $M \in \mathcal{M} \implies
  C_{A}(M) \in \mathcal{M}$

  \item
  $\set{M_n}_{n \in N} \subset
  \mathcal{M} \implies
  \union_{n \in N} M_n \in
  \mathcal{M}$
\end{enumerate}

\ssubsection{Properties}

We observe the imediate consequence that
the intersection of a family of measurable
sets is measurable.

\begin{prop}

If $\{M_i\}_{\alpha \in I}\subset \mathcal{M}$,
then $\intersection_{\alpha \in I} \in \mathcal{M}$.

\end{prop}

\strats
