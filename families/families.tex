
%!name:entire_functions
%!need:complex_analytic_functions
%!refs:yellow/IX/4

\section*{Definition}

An \t{entire function} is a complex function $f: \C  \to \C $ which is analytic for all $z \in \C $.

\blankpage
\sbasic


\sinput{../sets/macros.tex}
\sinput{../set_algebra/macros.tex}

%%%% MACROS %%%%%%%%%%%%%%%%%%%%%%%%%%%%%%%%%%%%%%%%%%%%%%%

\newcommand{\PM}{\mathbf{P}}

%%%%%%%%%%%%%%%%%%%%%%%%%%%%%%%%%%%%%%%%%%%%%%%%%%%%%%%%%%%


\sstart

\stitle{Families}

\ssection{Why}

We want to generalize operations beyond
two objects.

\ssection{Definition}

An \ct{family}{family} is a collection of objects
of some set indexed by an object ranging
in another set.
We call the indexing set the
\ct{index set}{indexset}.

If the elements indexed are something
in particular, like sets, we call it a
family of sets.
Often the indexing set is a subset of
the natural numbers, or the natural numbers
themselves, but it need not be.

If the index set is a finite set,
we call the family a
\ct{finite family}{finitefamily}.
If the index set a countable set,
we call the family a
\ct{countable family}{countablefamily}.
If the index set is an uncountable set,
we call the family a
\ct{uncountable family}{uncountablefamily}.

\ssubsection{Notation}
Let $S$ be a set.
We often denote the index set by $I$, though this is not required.
For each $\alpha \in I$, let $a_{\alpha} \in S$.
We denote the family of $a_{\alpha}$ indexed with $I$ by $\set{a_{\alpha}}_{\alpha \in I}$.
The $a_{\alpha}$ notation is meant to evoke function notation and remind that we have a correspondence between the index set and the indexed set.
We read this notation \say{a sub-alpha, alpha in I.}
If for each $\alpha \in I$ we have $A_{\alpha} \subset S$, we'd write $\set{A_{\alpha}}{\alpha \in I}$.

\ssection{Operations}

Given an algebra and a finite family of objects
in the \rt{ground set}{groundset} whose indexed
set is ordered, we can extend the operation by
applying it pairwise in order. We call this the
\ct{pairwiseextension}{pairwiseextension} of the
operation.

\ssubsection{Notation}

Let $(A, +)$ be an \rt{algebra}{algebra} and
$\set{A_{i}}_{i=1}^{n}$ a finite family
of elements of $A$. We denote the pairwise e
extension by
\[
  \overset{n}{\underset{i=1}{+}} A_i
\]

\ssection{Family Set Algebra}

We define the set whose elements are the objects
which are contained in at least one family member
the \ct{family union}{familyunion}.
We define the set whose elements are the objects
which are contained in all of the family members
the \ct{family intersection}{familyintersection}.

\ssubsection{Notation}

We denote the family union by $\union_{\alpha \in I} A_{\alpha}$.
We read this notation as \say{union over alpha in I of A sub-alpha.}
We denote family intersection by $\intersect_{\alpha \in I} A_{\alpha}$.
We read this notation as \say{intersection over alpha in I of A sub-alpha.}

\ssubsection{Results}

\begin{prop}
  For an indexed family $\set{A_{\alpha}}_{\alpha \in I}$ in $S$, if $I = \set{i, j}$ then
  \[
    \union_{\alpha \in I} A_{\alpha} = A_i \union A_j
  \]
  and
  \[
    \intersect_{\alpha \in I} A_{\alpha} = A_i \intersect A_j.
  \]
\end{prop}

\begin{prop}
  For an indexed family $\set{A_{\alpha}}_{\alpha \in I}$ in $S$, if $I = \emptyset$, then
  \[
    \union_{\alpha \in I} A_{\alpha} = \emptyset
  \]
  and
  \[
    \intersect_{\alpha \in I} A_{\alpha} = S.
  \]
\end{prop}

\begin{prop}
  For an indexed family $\set{A_{\alpha}}_{\alpha \in I}$ in $S$.
  \[
    C_S(\union_{\alpha \in I} A_{\alpha}) = \intersect_{\alpha \in I} C_S(A_{\alpha})
  \]
  and
  \[
    C_S(\intersect_{\alpha \in I} A_{\alpha}) = \union_{\alpha \in I} C_S(A_{\alpha}).
  \]
\end{prop}

\strats
