
%!name:entire_functions
%!need:complex_analytic_functions
%!refs:yellow/IX/4

\section*{Definition}

An \t{entire function} is a complex function $f: \C  \to \C $ which is analytic for all $z \in \C $.

\blankpage
\sbasic


\sinput{../sets/macros.tex}
\sinput{../set_algebra/macros.tex}

%%%% MACROS %%%%%%%%%%%%%%%%%%%%%%%%%%%%%%%%%%%%%%%%%%%%%%%

\newcommand{\PM}{\mathbf{P}}

%%%%%%%%%%%%%%%%%%%%%%%%%%%%%%%%%%%%%%%%%%%%%%%%%%%%%%%%%%%


\sstart

\stitle{Families}

\ssection{Why}

We want to generalize operations beyond
two objects.

\ssection{Definition}

Let $A,B$ be non-empty sets.
A \ct{family}{family} of elements of a first
set \ct{indexed}{familyindexing} by elements
of a second set is the range of a function
from the second set to the first set.
We call second set the \ct{index set}{indexset}.

If the index set is a finite set,
we call the family a
\ct{finite family}{finitefamily}.
If the index set a countable set,
we call the family a
\ct{countable family}{countablefamily}.
If the index set is an uncountable set,
we call the family a
\ct{uncountable family}{uncountablefamily}.

If the codomain is a set of sets, we
call the family a \ct{family of sets}{familyofsets}.
We often use a subset of the whole natural numbers
as the index set.
In this case, and for other indexed sets with
orders, we call the family an \ct{ordered family}.

\ssubsection{Notation}

Let $A$ be a non-empty set.
We denote the index set by $I$, a mnemonic
for index.
For $i in I$, let we denote the result of applying
the function to $i$ by $a_{i}$; the notation evokes
evokes function notation but avoids naming the
function.

We denote the family of $a_{\alpha}$ indexed with $I$
by $\set{a_{\alpha}}_{\alpha \in I}$, which is short-hand
for \rt{set-builder notation}{setbuildernotation}.
We read this notation \say{a sub-alpha, alpha in I.}
%If for each $\alpha \in I$ we have $A_{\alpha} \subset S$, we'd write $\set{A_{\alpha}}{\alpha \in I}$.

\ssection{Operations}

The \ct{pairwise extension}{pairwiseextension}
of a commutative operation is the function from
finite families of the \rt{ground set}{groundset}
to the \rt{ground set}{groundset} obtained by
applying the operation pairwise to elements.

The \ct{ordered pairwise extension}{orderedpairwiseextension}
of an operation is the function from finite families
\rt{ground set}{groundset} to the
\rt{ground set}{groundset} obtained by
applying the operation pairwise to elements in order.

\ssubsection{Notation}

Let $(A, +)$ be an \rt{algebra}{algebra} and
$\set{A_{i}}_{i=1}^{n}$ a finite family
of elements of $A$. We denote the pairwise
extension by
\[
  \overset{n}{\underset{i=1}{+}} A_i
\]

\ssection{Family Set Algebra}

We define the set whose elements are the objects
which are contained in at least one family member
the \ct{family union}{familyunion}.
We define the set whose elements are the objects
which are contained in all of the family members
the \ct{family intersection}{familyintersection}.

\ssubsection{Notation}

We denote the family union by $\union_{\alpha \in I} A_{\alpha}$.
We read this notation as \say{union over alpha in I of A sub-alpha.}
We denote family intersection by $\intersect_{\alpha \in I} A_{\alpha}$.
We read this notation as \say{intersection over alpha in I of A sub-alpha.}

\ssubsection{Results}

\begin{prop}
  For an indexed family $\set{A_{\alpha}}_{\alpha \in I}$ in $S$, if $I = \set{i, j}$ then
  \[
    \union_{\alpha \in I} A_{\alpha} = A_i \union A_j
  \]
  and
  \[
    \intersect_{\alpha \in I} A_{\alpha} = A_i \intersect A_j.
  \]
\end{prop}

\begin{prop}
  For an indexed family $\set{A_{\alpha}}_{\alpha \in I}$ in $S$, if $I = \emptyset$, then
  \[
    \union_{\alpha \in I} A_{\alpha} = \emptyset
  \]
  and
  \[
    \intersect_{\alpha \in I} A_{\alpha} = S.
  \]
\end{prop}

\begin{prop}
  For an indexed family $\set{A_{\alpha}}_{\alpha \in I}$ in $S$.
  \[
    C_S(\union_{\alpha \in I} A_{\alpha}) = \intersect_{\alpha \in I} C_S(A_{\alpha})
  \]
  and
  \[
    C_S(\intersect_{\alpha \in I} A_{\alpha}) = \union_{\alpha \in I} C_S(A_{\alpha}).
  \]
\end{prop}

\strats
