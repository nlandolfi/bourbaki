\documentclass[12pt]{extarticle}
%\usepackage[margin=1.75in]{geometry}
\usepackage[a5paper,margin=.72in]{geometry}
\usepackage{graphicx}
\renewcommand{\baselinestretch}{1.25}
\setlength{\parskip}{0.5em}

\usepackage{titlesec}
%\titleformat*{\section}{\large\bfseries\sffamily}
%\titleformat*{\subsection}{\normalsize\bfseries\sffamily}
%\titleformat*{\subsubsection}{\large\bfseries}
%\titleformat*{\paragraph}{\large\bfseries}
%\titleformat*{\subparagraph}{\large\bfseries}

\titleformat*{\section}{\bfseries\sffamily}
\titleformat*{\subsection}{\small\bfseries\sffamily}
\titleformat*{\subsubsection}{\large\bfseries}
\titleformat*{\paragraph}{\large\bfseries}
\titleformat*{\subparagraph}{\large\bfseries}
\titlespacing*{\section}{0pt}{0.3cm}{0.1cm}
\titlespacing*{\subsection}{0pt}{0.3cm}{0.1cm}

%\renewcommand{\title}[1]{
%\begin{center}
%  \includegraphics[width=0.05\textwidth]{../../trademark}
%  \\
%  \vspace{0.20cm}
%  {\large \textsf{ #1 }}
%\end{center}
%}

\newcommand{\bpage}{
\vspace*{\fill}
\begin{center}
\includegraphics[width=1cm]{../../trademark}
\end{center}
\vspace{\fill}
}

\newcommand{\gpage}{
\begin{center}
\vspace*{\fill}
\includegraphics{graph}
\vspace{\fill}
\end{center}
}


\newcommand*{\vcenteredhbox}[1]{\begin{tabular}{@{}c@{}}#1\end{tabular}}
\renewcommand{\title}[1]{
\begin{center}
\hspace*{-1cm}
\vcenteredhbox{\vspace*{-0.1cm} \includegraphics[height=0.5cm]{../../trademark}\hspace*{0.1cm}}
\vcenteredhbox{\large \textsf{#1}}
\end{center}
}

\usepackage{hyperref}
\hypersetup{
  colorlinks,
  citecolor=black,
  filecolor=black,
  linkcolor=black,
  urlcolor=black
}


%\usepackage{ccfonts}% http://ctan.org/pkg/{ccfonts}
%\usepackage[cm]{sfmath}
\usepackage[T1]{fontenc}
\DeclareMathAlphabet{\mathbfsf}{\encodingdefault}{\sfdefault}{bx}{n}
\def\mathword#1{\mathop{\textup{#1}}}

\usepackage{caption}
\usepackage{subcaption}

%%%% MACROS %%%%%%%%%%%%%%%%%%%%%%%%%%%%%%%%%%%%%%%%%%%%%%%

\newcommand{\PM}{\mathbf{P}}

%%%%%%%%%%%%%%%%%%%%%%%%%%%%%%%%%%%%%%%%%%%%%%%%%%%%%%%%%%%


\reference{sets}
\reference{set_algebra}

%%%% MACROS %%%%%%%%%%%%%%%%%%%%%%%%%%%%%%%%%%%%%%%%%%%%%%%

\newcommand{\PM}{\mathbf{P}}

%%%%%%%%%%%%%%%%%%%%%%%%%%%%%%%%%%%%%%%%%%%%%%%%%%%%%%%%%%%


\begin{document}
\title{Families}

\section{Why}

We want to generalize operations beyond two objects.

\section{Definition}

We often refer to a collection of elemenets of some set via an index which takes value in another set.
We call the indexing set the \definition{index set} and set of indexed sets an \definition{indexed family}.
We can think of a function from the index set to the indexed family.
If the elements indexed are something in particular, like sets, we call it an indexed family of sets.
Often the indexing set is a subset of the natural numbers, or the natural numbers themselves, but it need not be.

\subsection{Notation}
Let $S$ be a set.
We often denote the index set by $I$, though this is not required.
For each $\alpha \in I$, let $a_{\alpha} \in S$.
We denote the family of $a_{\alpha}$ indexed with $I$ by $\set{a_{\alpha}}_{\alpha \in I}$.
The $a_{\alpha}$ notation is meant to evoke function notation and remind that we have a correspondence between the index set and the indexed set.
We read this notation \say{a sub-alpha, alpha in I.}
If for each $\alpha \in I$ we have $A_{\alpha} \subset S$, we'd write $\set{A_{\alpha}}{\alpha \in I}$.

\section{Family Set Algebra}
We define the set whose elements are the objects which are contained in at least one family member the \definition{family union}.
We define the set whose elements are the obejcts which are contained in all of the family members the \definition{family intersection}.

\subsection{Notation}

We denote the family union by $\union_{\alpha \in I} A_{\alpha}$.
We read this notation as \say{union over alpha in I of A sub-alpha.}
We denote family intersection by $\intersect_{\alpha \in I} A_{\alpha}$.
We read this notation as \say{intersection over alpha in I of A sub-alpha.}

\subsection{Results}

\begin{prop}
  For an indexed family $\set{A_{\alpha}}_{\alpha \in I}$ in $S$, if $I = \set{i, j}$ then
  \[
    \union_{\alpha \in I} A_{\alpha} = A_i \union A_j
  \]
  and
  \[
    \intersect_{\alpha \in I} A_{\alpha} = A_i \intersect A_j.
  \]
\end{prop}

\begin{prop}
  For an indexed family $\set{A_{\alpha}}_{\alpha \in I}$ in $S$, if $I = \emptyset$, then
  \[
    \union_{\alpha \in I} A_{\alpha} = \emptyset
  \]
  and
  \[
    \intersect_{\alpha \in I} A_{\alpha} = S.
  \]
\end{prop}

\begin{prop}
  For an indexed family $\set{A_{\alpha}}_{\alpha \in I}$ in $S$.
  \[
    C_S(\union_{\alpha \in I} A_{\alpha}) = \intersect_{\alpha \in I} C_S(A_{\alpha})
  \]
  and
  \[
    C_S(\intersect_{\alpha \in I} A_{\alpha}) = \union_{\alpha \in I} C_S(A_{\alpha}).
  \]
\end{prop}

\end{document}
