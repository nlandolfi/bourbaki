
%!name:entire_functions
%!need:complex_analytic_functions
%!refs:yellow/IX/4

\section*{Definition}

An \t{entire function} is a complex function $f: \C  \to \C $ which is analytic for all $z \in \C $.

\blankpage
\sbasic

%%%% MACROS %%%%%%%%%%%%%%%%%%%%%%%%%%%%%%%%%%%%%%%%%%%%%%%

% use \set{stuff} for { stuff }
% use \set* for autosizing delimiters
\DeclarePairedDelimiter{\set}{\{}{\}}

% use \Set{a}{b} for {a | b}
% use \Set* for autosizing delimiters
\DeclarePairedDelimiterX{\Set}[2]{\{}{\}}{#1 \nonscript\;\delimsize\vert\nonscript\; #2}

% use \powerset{A} for power set of A
\newcommand{\powerset}[1]{2^{#1}}

\renewcommand{\emptyset}{\varnothing}

\newcommand{\SA}{\mathcal{A}}
\newcommand{\SB}{\mathcal{B}}
\newcommand{\SC}{\mathcal{C}}
\newcommand{\SD}{\mathcal{D}}
\newcommand{\SE}{\mathcal{E}}
\newcommand{\SF}{\mathcal{F}}
\newcommand{\SG}{\mathcal{G}}
\newcommand{\SH}{\mathcal{H}}
\newcommand{\SI}{\mathcal{I}}
\newcommand{\SJ}{\mathcal{J}}
\newcommand{\SK}{\mathcal{K}}
\newcommand{\SL}{\mathcal{L}}

%%%%%%%%%%%%%%%%%%%%%%%%%%%%%%%%%%%%%%%%%%%%%%%%%%%%%%%%%%%


\sstart

\stitle{Sets}

\ssection{Why}

We want to talk about things,
for which we will use the word
\textit{object},
and we want to talk
about none, one or
several things considered
as a whole,
for which we will use
the word \textit{set}.

\ssection{Definition}

We use the word \ct{object}{object}
with its usual sense in the English
language.
An object may be tangible, in that
we can hold or touch it,
or an object may be abstract,
in that we can do neither.

A \ct{set}{set} is an abstract
object which we think of as
several objects considered
at once.
Naturally, then,
we regularly consider the
objects a set contains.
We call the objects contained
in a set the
\ct{members}{members} or
\ct{elements}{element} of the set.
We say that an object
contained in a set is a
\ct{member}{memberof} of or an
\ct{element}{elementof} of the set.

We call the set which contains
no objects the
\ct{empty set}{emptyset}.
We call
a set which contains
only a single object a
\ct{singleton}{singleton}.
A singleton is not the
same as the object
it contains.
Besides these two cases,
we think of sets
as containing
two or more objects.



\ssection{Examples}

For familiar examples,
let us start
with some tangible
objects.
Find, or call to
mind,
a deck
of playing cards.

First, consider
the set of all
the cards.
This set contains
fifty-two elements.
Second, consider
the set of cards
whose suit is hearts.
This set contains
thirteen elements:
the ace, two, three, four, five,
six, seven, eight, nine, ten,
jack, queen, and
king of hearts.
Third, consider
the set of twos.
This set contains
four elements: the
the two of clubs,
the two of spades,
the two of hearts,
and the two of diamonds.

We can imagine many
more sets of cards.
If we are holding a deck,
each of these can be
made tangible: we can
touch the elements of
the set.
But the set itself
is always abstract:
we can not touch it.
It is the idea of the
group as distinct from
any individual member.

Moreover, the
elements of a set
need not be tangible.
First, consider the set
consisting of the suits of
the playing card:
hearts, diamonds, spades, and clubs.
This set has four elements.
Each element is a suit.
Second, consider the set
consisting of the card types.
This set has thirteen elements:
ace, two, three, four, five,
six, seven, eight, nine, ten,
jack, queen, king.
By this set we mean a different
set than the set
of hearts.

Of course, sets need have nothing to
do with playing cards.
For example, consider the set of
seasons: autumn, winter, spring,
and summer.
This set has four elements.
For another example,
consider the set of Latin letters:
a, b, c, \dots, x, y, z.
This set has twenty-six elements.

\ssubsection{Notation}

To aid discussing and
denoting objects, let
us tend to give them
short names.
A single Latin
letter regularly suffices:
for example,
$a$, $b$ or $c$.
Let us denote that
the object $a$ and
the object $b$ are
the same object
by $a = b$,
read aloud as
\say{a is b.}

For objects which
we imagine as containing
other objects, namely sets,
let us tend to use
upper case Latin letters:
for example,
$A$, $B$, and $C$.
To aid our memory,
let us tend to use the lower
case form of the letter for
an element of the set.
For example,
if $A$ is a set,
let us tend to denote by
$a$ an element of $A$.
Likewise, if $B$ is a set,
let us tend to denote
by $b$ an element of $B$.


Let us denote that
an object $a$
is an element of a set $A$
by $a \in A$.
We read the notation
$a \in A$ aloud as \say{a in A.}
The $\in$ is a stylized
lower case Greek letter: $\epsilon$.
It is
read aloud \say{ehp-sih-lawn} and
is a mnemonic for \say{element of}.
We write $a \not\in A$, read aloud
as \say{a not in A,} if $a$ is not
an element of $A$.

If we have named
the elements of a set,
and can list them,
let us do so between braces.
For example,
let $a$, $b$, and $c$
be three distinct objects.
Denote by $\set{a, b, c}$
the set containing theses
three objects and only these
three objects.
We can further compress notation,
and denote this set of
three objects by $A$:
so, $A = \set{a, b, c}$.
Then $a \in A$,
$b \in A$, and $c \in A$.
Moreover, if $d$
is an object and
$d \in A$, then $d = a$
or $d = b$ or $d = c$.

If the elements of a set are
well-known enough that we can
avoid ambiguity, then we can
describe the set in English.
To aid our memory,
let us tend to name it mnemonically.
For example,
let $L$ be the set of latin letters.

Often to be more precise, we should
explicitly deal with objects which
satisfy several conditions.
If the elements of a set satisfy
some common condition, then we use
the braces and include the condition.
For example, let $V$ be the set of
Latin vowels.
We can denote $V$ by
$\Set{l \in L}{l \text{ is a vowel}}$.
We read the symbol $\mid$ aloud as
\say{such that.}
We read the whole notation aloud as
\say{l in L such that l is a vowel.}
We call the notation
\ct{set-builder notation}{setbuildernotation}.
Set-builder notation is indispensable for
sets defined implicitly by some condition.
Here we could have alternatively denoted
$V$ by
$\set{\say{a},\say{e},\say{i},\say{o},\say{u}}$.
We prefer the former, slighly more concise notation.

%We develop herein a language
%for specifying things by either
%listing them explicitly or
%by listing their defining properties.

\strats
