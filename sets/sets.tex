\documentclass[12pt]{extarticle}
%\usepackage[margin=1.75in]{geometry}
\usepackage[a5paper,margin=.72in]{geometry}
\usepackage{graphicx}
\renewcommand{\baselinestretch}{1.25}
\setlength{\parskip}{0.5em}

\usepackage{titlesec}
%\titleformat*{\section}{\large\bfseries\sffamily}
%\titleformat*{\subsection}{\normalsize\bfseries\sffamily}
%\titleformat*{\subsubsection}{\large\bfseries}
%\titleformat*{\paragraph}{\large\bfseries}
%\titleformat*{\subparagraph}{\large\bfseries}

\titleformat*{\section}{\bfseries\sffamily}
\titleformat*{\subsection}{\small\bfseries\sffamily}
\titleformat*{\subsubsection}{\large\bfseries}
\titleformat*{\paragraph}{\large\bfseries}
\titleformat*{\subparagraph}{\large\bfseries}
\titlespacing*{\section}{0pt}{0.3cm}{0.1cm}
\titlespacing*{\subsection}{0pt}{0.3cm}{0.1cm}

%\renewcommand{\title}[1]{
%\begin{center}
%  \includegraphics[width=0.05\textwidth]{../../trademark}
%  \\
%  \vspace{0.20cm}
%  {\large \textsf{ #1 }}
%\end{center}
%}

\newcommand{\bpage}{
\vspace*{\fill}
\begin{center}
\includegraphics[width=1cm]{../../trademark}
\end{center}
\vspace{\fill}
}

\newcommand{\gpage}{
\begin{center}
\vspace*{\fill}
\includegraphics{graph}
\vspace{\fill}
\end{center}
}


\newcommand*{\vcenteredhbox}[1]{\begin{tabular}{@{}c@{}}#1\end{tabular}}
\renewcommand{\title}[1]{
\begin{center}
\hspace*{-1cm}
\vcenteredhbox{\vspace*{-0.1cm} \includegraphics[height=0.5cm]{../../trademark}\hspace*{0.1cm}}
\vcenteredhbox{\large \textsf{#1}}
\end{center}
}

\usepackage{hyperref}
\hypersetup{
  colorlinks,
  citecolor=black,
  filecolor=black,
  linkcolor=black,
  urlcolor=black
}


%\usepackage{ccfonts}% http://ctan.org/pkg/{ccfonts}
%\usepackage[cm]{sfmath}
\usepackage[T1]{fontenc}
\DeclareMathAlphabet{\mathbfsf}{\encodingdefault}{\sfdefault}{bx}{n}
\def\mathword#1{\mathop{\textup{#1}}}

\usepackage{caption}
\usepackage{subcaption}

%%%% MACROS %%%%%%%%%%%%%%%%%%%%%%%%%%%%%%%%%%%%%%%%%%%%%%%

\newcommand{\PM}{\mathbf{P}}

%%%%%%%%%%%%%%%%%%%%%%%%%%%%%%%%%%%%%%%%%%%%%%%%%%%%%%%%%%%


%%%% MACROS %%%%%%%%%%%%%%%%%%%%%%%%%%%%%%%%%%%%%%%%%%%%%%%

\newcommand{\PM}{\mathbf{P}}

%%%%%%%%%%%%%%%%%%%%%%%%%%%%%%%%%%%%%%%%%%%%%%%%%%%%%%%%%%%


\begin{document}
\title{Sets}

\section{Why}

We want to speak of a collection of objects, pre-specified or possessing some similar defining property.
%We here introduce and standardize a conceptual language of terms with notation to do so.

\section{Definition}

A \definition{set} is a collection of objects.
We use the term object in the usual sense of the English language.
So a set is itself an object, but of the peculiar nature that it contains other objects.
In thinking of a set, then, we regularly consider the objects it contains.
We call the objects contained in a set the \definition{members} or \definition{elements} of the set.
So we say that an object contained in a set is a \definition{member of} or an \definition{element of} the set.

\subsection{Notation}

We denote sets by upper case latin letters: for example, $A$, $B$, and $C$.
We denote elements of sets by lower case latin letters: for example, $a$, $b$, and $c$.
We denote that an object $A$ is an element of a set $A$ by $a \in A$.
We read the notation $a \in A$ aloud as \say{a in A.}
The $\in$ is a stylized $\epsilon$, which possesses the mnemonic for element.

If we can write down the elements of $A$, we do so using a brace notation.
For example, if the set $A$ is such that it contains only the elements $a, b, c$, we denote $A$ by $\set{a, b, c}$.
If the elements of a set are well-known we introduce the set in English and name it; often we select the name mnemonically.
For example, let $L$ be the set of latin letters.

If the elements of a set are such that they satisfy some common condition, we use the braces and include the condition.
For example, if $V$ is the set of vowels we denote $V$ by $\Set{l \in L}{l \text{ is a vowel}}$.
The $\mid$ is read aloud as \say{such that,} the notation reads aloud as \say{l in L such that l is a vowel.}
We call the notation $\Set{l \in L}{l \text{ is a vowel}}$ \textbf{set-builder notation}.
Set-builder notation is indispensable for sets definied implicitly by some condition.
Here we could have alternatively denoted $V$ by $\set{\say{a},\say{e},\say{i},\say{o},\say{u}}$.
We prefer the former, slighly more concise notation.

\section{Two Sets}

We now consider two sets at once.
If each element of a first set is an element of second set, we say that the first set is a \definition{subset} of or is \definition{contained in} the second set.
Conversely, we say that the second set is a \definition{superset} of or \definition{contains} the first set.
If a first set is a subset of a second set and the second set is a subset of the first set, we say the two sets are \definition{equal}.
We call the set which has no members the \definition{empty set}.
The empty set is contained in every other set.

For each set there is a natural collection of objects different from its elements that are related to it.
These objects are the subsets of the set.
We call the set consisting of the subsets of a set the \definition{powerset} of the set.
The powerset contains the original set and the empty set.

\subsection{Notation}
We denote that the set $A$ is a subset of the set $B$ by $A \subset B$.
We read the notation $A \subset B$ aloud as \say{A subset B}.
We denote that the set $A$ is equal to $B$ by $A = B$.
We read the notation $A = B$ aloud as \say{A equals B}.
We denote the empty set by $\set{}$ or by $\emptyset$; both notations are read aloud as \say{empty.}
We denote the power set of $A$ by $P_A$.

\section{Cartesian Products}

Given two sets, construct a new set that is the ordered pairs of elements of the sets: the first element of the pair is an element of the first set, the second element of the pair is an element of the second set.
We call the set of all such ordered pairs the \definition{cartesian product} of the first set with the second set.

Two ordered pairs are identical if they have identical elements in the same order.
If we have two sets, the cartesian product of the first with the second is not the same as the second with the first.
This asymmetry results from the ordering of the pairs.
We refer to an ordered pair as a \definition{tuple}.

\subsection{Notation}
For sets $A, B$ we denote the cartesian product by $A \cross B$.
We read the notation $A \cross B$ as \say{A cross B.}
We denote elements of $A \cross B$ by $(a, b)$ with the understanding that $a \in A$ and $b \in B$.
In this notation, we can write the observation that $A \cross B \neq B \cross A$.

\end{document}
