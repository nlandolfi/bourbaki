
%!name:entire_functions
%!need:complex_analytic_functions
%!refs:yellow/IX/4

\section*{Definition}

An \t{entire function} is a complex function $f: \C  \to \C $ which is analytic for all $z \in \C $.

\blankpage
\sbasic

%%%% MACROS %%%%%%%%%%%%%%%%%%%%%%%%%%%%%%%%%%%%%%%%%%%%%%%

% use \set{stuff} for { stuff }
% use \set* for autosizing delimiters
\DeclarePairedDelimiter{\set}{\{}{\}}

% use \Set{a}{b} for {a | b}
% use \Set* for autosizing delimiters
\DeclarePairedDelimiterX{\Set}[2]{\{}{\}}{#1 \nonscript\;\delimsize\vert\nonscript\; #2}

% use \powerset{A} for power set of A
\newcommand{\powerset}[1]{2^{#1}}

\renewcommand{\emptyset}{\varnothing}

\newcommand{\SA}{\mathcal{A}}
\newcommand{\SB}{\mathcal{B}}
\newcommand{\SC}{\mathcal{C}}
\newcommand{\SD}{\mathcal{D}}
\newcommand{\SE}{\mathcal{E}}
\newcommand{\SF}{\mathcal{F}}
\newcommand{\SG}{\mathcal{G}}
\newcommand{\SH}{\mathcal{H}}
\newcommand{\SI}{\mathcal{I}}
\newcommand{\SJ}{\mathcal{J}}
\newcommand{\SK}{\mathcal{K}}
\newcommand{\SL}{\mathcal{L}}

%%%%%%%%%%%%%%%%%%%%%%%%%%%%%%%%%%%%%%%%%%%%%%%%%%%%%%%%%%%


\sstart

\stitle{Sets}

\ssection{Why}

We speak of collections of objects
which we explicitly specify or which
we describe as possessing one or more
defining properties.

\ssection{Definition}


We use the words \ct{object}{object}
and \ct{collection}{collection}
with their usual sense in the English
language.
A \ct{set}{set} is a collection
of objects.
So a set is an object with
the property that it
contains other objects.

In thinking of a set, then,
we regularly consider the
objects it contains.
We call the objects contained
in a set the
\ct{members}{members} or
\ct{elements}{element} of the set.
So we say that an object
contained in a set is a
\ct{member of}{memberof} or an
\ct{element of}{elementof} the set.

For example, consider the set of
seasons. This set has four elements:
autumn, winter, spring and summer.
Consider the set playing card suits:
hearts, diamonds, spades, and clubs.
Consider the set of cards for each suit:
ace, two, three, four, so on, ten, jack, queen,
king.
Consider the set of fifty-two cards in a deck.

\ssubsection{Notation}

We denote sets by upper case
latin letters: for example,
$A$, $B$, and $C$.
We denote elements of sets
by lower case latin letters:
for example, $a$, $b$, and $c$.
We denote that an object $a$
is an element of a set $A$
by $a \in A$.
We read the notation
$a \in A$ aloud as \say{a in A.}
The $\in$ is a stylized $\epsilon$,
a mnemonic for \say{element of}.
We write $a \not\in A$, read aloud
as \say{a not in A,} if $a$ is not
an element of $A$.

If we can write down the elements
of $A$, we do so using brace notation.
For example, if the set $A$ is such
that it contains only the elements
$a, b, c$, we denote $A$ by
$\set{a, b, c}$.
If the elements of a set are well-known,
then we introduce the set in English
and name it; often we select the name mnemonically.
For example, let $L$ be the set of latin letters.

If the elements of a set satisfy
some common condition, then we use
the braces and include the condition.
For example, let $V$ be the set of
Latin vowels.
We can denote $V$ by
$\Set{l \in L}{l \text{ is a vowel}}$.
We read the symbol $\mid$ aloud as
\say{such that.}
We read the whole notation aloud as
\say{l in L such that l is a vowel.}
We call the notation
\ct{set-builder notation}{setbuildernotation}.
Set-builder notation is indispensable for
sets defined implicitly by some condition.
Here we could have alternatively denoted
$V$ by
$\set{\say{a},\say{e},\say{i},\say{o},\say{u}}$.
We prefer the former, slighly more concise notation.

\strats
