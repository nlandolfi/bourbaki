%!name:ordinary_reducer_factorization
%!need:ordinary_row_reductions
%!need:row_reducer_matrices
%!need:matrix_transpose
%!refs:trefethen1997numerical

\ssection{Why}

In ordinary reduction, we obtain a sequence of row reducers.

\ssection{Factorization of $A$ from a sequence of reducers}

Let $(A \in \R^{m \times m}, b \in \R^{m})$ be an ordinarily reducible linear system.
The \t{ordinary reducer sequence} is a sequence of reducer matrices $L_{1}, \dots L_{m-1}$ with $A_1 = L_1A$ and $A_i = L_iA_{i-1}$ for $2 \leq i \leq m-1$.
In other words, $U \in \R^{m \times m}$ defined by
\begin{equation}
	U = L_{m-1} \cdots L_2 L_1 A
	\label{equation:ordinary_reducer_factorization:main}
\end{equation}
is the ordinary row reduction of $A$.
$U$ is upper triangular.

If $L_{m-1}\cdots L_2 L_1$ in Equation~\eqref{equation:ordinary_reducer_factorization:main} is invertible, then we have
\[
  A = \invp{L_{m-1}\cdots L_2 L_1}U,
\]
which is a factorization of $A$.
Each $L_i$ is invertible, so
\[
  \invp{L_{m-1}\cdots L_2 L_1} = \inv{L}_1\inv{L}_2\cdots\inv{L}_{m-1}.
\]
So we are interested in the inverse of $L_i$ for $i \leq m-1$.
Recall that if $x_1$ is the first column of $A$, and $x_2$ is the second column of $L_1A$ and $x_k$ is the $k$th column of $L_{k-1}\cdots L_{1}A$ for $k = 2, \dots, m-1$, then
\[{\tiny
  L_k = \barray{
    1 & & & &  & \\
    & \ddots & & & & \\
    & & 1 & & & \\
    & & -\ell_{k+1,k} & 1 & &\\
    & & \vdots & & \ddots & & \\
    & & -\ell_{mk} & & & 1
  }}
\]
where $\ell_{jk} = \nicefrac{x_{jk}}{x_{kk}}$ for $k < j \leq m$.

\ssection{Properties}

The two important properties of the $L_i$ is that they have simple inverses and a simple product.
Define
\[\ell_k = (0,\cdots, 0,\ell_{k+1,k}, \cdots, \ell_{m,k})\]
so that $L_k = L_k - \ell_k\transpose{e}_k$ where $(e_k)_i$ is 1 if $k = i$ and 0 otherwise.

\noindent\noindent \textbf{Prop:} {\it $\inv{L}_i$ is $L_i$ with the subdiagonal entries negated.}
\textit{Proof.} From the sparsity pattern of $\ell_k$, we have $e_k^\tp \ell_k = 0$. So
\[
	(I - \ell_ke_k^\tp)(I + \ell_ke_k^\tp) = I - \ell_ke^\tp_k\ell_ke^\tp_k = I.
\]

\noindent\noindent \textbf{Prop:} {\it $\inv{L}_{k}\inv{L}_{k+1}$ is the unit lower-triangular matrix with the entries of both $\inv{L}_k$ and $\inv{L}_{k+1}$ in their usual places.}
\textit{Proof.} From the sparsity pattern of $\ell_{k+1}$ we have $e_k^\tp \ell_{k+1} = 0$ so that
\[
	\inv{L}_k \inv{L}_{k+1} = (I + \ell_ke^\tp_{k})(I + \ell_{k+1}e^\tp_{k+1}) = I + \ell_ke^*_k + \ell_{k+1}e^*_{k+1}.
\]
Combining these two results, we deduce that
\[
  \inv{L}_1\inv{L}_2\cdots\inv{L}_{m-1} = \barray{
    1 & & & & \\
    \ell_{21} & 1 &&& \\
    \ell_{31} & \ell_{32} & 1 & & \\
    \vdots & \vdots & \ddots &\ddots & \\
    \ell_{m1} & \ell_{m2} & \cdots & \ell_{m,m-1} & 1 \\
  }
\]

If we define $L = L_{1}^{-1}\cdots L_{m-1}^{-1}$ we obtain $A = LU$.
In other words, we have a factorization (the \t{ordinary reducer factorization}) of $A$ in terms of two matrices.
The first, $L$ is unit lower triangular.
The second, $U$, is upper triangular.
