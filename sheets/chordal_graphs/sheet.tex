
%!name:chordal_graphs
%!need:undirected_subgraphs
%!need:integral_line
%!need:integer_divisors
%!need:trees
%!refs:vandenberghe2014chordal

\section*{Why}

Many problems simplify if the graph involved is chordal.\footnote{Future editions will expand.}

\section*{Paths}

Let $G$ be an undirected graph.
A \t{chord} in a path $p$ of $G$ is an edge between two non-consecutive vertices of $p$.
So a chord of the path $(v_1, v_1, \dots , v_{k})$ is an edge $\set{v_i, v_j}$ with $\abs{j - i} > 1$.

We interpret a chord as a \say{one-edge shortcut} between two vertices of a path.
If a path $p$ has a chord, it can be reduced to a shorter path $p'$ by \say{skipping} vertices.
In other words, the shortest path between two vertices is chordless.
However, a chordless path need not be a shortest path.
See the figure below.

\section*{Graphs}

A chord of a cycle $(v_1, v_2, \dots , v_{k-1}, v_1)$ is an edge $\set{v_i, v_j}$ with $(j - i) \mod k > 1$.
An undirected graph $G$ is \t{chordal} if every cycle with more than three edges has a chord.

If $G$ is chordal, every cycle in $G$ can be reduced to a cycle of length three.
We sometimes call a cycle of length three a \t{triangle}.
For this reason, chordal graphs are also sometimes called \t{triangulated graphs}.
Other terminology includes \t{rigid-circuit graphs}, \t{triangulated graphs}, \t{perfect elimination graphs}, \t{decomposable graphs}.\footnote{See Vanenberghe and Anderson, 2014.}

The last graph in the figure below is not chordal because the cycle $(a, b, d, c, a)$ has length four and no chord.
Adding the edge $\set{b, c}$ or $\set{a, d}$ would make the graph chordal
An immediate consequence of the definition that $G$ be chordal is that any subgraph of $G$ is chordal.

\section*{Simple examples}

Since trees and forests have no cycles, they are chordal.
Similarly, any graph with no cycles longer than three edges are trivially chordal.
Such graphs are sometimes called \t{cactus graphs}.
The complete graphs are also trivially chordal.

\subsection*{Specific example}

\begin{center}\includegraphics[width=0.70\textwidth]{graphics/chords.png}\end{center}
The edge $\set{e, d}$ is a chord in the path $(a, e, c, d)$ of the first graph. The path $(a, v, d, c)$ is chordless.
The edge $\set{a, e}$ is a chord in the cycle $(a, v, e, c, a)$ of the second graph.
The cycle $(a, b, d, c, a)$ is chordless.

%<div data-littype='paragraph'>
% <div data-littype='run'> \begin{figure} </div>
% <div data-littype='run'> \centering </div>
% <div data-littype='run'> ∈cludegraphics[width=0.9\textwidth]{graphics_included/chords} </div>
% <div data-littype='run'> \caption{ </div>
% <div data-littype='run'> The edge $\set{e, d}$ is a chord in the path $(a, e, c,
%    d)$ of the first graph. The path $(a, v, d, c)$ is
%    chordless. </div>
% <div data-littype='run'> The edge $\set{a, e}$ is a chord in the cycle $(a, v, e,
%    c, a)$ of the second graph. </div>
% <div data-littype='run'> The cycle $(a, b, d, c, a)$ is chordless. </div>
% <div data-littype='run'> } </div>
% <div data-littype='run'> \label{figure:chordal_graphs:chords} </div>
% <div data-littype='run'> \end{figure” </div>
% <div data-littype='run'> </div>
%</div>
