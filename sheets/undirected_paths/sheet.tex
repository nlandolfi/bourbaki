%!name:undirected_paths
%!need:undirected_graphs
%!need:lists
%!ref:vandenberghe/chordal

\ssection{Why}
  \ifhmode\unskip\fi\footnote{
Future editions will include.
  }

\ssection{Definition}

Let $(V, E)$ be an undirected graph.
An (undirected) \t{path} between vertex $v \in V$ and vertex $w \neq v$ is a finite sequence of distinct vertices, whose first coordinate is $v$ and whose last coordinate is $w$, and whose consecutive coordinates are adjacent in the graph.
We call the first and last coordinate the \t{endpoints} of the path.
We say the path is \t{between} its endpoints.

The \t{length} of a path is one less than the number of vertices: namely, the number of edges.
Notice that this definition disagrees with the definition of the \say{length} of the sequence of vertices (namely, the number of vertices).
The length of a path is always at least one: there exists a path of length one between any two adjacent vertices.
If a path has length two or greater, we call a vertex which is not the first or last vertex an \t{interior vertex}.

Two vertices are \t{connected} in a graph if there exists at least one path between them.
A graph is \t{connected} if each pair of vertices is connected.
Recall that two vertices are \t{adjancent} if they are connected by a path of length one.
In contrast, two vertices are connected if they are connected by a path of any length.
In other words, all adjacent vertices are connected.

A \t{cycle} is a sequence whose first and last coordinate are identical, all other coordinates are distinct, and consecutive coordinates are adjacent.
An undirected graph is \t{acyclic} if there are no cycles of its vertices.

\ssubsection{Other Terminology}

Some authors allow paths to contain repeated vertices, and call a path with distinct vertices a \t{simple path}.
Similarly, some authors allow a cycle to contain repeated vertices, and call a path with distinct vertices a \t{simple cycle} or \t{circuit}.
Some authors use the term \t{loop} instead of \t{cycle}.

\ssubsection{Notation}

Let $G = (V, E)$ be a graph.
A path between $v$ and $w$ (with $v \neq w$) in $G$ is a sequence $(v_0, v_1, \dots , v_k)$ where $v_0 = v$ and $v_k = w$ and $\set{v_i, v_{i+1}} \in E$ for $i = 0, \dots , k-1$.
