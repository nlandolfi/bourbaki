%!name:undirected_paths
%!need:undirected_graphs
%!need:sequences

\ssection{Why}\footnote{Future editions will include.}

\ssection{Definition}

An (undirected) \t{path} between two distinct vertices of an undirected graph is a finite sequence of distinct vertices, whose first coordinate is the first vertex and whose last coordinate is the second vertex, and whose consecutive coordinates are adjacent in the graph.
We call the first and last coordinate the \t{endpoints} of the path.
We say that the path is \t{between} its endpoints.

The \t{length} of a path is one less than the number of vertices: namely, the number of edges.
The length of a path is always at least one: there exists a path of length one between any two adjacent vertices.
If a path has length two or greater, we call a vertex which is not the first or last vertex an \t{interior vertex}.

Two vertices are \t{connected} in a graph if there exists at least one path between them.
A graph is \t{connected} if each pair of vertices is connected.
A \t{connected component} of a graph is a maximal subset of vertices whose corresponding subgraph is connected.

A \t{cycle} is a sequence whose first and last coordinate are identical, all other coordinates are distinct, and consecutive coordinates are adjacent.
An undirected path is \t{acyclic} if it has no cycles.

\ssubsection{Other Terminology}

Some authors allow paths to contain repeated vertices, and call a path with distinct vertices a \t{simple path}.
Similarly, some authors allow a cycle to contain repeated vertices, and call a path with distinct vertices a \t{simple cycle} or \t{circuit}.
Some authors use the term \t{loop} instead of \t{cycle}.

\ssubsection{Notation}

Let $G = (V, E)$ be a graph.
A path between $v$ and $w$ (with $v \neq w$) in $G$ is a sequence $(v_0, v_1, \dots, v_k)$ where $v_0 = v$ and $v_k = w$ and $\set{v_i, v_{i+1}} \in E$ for $i = 0, \dots, k-1$.
The length of such a path is $k$.
