
%!name:entire_functions
%!need:complex_analytic_functions
%!refs:yellow/IX/4

\section*{Definition}

An \t{entire function} is a complex function $f: \C  \to \C $ which is analytic for all $z \in \C $.

\blankpage
\sbasic
%%%% MACROS %%%%%%%%%%%%%%%%%%%%%%%%%%%%%%%%%%%%%%%%%%%%%%%

\newcommand{\PM}{\mathbf{P}}

%%%%%%%%%%%%%%%%%%%%%%%%%%%%%%%%%%%%%%%%%%%%%%%%%%%%%%%%%%%

%%%% MACROS %%%%%%%%%%%%%%%%%%%%%%%%%%%%%%%%%%%%%%%%%%%%%%%

% use \set{stuff} for { stuff }
% use \set* for autosizing delimiters
\DeclarePairedDelimiter{\set}{\{}{\}}

% use \Set{a}{b} for {a | b}
% use \Set* for autosizing delimiters
\DeclarePairedDelimiterX{\Set}[2]{\{}{\}}{#1 \nonscript\;\delimsize\vert\nonscript\; #2}

% use \powerset{A} for power set of A
\newcommand{\powerset}[1]{2^{#1}}

\renewcommand{\emptyset}{\varnothing}

\newcommand{\SA}{\mathcal{A}}
\newcommand{\SB}{\mathcal{B}}
\newcommand{\SC}{\mathcal{C}}
\newcommand{\SD}{\mathcal{D}}
\newcommand{\SE}{\mathcal{E}}
\newcommand{\SF}{\mathcal{F}}
\newcommand{\SG}{\mathcal{G}}
\newcommand{\SH}{\mathcal{H}}
\newcommand{\SI}{\mathcal{I}}
\newcommand{\SJ}{\mathcal{J}}
\newcommand{\SK}{\mathcal{K}}
\newcommand{\SL}{\mathcal{L}}

%%%%%%%%%%%%%%%%%%%%%%%%%%%%%%%%%%%%%%%%%%%%%%%%%%%%%%%%%%%

%%%% MACROS %%%%%%%%%%%%%%%%%%%%%%%%%%%%%%%%%%%%%%%%%%%%%%%

\newcommand{\PM}{\mathbf{P}}

%%%%%%%%%%%%%%%%%%%%%%%%%%%%%%%%%%%%%%%%%%%%%%%%%%%%%%%%%%%

%%%% MACROS %%%%%%%%%%%%%%%%%%%%%%%%%%%%%%%%%%%%%%%%%%%%%%%

\newcommand{\PM}{\mathbf{P}}

%%%%%%%%%%%%%%%%%%%%%%%%%%%%%%%%%%%%%%%%%%%%%%%%%%%%%%%%%%%

%%%% MACROS %%%%%%%%%%%%%%%%%%%%%%%%%%%%%%%%%%%%%%%%%%%%%%%

\newcommand{\PM}{\mathbf{P}}

%%%%%%%%%%%%%%%%%%%%%%%%%%%%%%%%%%%%%%%%%%%%%%%%%%%%%%%%%%%

%%%% MACROS %%%%%%%%%%%%%%%%%%%%%%%%%%%%%%%%%%%%%%%%%%%%%%%

\newcommand{\PM}{\mathbf{P}}

%%%%%%%%%%%%%%%%%%%%%%%%%%%%%%%%%%%%%%%%%%%%%%%%%%%%%%%%%%%

%%%% MACROS %%%%%%%%%%%%%%%%%%%%%%%%%%%%%%%%%%%%%%%%%%%%%%%

\newcommand{\PM}{\mathbf{P}}

%%%%%%%%%%%%%%%%%%%%%%%%%%%%%%%%%%%%%%%%%%%%%%%%%%%%%%%%%%%

%%%% MACROS %%%%%%%%%%%%%%%%%%%%%%%%%%%%%%%%%%%%%%%%%%%%%%%

\newcommand{\PM}{\mathbf{P}}

%%%%%%%%%%%%%%%%%%%%%%%%%%%%%%%%%%%%%%%%%%%%%%%%%%%%%%%%%%%

%%%% MACROS %%%%%%%%%%%%%%%%%%%%%%%%%%%%%%%%%%%%%%%%%%%%%%%

\newcommand{\PM}{\mathbf{P}}

%%%%%%%%%%%%%%%%%%%%%%%%%%%%%%%%%%%%%%%%%%%%%%%%%%%%%%%%%%%

%%%% MACROS %%%%%%%%%%%%%%%%%%%%%%%%%%%%%%%%%%%%%%%%%%%%%%%

\newcommand{\PM}{\mathbf{P}}

%%%%%%%%%%%%%%%%%%%%%%%%%%%%%%%%%%%%%%%%%%%%%%%%%%%%%%%%%%%

%%%% MACROS %%%%%%%%%%%%%%%%%%%%%%%%%%%%%%%%%%%%%%%%%%%%%%%

\newcommand{\PM}{\mathbf{P}}

%%%%%%%%%%%%%%%%%%%%%%%%%%%%%%%%%%%%%%%%%%%%%%%%%%%%%%%%%%%

\sstart
\stitle{Cardinality}

\ssection{Why}

We want to speak of the number of elements
of a set. Subtetly arises when we can not finish
counting the set's elements.

\ssection{Finite Definition}

If a set $A$ is contained in a set $B$ and not equal to $B$,
we say that $B$ is a \ct{larger set}{largerset} than $A$.
Conversely, we say that $A$ is a \ct{smaller set}{smallerset} than $B$.
We reason that we could pair the elements of $B$ with themselves
in $A$ and still have some elements of $B$ left over.

A \ct{finite set}{finiteset} is one whose elements we can count
and the process terminates.
For example, $\set{1, 2, 3}$ or $\set{a,b,c,d}$.
The \ct{cardinality}{cardinality} of a finite set is the number of
elements it contains.
The cardinality of $\set{1, 2, 3}$ is 3 and the cardinality of
$\set{a,b,c,d}$ is 4.

\ssubsection{Notation}

Let $A$ be a non-empty set.
We denote the cardinality of $A$ by $\card{A}$.

\ssection{Infinite Definition}

Suppose we know that the counting process could never terminate.
This situation superficially seems bizarre, but is in fact built
in to some of our fundamental notions: namely, the natural numbers.
We defined the natural numbers in a manner which made them not
finite.

If we had a bag of natural numbers, we could use the total order
to find the largest, and then use the existence of a successor to
add a new largest number.
Therefore, bizarrely, the process of counting the natural numbers
can not terminate.

An \ct{infinite set}{infiniteset} is a non-empty set which is not finite.
So the natural numbers are an infinite set.
Alternatively we say that there are \ct{infinitely many}{infinitelymany}
natural numbers.
The negating prefix \say{in} emphasizes that we have defined
the nature of the size of the naturals indirectly: their size
is not something we understand from the simple intuition of
counting, but in contrast to the simple intuition of counting.

Still, we imagine that if we could go on forever, we could count
the natural numbers; so in an infinite sense, they are countable.
A \textbf{countable} set is one which is either (a) finite or
(b) one for which there exists a one-to-one function mapping
the natural numbers onto the set.

The natural numbers are countable: we exhibit the identity function.
Less obviously the integer numbers and rational numbers are countable.
Even more bizarre, the real numbers are not countable.
An \textbf{uncountable} set is one which is not countable.

\ssubsection{Notation}

We denote the cardinality of the natural numbers by $\aleph_0$.
\strats
