%!name:rooted_trees
%!need:trees
%!refs:vandenberghe2014chordal

\ssection{Why}

We want to talk about orienting the edges of a tree away from a given vertex.

\ssection{Definition}

A \t{rooted tree} is an ordered pair $((V, T), r)$ where $(V, T)$ is a tree and $r \in V$ is a distinguished vertex which we call the \t{root}.

Rooted trees are often displayed with the root at the top, as in Figure~\ref{figure:rooted_trees:rooted_tree}.

\ssubsection{Parents and Children}

Suppose $w$ is the first vertex on the path from the root to a non-root vertex $v$.
Since there is only one such path, $w$ is unique and we call it \t{the parent} of $v$.
Conversely, we call $v$ \t{a child} of $w$.
We denote the set of children by $\ch(v)$.
A vertex may have no children or it may have many children.
If it has no children we call it a \t{leaf}.

We define the \t{parent function} $\pa: V \to V$ with the convention that the \t{parent of the root} is the root.
The \t{parent of degree $k$} where $k$ is positive is $\pa^k(x)$ where $\pa^k$ is the composite of $\pa$ with itself $k$ times.
So, in particular, $\pa^{k+1}(v) = \pa(\pa^k(v))$.
We define the parent of degree $0$ of $v$ to be $v$, and denote it by $\pa^0(v) = v$.
For the tree visualized in Figure~\ref{figure:rooted_trees:rooted_tree}, $\pa(i) = g$, $\pa^2(i) = d$, $\pa^3(i) = a$.

If $w = \pa^k(v)$ for some $k \geq 0$, then $w$ is a \t{ancestor} of $v$ and $v$ is a \t{descendent} of $w$.
An ancestor $w$ of $v$ is a \t{proper ancestor} and a descendent $v$ of $w$ is a \t{proper descendent} if $w \neq v$.

The \t{depth} or \t{level} of a vertex $v$ in a rooted tree is the its distance (see \sheetref{trees}{Trees}) to the root.
We denote the level of a vertex $v$ by $\lev(v)$.
The level of the root is $0$.
If $\lev(v) = k > 0$, then $\pa^{k}(v)$ is the root.
The level function $\lev$ satisfies $\lev(v) = \lev(\pa(v)) + 1$.

% We associate with a rooted tree the directed graph constructed by orienting al edges away from the root.
% For a particular rooted tree, which is to say a particular choice of root, we obtain a directed graph.
% There are many choices of root, and so naturally we wonder if each choice of root corresponds to a different directed graph.
% Since trees are connected, each choice of root does lead to a different graph.
% If they were disconnected, some of the graphs may turn out to be the same.
% So there is a one-to-one correspondence between rooted trees and their associated directed graphs, and so we could have defined a rooted tree as a directed graph with particular properties.
%

\begin{figure}
  \centering
  \includegraphics[width=0.5\textwidth]{graphics_included/rooted_tree}
  \caption{A rooted tree.}
  \label{figure:rooted_trees:rooted_tree}
\end{figure}
