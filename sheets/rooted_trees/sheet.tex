%!name:rooted_trees
%!need:trees

\ssection{Why}

We want to talk about orienting the edges of a tree away from a given vertex.

\ssection{Definition}

A \t{rooted tree} is an ordered pair whose first object is a tree and whose second object is a vertex of the tree. We call the vertex the \t{root}.

Using the pair we can construct a directed graph by orienting all the edges away from the root.
There are as many choices of rooted trees as there are choices of root: namely, the number of vertices in the tree.
Each choice leads to a different graph since all trees are connected.
If they were disconnected, some of the graphs may turn out to be the same.

\ssubsection{Notation}

We denote the tree $T$ rooted at vertex $i$ by $(T, i)$.

\ssection{Properties}

\begin{prop}
Let $(T, i)$ be a rooted tree.
In the directed graph corresponding to this rooted tree every vertex has one parent.
\end{prop}

We denote the parent of vertex $i$ by $\pa{i}$.
