
%!name:rooted_trees
%!need:trees
%!refs:vandenberghe2014chordal
% below needed for parent and child functions, maybe change
%!need:directed_graphs

\section*{Why}

We want to talk rooting a tree at a given vertex.\footnote{Future editions will expand this intuition.}

\section*{Definition}

A \t{rooted tree} is an ordered pair $((V, T), r)$ where $(V, T)$ is a tree and $r \in V$ is a distinguished vertex which we call the \t{root}.
We visualize rooted trees with the root at the top (see the figure below).

\subsection*{Parents and children}

Suppose $w$ is the first vertex on the path from the root to a non-root vertex $v$.
Since there is only one such path, $w$ is unique and we call it \t{the parent} of $v$.
Conversely, we call $v$ \t{a child} of $w$.
We denote the set of children of $v$ by $\ch(v)$.
A vertex may have no children or it may have many children.
If it has no children we call it a \t{leaf}.

We define the \t{parent function} $\pa: V \to V$ with the convention that the \t{parent of the root} is the root.
The \t{parent of degree $k$} where $k > 0$ is $\pa^k(x)$ where $\pa^k$ is the composite of $\pa$ with itself $k$ times.
So, in particular, $\pa^{k+1}(v) = \pa(\pa^k(v))$.
We define the \t{parent of degree $0$} of $v$ to be $v$, and denote it by $\pa^0(v) = v$.
For the tree visualized in the figure below, $\pa(i) = g$, $\pa^2(i) = d$, $\pa^3(i) = a$.

If $w = \pa^k(v)$ for some $k \geq 0$, then $w$ is a \t{ancestor} of $v$ and $v$ is a \t{descendent} of $w$.
We use the term \t{proper ancestor} and \t{proper descendent} if $k > 0$ (i.e., $w \neq v$).

The \t{depth} or \t{level} of a vertex $v$ is its distance (see \sheetref{trees}{Trees}) to the root.
We denote the level of a vertex $v$ by $\lev(v)$.
The level of the root is $0$.
If $\lev(v) = k > 0$, then $\pa^{k}(v)$ is the root.
The level function $\lev$ satisfies $\lev(v) = \lev(\pa(v)) + 1$.

\begin{center}\includegraphics[width=0.50\textwidth]{./graphics/rootedtree.png}\end{center}