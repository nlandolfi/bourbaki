%!name:power_set
%!need:set_inclusion

\ssection{Why}

We want to consider the subsets of a given set.
Does a set exist which contains all the subsets.

\ssection{Definition}

We say yes.

\begin{principle}[Powers]
	For every set, there exists a set containing all of the subsets.
\end{principle}

We call the existence of this set the \t{principles of powers} and we call the set the \t{power set}.
As usual, the principle of extension gives uniqueness (see \sheetref{set_equality}{Set Equality}).
The power set of a set includes the set itself and the empty set.

\ssubsection{Notation}

We denote the power set of $A$ by $\powerset{A}$, read aloud as \say{powerset of A.}
$A \in \powerset{A}$ and $\emptyset \in \powerset{A}$.
However, $A \subset \powerset{A}$ is false.

\ssubsection{Example}

Let $a, b, c$ be distinct
objects. Let $A = \set{a, b ,c}$
and $B = \set{a, b}$. Then
$B \subset A$.
In other notation,
$B \in \powerset{A}$.
As always, $\emptyset \in \powerset{A}$
and $A \in \powerset{A}$ as well.
In this case, we can
list the elements (which are sets)
of the power set:
\[
  \powerset{A} = \set{
    \emptyset,
    \set{a},
    \set{b},
    \set{c},
    \set{a, b},
    \set{b, c},
    \set{a, c},
    \set{a, b, c}
  }.
\]

\blankpage
