%!name:powers_and_unions
%!need:set_powers
%!refs:paul_halmos/naive_set_theory/section_05

\s{Why}

How does the power set relate to a union?

\ss{Notation Preliminaries}

Let $E$ denote a set.
Let $\CA$ denote a set of subsets of the set denoted by $E$.
We define $\bigcup_{A \in \CA} A$ to mean $\cap \CA$.

\s{Basic Properties}

Here are some basic interactions between the powerset and unions.\footnote{Future editions will expand on these propositions and provide accounts of them.}

\begin{proposition}
$\powerset{E} \cup \powerset{F} \subset \powerset{\parens{E \cup F}}$
\end{proposition}

\begin{proposition}
$\bigcup_{X \in \CC} \powerset{X} \subset \powerset{\parens{\bigcup_{X \in \CC} X}}$
\end{proposition}

\begin{proposition}
  $E = \bigcup \powerset{E}$
\end{proposition}

\begin{proposition}
  $\powerset{\parens{\bigcup E}} \supset E$.
\end{proposition}
Typically $E \neq \powerset{\parens{\bigcup E}}$, in which case $E$ is a proper subset.

\blankpage
