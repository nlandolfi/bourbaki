
%!name:powers_and_unions
%!need:set_powers
%!refs:paul_halmos/naive_set_theory/section_05

\section*{Why}

How does the power set relate to a union?

\subsection*{Notation preliminaries}

Let $E$ denote a set.
Let $\mathcal{A} $ denote a set of subsets of the set denoted by $E$.
We define $\bigcup_{A \in \mathcal{A} } A$ to mean $\bigcup \mathcal{A} $.

\section*{Basic properties}

Here are some basic interactions between the powerset and unions.\footnote{Future editions will expand on these propositions and provide accounts of them.}

\begin{proposition}
$\powerset{E} \cup \powerset{F} \subset \powerset{(E \cup F)}$
\end{proposition}

\begin{proposition}
$\bigcup_{X \in \mathcal{C} } \powerset{X} \subset \powerset{(\bigcup_{X \in \mathcal{C} } X)}$
\end{proposition}

\begin{proposition}
$E = \bigcup \powerset{E}$
\end{proposition}

\begin{proposition}
$\powerset{(\bigcup E)} \supset E$.
\end{proposition}

Typically $E \neq \powerset{(\bigcup E)}$, in which case $E$ is a proper subset.
\blankpage