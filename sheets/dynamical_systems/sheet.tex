%!name:dynamical_systems
%!need:real_numbers

\ssection{Why}

We want to model natural phenomena.\footnote{Future editions will modify, and may develop dynamic systems via the genetic approach by appealing to their classical use in Newtonian physics for modeling celestial mechanics.}

\ssection{Definition}

Let $X_0, X_1, \dots, X_T$ be a sequence of sets and let $f_t: X_t \to X_{t+1}$ for $t = 0, \dots, T-1$.
We call $((X_0, \dots, X_T), (f_1, \dots, f_{T-1}$ a \t{deterministic discrete-time dynamical system}.

We call the index $t$ the \t{epoch}, the \t{stage} or the \t{period}.
We call $X_t$ the \t{state space} at period $t$.
We call $f_t$ the \t{transition function} or \t{dynamics function}.

Let $x_0 \in \CX_0$.  Define a state sequence $x_1 \in \CX_1, \dots, x_T \in \CX_T$ by
\[
    x_{t+1} = f_t(x_t, u_t).
\]
In this case we call $x_0$ the \t{initial state}.
We call the $x_t$ the \t{trajectory} associatd with initial state $x_0$.

We call $T$ the \t{horizon}.
In the case that we have an infinite sequence of state sets, input sets, and dynamics, then we refer to a \t{infinite-horizon} dynamical system.
To use language in contrast with this case, we refer to the dynamical system when $T$ is finite as a \t{finite-horizon} dynamical system.

\blankpage
