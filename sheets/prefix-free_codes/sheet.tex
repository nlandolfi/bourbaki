%!name:prefix-free_codes
%!need:codes
%!refs:david_huffman/a_method_for_the_construction_of_minimum_redundancy_codes

\ssection{Why}

Let $X$ and $A$ be finite sets.
Here are two practical considerations for constructing a code $C: X \to \strings(A)$.

First, the code should be nonsingular (injective).
In other words, no two objects in the base set have the same codewords.

Second, no additional information should be required to indicate where codewords start and end.
This second restriction indicates that no codeword should appear as the first part of another codeword of greater length.
This second implication motivates this sheet.

\ssection{Definition}

We call a string $s \in \strings(A)$ of length $m$ a \t{prefix} of a string $t \in \strings(A)$ of length $n$ if $m \leq n$ and $s_i = t_i$ for all $i \in \upto{m}$.

We call a code $c: X \to \strings(A)$ \t{prefix-free} if, for all $x \in X$, $c(x)$ is \textit{not} a prefix of $c(x')$ for all $x' \neq x$, $x' \in X$.
Otherwise, we call the code \t{prefixed}.
All prefix-free codes are uniquely decodable, but the converse is false.

\begin{proposition}
  There exists a set $X$, alphabet $A$, and prefixed code $C: X \to \CA$ such that $C$ is uniquely decodable.
  \begin{proof}
  Let $\alpha$ and $\beta$ be objects.
    Try $X = \set{\alpha, \beta}$, $A = \set{0, 1}$ and $c: X \to \strings(A)$ defined by $c(\alpha) = (0,)$ , $c(\beta) = (0,1)$.
  We proceed by induction on the length of encodings.
  Consider a length one encoding.
  It must be $(0,)$, which decodes as $(A,)$.
  Consider a length two encoding.
  It is either $(0,0)$, which decodes as $(A,A)$, or it is $(0,1)$ which decodes as $(B,)$.
  Now assume the cases $k-1$ and $k-2$.
  Now consider a length $k$ code $a \in \strings(A)$.
  It consists of $(a_{1:k-1},a_k)$.
  If $a_k = 0$, then the the code must be $(y, \alpha)$ where $y$ is the decoding of $a_{1:k-1}$.
  By the induction hypothesis, $a_{1:k-1}$ is of length $k-1$ and so uniquely decodable.
  Otherwise, $(a_{k-1}, a_k) = (0,1)$ and so the code must be $(y', \beta)$ where $y'$ is the decoding of $a_{a:k-2}$.
  By the induction hypothesis, $a_{1:k-2}$ is of length $k-2$ and so uniquely decodable.
  \end{proof}
\end{proposition}

In other words, the prefix-free codes are a \textit{strict} subset of the uniquely decodable codes.
However, our second consideration mentioned above indicates that the \say{practical} codes are prefix-free.

\blankpage
