
%!name:estimators
%!need:estimates

\section*{Why}

We have studied guessing random variables (see \sheetref{estimates}{Estimates}).
What if we can use another random variable in making our estimate?

\section*{Definition}

Let $(\Omega , \mathcal{A} , \mathbfsf{P} )$ be a probability space.
Let $U, V$ be sets.
Let $x: \Omega  \to V$ and let $y: \Omega  \to U$.
An \t{estimator} or \t{predictor} for $x$ given $y$ is a function from $U$ to $V$.
An estimate, then, corresponds to a constant estimator, and vice versa.
Some authors call the selection of an estimator \t{estimation} or an \t{estimation problem}.

\section*{Error function}

An error function is a function $e: U \times  V \to \R $ which quantifies the cost of an error.

\blankpage