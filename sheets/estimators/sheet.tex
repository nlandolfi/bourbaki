%!name:estimators
%!need:estimates

\ssection{Why}

We have studied guessing random variables (see \sheetref{estimates}{Estimates}).
What if we can use another random variable in making our estimate?

\ssection{Definition}

Let $(\Omega, \CA, \PM)$ be a probability space.
Let $U, V$ be sets.
Let $x: \Omega \to V$ and let $y: \Omega \to U$.
An \t{estimator} or \t{predictor} for $x$ given $y$ is a function from $U$ to $V$.
We use these words in contrast with the word estimate (see \sheetref{estimates}{Estimates}).
Notice, however, that we may consider a constant estimator.
Some authors call the selection of an estimator \t{estimation} or an \t{estimation problem}.

\ssection{Error function}

An error function is a function $e: U \times V \to \R$ which quantifies the cost of an error.




\blankpage
