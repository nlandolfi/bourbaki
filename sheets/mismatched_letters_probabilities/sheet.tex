
\section*{Why}

Here's a nice (surprising) example of computing an event probability.
Consider the following question:
We have $n$ letters to put into $n$ addressed envelopes, but we \textit{randomly} put them into envelopes.
What's the chance that no letter is in the correct envelope?

\section*{Example}

After numbering (see \sheetref{lists}{Lists}) the envelopes and letters, we model the uncertain outcome of assignments of letters to envelopes using the sample space $\Omega  = S_n$.
Here $S_n$ denotes the symmetric group of degree $n$, as usual (see \sheetref{permutations}{Permutations}).
We agree to interpret $\omega  \in \Omega $ so that $\omega (i)$ is the number of the \textit{letter} in the \textit{envelope} numbered $i$, where $i = 1,\dots , n$.
Suppose we put a distribution $p: \Omega  \to [0,1]$ on $\Omega $ so that every permutation is equally likely:
\[
p(\omega ) = \frac{1}{n!}
\]
We are interested in the event $W$ defined by
\[
W = \Set{\omega  \in \Omega }{\omega (s) \neq s \text{ for all } s = 1, \dots , n}
\]
which we interpret as the event that no letter is in the correct envelope.
To get a handle on this event, we express it as smaller events.

Define $A_i$ by
\[
A_i = \Set{\omega  \in \Omega }{\omega (i) = i}
\]
so that $A_i$ is the set of outcomes in which letter $i$ is in envelope $i$.
The even that at least one letter goes into the correct envelope is given
\[
\cup_{i = 1}^{n} A_i
\]
We can compute this probability using the genrealized inclusion-exclusion formula.

First, notice that the event
\[
\cap _{i = 1}^{n} A_i
\]
contains the single outcome in which all letters go into the correct envelope.
More generally, for any $r$ between $1$ and $n$, $\cap _{i = 1}^{n} A_i$ contains all outcomes in which the letters $1, \dots , r$ go into the correct envelope.
What is the size of $A_1 \cap  \cdots \cap  A_r$?
Given that the $\omega (1) = 1, \omega (2) = 2, \dots , \omega (r) = r$, there are $n-r$ envelopes and $n-r$ ways of assigning letters to them.
Thus, by the fundamental principle of counting
\[
\num{\cap _{i = 1}^{r} A_i} = (n-r)!
\]
Thus the probability of the event is
\[
P(\cap _{i = 1}^{r} A_i) = \sum_{\omega  \in \cap _{i = 1}^{r} A_i} p(\omega ) = \frac{(n-r)!}{n!}.
\]
where we have used the fact that $p(\omega ) = 1/n!$ for every $\omega  \in \Omega $.
A similar argument holds for any distinct $i_1, \dots , i_r$ indices, where $i_j$ are distinct integers between $1$ and $n$.
So $P(A_{i-1} \cap  \cdots \cap  A_{i_r}) = (n-r)!/n!$
Thus, each probability in the $r$th sum of the inclusion-exclusion formula is $(n-r)!/n!$, since the $r$th sum as ${n \choose r}$ terms, the $r$th sum is
\[
{n \choose r} \frac{(n-r)!}{n!} = \frac{n!}{r!(n-r)!}\frac{(n-r)!}{n!} = \frac{1}{r!}
\]

Finally, we apply the generalized inclusion-exclusian formula to obtain
\[
P(A_1 \cup \cdots \cup A_n) = \frac{1}{1!} - \frac{1}{2!} + \frac{1}{3!} + \cdots + (-1)^{n-1} \frac{1}{n!}.
\]
Hence, the probablty that no letter goes into the correct envelope $W = \Omega  - \cup_{i = 1}^{n} A_i$ is
\[
1 - P(A_1 \cup \cdots \cup A_n) = 1 + \frac{1}{2!} - \frac{1}{3!} + \cdots + (-1)^n\frac{1}{n!}
\]
If we take $n \to \infty$, the above series series converges to $1/e \approx 0.37$.\footnote{Future editions will define $e$.}

This is sometimes called the \t{secretary problem}.

% TODO: finish 

\blankpage