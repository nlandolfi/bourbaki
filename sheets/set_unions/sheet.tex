%!name:set_unions
%!need:set_specification

\ssection{Why}

Can we combine sets?

\ssection{Definition}

We say yes.

\begin{principle}[Union]
	Given a set of sets, there exists a set which contains all elements which belong to any of the sets.	
\end{principle}
We call this the \t{principle of union}.
If we have one set and another, the axiom of unions says that there exists a set which contains all the elements that belong to at least one of the former or the latter.

The set guaranteed by the principle of union may contain more elements than just those which are elements of a member of the the given set of sets.
No matter: apply the axiom of specification (see \sheetref{set_specification}{Set Specification}) to form the set which contains only those elements which are appear in at least one of any of the sets.
The set is unique by the principle of extension.
We call that unique set the \t{union} of the sets.

\ssection{Notation}

Let $\CA$ be a set of sets.
We denote the union of $\CA$ by $\union \CA$.
So 
\[
	(\forall x)((x \in (\union\CA)) \iff (\exists A)((A \in \CA) \land x \in A)).
\]

\ssection{Simple Facts}

It is reasonable for the union of the empty set to be empty and for the union of one set to be itself.

\begin{prop}
	$\union \emptyset = \emptyset$
\end{prop}
\begin{proof}
	\begin{caccount}
		\chave{}{$(\forall x)(x \in (\union \emptyset)) \iff (\exists y)(y \in \emptyset \land x \in y)$};
		\cthus{myref}{$\neg(\exists y)(y \in \emptyset) \implies (\forall x)(x \not\in (\union \emptyset))$}{contrapositive};
		\cthus{}{$\union\emptyset = \emptyset$}{\eqref{myref}}.
	\end{caccount}
\end{proof}

\begin{prop}
	$\union \set{A} = A$
\end{prop}