
\section*{Why}

We name and discuss the topological sigma algebra on the real numbers; the language and results generalize to $\R ^d$.

\section*{Definition}

The \t{Borel sigma algebra} is the topological sigma algebra for the real numbers with the usual topology; we call its members the \t{Borel sets}.

\subsection*{Notation}

We denote the Borel sigma algebra on $\R ^d$ by $\mathcal{B} (\R ^d)$.
We denote $\mathcal{B} (\R ^1)$ by $\mathcal{B} (\R )$.

\section*{Alternate Generations}

The Borel sigma algebra is useful because it contains interesting sets besides the open sets.
To make precise this statement, we show that the Borel sigma algebra is generated by, and therefore contains, other common subsets of $R$.

\begin{proposition}
If a sigma algebra $\mathcal{A} $ includes a particular set of subsets $\mathcal{B} $, then $\mathcal{A} $ includes $\sigma (\mathcal{B} )$.
\end{proposition}

\begin{proposition}

\label{borelalternategenerations}Each of
  \begin{enumerate}
    \item the collection of all closed subsets of $\R$,
    \item the collection of all subintervals of $\R$ of the form $(-\infty, b]$,
    \item the collection of all subintervals of $\R$ of the form $(a, b]$,
generate $\mathcal{B} (\R)$.
  \end{enumerate}
\end{proposition}

\begin{proof}Denote the sigma algebra which corresponds to (1) by $\mathcal{B} _1$, that which corresponds to (2) by $\mathcal{B} _2$, and that which corresponds to (3) by $\mathcal{B} _3$.
It suffices to establish $\mathcal{B} (R) \subset \mathcal{B} _3 \subset \mathcal{B} _2 \subset \mathcal{B} _1 \subset \mathcal{B} (R)$.

Start with $\mathcal{B} _1$.
Closed sets are the complement of open sets.
Thus $\mathcal{B} (\R )$ contains all closed sets and so contains the sigma algebra generated by all closed sets, namely $\mathcal{B} _1$.

Next, $\mathcal{B} _2$.
The intervals $(-\infty, b])$ are closed.
Thus $\mathcal{B} _1$ contains all such intervals, and so contains the sigma algebra generated by such intervals, namely $\mathcal{B} _2$.

Next, $\mathcal{B} _3$.
An interval $(a, b]$ is $(-\infty, b) \cap  C_R((-\infty, a])$.
Thus, all such intervals are contained in $\mathcal{B} _2$, and so $\mathcal{B} _2$ contains the sigma algebra generated by all such intervals, namely, $\mathcal{B} _3$.

Each open interval of $R$ is the union of a sequence of sets $(a, b]$; namely $(a, b-1/n]$.
So $\mathcal{B} _3$ contains all open intervals $(a, b)$.
Each open set of $R$ can be written as a countable union of open intervals.\footnote{Future editions will include a proof.}
Thus, $\mathcal{B} _3$ contains all open sets, and therefore contains the sigma algebra generated by the open subsets, namely $\mathcal{B} (R)$.
\end{proof}
\begin{proposition}
Each of:
  \begin{enumerate}
    \item the collection of all closed subsets of $\R^d$,
    \item the collection of all closed half-spaces of $\R^d$ of the form
\[
\Set{(x_1, \dots , x_d)}{x_i \leq b_i}
\]
for some index $i$ and some $b$ in $\R$, and
    \item the collection of all rectangles of $\R^d$ of the form
\[
\Set{(x_1, \dots , x_d)}{a_i < x_i \leq b_i}
\]
  \end{enumerate}
generate $B(R^d)$.
\end{proposition}

\begin{proof}Follow the proof of Proposition~\ref{borelalternategenerations}.

The complement of open sets are closed.
Closed half spaces are closed.
A strip of the form $\Set{(x_1, \dots , x_d)}{a_i < x_i \leq b_i}$ is the intersection of two half-spaces in (b).
Each rectangle in (c) is the union of $d$ such strips.\footnote{Future editions will complete the proof:
open rectangles are unions of
rectangles in (c) and
open sets are union of open rectangles.}
\end{proof}