%!name:graphs
%!need:relations
%% for child and parent functions?
%%!need:set_inclusion

\ssection{Why}

We want to visualize relations.

\ssection{Definition}

A \ct{graph}{graph} is a set and a relation
on the set.
The graph is \ct{undirected}{undirectedgraph}
if the relation is symmetric; otherwise the
graph is \ct{directed}{directedgraph}.

A \ct{vertex}{vertex} of the graph is an element
of the set.
The set is called the \ct{vertex set}{vertexset}.
An \ct{edge}{edge} of the graph is an element of
the relation.
The relation is called the \ct{edge set}{edgeset}.

If the graph is directed, we call the first element of an edge the \t{parent} of the second element. We call the second element of an edge the \t{child} of the first element. So we can discuss the set of parents or set of children of a particular vertex (these sets may be empty).


%If we have a relation between two sets, the
%standard corresponding graph is that obtained
%by considering the relation as defined on the
%union of the two sets. In this case we say the
%graph is \ct{bipartite}{bipartite}.

\ssubsection{Notation}

We denote the vertex set by $V$, a mnemonic for
vertex.
We denote the edge set by $E$, a mnemonic for
edge.
We denote a graph by $(V, E)$.
If the vertex set is assumed, or if every vertex appears in $E$  we can unambiguously refer to the graph by $E$.

%Let $\pa: V \to \powerset{V}$ and $\ch: V \to \powerset{V}$ be the functions associating to each vertex its set of parents and set of children, respectively.
%As usual, we denote the parents of vertex $v$ by $\pa_v$ and the children by $\ch_v$.

\ssubsection{Visualization}

We visualize a graph by drawing a
point for each vertex.
If two vertices $u$ and $v$ are in relation,
we draw a line from the point corresponding
to $u$ to the point corresponding to $v$ with
an arrow at the point corresponding to $v$.
If the graph is undirected, we omit arrows.
Here are all undirected graphs on three
vertices.

%\includegraphics[width=0.9\textwidth]{../graphs/figs/graphs}

\ssection{Paths}

A path in a graph is a sequence of
vertices with the property that
consecutive vertices are related.
A path \ct{cycles}{cycles} if a vertex
appears more than once.
A path is \ct{finite}{pathfinite} if the
sequence is finite.
A \ct{loop}{loop} is a finite path
that cycles once.
A finite path from vertex $u$ to vertex $v$
is a path starting with $u$ and
ending with $v$.
The \ct{length}{length} of a finite path is
the length of the sequence.

\ssection{Properties}

A graph is \ct{connected}{connected}
if there is a path between every pair
of vertices. A graph is
\ct{acyclic}{acyclic} if none of its
paths cycle.
