%!name:singular_measures
%!need:measures

\ssection{Why}

TODO

\ssection{Definition}

A measure is
\ct{concentrated}{}
on a set if the measure
on the complement of the
set is zero.
A signed or complex measure
is \ct{concentrated}{}
on a set if its variation
is concentrated on the set.

Two measures
(or signed or complex measures)
are
\ct{mutually singular}{}
if there exists a set on which
one is concentrated and on whose
complement the other is concentrated.

\ssubsection{Notation}

Let $(X, \SA)$ be a measurable
space and let $\mu$ be a measure.
Then $\mu$ is concentrated on a
set $C \in \SA$ if
$\mu(X - C) = 0$.
If $\nu$ is a signed or complex
measure, then $\nu$ is concentrated
on $C \in \SA$ if $\abs{\nu}$
is concentrated on $C$; in which
case $\abs{\nu}(X - C) = 0$.

Let $\mu$ and $\nu$ be measures
on $(X, \SA)$. Then $\mu$ and $\nu$
are mutually singular if there exists
a set $A\in\SA$ so that $\mu$ is
concentrated on $A$ and $\nu$ is
concentrated on $X - A$.
We denote that two measures are singular
by $\mu \perp \nu$, read aloud as
\say{mu perp nu}.
