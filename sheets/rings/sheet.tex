
\section*{Why}

We generalize the algebraic structure of \textit{addition} and \textit{multiplication} over the integers.\footnote{Future editions will likely modify this sheet, and give a genetic treatment involving the solution of polynomial equations by Galois.}

\section*{Definition}

A \t{ring} (or \t{ring with identity}) $(R, f, g)$ is a set $A$ and two binary operations on $R$ satisfying the following set of conditions.

(A)
(i) $f$ is \textit{associative}.
(ii) $f$ is \textit{commutative},
(iii) $A$ has an \textit{identity element} for $f$ (i.e., there is $e \in R$ with $f(r, e) = f(e, r) = r$ for all $r \in R$
(iv) $R$ has \textit{inverse elements} for $f$ (i.e., for any $r \in R$, there is $\tilde{r} $ satisfying $f(r,\tilde{r}) = f(\tilde{r}, r) = e$)

(B)
(i) $g$ is \textit{associative};
(ii) $R$ has an \textit{identity element} for $g$ (i.e., for any $r \in R$, there is $\tilde{e} \in A$ satisfying $g(r, \tilde{e}) = g(\tilde{e},r) = r$)

(C)
(i) $g$ \textit{left distributes}:
\[
g(f(x, y), \alpha ) = f(g(\alpha ,x), g(\alpha ,y)) \quad \text{for all } x, y, \alpha  \in R
\]
(ii) $g$ \textit{right distributes}:
\[
g(\alpha , f(x,y)) = f(g(\alpha ,x), g(\alpha ,y)) \quad \text{for all } x, y, \alpha  \in R
\]

Conditions (A) concern $f$, conditions (B) concern $g$, and conditions (C) relate the two.

Clearly, $\Z $ with addition and multiplication is a ring.
The element referred to in (A.2) is $0 \in \Z $, so we refer to this element in any ring as the \t{additive identity}.
That referred to (A.3) is $1 \in \Z $, so we refer to this element in any ring as the \t{multiplicative identity}.
We refer to the elements mentioned in (A.4) as \t{additive inverses}.
We call to $f$ \t{ring addition} and $g$ \t{ring multiplication}.


% <div data-littype='run'> Although integer products are commutative, we have not
%    required this aspect (future editions will elaborate). </div>
%    

A ring which for which multiplication is commutative is called a \t{commutative ring}.
Note that a ring is \textit{always} commutative with respect to addition, here the term commutative refers to multiplication.
A ring for which there are inverse elements, excepting 0, is called a \t{division ring}).

Of course, the integers form a ring with the usual notion of addition and multiplication.
For another trivial example, consider $\set{0}$ with $0+0 = 0$ and $0\cdot 0 = 0$; this is called the \t{zero ring} (any ring isomorphic to this one is called a \t{trivial ring} or \t{zero ring}).

% TODO: define isomorphic??? 

\subsection*{Notation}

The notation commonly adopted in discussing rings relies on analogy with the set of integers $\Z $.
We denote the ring addition by $+$ and ring multiplication by $\cdot $.
Moreover, we denote the ring's additive identity by $0$ and the ring's multiplicative identity by $1$.
Finally, we denote the additive inverse of $a \in A$ by $-a$.

Rewriting the conditions (A), (B), (C) in this notation gives familiar-looking relations, from when the objects involved were integers.
(A) (1) $a+(b + c) = (a+b)+c$; (2) $a+b = b+a$; (3) $a + 0 = 0 + a = a$; (4) $a + (-a) = 0$.
(B) (1) $a(bc) = (ab)c$; (2) $1a = a1 = a$.
(C) (1) $(a+b)c = ac + bc$; (2) $c(a+b) = ca + cb$.

\subsection*{Immediate consequences}

We need not require that $0x = 0$, because we can deduce it:
\[
0x + x = (0 + 1)x = 1x = x.
\]
Similarly, $(-a)b = -(ab)$ since
\[
ab + (-a)b = (a + (-a))b = 0b = 0.
\]
Other familiar relations among the integers, e.g. $(-a)(-b) = ab$, may be deduced.

%macros.tex
%%%%% MACROS %%%%%%%%%%%%%%%%%%%%%%%%%%%%%%%%%%%%%%%%%%%%%%%
%%%%%%%%%%%%%%%%%%%%%%%%%%%%%%%%%%%%%%%%%%%%%%%%%%%%%%%%%%%%
