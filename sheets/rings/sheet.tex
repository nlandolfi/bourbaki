%!name:rings
%!need:integer_arithmetic
%!refs:peter_cameron/introduction_to_algebra

\ssection{Why}

We generalize the algebraic structure of addition and multiplication over the integers.
  \ifhmode\unskip\fi\footnote{
Future editions will likely modify this sheet, and give a genetic treatment involving the solution of polynomial equations by Galois.
  }

\ssection{Definition}

A \t{ring} $(A, f, g)$ is a set $A$ and two operations on $A$ satisfying the following set of conditions.

(A)
(i) $f$ is associative.
(ii) $f$ is \textit{commutative},
(iii) $A$ has an identity for $f$ (i.e., is $e \in A$ with $f(a, e) = f(e, a) = a$ for all $a \in A$
(iv) $A$ has inverse elements for $f$ (i.e., for any $a \in A$, there is $\tilde{a} $ satisfying $f(a,\tilde{a}) = f(\tilde{a}, b) = e$)

(B)
(i) $g$ is associative;
(ii) $A$ has an idenity element for $g$ (i.e., there is $\tilde{e} \in A$ satisfying $g(a, \tilde{e}) = g(\tilde{e},a) = a$)

(C)
(i) $g$ left distributes: $g(f(a, b), c) = f(g(a,c), g(b,c)$;
(ii) $g$ right distributes: $g(c,f(a,b)) = f(g(c,a), g(c,b))$.

Conditions (A) concern $f$, conditions (B) concern $g$, and conditions (C) relate the two.
Define $\psi : \Z  \times \Z  \to \Z $ by $\psi (a, b) = a+b$ and $\pi : \Z  \times \Z  \to \Z $ by $\pi(a, b) = a\cdot b$.
We have defined a ring so that $(\Z , \psi , \pi )$ is one.
The element referred to in (A.2) is $0 \in \Z $, so we refer to this element in any ring as the \t{additive identity}.
That referred to (A.3) is $1 \in \Z $, so we refer to this element in any ring as the \t{multipliciative identity}.
We refer to the elements mentioned in (A.4) as \t{additive inverses}.
We call to $f$ \t{ring addition} and $g$ \t{ring multiplication}.
Although integer products are commutative, we have not required this aspect (future editions will elaborate).

\ssubsection{Notation}
Our notation furthers this analogy with $\Z $.
We denote the ring addition by $+$ and ring multiplication by $\cdot $.
Moreover, we denote the ring's additive identity by $0$ and the ring's multiplicative identity by $1$.
Finally, we denote the additive inverse of $a \in A$ by $-a$.

These notational conventions make the condtions (A), (B), (C) familiar relations among integers.
(A) (1) $a+(b + c) = (a+b)+c$; (2) $a+b = b+a$; (3) $a + 0 = 0 + a = a$; (4) $a + (-a) = 0$.
(B) (1) $a(bc) = (ab)c$; (2) $1a = a1 = a$.
(C) (1) $(a+b)c = ac + bc$; (2) $(c(a+b) = ca + cb$.

\ssubsection{Immediate consequences}
We needn't require that $0x = 0$, because we can deduce it:
  \[
0x + x = (0 + 1)x = 1x = x.
  \]
Similarly, $(-a)b = -(ab)$ since
  \[
ab + (-a)b = (a + (-a))b = 0b = 0.
  \]
Other familiar relations among the integers, e.g. $(-a)(-b) = ab$, may be deduced.
