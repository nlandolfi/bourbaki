%!name:smooth_manifolds
%!need:smooth_functions
%!need:real_neighborhoods
%!refs:milnor/1
%!refs:guillemon/1

\ssection{Why}\footnote{Future editions will include.}

\ssection{Definition}

A subset $M \subset \R^n$ is a \t{smooth manifold} of dimension $d$ if for every $x \in M$, there exists a neighborhood $V$ of $x$ in $X$ that is diffeomorphic to an open subset $U$ of $\R^d$.
In this case we say that the set is \t{locally diffeomorphic} to $\R^d$.

A diffeomorphism $\phi: U \to V$ is called a \t{parameterization} of the neighborhood of $V$.
% Recall $V \subset X$ a neighborhood of $x$ means that there exists an open set in $V$ around $x$.
Its inverse diffeomorphism $\phi^{-1}$ is called a \t{coordinate system} (or system of \t{coordinates}) on $V$.

\ssubsection{Notation}

We denote the dimension of a manifold $M$ by $\dim M$.

\ssubsection{Submanifolds}

If $X$ and $Z$ are both manifolds in $R^n$ and $Z \subset X$, then we call $Z$ a \t{submanifold} of $X$.
In particular, $X$ is a submanifold of $R^n$.
Any open set of a manifold $X$ is a submanifold $X$.\footnote{Future editions will expand.}


\blankpage
