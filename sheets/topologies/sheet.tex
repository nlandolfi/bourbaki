
\section*{Why}

We want to generalize the notion of continuity.

\section*{Definition}

Given a set $X$, a \t{topology} on $X$ is a set of subsets of $X$ for which (1) the empty set base set are distinguished (2) the intersection of a \textit{finite} family of distinguished subsets is distinguished, and (3) the union of a family of distinguished subsets is distinguished.
We call the elements of the topology the \t{open sets}.

A \t{topological space} is an ordered pair: a base set and a set distinguished subsets of the base set which are a topology.

\subsection*{Notation}

Let $X$ be a non-empty set.
For the set of distinguished sets, we tend to use $\mathcal{T} $, a mnemonic for topology, read aloud as ``script T''.
We tend to denote elements of $\mathcal{T} $ by $O$, a mnemonic for open.
We denote the topological space with base set $X$ and topology $\mathcal{T} $ by $(X, \mathcal{T} )$.
We denote the properties satisfied by elements of $\mathcal{T} $:
    \begin{enumerate}
      \item $X, \varnothing \in \mathcal{T} $
      \item if $O_1, \dots , O_n \in \mathcal{T} $, then $\bigcap_{i = 1}^{n} O_i \in \mathcal{T} $
      \item if $O_\alpha  \in \mathcal{T} $ for all $\alpha  \in I$, then $\bigcup_{\alpha  \in I} \in \mathcal{T} $
    \end{enumerate}

\section*{Examples}

$\R $ with the open intervals as the open sets is a topological space.

\blankpage