%!name:bit_strings
%!need:sequences
%!need:natural_exponents

\ssection{Why}

We name sequences in $\set{0, 1}$.\footnote{Why we want to name these requires more explanation, to be included in future editions.}

\ssection{Definition}

A \t{bit string} is a finite sequence in the set $\set{0, 1}$.

If a bit string has length one, we refer to it as a \t{bit}.
Using this terminology, it is natural to call the sequence terms \t{bits}.
Other terminology for bit strings includes \t{binary sring}, \t{bit sequence} and \t{digital datum}.

If the bit string has length eight, we refer to it as a \t{byte}.
Using this terminology, a \t{kilobyte} is a length $8 * 2^{10}$ bit string.
In other words, a kilobyte is $2^{10} = 1024$ bytes, or roughly one thousand bytes.
Likewise a \t{megabyte} is a length $8 * 2^{20}$ bit string.
A megabyte is $2^{20} = 1048576$ bytes, or roughly one million bytes.
Similarly a \t{gigabyte} is $2^{30}$ bytes and a \t{terabyte} is $2^{40}$ bytes.\footnote{Warning: some authors use these monikers with powers of ten. For example, a kilobyte is exactly one thousand bytes. etc.}

\blankpage
