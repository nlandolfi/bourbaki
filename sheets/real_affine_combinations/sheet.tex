
\section*{Definition}

An \t{affine combination} from $\R ^n$ is a linear combination whose scalars sum to 1.
As with linear combinations, we say that $y$ is an \t{can be written as an affine combination of} of the vectors $x_1, \dots , x_k \in \R ^n$ if there exists $\lambda _1, \dots , \lambda _k \in \R $ so that
\[
y = \sum_{i =1 }^{k} \lambda _i x_i
\]
and $\sum_{i = 1}^{k} \lambda _i = 1$.

All affine combinations of two distint vectors $x, y \in \R ^n$ is the line through $x$ and $y$, which we denote as usual $L(x, y)$.
In other words,
\[
L(x, y) = \Set{(1-\lambda x + \lambda y}{\lambda  \in \R }
\]

A set of vectors $\set{v_1, \dots , v_k}$ is \t{affinely dependent} if one can be written as an affine combination of the others.
A set of vectors which is not affinely dependent is called an \t{affinely independent set of vectors}.
An equivalent condition is that there exist an affine combination in these vectors in which at least one scalar is nonzero and the sum of all the scalars is 1.

\begin{proposition}
Suppose $y \in \R ^n$.
The set $X = \set{x_1, \dots , x_k} \subset \R ^n$ is affinely dependent if and only if $y + X$ is.
\end{proposition}

\begin{proposition}
Suppose $y \in \R ^n$.
The set $X = \set{x_1, \dots , x_k} \subset \R ^n$ is affinely dependent if and only if $y + X$ is.
% from Grunbaum p 3 

\end{proposition}

\blankpage