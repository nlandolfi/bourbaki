%!name:set_equality
%!need:identity
%!need:sets

\ssection{Why}

When are two
sets the same?

\ssection{Definition}

Consider the sets
$A$ and $B$.
If $A$ is $B$,
then every element of
$A$ is an element of
$B$ and every element
of $B$ is an element of
$A$.

What of the converse?
If every element of $A$
is an element of $B$
and vice versa is $A$
the same as $B$?
We declare the affirmative.

Thus:
Two sets are
the same if and only
if they have the same
elements.
We call the elements of a set
its \ct{extension}{} and
this above statement
is sometimes called the
\ct{axiom of extension}{}.

The importance is that we have
given ourselves a way to argue
two sets are equivalent. Argue
the consequence of the first
paragraph, and the use the
axiom of extension to conclude
that the sets are the same.

\ssubsection{Notation}

Let $A$ and $B$ be
two sets.
As with any objects,
we denote that
$A$ and $B$
are equal
by $A = B$.
The axiom of extension is
\[
  A = B \Leftrightarrow (a \in A \implies a \in B) \land (b \in B \implies b \in A).
\]

\ssection{A Contrast}

It is useful to compare the axiom
of extension for sets with elements
to a an analogue for human beings
with ancesors (parents, grandparents,
and so on).
If two human beings are equal,
then they have the same set of ancestors.
(this is analogue of the the \say{only if} part of
the axiom of extension).
However, if two human beings have
the same set of ancestors, they need
not be the same (this is the analogue of the
\say{if} part of the axiom
of extension).
Siblings have the same ancestors,
but are different people.

So we conclude that the axiom of extension
is not just a logically necessary consequence
of equality.
In fact, it contains aspects of our notion
of belonging.
