%!name:set_equality
%!need:identity
%!need:sets

\ssection{Why}

When are two
sets the same?

\ssection{Definition}

Consider the sets
$A$ and $B$.
If $A$ is $B$,
then every element of
$A$ is an element of
$B$ and every element
of $B$ is an element of
$A$.

What of the converse?
If every element of $A$
is an element of $B$
and vice versa is $A$
the same as $B$?
We axiomatically answer
this question in the affirmative.

Thus:
Two sets are
the same if and only
if they have the same
elements.
We call the elements of a set
its \ct{extension}{} and
this above statement
is sometimes called the
\ct{axiom of extension}{}.

The importance is that we have
given ourselves a way to argue
two sets are equivalent. Argue
the consequence of the first
paragraph, and the use the
axiom of extension to conclude
that the sets are the same.

\ssubsection{Notation}

Let $A$ and $B$ be
two sets.
As with any objects,
we denote that
$A$ and $B$
are equal
by $A = B$.
The axiom of extension is
\[
  A = B \Leftrightarrow (a \in A \implies a \in B) \land (b \in B \implies b \in A).
\]
