%!name:set_equality
%!need:set_inclusion
%!refs:bert_mendelson∕introduction_to_topology∕theory_of_sets∕sets_and_subsets
%!refs:paul_halmos∕naive_set_theory

\section*{Why}

When are two sets the same?

\section*{Definition}

Let $A$ and $B$ denote sets.
If $A = B$ then every element of $A$ is an element of $B$ and every element of $B$ is an element of $A$.
In other words, $(A = B) \implies ((A \subset B) \land (B \subset A))$.

%\begin{account}[Joint Membership]
%  \nameee{$A$}{$B$}{$x$}
%  \have{set_equality:joint_membership:equality}{$A = B$}
%  \have{set_equality:joint_membership:belonging}{$x \in B$}
%  \thus{set_equality:joint_membership:conclusion}{$x \in A$}{\ref{set_equality:joint_membership:equality},\ref{set_equality:joint_membership:belonging}}
%\end{account}

What of the converse?
Suppose every element of $A$ is an element of $B$ and every element of $B$ is an element of $A$.
Then $A = B$?
We define it to be so.
Sets are determined by their members.

\begin{principle}[Extension]
Sets are the same if every member of one is a member of the other and vice versa.
\end{principle}

In other words, two sets are identical if and only if every element of one is an element of the other.
This principle is sometimes called the \t{principle of extension}.
We refer to the elements of a set as its \t{extension}.
Roughly speaking, we have declared that if we know the extension then we know the set.
A set is determined by its extension.

\subsection*{Deductive principle}

We can use this definition to deduce $A= B$ if we first deduce $A \subset B$ and $B \subset A$.
With these two implications, we use the principle of extension to conclude that the sets are the same.
In other words, $(A = B) \iff ((A \subset B) \land (B \subset A))$.
We also describe this fact by saying that inclusion ($\subset)$ is \t{antisymmetric}.

\subsection*{Belonging and sets compared with ancestry and humans}

Compare the principle of extension for identifying sets from their elements with an analogous principle for identifying people from their ancestors.

We can consider a person's ancestors.
Namely, the person's parents, grandparents, great grandparents and so on.
It is clear that if we label the same human with two names $A$ and $B$, then $A$ and $B$ have the same ancestors.
In other words, same human implies same ancestors.
This is the analog of \say{if two sets are equal they have the same members}.

On the other hand, if we have two people denoted by $A$ and $B$, and we know that $A$ has the same ancestors as $B$, we can not conclude that $A$ and $B$ denote the same human.
For example, siblings have the same ancestors but are different people.
This direction, same ancestors implies same human, is the analogue of \say{if they have the same elements, two sets are the same}.
It is false for humans and ancestors, but we define it to be true for sets and members.

The principle of extension is more than a statement about equality.
It is also a statement about our notion of belonging, of what it means to be an element of a set, and what a set is.

%TODO(next edition) discuss uniqueness
%The principle tells us that if we are talking about a set via some statement about its elements, we are talking about a unique set.

