%!name:set_equality
%!need:identities
% wierd because statements needs sets
% %!need:sets
%!need:standardized_accounts
%!need:set_inclusion


\ssection{Why}

When are two sets the same?

\ssection{Definition}

Given sets $A$ and $B$, if $A = B$ then every element of $A$ is an element of $B$ and every element of $B$ is an element of $A$.

\begin{account}[Joint Membership]
  \nameee{$A$}{$B$}{$x$}
  \have{set_equality:joint_membership:equality}{$A = B$}
  \have{set_equality:joint_membership:belonging}{$x \in B$}
  \thus{set_equality:joint_membership:conclusion}{$x \in A$}{\ref{set_equality:joint_membership:equality},\ref{set_equality:joint_membership:belonging}}
\end{account}



What of the converse?
Suppose every element of $A$ is an element of $B$ and every element of $B$ is an element of $A$.
Then $A = B$?
We define it to be so.
\begin{principle}[Determined by Members]
	Sets are the same if every member of one is a member of the other and vice versa.
\end{principle}
In other words, two sets are identical if and only if every element of one is an element of the other.
%The elements of the set are sometimes called the \t{extension}.
This principle is sometimes called the \t{principle of extension}.
%Two sets are \t{equal} if and only if every element of one is an element of the other.
%This statement is sometimes called the \t{axiom of extension}.
Roughly speaking, if we refer to the elements of a set as its \t{extension}, then we have declared that if we know the extension then we know the set.
A set is determined by its extension.

\ss{Deductive principle}

This definition allows us to deduce $A = B$ if we first deduce that each element of $A$ is an element of $B$ and then then that each element of $B$ is an element of $A$.
With these two implications, we use the principle of extension to conclude that the sets are the same.
In our notation, with the quantifiers for all sets denoted here by $A$ and $B$, 
\[
	(\forall x)(((A \subset B) \land (B \subset A)) \iff (A = B))
\]

\ssection{Belonging and sets compared with ancestry and humans}

Compare the principle of extension for identifying sets from their elements with an analogous principle for identifying people from their ancestors.

On the one hand, if two human beings are equal then they have the same ancestors.
The ancestors being the person's parents, grandparents, greatgrandparents, and so on.
This direction, same human implies same ancestors, is the analogue of the \say{only if} part of the axiom of extension.
It is true.
On the other hand, if two human beings have the same set of ancestors, they need not be the same human.
This direction, same ancestors implies same human, is the analogue of the \say{if} part of the axiom of extension.
It is false.
For example, siblings have the same ancestors but are different people.

We conclude that the axiom of extension
is more than a statement about equality.
It is also a statement about our notion of
belonging, of what it means
to be an element of a set, and what a set is.

\blankpage
