
%!name:entire_functions
%!need:complex_analytic_functions
%!refs:yellow/IX/4

\section*{Definition}

An \t{entire function} is a complex function $f: \C  \to \C $ which is analytic for all $z \in \C $.

\blankpage
\sbasic
%%%% MACROS %%%%%%%%%%%%%%%%%%%%%%%%%%%%%%%%%%%%%%%%%%%%%%%

\newcommand{\PM}{\mathbf{P}}

%%%%%%%%%%%%%%%%%%%%%%%%%%%%%%%%%%%%%%%%%%%%%%%%%%%%%%%%%%%

%%%% MACROS %%%%%%%%%%%%%%%%%%%%%%%%%%%%%%%%%%%%%%%%%%%%%%%

% use \set{stuff} for { stuff }
% use \set* for autosizing delimiters
\DeclarePairedDelimiter{\set}{\{}{\}}

% use \Set{a}{b} for {a | b}
% use \Set* for autosizing delimiters
\DeclarePairedDelimiterX{\Set}[2]{\{}{\}}{#1 \nonscript\;\delimsize\vert\nonscript\; #2}

% use \powerset{A} for power set of A
\newcommand{\powerset}[1]{2^{#1}}

\renewcommand{\emptyset}{\varnothing}

\newcommand{\SA}{\mathcal{A}}
\newcommand{\SB}{\mathcal{B}}
\newcommand{\SC}{\mathcal{C}}
\newcommand{\SD}{\mathcal{D}}
\newcommand{\SE}{\mathcal{E}}
\newcommand{\SF}{\mathcal{F}}
\newcommand{\SG}{\mathcal{G}}
\newcommand{\SH}{\mathcal{H}}
\newcommand{\SI}{\mathcal{I}}
\newcommand{\SJ}{\mathcal{J}}
\newcommand{\SK}{\mathcal{K}}
\newcommand{\SL}{\mathcal{L}}

%%%%%%%%%%%%%%%%%%%%%%%%%%%%%%%%%%%%%%%%%%%%%%%%%%%%%%%%%%%

%%%% MACROS %%%%%%%%%%%%%%%%%%%%%%%%%%%%%%%%%%%%%%%%%%%%%%%

\newcommand{\PM}{\mathbf{P}}

%%%%%%%%%%%%%%%%%%%%%%%%%%%%%%%%%%%%%%%%%%%%%%%%%%%%%%%%%%%

%%%% MACROS %%%%%%%%%%%%%%%%%%%%%%%%%%%%%%%%%%%%%%%%%%%%%%%

\newcommand{\PM}{\mathbf{P}}

%%%%%%%%%%%%%%%%%%%%%%%%%%%%%%%%%%%%%%%%%%%%%%%%%%%%%%%%%%%

\sstart
\stitle{Set Equality}

\ssection{Why}

When are two
sets the same?

\ssection{Definition}

Consider the sets
$A$ and $B$.
If $A$ is $B$,
then every element of
$A$ is an element of
$B$ and every element
of $B$ is an element of
$A$.

What of the converse?
If every element of $A$
is an element of $B$
and vice versa is $A$
the same as $B$?
We declare the affirmative.
Thus we can assert equality
of sets.

Two sets are
\ct{equal}{setequal}
if and only
if every element of one
is an element of the other.
In other words, two sets are
the same if they have the same
elements.
This statement
is sometimes called the
\ct{axiom of extension}{}.

The importance is that we have
given ourselves a way to argue
two sets are equivalent. Argue
the consequence of the first
paragraph, and the use the
axiom of extension to conclude
that the sets are the same.

An immediate consequence of the axiom
of extension.
There is only one set that
is empty,
since every empty set
is equal to
every other empty set.
Thus we speak of
the \ct{empty set}{}.

\ssubsection{Notation}

Let $A$ and $B$ be
two sets.
As with any objects,
we denote that
$A$ and $B$
are equal
by $A = B$.
The axiom of extension is
\[
  A = B \Leftrightarrow (a \in A \implies a \in B) \land (b \in B \implies b \in A).
\]

We denote the empty set
by $\emptyset$.

\ssection{A Contrast}

We can compare the axiom of extension
for sets and their elements with an
analogous statement
for human beings and their ancestors.

If two human beings are equal,
then they have the same ancestors.
The ancestors being the person's parents,
grandparents, greatgrandparents,
and so on.
This direction, same human implies
same ancestors, is the analogue of
the \say{only if} part of the axiom
of extension.
It is true.

In contrast, if two human beings have
the same set of ancestors, they need not be
be equal.
This direction, same ancestors implies
same human, is the analogue of the
\say{if} part of the axiom
of extension.
It is false:
siblings have the same ancestors,
but are different people.

We conclude that the axiom of extension
is more than a statement about equality.
It is also a statement about our notion of
belonging, of what it means
to be an element of a set, and what a set is.
\strats
