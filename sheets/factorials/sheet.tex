
\section*{Why}

How many ways are there to arrange $n$ objects in order?

\section*{Definition}

By the fundamental principle of counting, there are $n$ ways to select the first card, $n-1$ ways to select the second, and so on.
Thus, the number of ways of stacking $n$ cards in a deck is
\[
n(n-1)(n-2)\cdots1
\]
We call this number the \t{factorial} of $n$, or \t{$n$-factorial}.

%<div data-littype='section' data-litsectionlevel='1' data-litsectionnumbered='false'> Discussion </div>
%<div data-littype='paragraph'>
% <div data-littype='run'> There is one way to arrange a single person and there are
%    two ways to arrange two people. </div>
% <div data-littype='run'> What of three? </div>
% <div data-littype='run'> First, we can pick a seat, and decide which person to put
%    in that chair. </div>
% <div data-littype='run'> We have three choices of people. </div>
% <div data-littype='run'> Then, whomever we choose, there are only two ways to seat
%    the other two people in the other two chairs. </div>
% <div data-littype='run'> Since there were three ways of picking whom to sit in the
%    firs seat, there are $3×2 = 6$ ways to order the three
%    people. </div>
% <div data-littype='run'> We reason similarly for four people. </div>
%</div>

\textit{Factorial function.}
Define $f: \N   \to \N  $ recursively by $f(1) = 1$ and $f(2) = 2f(1)$, and $f(n) = nf(n-1)$ for $n \in \N  $ ($f$ exists by the the recursion theorem---see \sheetref{recursion_theorem}{Recursion Theorem}).
$f$ is defined such that $f(n)$ is $n$ factorial, for which reason we call $f$ the \t{factorial function}.
For convenience, we extend $f$ to $\omega $\footnote{See \sheetref{natural_numbers}{Natural Numbers}.}
by defining $f(0) = 1$.

\subsection*{Notation}

We denote the factorial of $n$ by $n!$, read aloud ``n factorial''.
So for example, $5! = 5\cdot 4\cdot 3\cdot 2\cdot 1$ and $0! = 1$.

\blankpage