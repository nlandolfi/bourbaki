
%!name:product_distributions
%!need:marginal_distributions

\section*{Why}

How do we get a joint probability distribution function from marginals?\footnote{Future editions will modify.}

\section*{Definition}

The \t{product distribution} of a sequence $p_1, \dots , p_n: A_i \to [0, 1]$ of distributions is the function $p: \prod_{i} A_i \to [0, 1]$ defined by
\[
p(x) = \prod_{i = 1}^{n} p_i(x_i).
\]

\begin{proposition}
Let $p_i: A_i \to [0, 1]$ be probability distributions.
Then the product distribution of $p_1, \dots , p_n$ is a probability distribution with marginals $p_1, \dots , p_n$.\footnote{Future editions will include.}
\end{proposition}

\section*{Example: fair coin repeated flips}

Suppose we want to model a coin flipped $n$ times.
Supposing the coin is fair (see \sheetref{probability\_distribtuions}{Probability Distributions}) we might use our probability distribution $p: \set{0, 1} \to [0, 1]$ which assigned $p(0) = p(1) = \nicefrac{1}{2}$.
Then the probability of obtaining a sequence of flips $x \in \set{0, 1}$ is
\[
\textstyle
p(x) = \prod_{x_i = 1} p_i(1) \prod_{x_i = 0} p_i(0).
\]
Notice that if we know $p_i(1) = \rho _i$, then we know $p(0) = (1-\rho _i)$ and so we can write the above as
\[
\textstyle
p(x) = \prod_{x_i = 1} \rho _i \prod_{x_i = 0} (1-\rho _i).
\]
