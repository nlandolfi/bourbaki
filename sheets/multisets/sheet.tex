
\section*{Why}

We often want to discuss lists without regard to order.
This is natural, for instance, in discussing number factorizations.
There, the associativity and commutativity of natural multipliciation mean that the factorization $(2, 3)$ and $(3,2)$ are equivalent, in the sense that they both factor 6.

\section*{Definition}

Given a set $A$, a \t{multiset} (or \t{mset}, \t{bag}) of elements \textit{of} or \textit{in} $A$ is a function $m: A \to \N  $.
In this case, the result $m(a)$ is called the \t{multiplicity} of $a \in A$.
For this reason, another term for the object we have named a multiset is \t{multiplicity function}.
An $a \in A$ with $m(a) \neq 0$ is called an \t{element} of the multiset.

The \t{size} (or \t{cardinality}) of a multiset $m$ is the sum of the multiplicities of its elements.
In summation notation, the size is defined to be
\[
\sum_{a \in A} m(a)
\]

The set
\[
\Set{a \in A}{m(a) > 0}
\]
is called the \t{support} of the multiset.

\section*{Examples}

A natural use for multisets is in factorizations of natural numbers.

\subsection*{Notation}

Notation for multisets is nonstandard---here are some common in use.
If $a$ and $b$ are two distinct objects and $A = \set{a, b}$ is the unordered pair containing them, the multiset $m: A \to \N  $ defined by $m(a) = 2$ and $m(b) = 3$ is sometimes denoted
\[
\lbag a,a,b,b,b \rbag
\]
or
\[
\{ \{ a,a,b,b,b\} \}
\]
or alternatively
\[
[a,a,b,b,b]
\]
or
\[
\set{a^2, b^3}
\]
As it appears to the present authors, none of these is desirable.
This last notation seems to be a reference to the notation for natural powers (a sheet not required by this one) and the use of multisets in number factorizations .

\subsection*{Other terminology}

Other authors refer to the ordered pair $(A, m)$ as the multi set, and call $A$ the \t{underlying set}.
We avoid this terminology, for the reason that $\dom m = A$.

\blankpage