
\section*{Definition}

An operator $T \in \mathcal{L} (V)$ is called \t{self-adjoint} (or \t{Hermitian}) if the adjoint of $T$ is itself.
In symbols, $T$ is self-adjoint if $T = T^*$.
In other words, $T$ is self-adjoint if and only if
\[
\ip{Tv, w} = \ip{v, Tw} \quad \text{for all } v, w \in V
\]

\subsection*{Properties}

\begin{proposition}
Suppose $S, T \in \mathcal{L} (V)$ are self-adjoint.
The $S + T$ are self-adjoint.
Also $\lambda T$ is adjoint for all \textit{real} $\lambda $.
\end{proposition}

\subsection*{Notation}

We will see that the adjoint on $\mathcal{L} (V)$ plays a role similar to complex conjugation on $\C $.
The self-adjoint operators will seen to be analogous to the real numbers.
A complex number is real if and only if $z = \Cconj{z}$.
Similarly, an operator is self-adjoint if and only if $T = T^*$.

\subsection*{Characterization for complex space}

% the following is axler 7.15 

\begin{proposition}
Suppose $V$ is a complex inner product space and let $T \in \mathcal{L} (V)$.
Then
\[
T = T^* \quad \iff \quad \ip{Tv, v} \in \R  \; \text{for all } v \in V
\]
\end{proposition}

\blankpage