
%!name:least_upper_bounds
%!TODO(next edition) fix needs
%!need:real_order

\section*{Definition}

Suppose $(A, \leq)$ is a partially ordered set.

An \t{upper bound} for $B \subset A$ is an element $a \in A$ so that $b \leq a$ for all $b \in B$.
A set is \t{bounded from above} if it has a least upper bound.
A \t{least upper bound} for $B$ is an element $c \in A$ so that $c$ is an upper bound and $c < a$ for all other upper bounds $a$.
\begin{proposition}
If there is a least upper bound it is unique.\footnote{Proof in future editions.}
\end{proposition}

We call the unique least upper bound of a set (if it exists) the \t{supremum}.

\section*{Notation}

We denote the supremum of a set $B \subset A$ by $\sup A$.

\blankpage