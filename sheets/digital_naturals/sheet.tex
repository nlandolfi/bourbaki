
%!name:digital_naturals
%% gets naturals through lists (strings)
%%!need:natural_numbers
%!need:bit_strings

\section*{Why}

We want to associate the natural numbers with bit strings for use on digital computers.\footnote{Future editions will expand.}

\section*{Definition}

A \t{digital natural} is a bit string.
The set of \t{$d$-bit digtal natural numbers} is the set of length-$d$ bit strings $\set{0, 1}^d$.
For example, the set of 8-bit digital naturals is the set $\set{0, 1}^8$.

\section*{Correspondence with $\N  \cup \set{0}$}

We associate $x \in \set{0, 1}^d$ corresponds to the number $\sum_{i = 1}^{d} x_i 2^i$.
For example, the bit string $(0,0,0) \in \set{0, 1}^3$ corresponds to the natural number $0 \in \omega $.
Likewise, $(1, 0, 0)$ corresponds to $1 \in \N $, $(0, 1, 0)$ corresponds to 2, $(1, 1, 0)$ corresponds to 3, etc.

Call the function so defined the \t{digital natural decoder}, and denote it by $f: \set{0, 1}^d \to \N  \cup \set{0}$.
In other words $f((0, 0, 0)) = 0$, $f((0, 1, 0)) = 2$, etc.
Call the set $f(\set{0, 1}^d)$ the set of naturals \t{representable} by length-$d$ bit strings.

Specifically, if, for $n \in \N  \cup \set{0, 1}$, there exists $x \in \set{0, 1}^d$ so that $f(x) = n$, we say that $x$ is \t{representable in $d$ bits}.

\section*{Correspondence between $d$ and $k > d$ bit naturals}

Let $x \in \set{0, 1}^d$ and $y \in \set{0, 1}^k$ with $k > d$.
Although $\set{0, 1}^d \not\subset \set{0, 1}^k$, $f(\set{0, 1}^d) \subset f(\set{0, 1}^k)$.
We can identify $x \in \set{0, 1}^d$ with $x' \in \set{0, 1}^k$ where $x' = (x_1, \dots , x_d, 0, \dots , 0)$ so that $f(x) = f(x')$.
Clearly then, if $x$ is representable in $d$ bits, it is representable in $k > d$ bits.

\section*{Addition}


% right now, natural sums comes as a need through bit strings, but this could change in the future.

We want to define addition $\oplus: \set{0, 1}^d \times \set{0, 1}^d \to \set{0, 1}^d$ so that $f(x \oplus x') = f(x) + f(x')$.
In general, we are stuck, because $x + x'$ may not be representable in $d$ bits.
Suppose, however and for the time being, that it is.\footnote{Future editions will complete.}
