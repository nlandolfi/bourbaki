
\section*{Why}

The set of real numbers is large, and so we can often embed other sets within it.
Also, we are often interested in modelling random quantities.

\section*{Definition}

A \t{real outcome variable} (or \t{real random variable}, \t{random variable}) is an outcome variable whose codomain is a subset of the real numbers.
Such variables are often called \t{quantitative}.
Caution: many authorities \textit{reserve} the term \textit{random variable} for outcome variables whose domain is $\R $.

The \t{probability mass function} (or \t{p.m.f.}, \t{pmf}) of a random variable $X: \Omega  \to \R $ is the function $f: \R \to \R $ defined by
\[
f(x) = P(X = x)
\]
If $\Omega $ is finite, then $\range X$ is a finite set, and so the probability mass function is the extension to $\R $ of the induced distribution $p: \range(X) \to \R $ of $X$.
If $\range(X)$ is finite or countable, we call $X$ a \t{discrete random variable}.

\subsection*{Notation}

For a real-valued random variable $X: \Omega  \to \R $ and $\alpha  \in \R $, we often abbreviate the sets
\[
\Set{\omega  \in \Omega }{X(\omega ) \leq \alpha } \text{ and } \Set{\omega  \in \Omega }{X(\omega ) \geq \alpha }
\]
by $\set{X \leq \alpha }$ and $\set{X \geq \alpha }$ respectively.
Also, given a probability measure $P$, we denote the probabilities of these events by $P(X \leq \alpha )$ and $P(X \geq \alpha )$, respectively.
Similar to before, the notation $X \sim f$ is shorthand for the random variable $X: \Omega  \to \R $ has probability mass function $f: \R  \to \R $.

\section*{Examples}

\textit{Tossing a fair coin $n$ times}.
Suppose we model $n$ tosses of a fair coin as usual, so that $\Omega  = \set{0,1}^n$ and $p: \Omega  \to \R $ is defined by
\[
p(\omega ) = 2^{-n} \quad \text{for all } \omega  \in \Omega
\]
Recall that we have calculuated $P(X = k)$ to be ${n \choose k} 2^{-n}$.
Thus, the probability mass function $f: \R  \to \R $ of $X$ satisifes
\[
f(k) = {n \choose k} 2^{-n} \quad \text{for } k = 0, \dots , n
\]
and $f(x) = 0$ for $x \neq 0, \dots , n$.

\section*{Cumulative distribution function}

Given a random variable $X: \Omega  \to \R $ and probability measure $P$ on $\pow{\Omega }$, the function $F: \R  \to \R $ defined by
\[
F(x) = P(X \leq x)
\]
is called the \t{cumulative distribution function} of $X$.

\blankpage