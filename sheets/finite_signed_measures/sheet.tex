
%!name:finite_signed_measures
%!need:signed_measures

\section*{Why}

For the difference of two (signed) measures to be well-defined, we need one of the two to be finite.
Otherwise, the measure of the difference on the base set involves subtracting $\infty$ from $\infty$.

\section*{Definition}

A \t{finite} signed measure is one for which the measure of every set is finite.
This condition is equivalent to the base set having finite measure (see below).

\section*{Result}

\begin{proposition}

\label{prop:finitesignedmeasures}A signed measure is finite if and only if it is finite on the base set.

\end{proposition}

\begin{proof}Let $(X, \mathcal{A} )$ be a measurable space.
Let $\mu : \mathcal{A}  \to \eri$ be a signed measure.

($\Rightarrow$) If $\mu $ is finite, then $\mu (X)$
is finite since $X \in \mathcal{A} $.

($\Leftarrow)$
Next, suppose $\mu (X)$ is finite.
Let $A \in \mathcal{A} $.
Then $X = A \union (X - A)$,
with these sets disjoint,
so by countable additivity
of $\mu $,
$\mu (X) = \mu (A) + \mu (X - A)$.
Since $\mu (X)$ finite, $\mu (A)$
and $\mu (X - A)$ are both finite.
\end{proof}