
%!name:circular_coordinates
%!need:real_plane

\section*{Why}

We identified points in $\R ^2$ with elements of the plane in a natural way.\footnote{Future editions will expand on this in the genetic approach, and likely reference celestial motion.}

\section*{Definition}

Let $(x, y) \in \R ^2$.
Then $(r, \theta ) \in \R ^2$ is the \t{polar form} or \t{circular form} of $(x, y)$ if
    \[
x = r \cos \theta  \quad \text{ and } \quad y = r \sin \theta  .
    \]
In this case we call $r$ and $\theta $ the \t{circular coordinates} or \t{polar coordinates}.

Since $\sin$ and $\cos$ polar coordinates are not unique.

\section*{Non-uniqueness}

A difficulty with polar coordinates is that there are many elements of $\R ^2$ that correspond to the same point in the plane.
For example, consider the points
    \[
(5, \pi /3), (5, -5\pi /3), (-5, 4\pi /3), (-5, -2\pi /3).
    \]
Each of these specifies the same point in $\R ^2$.

\blankpage