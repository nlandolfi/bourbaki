
\section*{Why}

We can construct functions on the ground set of an algebra by fixing an element in the ground set and defining a function which maps elements to the result of the operation applied to the fixed element and the given element.

\section*{Definition}

Let $(A, +)$ be an algebra.
For each $a \in A$, denote by $+_a: A \to A$ the function defined by
\[
+_a(b) = a + b.
\]
If $+_a$ is the identity function on $A$ then we call $a$ a \t{left identity element} of the algebra.

Similarly, denote by $+^a: A \to A$ the function defined by
\[
+^{a}(b) = b + a.
\]
If $+^a$ is the identity function on $A$ then we call $a$ a \t{right identity element} of the algebra.

An \t{identity element} of the algebra is an element which is both a left and right identity.
If the operation commutes, then a left identity and right identities are the same.

\blankpage