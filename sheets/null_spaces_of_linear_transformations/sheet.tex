
\section*{Why}

When is a linear transformation between $V$ and $W$ one-to-one?
In other words, when does
\[
Tx = Ty \Rightarrow x = y
\]
Rearranging, and using additivity, we ask when
\[
T(x - y) = 0 \Rightarrow x - y = 0
\]
Clearly we are interested in vectors $z$ for which $Tz = 0$.

\section*{Definition}

Suppose $T \in \mathcal{L} (V, W)$.
The \t{null space} (or \t{kernel}) of $T$ is the set of vectors in $V$ which are mapped to $0$ under $T$.
In symbols, the null space of $T$ is the set
\[
\null T = \Set{v \in V}{Tv = 0}
\]
The word ``null'' means ``zero'' in German.

\subsection*{A subspace}

Why use the term \textit{space}?
Well, $\null T$ is a \textit{subspace} of $V$.

\begin{proposition}
Suppose $T \in \mathcal{L} (V, W)$.
Then $\null(T)$ is a subspace of $V$.
\end{proposition}

\begin{proof}We verify that $\null(T)$ contains $0$ and is closed under vector addition and scalar multiplication.
First, $0 \in \null T$ since $T0 = 0$ by homogeneity.
Second, by additivity, if $x, y \in \null T$, then
\[
T(x + y) = Tx + Ty = 0 + 0 = 0
\]
Third, if $u \in \null T$ and $\alpha  \in \F $, then
\[
T(\lambda u) = \lambda (Tu) = \lambda 0 = 0
\]\end{proof}
\subsection*{Characterization of injectivity}

\begin{proposition}
Suppose $T \in \mathcal{L} (V, W)$.
Then
\[
\null T = \set{0} \iff T \text{ is one-to-one}
\]
\end{proposition}

If $\null T = \set{0}$ we say that $T$ has \t{zero nullspace} or \t{trivial nullspace}.

\subsection*{Examples}

\textit{Zero map.}
Suppose $T$ is the zero map from $V$ to $W$.
In other words,
\[
Tv = 0 \quad \text{for all } v \in V
\]
Then $\null T = V$.
I.e., the null space is the whole space.

\textit{Simple function on $\C ^3$.}
Define $\phi  \in \mathcal{L} (\C ^3, \C )$ by
\[
\phi (z_1, z_2, z_3) = z_1 + 2z_2 + 3z_3
\]
Then $\null \phi $ is
\[
\Set{(z_1,z_2,z_3) \in \C ^3}{z_1 + 2z_2 + 3z_3 = 0}
\]
This is the \textit{solution set} of a linear equation.

\textit{Polynomial differentiation.}
Suppose $D \in \mathcal{L} (\mathcal{P} (\R ), \mathcal{P} (\R ))$ is the linear map defined by
\[
Dp = p' \quad \text{for all } p \in \mathcal{P} (\R )
\]
In other words, $Dp$ is the derivative of the polynomial $p$.
Then $\null(D)$ is the set of constant functions.

\textit{Multiplication by $x^2$.}
Define $T \in \mathcal{L} (\mathcal{P} (\R ), \mathcal{P} (\R ))$ by
\[
(Tp)(x) = x^2p(x) \quad \text{for all } x \in \R  \text{ and } p \in \mathcal{P} (\R )
\]
Then $\null(T) = \set{0}$, since no other polynomial satisfies $x^2p(x) = 0$ for all $x \in \R $.

\textit{Backward shift.}
Define $T \in \mathcal{L} (\F ^\N  , \F ^\N  )$ by
\[
T(x_1, x_2, x_3, \dots , ) = (x_2, x_3, \dots , )
\]
so that $T$ is the backward shift
Then $T(x_1, x_2, x_3, \dots ) = 0$ if and only $x_2 = x_3 = \cdots = 0$.
So
\[
\null T = \Set{(\alpha , 0, 0, \dots ,)}{\alpha  \in \F }
\]

\blankpage