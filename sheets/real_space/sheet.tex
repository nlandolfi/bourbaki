%!name:real_space
%!need:real_order
%!need:direct_products
%!need:geometry

% \ssection{Why}
%
% We often associate points in space with elements of $\R^3$.

\ssection{Why}

We are constantly thinking of the $\R^3$ as points of space.\footnote{Future editions will modify this sheet.}

\ssection{Discussion}

We commonly associate elements of $\R^3$ with points in space. (see \sheetref{geometry}{Geometry}).

\begin{principle}[Plane Sets]
  There exists a set of all planes.
\end{principle}

\begin{principle}[Real Space Correspondence]
  Let $P$ be the set of all planes of space.
  Then $\union P$ is the set of all lines and $\union \union P$ is the set of all points.
  There exists a one-to-one correspondence mapping elements of $\union \union P$ onto elements of $\R^3$.
\end{principle}
For this reason, we sometimes call elements of $\R^3$ \t{points}.
We call the point associated with $(0, 0, 0)$ the \t{origin}.
We call the element of $\R^3$ which corresponds to a point the \t{coordinates} of the point.

\ssection{Visualization}

To visualize the correspondence we draw three perpendicular lines.
We call these \t{axes}.
We then associate a point of the line with $(0, 0, 0) \in \R^2$.
We can label it so.
We then pick a unit length.
And proceed as usual.\footnote{Future editions will expand this.}

\blankpage
