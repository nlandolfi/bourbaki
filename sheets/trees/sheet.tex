%!name:trees
%!need:graphs
%!need:set_unions

TODO: make this not jsut the edge set...
\ssection{Why}

Tree branches split and do not recombine.
We formalize this property in the language of graphs.

\ssection{Definition}

A \t{tree} is a connected and acyclic undirected graph.
% Since the tree is connected, the union of its family of edges is the vertex set, and so each tree is fully specified by its edge set.
% For this reason, when speaking of trees we often reference only the edge set.
% But a tree is an undirected graph, and so is an ordered pair.
A \t{forest} is an undirected graph if it does not contain any cycles.

% Thus every tree corresponds to an undirected, connected and acyclic graph.
% We avoid defining a tree to be that, though, because we want to keep around only the most important object: the set of two-element sets.
% From that object we can get the vertex set and edge set of the graph.
% We need the concepts from graphs to talk about which sets of two-element sets are trees, but we want the notation from trees to be amenable to their nature.
% The notation we use will bear us out.


\ssubsection{Notation}

% TODO: sheet on connected graphs? move this reasoning there?
Let $T = (V, E)$ be a tree, a mnemonic for \say{tree.}
% Let $V$ be the vertex set associated with $T$.
% In other words, $$V = \union_{e \in T} e$$.
% Let $E$ be so that $(u, v) \in E$ if and only if $\set{u, v} \in T$ for every $u, v \in V$.
% By construction, the graph $(V, E)$ is undirected.
% $T$ a tree means $(V, E)$ is connected and acyclic.
%
% A major motivator for our definition of trees is so that we can write $\set{u, v} \in T$. TODO

% Since the graph $(V, E)$ is undirected, it must be the case that $(u, v) \in E$ if and only if $(v, u) \in E$ for $u, v \in V$.
% Using this and that $(V, E)$ is connected, we conclude that for each $v \in V$ there exists $e \in E$ such that $v$ is the first coordinate of $e$.
% So then, if we unite the first coordinates of elements of $E$ we obtain the vertex set.
% Thus, from the edge set we know the vertex set.
%
% In the case of trees, use the letter $T$ for the edge set (a mnemonic for \say{tree}).
% We say \say{let $T$ be a tree} to mean that \say{let $(V, T)$ be a tree where $V$ is the vertex set given by the set of edges $T$.}
% As usual, if $u T v$ then $\op{u, v} \in T$.
% So $\op{u, v} \in T$ is a shorthand for \say{$\op{u, v}$ is an edge of $T$.}
%
\ssection{Properties}

\begin{prop}

A unique path exists between any two vertices of a tree.

\begin{proof}
The path exists because a tree is connected.
The path is unique, since were it not, we could create a cycle by combining these paths.
\end{proof}

\end{prop}
