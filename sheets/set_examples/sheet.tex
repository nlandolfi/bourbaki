%!name:set_examples
%!need:sets

\ssection{Why}

We give some examples of objects and sets.

\ssection{Examples}

For familiar examples,
let us start
with some tangible
objects.
Find, or call to
mind,
a deck
of playing cards.

First, consider
the set of all
the cards.
This set contains
fifty-two elements.
Second, consider
the set of cards
whose suit is hearts.
This set contains
thirteen elements:
the ace, two, three, four, five,
six, seven, eight, nine, ten,
jack, queen, and
king of hearts.
Third, consider
the set of twos.
This set contains
four elements:
the two of clubs,
the two of spades,
the two of hearts,
and the two of diamonds.

We can imagine many
more sets of cards.
If we are holding a deck,
each of these can be
made tangible: we can
touch the elements of
the set.
But the set itself
is always abstract:
we can not touch it.
It is the idea of the
group as distinct from
any individual member.

Moreover, the
elements of a set
need not be tangible.
First, consider the set
consisting of the suits of
the playing card:
hearts, diamonds, spades, and clubs.
This set has four elements.
Each element is a suit, whatever that is.

Second, consider the set
consisting of the card types.
This set has thirteen elements:
ace, two, three, four, five,
six, seven, eight, nine, ten,
jack, queen, king.
The subtlety here is that
this set is different than
the set of hearts, namely
those thirteen cards which are
hearts.
However these sets are
similar: they both have thirteen
elements, and there is a natural
correspondence between their
elements: the ace of hearts
with the type ace, the two
of hearts with the type two,
and so on.

Of course, sets need have nothing to
do with playing cards.
For example, consider the set of
seasons: autumn, winter, spring,
and summer.
This set has four elements.
For another example,
consider the set of lower case latin letters (introduced in \sheetref{letters}{Letters}): a, b, c, \dots, x, y, z.
This set has twenty-six elements.
Finally, consider a pack of wolves, or a bunch of grapes, or a flock of pigeons.

%We develop herein a language
%for specifying things by either
%listing them explicitly or
%by listing their defining properties.
