%!name:pointwise_vs_measure_limits
%!need:real_limits
%!need:absolute_value
%!need:set_operations

\ssection{Why}

How does convergence pointwise
(or almost everywhere pointwise)
relate to convergence in measure?

\ssection{Results}

\begin{prop}
  There exists
  a measure space
  and a sequence of
  measurable
  real-valued functions
  on that space
  converging everywhere
  (and so almost everywhere)
  but not converging
  in measure.
\end{prop}

\begin{prop}
  There exists
  a measure space
  and a sequence of
  measurable
  real-valued functions
  on that space
  converging in measure
  but not converging almost
  everywhere (nor everywhere).
\end{prop}

\begin{prop}
On finite measure spaces,
all sequences of measurable
real-valued functions
converging almost
everywhere converge in
measure.

\begin{proof}

Let $(X, \SA, \mu)$
be a measure space.
Let $\seq{f}$ be a sequence
of measurable functions on $X$
such that $\seqt{f} \goesto f$
almost everywhere.
Let $\epsilon > 0$.

For each $x \in X$, if
$\abs{f_n(x) - f(x)} > \epsilon$
for infinitely many $n$,
then $f_n(x) \not\goesto f(x)$.
Let $A$ be the set of such $x$.
and let $B =
\Set*{
x \in X
}{
f_n(x) \not\goesto f(x)
}$.
$A$ is a subset
of $B$.
The measure of $B$
is zero
since $f_n \goesto f$.
Use the the monotonicity
of measure to conclude.
$\mu(A) \leq \mu(B) = 0$.
Since $\mu(A) \geq 0$,
$\mu(A) = 0$.

For natural
$k$, let $E_k$
be the
$\Set*{x \in X}
{\abs{f_k(x) - f(x)} > \epsilon}$.
Then $x \in A$ means that
for every natural $n$, there exists
a $k \geq n$ such that $x \in E_k$.
In particular, for every $n$,
$x$ is in $\union_{k = n}^{\infty}E_k$;
denote this set by $B_n$.
If $x$ is in $B_n$ for every
$n$, then
$x \in \intersection_{n = 1}^{\infty} B_n$.
So we can write
\[
  A = \bintersection_{n = 1}^{\infty}
  \bunion_{k = n}^{\infty}
  E_k
  = \bintersection_{n = 1}^{\infty}
  B_n
  .
\]
The sequence of sets $\seq{B}$
is decreasing. So since $\mu$ is finite,
\[
  \lim_{n \to \infty} \mu(B_n) = \mu(A) = 0.
\]
For every $n$, the set $B_n$ contains
$\Set*{x \in X}{\abs{f_n(x) - f(x)} >
\epsilon}$,
namely $E_n$, the first
set in the union.
So then $\mu(E_n) \leq \mu(B_n)$
by monotonicity and so
\[
  0
  \leq \lim_{n \to \infty} \mu(E_n)
  \leq \lim_{n \to \infty} \mu(B_n)
  = 0,
\]
and we conclude $\lim_{n} E_n = 0$.
Since $\epsilon$ was arbitrary,
we conclude $f_n \goesto f$ in
measure.

\end{proof}

\end{prop}

\begin{prop}
  On any measure space,
  for a sequence
  of measurable real-valued
  functions converging
  in measure to a measurable
  real-valued limit function,
  there exists a subsequence
  convergeng to the limit
  function almost everywhere.

  \begin{proof}
Let $(X, \SA, \mu)$
be a measure space.
Let $\seq{f}$ be a sequence
of measurable functions on $X$
such that $\seqt{f} \goesto f$
in measure.

There exists $n_1$
so that
    \[
      \mu(\Set*{x \in X}{\abs{f_{n_1}(x) - f(x)} > 1}) < \frac{1}{2}.
    \]
    Can find $n_2 > n_1$ so that
    \[
      \mu(\Set*{x \in X}{\abs{f_{n_2}(x) - f(x)} > \frac{1}{2}}) < \frac{1}{4}.
    \]
    We can inductively find
    a sequence
    $\set{n_k}_k$
    so that:
    \[
      \mu\parens*{\Set*{
        x \in x
      }{
        \abs{f_{n_k}(x) - f(x)} > \frac{1}{k}
      }} \leq \frac{1}{2^k}.
    \]
  \end{proof}
\end{prop}
