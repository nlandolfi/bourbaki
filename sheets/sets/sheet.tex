%!name:sets
%!need:objects

\ssection{Why}

We want to talk about none, one, or several objects considered as an abstract whole.

\ssection{Definition}

A \t{set} is an abstract object. 
We think of it as several objects considered as a whole.
A set \t{contains} the objects so considered.
These objects are the \t{members} or \t{elements} of the set.
The objects a set contains may be other sets.
This may be subtle at first glance, but becomes familiar with experience.

We call a set which contains no objects \t{empty}.
Otherwise we call a set \t{nonempty}.
%We call a set which contains only a single object a \t{singleton}.
%A singleton is not the same as the object it contains.
%Besides these two cases, we think of sets as containing two or more objects.


\ssubsection{Notation}

We tend to denote sets by upper case Latin letters: for example, $A$, $B$, and $C$.
To aid our memory, we tend to use the lower case form of the letter for an element of the set.
For example, let $A$ and $B$ be nonempty sets.
We tend to denote by $a$ an element of $A$, and similarly, by $b$ an element of $B$

We denote that an object $a$ is an element of a set $A$ by $a \in A$.
We read the notation $a \in A$ aloud as \say{a in A.}
The $\in$ is a stylized lower case Greek letter $\epsilon$.
It is read aloud \say{ehp-sih-lawn} and is a mnemonic for \say{element of}.
If $A$ is not an element of $A$, we write $a \not\in A$, read aloud as \say{a not in A.}

Suppose a set has few elements, and we can list them.
If we give the objects names, then let us denote the set by listing the names of its elements between braces.
For example, let $a$, $b$, and $c$ be distinct objects.
Denote by $\set{a, b, c}$ the set containing these objects and only these objects.
We can further compress notation, and denote this set of objects by $A$: so, $A = \set{a, b, c}$.
Then $a \in A$, $b \in A$, and $c \in A$.
Moreover, if $d$ is an object and $d \in A$, then $d = a$ or $d = b$ or $d = c$.

Let $a$ be an object.
Note that $a \neq \set{a}$.
The left hand side, $a$, is the object $a$.
The right hand side, $\set{a}$, is the set whose element is the object $a$.
We distinguish the set containing one element from the element itself.

%If the elements of a set are
%so well-known that we can
%avoid ambiguity, then we can
%describe the set in English.
%To aid our memory,
%let us tend to name such sets
%mnemonically.
%For example,
%let $L$ be the set of Latin letters.

%We develop herein a language
%for specifying things by either
%listing them explicitly or
%by listing their defining properties.
