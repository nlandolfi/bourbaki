%!name:sets
%!need:names
%!see:https://people.maths.ox.ac.uk/knight/lectures/b1st.html
%!refs:bert_mendelson/introduction_to_topology/theory_of_sets/sets_and_subsets
%!refs:bert_mendelson/introduction_to_topology/theory_of_sets/set_operations

\ssection{Why}

We want to talk about none, one, or several objects considered together, as an aggregate.

\ssection{Definition}

When we think of several objects considered as an intangible whole, or group, we call the intangible object which is the group a \t{set}.
We say that these objects \t{belong} to the set.
They are the set's \t{members} or \t{elements}.
They are \t{in} the set.

% \begin{principle}[Existence of Sets]
% 	Intangible groups exist.
% \end{principle}

A set may have other sets as its members.
This is subtle but becomes familiar.
We call a set which contains no objects \t{empty}.
Otherwise we call a set \t{nonempty}.

\ssection{Denoting a set}

Let $A$ denote a set.
Then $A$ is a name for an object.
That object is a set.
So $A$ is a name for an object which is a grouping of other objects.

\s{Belonging}

Let $a$ denote an object and $A$ denote a set.
So we are using the names $a$ and $A$ as placeholders for some object and some set, we do not particularly know which.
Suppose though, that whatever this object and set are, it is the case that the object belongs to the set.
In other words, the object is a member or an element of the set.
We say \say{The object denoted by $a$ belongs to the set denoted by $A$}.

\ssubsection{Not symmetric}

Notice that belonging is not symmetric.
Saying \say{the object denoted by $a$ belongs to the set denoted by $A$} does not mean the same as \say{the set denoted by $A$ belongs to the object denoted by $a$}
In fact, the latter sentence is nonsensical unless the object denoted by $a$ is also a set.

\ssubsection{Not transitive}

Let $a$ denote an object and let $A$ and $B$ both denote sets.
If the object denoted by $a$ is \say{a part of} the set denoted by $A$, and the set denoted by $A$ is \say{a part of} the set denoted by $B$, then usual English usage would suggest that $a$ is \say{a part of} the set denoted by $B$.
In other words, if a thing is a part of a second thing, and the second thing is part of a third thing, then the first thing is often said to be a part of the third thing.

The relation of belonging does not follow this familiar usage.
In contrast, if an object is an element of a set, that set may be an element of another set, but this does not mean that the the first object is also an element of that other set.
The upshot is that sets are nested: we can have intangible groups of intangible groups, and have them be different than the intangible group of all the members of each group.

% \blankpage

%Besides these two cases, we think of sets as containing two or more objects.

% \s{Equality of Sets}
%
% If two sets are e
%
% We denote the ob

%
%Suppose a set has few elements, and we can list them.
%If we give the objects names, then let us denote the set by listing the names of its elements between braces.
%For example, let $a$, $b$, and $c$ be distinct objects.
%Denote by $\set{a, b, c}$ the set containing these objects and only these objects.
%We can further compress notation, and denote this set of objects by $A$: so, $A = \set{a, b, c}$.
%Then $a \in A$, $b \in A$, and $c \in A$.
%Moreover, if $d$ is an object and $d \in A$, then $d = a$ or $d = b$ or $d = c$.
%
%Let $a$ be an object.
%Note that $a \neq \set{a}$.
%The left hand side, $a$, is the object $a$.
%The right hand side, $\set{a}$, is the set whose element is the object $a$.
%We distinguish the set containing one element from the element itself.

%If the elements of a set are
%so well-known that we can
%avoid ambiguity, then we can
%describe the set in English.
%To aid our memory,
%let us tend to name such sets
%mnemonically.
%For example,
%let $L$ be the set of Latin letters.

%We develop herein a language
%for specifying things by either
%listing them explicitly or
%by listing their defining properties.
