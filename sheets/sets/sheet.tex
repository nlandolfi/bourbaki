%!name:sets
%!need:objects

\ssection{Why}

We want to talk about none, one, or several objects considered as an aggregate.

\ssection{Definition}

A \t{set} is an intangible object.
We think of it as several objects considered as a whole.
We say that these objects \t{belong} to the set.
They are the set's \t{members} or \t{elements}.

The objects a set contains may be other sets.
In other words, an element of a set may be another set.
This may be subtle at first glance, but becomes familiar with experience.

We call a set which contains no objects \t{empty}.
Otherwise we call a set \t{nonempty}.

\blankpage
%We call a set which contains only a single object a \t{singleton}.
%A singleton is not the same as the object it contains.
%Besides these two cases, we think of sets as containing two or more objects.

% \s{Equality of Sets}
%
% If two sets are e
%
% We denote the ob

%
%Suppose a set has few elements, and we can list them.
%If we give the objects names, then let us denote the set by listing the names of its elements between braces.
%For example, let $a$, $b$, and $c$ be distinct objects.
%Denote by $\set{a, b, c}$ the set containing these objects and only these objects.
%We can further compress notation, and denote this set of objects by $A$: so, $A = \set{a, b, c}$.
%Then $a \in A$, $b \in A$, and $c \in A$.
%Moreover, if $d$ is an object and $d \in A$, then $d = a$ or $d = b$ or $d = c$.
%
%Let $a$ be an object.
%Note that $a \neq \set{a}$.
%The left hand side, $a$, is the object $a$.
%The right hand side, $\set{a}$, is the set whose element is the object $a$.
%We distinguish the set containing one element from the element itself.

%If the elements of a set are
%so well-known that we can
%avoid ambiguity, then we can
%describe the set in English.
%To aid our memory,
%let us tend to name such sets
%mnemonically.
%For example,
%let $L$ be the set of Latin letters.

%We develop herein a language
%for specifying things by either
%listing them explicitly or
%by listing their defining properties.
