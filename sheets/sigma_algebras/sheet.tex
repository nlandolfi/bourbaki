
\section*{Why}

For general measure theory, we need an algebra of sets closed under countable unions; we define such an object.\footnote{Future editions will make no reference to measure theory. The entire development will follow the genetic approach, and so roughly follow the historical development for handling integration.}

\section*{Definition}

A \t{countably summable subset algebra} is a subset algebra for which (1) the base set is distinguished (2) the complement of a distinguished set is distinguished (3) the union of a sequence of distinguished sets is distinguished.

The name is justified, as each countably summable subset algebra is a subset algebra, because the union of $A_1, \dots , A_n$ coincides with the union of $A_1, \cdots, A_n, A_n, A_n \cdots$.

We call the set of distinguished sets a \t{sigma algebra} (or \t{sigma field}) \t{on} the base set.
This language is justified (as for a regular subset algebra) by the closure properties of the sigma algebra under the usual set operations.
We sometimes write are $\sigma $-algebra and $\sigma $-field.

A \t{sub-$\sigma $-algebra} (\t{sub-sigma-algebra}) is a subset of a sigma algebra which is itself a sigma algebra.

\subsection*{Notation}


% The notation follows that of a subset space. 

We often denote a sigma algebra by $\mathcal{A} $ or $\mathcal{F} $; the former is a mnemonic for ``algebra'' and the second is a mnemonic for ``field''.
The calligraphic typeface is meant as a reminder that the object so denoted is a set \textit{of sets}.

A common pattern is also to use the calligraphic font of whichever letter is being used for the base set.
Thus, if $A$ is a set, then we might choose to denote a sigma algebra on $A$ by $\mathcal{A} $.

Often, instead of saying ``let $(A, \mathcal{A} )$ be a countably summable subset algebra'' we say instead ``let $\mathcal{A} $ be a sigma algebra on $A$.''
Since the largest element of the sigma algebra is the base set, we can also say (without ambiguity): ``let $\mathcal{A} $ be a sigma algebra.''
The base set is $\cup \mathcal{A} $.

\section*{Examples}

\begin{example}
For any set $A$, $2^{A}$ is a sigma algebra.
\end{example}

\begin{example}
For any set $A$,
$\set{A, \varnothing}$ is a sigma algebra.
\end{example}

\begin{example}
Let $A$ be an infinite set.
Let $\mathcal{A} $ the collection of finite subsets of $A$.
$\mathcal{A} $ is not a sigma algebra.
\end{example}

\begin{example}
Let $A$ be an infinite set.
Let $\mathcal{A} $ be the collection
subsets of $A$ such that the set or its
complement is finite.
$\mathcal{A} $ is not a sigma algebra.
\end{example}

\begin{proposition}

\label{sigma_algebra:sigmaintersection}The intersection of a family of sigma algebras is a sigma algebra.
\end{proposition}

\begin{example}
For any infinite set $A$, let $\mathcal{A} $ be the set
\[
\Set*{
B \subset A
}{
\card{B} \leq \aleph_0 \lor
\card{C_{A}(B)} \leq \aleph_0
}.
\]
$\mathcal{A} $ is an algebra; the \t{countable/co-countable algebra}.\footnote{Future editions will clean up and modify these examples.}
\end{example}
