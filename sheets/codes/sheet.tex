%!name:codes
%!need:sequences

\ssection{Why}\footnote{Future editions will include.}

\ssection{Definition}

Let $X$ be a set and let $A$ be a finite set.
We denote the set of all finite sequences (strings) in $A$ by $\CS(A)$.
We read $\CS(A)$ aloud as \say{the strings in $A$.}

A \t{code} for $X$ \t{in} $A$ is a function from $X$ to $\CS(A)$.
In this context, we refer to the finite set $A$ as an \t{alphabet}.
The \t{length} of an object (w.r.t to a code $c: X \to \CS(A)$) is the length of the sequence $c(x)$.
We call a code \t{nonsingular} if it is injective.

\ssection{Examples}

\ssection{Extensions}

We can extend a code $c: X \to \CS(A)$ to a code for $\CS(X)$ in a naural way.
The \t{extension} of $c$ is the function $C: \CS(X) \to \CS(A)$ defined, for $\xi = (\xi_1, \dots, \xi_n) \in \CS(X)$, by
\[
  \CS(\xi) = (c(\xi_1), \dots, c(\xi_n)).
\]
We call an code \t{uniquely decodable} if its extension is injective.
In other words, given the code $C(\xi)$ for a sequence $\xi \in \CS(X)$, we can recover $\xi$.

\ssection{Prefix-free codes}

A code $C: X \to \CA$ is \t{prefix-free} if, for all $x \in X$, $C(x)$ is not a prefix\footnote{To be defined.} of $C(x')$ for all $x' \neq x \in X$.
Prefix-free codes are nice because they are uniquely decodable.
The converse, is not true.

\begin{proposition}
  There exists a set $X$, alphabet $A$, and \textit{not} prefix-free code $C: X \to \CA$ such that $C$ is uniquely decodable.
  \begin{proof}
  Try $X = \set{A, B}$, $D = \set{0, 1}$ and $C(A) = (0)$ , $C(B) = 01$.
  Proof by induction on the length of the sequence, base case length 1 and length 2 sequences.\footnote{Future editions will expand on this account.}
  \end{proof}
\end{proposition}

Example...


\blankpage
