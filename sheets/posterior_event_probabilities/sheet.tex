
\section*{Result}

\begin{proposition}
Suppose $\mathbfsf{P} $ is a \sheetref{event_probabilities}{finite probability measure}on a set of \sheetref{uncertain_outcomes}{outcomes}$\Omega $.
For any two \sheetref{uncertain_outcomes}{events}$A, B$ with $\mathbfsf{P} (A), \mathbfsf{P} (B) > 0$, we have
\[
\mathbfsf{P} (A \mid  B) = \frac{ \mathbfsf{P} (B \mid A)\mathbfsf{P} (A) }{ \mathbfsf{P} (B) }.
\]
\begin{proof}By definition, we have
\[
\mathbfsf{P} (A \mid  B) = \frac{ \mathbfsf{P} (A \cap  B) }{ \mathbfsf{P} (B) }.
\]
And also symmetrically,
\[
P(B \mid  A) = \frac{ \mathbfsf{P} (A \cap  B) }{ \mathbfsf{P} (A) }.
\]
From this second equation we have $\mathbfsf{P} (A \cap  B) = \mathbfsf{P} (B \mid  A) \mathbfsf{P} (A)$, which we can substitute into the numerator of the first expression to obtain the result.\end{proof}
\end{proposition}

This result is known by many names including \t{Bayes' rule}, \t{Bayes rule} (no possessive), \t{Bayes' formula}, and \t{Bayes' theorem}.

It is a \textit{basic} consequence of the \textit{definition} of conditional probability, but it is \textit{useful} in the case that we are given problem data in terms of the probabilities on the right hand side of the above equation.

\subsection*{Compound form}

More is true.

\begin{proposition}
Suppose $\mathbfsf{P} $ is a \sheetref{event_probabilities}{finite probability measure}on a set of \sheetref{uncertain_outcomes}{outcomes}$\Omega $.
For any three \sheetref{uncertain_outcomes}{events}$A, B, C$ with $\mathbfsf{P} (A), \mathbfsf{P} (B), \mathbfsf{P} (C) > 0$, we have
\[
\mathbfsf{P} (A \mid  B \cap  C) = \frac{ \mathbfsf{P} (B \mid A \cap  C)\mathbfsf{P} (A \mid  C) }{ \mathbfsf{P} (B \mid  C) }.
\]
\begin{proof}Future editions will include, the strategy is the same as above.\end{proof}
\end{proposition}

\blankpage