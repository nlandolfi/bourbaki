
\section*{Why}

We can embed the undirected graphs as certain garphs directed graphs.

\section*{Definition}

Suppose that $G = (V, E)$ is a directed graph.
If $(v, w) \in E$ \textit{and} $(w, v) \in E$ we call $(v, w)$ (and $(w, v)$) and \t{bidirected edge} (or \t{undirected edge}).
In other words, an edge $(v, w) \in E$ is undirected if $(v, w)$ is also in $E$.

If every edge of a directed graph is undirected, then we call the graph \t{undirected}.
If some edges are an some are not, we call $G$ a \t{partially directed graph}.

\subsection*{Notation}

Suppose $(V, E)$ is a directed graph.
It is common to write $v \to w$ for
\[
(v, w) \in E \text{ and } (w, v) \not\in E.
\]
It is common to write $a \sim \beta $ if
\[
(v, w) \in E \text{ and } (w, v) \in E
\]
Similarly, we write $v \not\to w$ if $(v, w) \not\in E$ and $v \not\sim w$ if
\[
(v, w) \not\in E \text{ and } (w, v) \not\in E
\]

\subsection*{Undirected version}

As before, the \t{undirected version} (or \t{skeleton}) of $G$ is the \textit{undirected} partially directed graph defined satisfying $u \sim v$ if $u \to v$ or $u \sim v$.

\subsection*{Subgraphs}

Given a graph $G = (V, E)$ and a set $A \subset V$, the \t{subgraph} of $G = (V, E)$ corresponding to $A$ is the graph denoted $G_{A}$ defined by $(A, E \cap  (A \times  A)$.

\subsection*{Completeness}

Suppose $(V, E)$ is a partially directed graph.
In the context of partially directed graphs, the graph is \t{complete} if $(u,v) \in E$ or $(v, u) \in E$ for any two vertices $u$ and $v$.
A subset $A$ is \t{complete} if the subgraph $G_A$ is complete.
A complete subset that is maximal (w.r.t. $\subseteq$) is called a \t{clique}.

\blankpage