%!name:linear_equations
%!need:natural_equations
%!need:linear_functions
%!need:arrays

\section*{Why}

Linear equations are ubiquitous.

\section*{Definition}

Given $a \in \R ^n$ and $y \in \R $, suppose we want to find $x \in \R ^n$ satisfying
  \[
a_1x_1 + a_2x_2 + \cdots + a_nx_n = y.
  \]
We refer to this expression as a \t{real linear equation} or \t{linear equation}.
We treat each component $x_i \in \R $ as a variable and we treat each component $a_i \in \R $ and $y \in \R $ as constants.
We call the pair $(a, y)$ the \t{real linear equation constants}.
  \ifhmode\unskip\fi\footnote{
Future editions will clarify.
  }

The source of the terminology \say{linear} is by viewing the left hand side as a function.
Define $f: \R ^n \to \R $ by $f(x) = \sum_{i}a_ix_i$.
We want to find $x \in \R ^n$ to satisfy $f(x) = b$.
Notice that $f$ is a \textit{linear} real function.
  \ifhmode\unskip\fi\footnote{
Future editions may require a sheet here.
  }

Moreover, to each linear function $f: \R ^d \to \R $ there exists a vector $a \in \R ^d$ so that $f(x) = \sum_{i} a_ix_i$.
For this reason, if we are given several linear function $f_1, \dots , f_m$, then we can think of these as several vectors $a^1, \dots , a^n$.
If we are also given $b_i \in \R $ for each $i = 1, \dots , m$, then we have the vector $b \in \R ^m$

We can define the two-dimensional array $A \in \R ^{m \times n}$ by $A_{ij} = a^{i}_j$.
For this reason, a \t{linear system of equations} is a pair $(A, b)$.
A solution of a linear system of equations is a vector $x \in \R ^n$ satisfying the equations
  \[
\begin{aligned}
A_{11}x_1 & + & A_{12}x_2 & + & \cdots \, & + & A_{1n}x_n & = \, & b_1 \\
A_{21}x_1 & + & A_{22}x_2 & + & \cdots & + & A_{2n}x_n & = & b_2 \\
\vdots & & \vdots & & & & \vdots & & \vdots \\
A_{m1}x_1 & + & A_{m2}x_2 & + & \cdots & + & A_{mn}x_n & = & b_n \\
\end{aligned}
  \]
Other terminology includes a \t{system of linear equations} or \t{linear system} or \t{simultaneous linear equations}
