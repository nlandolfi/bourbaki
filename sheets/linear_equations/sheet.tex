%!name:linear_equations
%!need:linear_combinations
%!need:natural_equations
%!need:real_summation
%!need:arrays

\ssection{Why}\footnote{Future editions will include.}

\ssection{Definition}

Let $a \in \R^n$ and $b \in \R$.
Suppose we want to find $x \in \R^n$ to satisfy
\[
  a_1x_1 + a_2x_2 + \cdots + a_nx_n = b.
\]
We refer to this expression as a \t{real linear equation} or \t{linear equation}.
We treat $x_i$s as a variables and we treat the $a_i$s and $b$ as constants.
We call the pair $(a, b)$ the \t{real linear equation constants}.\footnote{Future editions will clarify.}

The source of the terminology \say{linear} is by viewing the left hand side as a function.
Define $f: \R^n \to \R$ by $f(x) = \sum_{i}a_ix_i$.
We want to find $x \in \R^n$ to satisfy $f(x) = b$.
Notice that $f$ is a \textit{linear} real function.\footnote{Future editions may require a sheet here.}

Moreover, to each linear function $f: \R^d \to \R$ there exists a vector $a \in \R^d$ so that $f(x) = \sum_{i} a_ix_i$.
For this reason, if we are given several linear function $f_1, \dots, f_m$, then we can think of these as several vectors $a^1, \dots, a^n$.
If we are also given $b_i \in \R$ for each $i = 1, \dots, m$, then we have the vector $b \in \R^m$

We can define the two-dimensional array $A \in \R^{m \times n}$ by $A_{ij} = a^{i}_j$.
For this reason, a \t{linear system of equations} is a pair $(A, b)$.
A solution of a linear system of equations is a vector $x \in \R^n$ satisfying the equations
\[
  \begin{aligned}
    A_{11}x_1 & + & A_{12}x_2 & + & \cdots \, & + & A_{1n}x_n & = \, & b_1 \\
    A_{21}x_1 & + & A_{22}x_2 & + & \cdots & + & A_{2n}x_n & = & b_2 \\
    \vdots    &   & \vdots    &   &        &   & \vdots    &   & \vdots \\
    A_{m1}x_1 & + & A_{m2}x_2 & + & \cdots & + & A_{mn}x_n & = & b_n \\
  \end{aligned}
\]
Other terminology includes a \t{system of linear equations} or \t{linear system} or \t{simultaneous linear equations}

\blankpage
