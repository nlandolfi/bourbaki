%!name:vectors_as_matrices
%!need:real_vectors
%!need:real_norm
%!need:real_inner_product
%!need:real_matrices

\ssection{Why}
Vectors can be identified with matrices of width 1.

\ssection{Canonical Identification}
We identify $\R^{n}$ with $\R^{n \times 1}$ in the obvious way.
For this reason, we call $x \in \R^{n \times 1}$ (meaning $x \in \R^{n}$) a \t{column vector}.

For the reasons that we identify $\R^n$ with $\R^{n \times 1}$, we write the vector $a = (a_1, a_2, a_3) \in \R^3$ as
  \[
\bmat{a_1 \\ a_2 \\ a_3}
\text{ or }
\pmat{ a_1 \\ a_2 \\ a_3}.
  \]

We could as easily also identify $\R^{n}$ with $\R^{1 \times n}$.
We avoid this convention.
However, by analogy with the language \say{column vector,} we refer to the \textit{matrix} $y \in \R^{1 \times n}$ as a \t{row vector}.

\ssection{Matrix transpose}
We frequently move from $\R^{n \times 1}$ and $\R^{1 \times n}$.
If $a \in \R^{n \times 1}$, we denote $b \in \R^{1 \times n}$ defined by $b_i = a_i$ by $a^\tp$.

More generally, given a matrix $A \in \R^{m \times n}$, we denote the matrix $B \in \R^{m \times n}$ defined by $B_{ij} = A_{ji}$ by $A^\tp$.
Notice that the entries of $i$ and $j$ have swapped.
We call the matrix $B$ the \t{transpose} of $A$, and similarly call $a^\tp$ the \t{transpose} of the vector $a$.
Clearly, $(A^\tp)^\tp = A$, which includes $(a^\tp)^\tp = a$.

\ssubsection{Reals as vectors}
There is a similar, and similarly obvious, identification of scalars $a \in \R$ with the 1-vectors $\R^{1}$ (and so with the 1 by 1 matrices $\R^{1 \times 1}$).
Given our definition of matrix-vector products, if we identify $a \in \R$ with $A \in \R^{1 \times 1}$ where $A_{11} = a$, then $Ax = ax$.

\ssection{Familiar concepts, new notation}
These identifications and the notation of transposition give allow us to write several familiar concepts in a compact notation.
We write the norm $\norm{x}$ as
  \[
\norm{x} = \sqrt{x_1^2 + x_2^2 + \cdots + x_n^2} = \sqrt{x^\tp x}.
  \]
We write the inner product as
  \[
\ip{x,y} =
x_1y_1 + x_2y_2 + \cdots + x_ny_n
= x^\tp y.
  \]
We express the symmetry of the inner product by $ x^\tp y = y^\tp x$.

\blankpage
