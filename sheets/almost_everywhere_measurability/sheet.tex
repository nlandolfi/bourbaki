%!name:almost_everywhere_measurability
%!need:negligible_sets

\ssection{Why}

Does convergence almost everywhere of a sequence of measurable functions guarantee measurability of the limit function?
It does on complete measure spaces, and we can use this result to \say{weaken} the hypotheses of many theorems.

\ssection{Results}

A measure is \t{complete} if every subset of a measurable set of measure zero is measurable.
If the measure is complete, then every negligible set must be measurable.

We begin with a transitivity property: almost everywhere equality of two functions allows us to infer measurability of one from the other.

\begin{prop}
Let $(X, \mathcal{A}, \mu)$ be a measure space
Let $f,g: X \to [-\infty, \infty]$
with $f = g$ almost everywhere.
If $\mu$ is complete
and $f$ is $\mathcal{A}$-measurable,
then $g$ is $\mathcal{A}$-measurable.
\begin{proof}
\end{proof}
\end{prop}

\begin{prop}
Let $(X, \mathcal{A}, \mu)$ be a measure space.
Let $\seqt{f}: X \to [-\infty, \infty]$ for all
natural numbers $n$
and $f: X \to [-\infty, \infty]$
with
$\seq{f}$ converging to $f$ almost everywhere.
If $\mu$ is complete and
and $\seqt{f}$ is measurable for each $n$,
then $f$ is $\mathcal{A}$-measurable.
  \begin{proof}\footnote{Future editions will include.}
\end{proof}
\end{prop}

\blankpage
