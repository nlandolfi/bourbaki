
\section*{Why}

Does convergence almost everywhere of a sequence of measurable functions guarantee measurability of the limit function?
It does on complete measure spaces, and we can use this result to ``weaken'' the hypotheses of many theorems.

\section*{Results}

A measure is \t{complete} if every subset of a measurable set of measure zero is measurable.
If the measure is complete, then every negligible set must be measurable.

We begin with a transitivity property: almost everywhere equality of two functions allows us to infer measurability of one from the other.

\begin{proposition}
Suppose $(X, \mathcal{A} , \mu )$ is measure space and $f,g: X \to [-\infty, \infty]$ are such that $f = g$ almost everywhere.
If $\mu $ is complete and $f$ is $\mathcal{A} $-measurable, then $g$ is $\mathcal{A} $-measurable.
\end{proposition}

\begin{proposition}
Let $(X, \mathcal{A} , \mu )$ be a measure space.
Let $\seqt{f}: X \to [-\infty, \infty]$ for all
natural numbers $n$
and $f: X \to [-\infty, \infty]$
with
$\seq{f}$ converging to $f$ almost everywhere.
If $\mu $ is complete and
and $\seqt{f}$ is measurable for each $n$,
then $f$ is $\mathcal{A} $-measurable.
\end{proposition}

\blankpage