%!name:affine_sets_and_subspaces
%!need:affine_sets
%!need:subspaces

\ssection{Why}

How do affine sets relate to subspaces?

\ssection{Main Results}

First, we will see that the subspaces of $\R^n$ are the affine sets which contain the origin.

\begin{prop}

A set is a subspace if and only if it is an affine set which contains the zero vector.

\begin{proof}

($\rightarrow$) Every subspace of contains the zero vector. Moreover, it is closed under addition and scalar multiplication, so it must be an affine set, since the points of which the line consists between two points is expressed as a sum and scalar multiplication.

($\leftarrow$) Let $M$ be an affine set containing the zero vector. First, notice that $M$ is closed under scalar multiplication: if $x \in M$ and $a \in \R$, the vector
$$
  ax = (1 - a)0 + ax
$$
is in the line through $0$ and $x$.
Second, notice that $M$ is closed under vector addition: if $x, y \in M$, the vector
$$
  \frac{1}{2}(x + y) = \frac{1}{2}x + \parens*{1 - \frac{1}{2}}y
$$
is in the line $M$.
  Since $\frac{1}{2}(x + y)$ is in $M$, and $M$ is closed under scalar multiplication,
$$
  x + y = 2\parens*{\frac{1}{2}\parens*{x + y}}
$$
is also in $M$.

\end{proof}

\end{prop}

Next we will see how all other affine sets can be obtained by those which contain the origin.

\begin{prop}

Each nonempty affine set $M$ is parallel to a unique subspace $L$The subspaces of $\R^n$ are the affine sets which contain the origin.

\end{prop}
