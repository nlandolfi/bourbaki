
%!name:entire_functions
%!need:complex_analytic_functions
%!refs:yellow/IX/4

\section*{Definition}

An \t{entire function} is a complex function $f: \C  \to \C $ which is analytic for all $z \in \C $.

\blankpage
\sbasic
%%%% MACROS %%%%%%%%%%%%%%%%%%%%%%%%%%%%%%%%%%%%%%%%%%%%%%%

\newcommand{\PM}{\mathbf{P}}

%%%%%%%%%%%%%%%%%%%%%%%%%%%%%%%%%%%%%%%%%%%%%%%%%%%%%%%%%%%

%%%% MACROS %%%%%%%%%%%%%%%%%%%%%%%%%%%%%%%%%%%%%%%%%%%%%%%

\newcommand{\PM}{\mathbf{P}}

%%%%%%%%%%%%%%%%%%%%%%%%%%%%%%%%%%%%%%%%%%%%%%%%%%%%%%%%%%%

%%%% MACROS %%%%%%%%%%%%%%%%%%%%%%%%%%%%%%%%%%%%%%%%%%%%%%%

\newcommand{\PM}{\mathbf{P}}

%%%%%%%%%%%%%%%%%%%%%%%%%%%%%%%%%%%%%%%%%%%%%%%%%%%%%%%%%%%

%%%% MACROS %%%%%%%%%%%%%%%%%%%%%%%%%%%%%%%%%%%%%%%%%%%%%%%

% use \set{stuff} for { stuff }
% use \set* for autosizing delimiters
\DeclarePairedDelimiter{\set}{\{}{\}}

% use \Set{a}{b} for {a | b}
% use \Set* for autosizing delimiters
\DeclarePairedDelimiterX{\Set}[2]{\{}{\}}{#1 \nonscript\;\delimsize\vert\nonscript\; #2}

% use \powerset{A} for power set of A
\newcommand{\powerset}[1]{2^{#1}}

\renewcommand{\emptyset}{\varnothing}

\newcommand{\SA}{\mathcal{A}}
\newcommand{\SB}{\mathcal{B}}
\newcommand{\SC}{\mathcal{C}}
\newcommand{\SD}{\mathcal{D}}
\newcommand{\SE}{\mathcal{E}}
\newcommand{\SF}{\mathcal{F}}
\newcommand{\SG}{\mathcal{G}}
\newcommand{\SH}{\mathcal{H}}
\newcommand{\SI}{\mathcal{I}}
\newcommand{\SJ}{\mathcal{J}}
\newcommand{\SK}{\mathcal{K}}
\newcommand{\SL}{\mathcal{L}}

%%%%%%%%%%%%%%%%%%%%%%%%%%%%%%%%%%%%%%%%%%%%%%%%%%%%%%%%%%%

%%%% MACROS %%%%%%%%%%%%%%%%%%%%%%%%%%%%%%%%%%%%%%%%%%%%%%%

\newcommand{\PM}{\mathbf{P}}

%%%%%%%%%%%%%%%%%%%%%%%%%%%%%%%%%%%%%%%%%%%%%%%%%%%%%%%%%%%

%%%% MACROS %%%%%%%%%%%%%%%%%%%%%%%%%%%%%%%%%%%%%%%%%%%%%%%

\newcommand{\PM}{\mathbf{P}}

%%%%%%%%%%%%%%%%%%%%%%%%%%%%%%%%%%%%%%%%%%%%%%%%%%%%%%%%%%%

%%%% MACROS %%%%%%%%%%%%%%%%%%%%%%%%%%%%%%%%%%%%%%%%%%%%%%%

\newcommand{\PM}{\mathbf{P}}

%%%%%%%%%%%%%%%%%%%%%%%%%%%%%%%%%%%%%%%%%%%%%%%%%%%%%%%%%%%

%%%% MACROS %%%%%%%%%%%%%%%%%%%%%%%%%%%%%%%%%%%%%%%%%%%%%%%

\newcommand{\PM}{\mathbf{P}}

%%%%%%%%%%%%%%%%%%%%%%%%%%%%%%%%%%%%%%%%%%%%%%%%%%%%%%%%%%%

%%%% MACROS %%%%%%%%%%%%%%%%%%%%%%%%%%%%%%%%%%%%%%%%%%%%%%%

\newcommand{\PM}{\mathbf{P}}

%%%%%%%%%%%%%%%%%%%%%%%%%%%%%%%%%%%%%%%%%%%%%%%%%%%%%%%%%%%

%%%% MACROS %%%%%%%%%%%%%%%%%%%%%%%%%%%%%%%%%%%%%%%%%%%%%%%

\newcommand{\PM}{\mathbf{P}}

%%%%%%%%%%%%%%%%%%%%%%%%%%%%%%%%%%%%%%%%%%%%%%%%%%%%%%%%%%%

%%%% MACROS %%%%%%%%%%%%%%%%%%%%%%%%%%%%%%%%%%%%%%%%%%%%%%%

\newcommand{\PM}{\mathbf{P}}

%%%%%%%%%%%%%%%%%%%%%%%%%%%%%%%%%%%%%%%%%%%%%%%%%%%%%%%%%%%

%%%% MACROS %%%%%%%%%%%%%%%%%%%%%%%%%%%%%%%%%%%%%%%%%%%%%%%

\newcommand{\PM}{\mathbf{P}}

%%%%%%%%%%%%%%%%%%%%%%%%%%%%%%%%%%%%%%%%%%%%%%%%%%%%%%%%%%%

%%%% MACROS %%%%%%%%%%%%%%%%%%%%%%%%%%%%%%%%%%%%%%%%%%%%%%%

\newcommand{\PM}{\mathbf{P}}

%%%%%%%%%%%%%%%%%%%%%%%%%%%%%%%%%%%%%%%%%%%%%%%%%%%%%%%%%%%

\newcommand{\union}{\,\cup\,}
\newcommand{\bunion}{\bigcup}
\newcommand{\intersect}{\cap}
\newcommand{\intersection}{\cap}
\newcommand{\bintersection}{\bigcap}
\newcommand{\symdiff}{\Delta}

%%%% MACROS %%%%%%%%%%%%%%%%%%%%%%%%%%%%%%%%%%%%%%%%%%%%%%%

\newcommand{\PM}{\mathbf{P}}

%%%%%%%%%%%%%%%%%%%%%%%%%%%%%%%%%%%%%%%%%%%%%%%%%%%%%%%%%%%

%%%% MACROS %%%%%%%%%%%%%%%%%%%%%%%%%%%%%%%%%%%%%%%%%%%%%%%

\newcommand{\PM}{\mathbf{P}}

%%%%%%%%%%%%%%%%%%%%%%%%%%%%%%%%%%%%%%%%%%%%%%%%%%%%%%%%%%%

%%%% MACROS %%%%%%%%%%%%%%%%%%%%%%%%%%%%%%%%%%%%%%%%%%%%%%%

\newcommand{\PM}{\mathbf{P}}

%%%%%%%%%%%%%%%%%%%%%%%%%%%%%%%%%%%%%%%%%%%%%%%%%%%%%%%%%%%

%%%% MACROS %%%%%%%%%%%%%%%%%%%%%%%%%%%%%%%%%%%%%%%%%%%%%%%

\newcommand{\PM}{\mathbf{P}}

%%%%%%%%%%%%%%%%%%%%%%%%%%%%%%%%%%%%%%%%%%%%%%%%%%%%%%%%%%%

%%%% MACROS %%%%%%%%%%%%%%%%%%%%%%%%%%%%%%%%%%%%%%%%%%%%%%%

\newcommand{\PM}{\mathbf{P}}

%%%%%%%%%%%%%%%%%%%%%%%%%%%%%%%%%%%%%%%%%%%%%%%%%%%%%%%%%%%

\sstart
\stitle{Measures}

\ssection{Why}

We want to generalize the notion of length, area, volume beyond the Lebesgue measure on the product spaces of real numbers.

\ssection{Definition}

An extended-real-valued non-negative
function on an algebra is
\ct{finitely additive}{nnfinitelyadditive}
if the result of the function applied to
the union of a disjoint finite family of
distinguished sets is the sum of the
results of the function applied to each
of the sets individually.

An extended-real-valued non-negative
function on a sigma algebra is
\ct{countably additive}{nncountablyadditive}
if the result of the function applied to
the union of a disjoint countable family of
distinguished sets is the limit of the partial
sums of the results of the function applied
to each of the sets individually.

A
\ct{finitely additive measure}{finitelyadditivemeasure}
is an extended-real-valued non-negative
finitely additive function which associates the empty
set with the real number $0$.
A
\ct{countably additive measure}{finitelyadditivemeasure}
is an extended-real-valued non-negative
countably additive function which associates the empty
set with the real number $0$.
We call countably additive measures
\ct{measures}{measures}, for short.

Every countably additive measure is finitely additive.
On the other hand, there exist finitely additive measures
which are not countable additive.

In the context of measure,
we call a countably unitable subset algebra
a \t{measurable space}.
We call the distinguished sets
\ct{measurable}{measurable}
sets.
A
\ct{measure space}{measurespace}
is triple.
As a pair, the first two
objects are a measurable space.
The third object is a measure defined
on the sigma algebra of teh measurable space.

\ssection{Notation}

Let $A$ a set.
Let $\mathcal{A}$ a sigma algebra on $A$.
The pair $(A, \mathcal{A})$ is a measurable space.

Let $\mu: \mathcal{A} \to [0, \infty]$ a measure;
thus:
(a) $\mu(\emptyset) = 0$ and
(b) for disjoint $\set{A_n} \subset \mathcal{A}$,
$\mu(\cup_{n = 1}^{\infty} A_n)
  = \sum_{n = 1}^{\infty} \mu(A_n)$
The triple $(A, \mathcal{A}, \mu)$ is a
measure space.

We use $\mu$ since it
is a mnemonic for \say{measure}.
We often also us
$\nu$ to denote measures,
since it is after
$\mu$ in the Greek alphabet,
and $\lambda$, since
it is before
$\mu$ in the Greek alphabet.

\ssection{Examples}

\begin{expl}
Let $(A, \mathcal{A})$ a measurable space.
Let $\mu: \mathcal{A} \to [0, +\infty]$ such that
$\mu(A)$ is $\card{A}$ if $A$ is finite
and $\mu(A)$ is $+\infty$ otherwise.
Then $\mu$ is a measure.
We call $\mu$ the
\ct{counting measure}{countingmeasure}.
\end{expl}

\begin{expl}
Let $(A, \mathcal{A})$ measurable.
Fix $a \in A$.
Let $\mu: \mathcal{A} \to [0, +\infty]$ such
that $\mu(A)$ is $1$ if $a \in A$ and
$\mu(A)$ is $0$ otherwise.
Then $\mu$ is a measure.
We call $\mu$ the
\ct{point mass}{pointmassmeasure}
concentrated at $a$.
\end{expl}

\begin{expl}
Let $R$ denote the real numbers.
The Lebesgue measure on the measurable
space $(R, \mathcal{B}(R))$ is a measure.
\end{expl}

\begin{expl}
Let $N$ be the natural numbers.
Let $\mathcal{A}$ the finite
co-finite algebra on $N$.
Let $\mu: \mathcal{A} \to [0, +\infty]$
be such that $\mu(A)$ is 1 if
$A$ is infinite or 0 otherwise.
Then $\mu$ is a finitely additive measure.
However it is impossible to extend
$\mu$ to be a countably additive measure.
Observe that if $A_n = \set{n}$ the
$\mu(\union_{n} A_n) = 1$ but
$\sum_{n} \mu(A_n) = 0$.
\end{expl}

\begin{expl}
Let $(A, \mathcal{A})$ a measurable
space.
Let $\mu: \mathcal{A} \to [0, +\infty]$
be $0$ if $A = \emptyset$ and
$\mu(A)$ is $+\infty$ otherwise.
Then $\mu$ is a measure.
\end{expl}

\begin{expl}
Let $A$ be set with at least
two elements ($\card{A} \geq 2$).
Let $\mathcal{A} = \powerset{A}$.
Let $\mu: \mathcal{A} \to [0, +\infty]$
such that $\mu(A)$ is $0$ if $A = \emptyset$
and $\mu(A) = 1$ otherwise.
Then $\mu$ is not a measure,
nor is $\mu$ finitely additive.
\begin{proof}
Let $B, C \in \mathcal{A}$,
$B \intersect C = \emptyset$
then using finite additivity
we obtain a contradiction
$
1 = \mu(B \union C) = \mu(B) + \mu(C) = 2
$.
\end{proof}
\end{expl}
\strats
