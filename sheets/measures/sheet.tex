
\section*{Why}

We extend our notion of length, area, and volume beyond the Lebesgue measure on the product spaces of real numbers.

\section*{Definition}

Suppose $\mathcal{A} $ is an algebra of sets.
A function $f: \mathcal{A}  \to \Rbar_+$ is \t{finitely additive} if
\[
f(\cup_{i = 1}^{n} A_i) = \sum_{i = 1}^{n} f(A_i) \quad \text{for all disjoint } A_1, \dots , A_n \in \mathcal{A}
\]

Similarly, suppose $\mathcal{F} $ is a $\sigma $-algebra.
Then $f$ is \t{countably additive} if
\[
f(\cup_{i = 1}^{\infty} F_i) = \sum_{i =1 }^{\infty} f(F_i) \quad \text{for all disjoint sequences } \set{F_i}_{i \in \N  } \text{ in } \mathcal{F}
\]

If, in addition, $f(\varnothing) = 0$, then $f$ is called a \t{finitely additive measure} or \t{countably additive measure} respectively.
Since a countably additive measure is finitely additive (the converse is false!), when we speak of a \t{measure} we mean a countable additive one.

When $(X, \mathcal{F} )$ is a countably unitable subset algebra and $\mu : \mathcal{F}  \to \Rbar_+$, then we call $(X, \mathcal{F} )$ a \t{measurable space} and call $(X, \mathcal{F} , \mu )$ a \t{measure space}.
The word ``space'' is natural, since the notion of a measure generalized the notion of volume in real space (see \sheetref{real_space}{Real Space}and \sheetref{n-dimensional_space}{N-Dimensional Space}).
We often call $\mathcal{F} $ the \t{measurable sets}.
In other words, a measure space is a triple: a base set, a sigma algebra, and a measure.

\subsection*{Notation}

We often use $\mu $ for a measure since it is a mnemonic for ``measure''.
We often also us $\nu $ and $\lambda $ since these letters are near $\mu $ in the Greek alphabet.

\subsection*{Examples}

\begin{example}
Let $(A, \mathcal{A} )$ a measurable space.
Let $\mu : \mathcal{A}  \to [0, +\infty]$ such that $\mu (A)$ is $\card{A}$ if $A$ is finite and $\mu (A)$ is $+\infty$ otherwise.
Then $\mu $ is a measure.
We call $\mu $ the \t{counting measure}.
\end{example}

\begin{example}
Let $(A, \mathcal{A} )$ measurable.
Fix $a \in A$.
Let $\mu : \mathcal{A}  \to [0, +\infty]$ such that $\mu (A)$ is $1$ if $a \in A$ and $\mu (A)$ is $0$ otherwise.
Then $\mu $ is a measure.
We call $\mu $ the \t{point mass} concentrated at $a$.
\end{example}

\begin{example}
The Lebesgue measure on the measurable space $(\R , \mathcal{B} (\R ))$ is a measure.
\end{example}

\begin{example}
Let $\mathcal{A} $ the co-finite algebra on $N$.
Let $\mu : \mathcal{A}  \to [0, +\infty]$ be such that $\mu (A)$ is 1 if $A$ is infinite or 0 otherwise.
Then $\mu $ is a finitely additive measure.
However it is impossible to extend $\mu $ to be a countably additive measure.
Observe that if $A_n = \set{n}$ the $\mu (\cup_{n} A_n) = 1$ but $\sum_{n} \mu (A_n) = 0$.
\end{example}

\begin{example}
Let $(A, \mathcal{A} )$ a measurable space.
Let $\mu : \mathcal{A}  \to [0, +\infty]$ be $0$ if $A = \varnothing$ and $\mu (A)$ is $+\infty$ otherwise.
Then $\mu $ is a measure.

\end{example}

\begin{example}
Let $A$ be set with at least two elements ($\card{A} \geq 2$).
Let $\mathcal{A}  = \powerset{A}$.
Let $\mu : \mathcal{A}  \to [0, +\infty]$ such that $\mu (A)$ is $0$ if $A = \varnothing$ and $\mu (A) = 1$ otherwise.
Then $\mu $ is not a measure, nor is $\mu $ finitely additive.
\begin{proof}Let $B, C \in \mathcal{A} $,
$B \cap  C = \varnothing$
then using finite additivity
We obtain a contradiction
\[
1 = \mu (B \cup C) \neq \mu (B) + \mu (C) = 2
\]\end{proof}
\end{example}

\subsection*{Properties}

\begin{proposition}[monotonicity]
Suppose $(A, \mathcal{A} , \mu )$ is measure space.
Then
\[
\mu (B) \leq \mu (C) \quad \text{for all } B \subset C \subset A
\]
\end{proposition}

\begin{proposition}[subaddivity]
Suppose $(A, \mathcal{A} , m)$ is a measure space and $\set{A_n} \subset \mathcal{A} $ is a countable family.
Then $m(\cup A_n) \leq \sum_{i} m(A_i)$.
\end{proposition}

\begin{proposition}
For a measure space $(A, \mathcal{A} , m)$.
\[
m(\cup_{n = 1}^{\infty} A_i) = \lim_{n \to \infty} m(A_i)
\]
\end{proposition}

\begin{proposition}
For a measure space $(A, \mathcal{A} , m)$.
\[
m(\cap _{n = 1}^{\infty} A_i) = \lim_{n \to \infty} m(A_i)
\]
\end{proposition}

%macros.tex
%%%%% MACROS %%%%%%%%%%%%%%%%%%%%%%%%%%%%%%%%%%%%%%%%%%%%%%%
%%%%%%%%%%%%%%%%%%%%%%%%%%%%%%%%%%%%%%%%%%%%%%%%%%%%%%%%%%%%
