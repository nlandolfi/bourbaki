
\section*{Why}

We extend our notion of length, area, and volume beyond the Lebesgue measure on the product spaces of real numbers.

\section*{Definition}

An extended-real-valued non-negative function on an \textit{algebra} is \t{finitely additive} if the result of the function applied to the union of a disjoint finite family of distinguished sets is the sum of the results of the function applied to each of the sets individually.

An extended-real-valued non-negative function on a \textit{sigma algebra} is \t{countably additive} if the result of the function applied to the union of a disjoint countable family of distinguished sets is the limit of the partial sums of the results of the function applied to each of the sets individually.

A \t{finitely additive measure} is an extended-real-valued non-negative finitely additive function which associates the empty set with the real number $0$.
A \t{countably additive measure} is an extended-real-valued non-negative countably additive function which associates the empty set with the real number $0$.
We call countably additive measures \t{measures}, for short.

Every countably additive measure is finitely additive.
On the other hand, there exist finitely additive measures which are not countably additive.

In the context of measure, we call a countably unitable subset algebra a \t{measurable space}.
We call the distinguished sets \t{measurable} sets.
A \t{measure space} is triple.
As a pair, the first two objects are a measurable space.
The third object is a measure defined on the sigma algebra of the measurable space.

\subsection*{Notation}

Suppose $A$ a nonempty set and $\mathcal{A} $ is a sigma algebra on $A$ so that the pair $(A, \mathcal{A} )$ is a measurable space.

Let $\mu : \mathcal{A}  \to [0, \infty]$ a measure;
thus:
(a) $\mu (\varnothing) = 0$ and
(b) for disjoint $\set{A_n} \subset \mathcal{A} $,
$\mu (\cup_{n = 1}^{\infty} A_n) = \sum_{n = 1}^{\infty} \mu (A_n)$
The triple $(A, \mathcal{A} , \mu )$ is a measure space.

We use $\mu $ since it is a mnemonic for ``measure''.
We often also us $\nu $ to denote measures, since it is after $\mu $ in the Greek alphabet, and $\lambda $, since it is before $\mu $ in the Greek alphabet.

\subsection*{Examples}

\begin{example}
Let $(A, \mathcal{A} )$ a measurable space.
Let $\mu : \mathcal{A}  \to [0, +\infty]$ such that $\mu (A)$ is $\card{A}$ if $A$ is finite and $\mu (A)$ is $+\infty$ otherwise.
Then $\mu $ is a measure.
We call $\mu $ the \t{counting measure}.
\end{example}

\begin{example}
Let $(A, \mathcal{A} )$ measurable.
Fix $a \in A$.
Let $\mu : \mathcal{A}  \to [0, +\infty]$ such that $\mu (A)$ is $1$ if $a \in A$ and $\mu (A)$ is $0$ otherwise.
Then $\mu $ is a measure.
We call $\mu $ the \t{point mass} concentrated at $a$.
\end{example}

\begin{expl}
The Lebesgue measure on the measurable space $(\R , \mathcal{B} (\R ))$ is a measure.
\end{expl}

\begin{expl}
Let $\mathcal{A} $ the co-finite algebra on $N$.
Let $\mu : \mathcal{A}  \to [0, +\infty]$ be such that $\mu (A)$ is 1 if $A$ is infinite or 0 otherwise.
Then $\mu $ is a finitely additive measure.
However it is impossible to extend $\mu $ to be a countably additive measure.
Observe that if $A_n = \set{n}$ the $\mu (\cup_{n} A_n) = 1$ but $\sum_{n} \mu (A_n) = 0$.
\end{expl}

\begin{expl}
Let $(A, \mathcal{A} )$ a measurable space.
Let $\mu : \mathcal{A}  \to [0, +\infty]$ be $0$ if $A = \varnothing$ and $\mu (A)$ is $+\infty$ otherwise.
Then $\mu $ is a measure.
\end{expl}

\begin{expl}
Let $A$ be set with at least two elements ($\card{A} \geq 2$).
Let $\mathcal{A}  = \powerset{A}$.
Let $\mu : \mathcal{A}  \to [0, +\infty]$ such that $\mu (A)$ is $0$ if $A = \varnothing$ and $\mu (A) = 1$ otherwise.
Then $\mu $ is not a measure, nor is $\mu $ finitely additive.
\begin{proof}
Let $B, C \in \mathcal{A} $,
$B \cap  C = \varnothing$
then using finite additivity
we obtain a contradiction
$
1 = μ(B \cup C) = μ(B) + μ(C) = 2
$.
\end{proof}
\end{expl}

\ssection{Why}

We want to generalize the notions of length, area, and volume.

\ssection{Definition}

A \t{measurable space}
is a sigma algebra.
We call the
\rt{distinguished subsets}{distinguishedsubsets}
the \ct{measurable sets}{measurablesets}.
A
\ct{measure}{measure}
on a measurable space
is a function from the sigma algebra
to the positive extended reals.
A
\ct{measure space}{measurespace}
is a measurable space and a measure.

\ssubsection{Notation}

\ssubsection{Properties}

\begin{prop}
Let $(A, \mathcal{A} )$ be a measurable space and
$m: \mathcal{A}  \to [0, \infty]$ be a measure.

If $B \subset C \subset A$, then $m(B) \leq m(C)$.
We call this property the of measures
\ct{monotonicity of measure}{}.
\end{prop}

\begin{prop}
For a measure space $(A, \mathcal{A} , m)$.

If $B \subset C \subset A$, then $m(B) \leq m(C)$.

We call this property the
\ct{monotonicity of measure}{}.
\end{prop}

\begin{prop}
For a measure space $(A, \mathcal{A} , m)$.

If $\set{A_n} \subset \mathcal{A} $ a countable family,
then $m(\cup A_n) \leq \sum_{i} m(A_i)$.

We this property the
\ct{sub-additivty of measure}{}.
\end{prop}

\begin{prop}
For a measure space $(A, \mathcal{A} , m)$.

If $\set{A_n} \subset \mathcal{A} $ a countable family,
then $m(\cup A_n) \leq \sum_{i} m(A_i)$.

We this property the
\ct{sub-additivty of measure}{}.
\end{prop}

\begin{prop}
For a measure space $(A, \mathcal{A} , m)$.
\[
m(\cup_{n = 1}^{\infty} A_i) = \lim_{n \to \infty} m(A_i)
\]
\end{prop}

\begin{prop}
For a measure space $(A, \mathcal{A} , m)$.
\[
m(\cap _{n = 1}^{\infty} A_i) = \lim_{n \to \infty} m(A_i)
\]
\end{prop}

