%!name:measures
%!need:cardinality
%!need:subset_algebras
%!need:power_set

\ssection{Why}

We want to generalize the notion of length, area, volume beyond the Lebesgue measure on the product spaces of real numbers.

\ssection{Definition}

An extended-real-valued non-negative
function on an algebra is
\ct{finitely additive}{nnfinitelyadditive}
if the result of the function applied to
the union of a disjoint finite family of
distinguished sets is the sum of the
results of the function applied to each
of the sets individually.

An extended-real-valued non-negative
function on a sigma algebra is
\ct{countably additive}{nncountablyadditive}
if the result of the function applied to
the union of a disjoint countable family of
distinguished sets is the limit of the partial
sums of the results of the function applied
to each of the sets individually.

A
\ct{finitely additive measure}{finitelyadditivemeasure}
is an extended-real-valued non-negative
finitely additive function which associates the empty
set with the real number $0$.
A
\ct{countably additive measure}{finitelyadditivemeasure}
is an extended-real-valued non-negative
countably additive function which associates the empty
set with the real number $0$.
We call countably additive measures
\ct{measures}{measures}, for short.

Every countably additive measure is finitely additive.
On the other hand, there exist finitely additive measures
which are not countable additive.

In the context of measure,
we call a countably unitable subset algebra
a \t{measurable space}.
We call the distinguished sets
\ct{measurable}{measurable}
sets.
A
\ct{measure space}{measurespace}
is triple.
As a pair, the first two
objects are a measurable space.
The third object is a measure defined
on the sigma algebra of the measurable space.

\ssection{Notation}

Let $A$ a set.
Let $\mathcal{A}$ a sigma algebra on $A$.
The pair $(A, \mathcal{A})$ is a measurable space.

Let $\mu: \mathcal{A} \to [0, \infty]$ a measure;
thus:
(a) $\mu(\emptyset) = 0$ and
(b) for disjoint $\set{A_n} \subset \mathcal{A}$,
$\mu(\cup_{n = 1}^{\infty} A_n)
  = \sum_{n = 1}^{\infty} \mu(A_n)$
The triple $(A, \mathcal{A}, \mu)$ is a
measure space.

We use $\mu$ since it
is a mnemonic for \say{measure}.
We often also us
$\nu$ to denote measures,
since it is after
$\mu$ in the Greek alphabet,
and $\lambda$, since
it is before
$\mu$ in the Greek alphabet.

\ssection{Examples}

\begin{expl}
Let $(A, \mathcal{A})$ a measurable space.
Let $\mu: \mathcal{A} \to [0, +\infty]$ such that
$\mu(A)$ is $\card{A}$ if $A$ is finite
and $\mu(A)$ is $+\infty$ otherwise.
Then $\mu$ is a measure.
We call $\mu$ the
\ct{counting measure}{countingmeasure}.
\end{expl}

\begin{expl}
Let $(A, \mathcal{A})$ measurable.
Fix $a \in A$.
Let $\mu: \mathcal{A} \to [0, +\infty]$ such
that $\mu(A)$ is $1$ if $a \in A$ and
$\mu(A)$ is $0$ otherwise.
Then $\mu$ is a measure.
We call $\mu$ the
\ct{point mass}{pointmassmeasure}
concentrated at $a$.
\end{expl}

\begin{expl}
Let $R$ denote the real numbers.
The Lebesgue measure on the measurable
space $(R, \mathcal{B}(R))$ is a measure.
\end{expl}

\begin{expl}
Let $N$ be the natural numbers.
Let $\mathcal{A}$ the finite
co-finite algebra on $N$.
Let $\mu: \mathcal{A} \to [0, +\infty]$
be such that $\mu(A)$ is 1 if
$A$ is infinite or 0 otherwise.
Then $\mu$ is a finitely additive measure.
However it is impossible to extend
$\mu$ to be a countably additive measure.
Observe that if $A_n = \set{n}$ the
$\mu(\union_{n} A_n) = 1$ but
$\sum_{n} \mu(A_n) = 0$.
\end{expl}

\begin{expl}
Let $(A, \mathcal{A})$ a measurable
space.
Let $\mu: \mathcal{A} \to [0, +\infty]$
be $0$ if $A = \emptyset$ and
$\mu(A)$ is $+\infty$ otherwise.
Then $\mu$ is a measure.
\end{expl}

\begin{expl}
Let $A$ be set with at least
two elements ($\card{A} \geq 2$).
Let $\mathcal{A} = \powerset{A}$.
Let $\mu: \mathcal{A} \to [0, +\infty]$
such that $\mu(A)$ is $0$ if $A = \emptyset$
and $\mu(A) = 1$ otherwise.
Then $\mu$ is not a measure,
nor is $\mu$ finitely additive.
\begin{proof}
Let $B, C \in \mathcal{A}$,
$B \intersect C = \emptyset$
then using finite additivity
we obtain a contradiction
$
1 = \mu(B \union C) = \mu(B) + \mu(C) = 2
$.
\end{proof}
\end{expl}

\ssection{Why}

We want to generalize the notions
of length, area, and volume.

\ssection{Definition}

A
\ct{measurable space}{measurablespace}
is a sigma algebra.
We call the
\rt{distinguished subsets}{distinguishedsubsets}
the \ct{measurable sets}{measurablesets}.

A
\ct{measure}{measure}
on a measurable space
is a function from the sigma algebra
to the positive extended reals.
A
\ct{measure space}{measurespace}
is a measurable space and a measure.

\ssubsection{Notation}

\ssubsection{Properties}

\begin{prop}
  Let $(A, \mathcal{A})$ be a measurable space and
  $m: \mathcal{A} \to [0, \infty]$ be a measure.

  If $B \subset C \subset A$, then $m(B) \leq m(C)$.
  We call this property the of measures
  \ct{monotonicity of measure}{}.
\end{prop}

\begin{prop}
  For a measure space $(A, \mathcal{A}, m)$.

  If $B \subset C \subset A$, then $m(B) \leq m(C)$.

  We call this property the
  \ct{monotonicity of measure}{}.
\end{prop}

\begin{prop}
  For a measure space $(A, \mathcal{A}, m)$.

  If $\set{A_n} \subset \mathcal{A}$ a countable family,
  then $m(\union A_n) \leq \sum_{i} m(A_i)$.

  We this property the
  \ct{sub-additivty of measure}{}.
\end{prop}

\begin{prop}
  For a measure space $(A, \mathcal{A}, m)$.

  If $\set{A_n} \subset \mathcal{A}$ a countable family,
  then $m(\union A_n) \leq \sum_{i} m(A_i)$.

  We this property the
  \ct{sub-additivty of measure}{}.
\end{prop}

\begin{prop}
  For a measure space $(A, \mathcal{A}, m)$.

  $$
    m(\union_{n = 1}^{\infty} A_i) = \lim_{n \to \infty} m(A_i)
  $$
\end{prop}

\begin{prop}
  For a measure space $(A, \mathcal{A}, m)$.

  $$
    m(\intersection_{n = 1}^{\infty} A_i) = \lim_{n \to \infty} m(A_i)
  $$
\end{prop}

\ssubsection{Examples}

\begin{expl}
  \ct{counting measure}{}
\end{expl}
