%!name:topological_spaces
%!need:metrics
%!refs:bert_mendelson∕introduction_to_topology∕theory_of_sets∕introduction

\section*{Why}

We want to generalize the notion of continuity.

\section*{Definition}

A \t{topological space} is a base set and a set distinguisehed subsets of this set for which:
(1) the empty set base set are distinguised (2) the intersection of a finite family of distinguished subsets is distinguished, and (3) the union of a family of distinguished subsets is distinguished.
We call the set of distinguished subsets the \t{topology} and we call its members the \t{open sets}.

\subsection*{Notation}

Let $X$ be a non-empty set.
For the set of distinguished sets, we tend to use $\mathcal{T} $, a mnemonic for topology, read aloud as \say{script T}.
We tend to denote elements of $\mathcal{T} $ by $O$, a mnemonic for open.
We denote the topological space with base set $X$ and topology $\mathcal{T} $ by $(X, \mathcal{T} )$.
We denote the properties satisfied by elements of $\mathcal{T} $:
  \begin{enumerate}
  \item $X, \varnothing \in \mathcal{T} $
  \item $\set{O_i}_{i = 1}^{n} \subset \mathcal{T}  \implies \cap_{i = 1}^{n} O_i \in \mathcal{T} $
  \item $\set{O_\alpha }_{\alpha  \in I} \subset \mathcal{T} \implies \cup_{\alpha  \in I} \in \mathcal{T} $
  \end{enumerate}

\section*{Examples}

$\R $ with the open intervals as the open sets is a topological space.
