%!name:nondeterministic_finite_automata
%!need:finite_automata
%!need:regular_languages
%%!refs:sipser/1

\section*{Definition}

A \t{nondeterministic finite automata} $N = (Q, \Sigma , \delta , q_0, F)$ is a list where $Q$ and $\Sigma $ are finite sets, $\delta : Q \times \Sigma  \to \powerset{Q}$, $q_0 \in Q$ and $F \subset Q$.
A \t{nondeterministic finite automata with empty moves} $N = (Q, \Sigma , \delta , q_0, F)$ is a list where $Q$ and $\Sigma $ are finite sets, $\delta : Q \times (\Sigma  \cup \set{\varnothing}) \to \powerset{Q}$, $q_0 \in Q$ and $F \subset Q$.

As with finite automata, we call $Q$ the \t{states}, $\Sigma $ the \t{alphabet}, $\delta $ the \t{transition function}, $q_0$ the \t{start state}, and $F$ the \t{accept states} (or \t{final states}).
An input $u \in \str(\Sigma )$ results in a state sequence $x \in \str(Q)$ with $x_1 = q_0$ and $x_{i+1} = \delta (x_i, u_i)$ for $i = 1, \dots , \num{u}$.

\section*{Main result}

For any automata $M$, there exists a nondeterminstic finite automata $N$ such that $N$ accepts the same languages as $M$.
  \ifhmode\unskip\fi\footnote{
Future editions will include an account.
  }
For this reason, a language is regular if and only if some nondeterministic finite automaton accepts it.

\blankpage