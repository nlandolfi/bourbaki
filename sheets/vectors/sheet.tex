%!name:vectors
%!need:fields
%!need:n-dimensional_space
%!need:real_vectors

\ssection{Why}

We speak of objects which we can add and scale. Of course, our model is the real vectors.

\ssection{Definition}

A \t{vector space} (or \t{linear space}, or \t{linear class}) is an ordered triple consisting of
(1) a commutative group, (2) a field, and (3) a function from the product of the field ground set and group ground set to the group ground set.

We call the ground set of the commutative group the \t{vectors} and we call the group operation \t{vector addition}.
We call the field the \t{scalars}.
The field has addition and multiplication of scalars which we call \t{scalar addition} and \t{scalar multiplication}, respectively.
The \t{scaling map} is the map on pairs of scalars and vectors.

The first property is that the result of scaling a vector by a scalar, and then scaling the result by a second scalar is the same as first multiplying the scalars and then scaling the vector.
The second property is that the result of scaling a vector by the scalar sum of two scalars is the same as scaling the vector by each scalar separately and then adding the vectors.
The third property is that the result of scaling the vector sum of two vectors is the same as scaling each individually and then adding the two vectors.

\ssubsection{Notation}

Let $(V,+)$ be a commutative group and $(F, \tilde{+}, \cdot)$ be a field.
Let $s: F \times V \to V$. $((V, +), (F, \tilde{+}, \cdot), s)$ is a vector space if
\begin{enumerate}
  \item $s(b,s(a, v)) = s(b \cdot a, v)$ for all $a, b \in F$ and $v \in V$.
  \item $s(a\tilde{+}b, v) = s(a, v) + s(b, v)$ for all $a, b \in F$ and $v \in V$.
  \item $s(a, u + v) = s(a,u) + s(a, b)$ for all $a \in F$ and $u, v \in V$.
\end{enumerate}

We can simplify (contract) the notation above with four conventions.
First, we denote both vector and scalar addition by $+$.
So $a + b$ denotes $a \tilde{+} b$ for all $a, b \in F$.
Second, we contract $\cdot$.
So $ab$ denotes $a \cdot b$ for all $a, b \in F$.
Third, we contract the scaling map.
So $av$ denotes $s(a, v)$ for $a \in F$ and $v \in V$.
Fourth, we conventionally take the scaling map to have precedence so that $au + av$ means $(au) + (av)$ for all $a \in F$ and $u,v \in V$.

We denote:
(1)
$b(av) = (ba)v$;
(2)
$(a+b)v = av + bv$;
and (3)
$a(u + v) = au + av$
for all
$a,b \in F$ and $u, v \in V$.

Finally, we say: let $(V, F)$
be a vector space, assuming the
above notational conventions.
Often the field will be assumed, in which case we will refer to the vector space $V$ and define functions on $V$.
