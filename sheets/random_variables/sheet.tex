%!name:random_variables
%!need:probability_measures
%!need:topological_sigma_algebra
%!need:measurable_functions
%!need:outcome_variables

\ssection{Why}
  \ifhmode\unskip\fi\footnote{
Future editions will include this.
  }

\ssection{Definition}

A \t{random variable} is a measurable map from a probability space to a measurable space.

A \t{real-valued random variable} is a measurable map between the probability space and the set of real numbers with its topological sigma algebra.

\ssubsection{Notation}

Let $(X, \SA, \PM)$ be a probability space.
Let $(Y, \SB)$ a measurable space.
Then a random variable is a measurable function $f: X \to Y$.

Some authors denote real-valued random variables by upper case Latin letters:
for example, $X, Y, Z$. In this case, the base probability space is denoted by $\Omega$, a mnemonic for \say{outcomes.}
Let $(\Omega,\SA,\PM)$ be a probability space.
Let $X: \Omega \to \R$ be measurable.
Then $X$ is a real-valued random variable.

Some authors use notation for the probability of particular, common sets.
Although the authors often use regular parentheses, we will use \say{$[$} and \say{$]$} for precision.
Let $A \in \SB(\R)$.
Let $\PM[X \in A]$ denote $\PM(X^{-1}(A))$.
These are equivalent to
  \[
\PM(\Set*{\omega \in \Omega}{X(\omega) \in A}).
  \]
As we mentioned, some authors use $P(X \in A)$ for $P[X \in A]$, which is mostly harmless.

Next, let $Y: \Omega \to \R$ a measurable function and let $B \in \SB(\R)$.
Similar to the above,
let $\PM[X \in A,Y \in B]$
denote $\PM(X^{-1}(A) \intersect Y^{-1}(B))$.
These are equivalent to
  \[
\PM(\Set*{\omega \in \Omega}{X(\omega) \in A \text{ and } Y(\omega) \in B}).
  \]
Similarly for $n$ random variables $X_1, \dots, X_n: \Omega \to \R$,
  \[
\PM[X_1 \in A_1, \dots, X_n \in A_n] = \PM(\intersect_{i = 1}^{n} X_i^{-1}(A_i)).
  \]
