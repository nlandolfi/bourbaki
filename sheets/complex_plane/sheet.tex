%!name:complex_plane
%!need:complex_numbers
%!need:real_plane
%!refs:elias_stein/complex_analysis/01_preliminaries_to_complex_analysis

\ssection{Why}

We are regularly thinking of the set $\C$ as identified with the plane.

\ssection{Definition}

Since $\C = \R^2$, we can identify $z \in \C$ with a point in the plane, as we did in \sheetref{real_plane}{Real Plane}.
In this case, if $z = x + iy \in \C$ (i.e, $z = (x, y) \in \R^2$), we can visualize this identification in the following figure.
We can identify the origin with the complex number $(0,0) = 0 \in \C$.
For this reason we call $0 \in \C$ the \t{complex origin}.
Likewise, the imaginary number $(0, 1) = i \in \C$ corresponds to $(0,1)$
Clearly, the horizontal axis corresponds to the purely real numbers and the vertical axis corresponds to the purely imaginary numbers.
For these reasons, we refer to these axes as the \t{real axis} and \t{imaginar axis}, respectively.

\begin{figure}[h]
\centering
\vspace{0.5cm}
  \includegraphics[width=0.9\textwidth]{\figpath{complex_plane}{complex_plane}}
\caption{The complex plane}
\label{complex_plane:figure:complex_plane}
\end{figure}


\ssection{Modulus and argument}

The \t{modulus} of $z \in \C$ is the distance of $z$ to the origin.
If $z \in \C$, then the modulus of $z$ is
\[
  \sqrt{\re{z}^2 + \im{z}^2}.
\]
We denote the modulus of $z$ by $\Cmod{z}$.

The \t{argument} of $z \in \C$ is $\tan^{-1}(\im{z}/\re{z})$.
We denote the argument of $z$ by $\arg z$.\footnote{Future editions will include the geometric interpretations.}

\ssection{Complex disc}

The \t{complex disc} is the set $\Set*{z \in \C}{\Cmod{z} M 1}$.
We denote it by $\D$, a mnemonic for disc.
The \t{complex unit circle} is the set $\Set*{z \in \C}{\Cmod{z} = 1}$.
We denote it by $\T$, a mnemonic for torus.

\blankpage
