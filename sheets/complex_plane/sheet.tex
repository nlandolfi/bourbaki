
%!name:complex_plane
%!need:complex_numbers
%!need:real_plane
%!refs:elias_stein∕complex_analysis∕01_preliminaries_to_complex_analysis

\section*{Why}

We are regularly thinking of $\C $ as plane.

\section*{Definition}

Since $\C  = \R ^2$, we can identify elements of $\C $ with points in the plane (as we did in \sheetref{real_plane}{Real Plane}).
In this case, if $z = x + iy \in \C $ (i.e, $z = (x, y) \in \R ^2$), we can visualize this identification in the following figure.

\begin{center}\includegraphics[width=0.90\textwidth]{./graphics/complex_plane.png}\end{center}
% TODO bring back figure 

%<div data-littype='paragraph'>
% <div data-littype='run'> \begin{figure}[h] </div>
% <div data-littype='run'> \centering </div>
% <div data-littype='run'> \vspace{0.5cm} </div>
% <div data-littype='run'> \includegraphics[width=0.9\textwidth]{\figpath{complex_plane}{complex_plane}} </div>
% <div data-littype='run'> \caption{The complex plane} </div>
% <div data-littype='run'> \label{complex_plane:figure:complex_plane} </div>
% <div data-littype='run'> \end{figure} </div>
%</div>

We can identify the origin with the complex number $(0,0) = 0 \in \C $.
For this reason we call $0 \in \C $ the \t{complex origin}.
Likewise, the imaginary number $(0, 1) = i \in \C $ corresponds to $(0,1)$
Clearly, the horizontal axis corresponds to the purely real numbers and the vertical axis corresponds to the purely imaginary numbers.
For these reasons, we refer to these axes as the \t{real axis} and \t{imaginary axis}, respectively.
We refer to the above figure as the \t{complex plane} (or \t{Argand diagram}).

\blankpage