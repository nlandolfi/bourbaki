%!name:complex_plane
%!need:complex_numbers
%!need:real_plane
%!refs:elias_stein/complex_analysis/01_preliminaries_to_complex_analysis

\ssection{Why}

We are regularly thinking of $\C$ as plane.

\ssection{Definition}

Since $\C = \R^2$, we can identify elements of $\C$ with points in the plane (as we did in \sheetref{real_plane}{Real Plane}).
In this case, if $z = x + iy \in \C$ (i.e, $z = (x, y) \in \R^2$), we can visualize this identification in the following figure.

\begin{figure}[h]
\centering
\vspace{0.5cm}
  \includegraphics[width=0.9\textwidth]{\figpath{complex_plane}{complex_plane}}
\caption{The complex plane}
\label{complex_plane:figure:complex_plane}
\end{figure}

We can identify the origin with the complex number $(0,0) = 0 \in \C$.
For this reason we call $0 \in \C$ the \t{complex origin}.
Likewise, the imaginary number $(0, 1) = i \in \C$ corresponds to $(0,1)$
Clearly, the horizontal axis corresponds to the purely real numbers and the vertical axis corresponds to the purely imaginary numbers.
For these reasons, we refer to these axes as the \t{real axis} and \t{imaginary axis}, respectively.

\blankpage
