%!name:nearest_neighbor_predictors
%!need:inductors

\ssection{Why}

We might expect similar
precepts to lead to
similar postcepts.

\ssection{Definition}

Consider a set of precepts.
We call a real-valued function on ordered
pairs of precepts that is nonnegative and
zero when applied to the a pair of the same
precept a similarity function.

\ssubsection{Notation}

Let $n$ be a natural number.
Let $\Xi$ be a length $n$ paired record sequence
in $\CU \cross \CV$; so
\[
  \Xi = ((u^1, v^1), \dots, (u^n, v^n))
\]
with $u^i \in \CU$ and $v^i \in \CV$ for $i = 1,\dots,n$.

The nearest neighbor induction associates
$\Xi$ with the function $f_{\Xi}$ such that
\[
  f_{\Xi}(u) = v^j
\]
where $j < n$ is the largest integer such that
\[
  d(u, u^j) = \min_{i} \set{d(u, u^i)}.
\]
