
%!name:contingency_tables
%!need:arrays
%!need:partitions

\section*{Why}

We want to summarize the interaction between two binary traits.

\section*{Definition}

The \t{contingency table} of a population $\set{1,\dots ,n}$ with respect to binary traits $a, b : \upto{n} \to \set{0,1}$ is the array $A \in \N  ^{2 \times  2}$ of natural numbers defined by
\[
A = \barray{ \num{a^{-1}(0) \cap  b^{-1}(0)} & \num{a^{-1}(0) \cap  b^{-1}(1)} \\ \num{a^{-1}(1) \cap  b^{-1}(0)} & \num{a^{-1}(0) \cap  b^{-1}(1)}}.
\]
We interpret $A_{11}$ as the number of individuals which have neither trait, $A_{12}$ as the individuals which have trait $b$ but not trait $a$, and so on.

\section*{Normalization}

These four sets partition $\upto{n}$, so that if we divide the elements by $n$, we obtain four numbers which sum to 1, the 2 by 2 table with these entries is called the \t{normalized contingency table}.

\section*{Contingency arrays}

In general, we have $k$ binary traits, each of which an individual may or may not have.
We encode these traits using $k$ functions
\[
a_1, \dots , a_k: \upto{n} \to \set{0,1}.
\]
The contingency array is $k$-dimensional array $A$, with
\[
A_x = \cap _{j = 1}^{k} a_j^{-1}(x_j),
\]
where $x \in \set{0,1}^k$.

\blankpage