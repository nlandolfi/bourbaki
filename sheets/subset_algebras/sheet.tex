
%!name:subset_algebras
%!need:operations
%!need:cardinality
%!need:subset_systems

\section*{Why}

We often want a subset system whose set of distinguished subsets is closed under the usual set operations.

\section*{Definition}

A \t{subset algebra} is a subset system for which (1) the base set is distinguished (2) the complement of a distinguished set is distinguished (3) the union of two distinguished sets is distinguished.

We call the set of distinguished sets an \t{algebra} on the the base set.
We justify this language by showing that the standard set operations applied to distinguished sets result in distinguished sets.

If a set of subsets is closed under complements it contains the base set if and only if it contains the empty set.
So we can replace condition (1) by insisting that the algebra contain the empty set.
Similarly, if a non-empty set of subsets is closed under complements and unions then it contains the base set: the union of a distinguished set and its complement.
Thus we can replace condition (1) by insisting that the algebra be non-empty.

\subsection*{Notation}

The notation follows that of a subset system.
Let $(A, \mathcal{A} )$ be a subset algebra.
We also say \say{let $\mathcal{A} $ be an algebra on $A$.}
Moreover, since the largest element of the algebra is the base set, we can say without ambiguity: \say{let $\mathcal{A} $ be an algebra.}

\section*{Properties}

\begin{proposition}
For any subset algebra, $\varnothing$ is distinguished.\end{proposition}
\begin{proposition}
For any subset algebra,
for any distinguished sets,
(a) the intersection is distinguished and
(b) their symmetric difference is distinguished.
So, if one contains the other, the complement
of the smaller in the larger is distinguished.\end{proposition}
\begin{proposition}
For any subset algebra,
for any finite family of distinguished sets,
(a) the finite family union and
(b) the finite family intersection
are both distinguished.\end{proposition}
So we could have defined an algebra
by insisting it be closed under finite
intersections.

\section*{Examples}

\begin{example}
For any set $A$, $(A, 2^{A})$ is a subset algebra.\end{example}
\begin{example}
For any set $A$, $(A, \set{A, \emptyset})$ is a subset algebra.\end{example}
\begin{example}
For any infinite set $A$, let $\mathcal{A} $ be the set
  \[
\Set*{
B \subset A
}{
\card{B} < \aleph_0 \lor
\card{C_{A}(B)} < \aleph_0
}.
  \]
$\mathcal{A} $ is an algebra;
the
\t{finite/co-finite algebra}.\end{example}
\begin{example}
For any infinite set $A$, let $\mathcal{A} $ be the set
  \[
\Set*{
B \subset A
}{
\card{B} \leq \aleph_0 \lor
\card{C_{A}(B)} \leq \aleph_0
}.
  \]
$\mathcal{A} $ is an algebra; the \t{countable/co-countable algebra}.\end{example}
\begin{example}
For any infinite set $A$, let $\mathcal{A} $ be the set
  \[
\Set*{
B \subset A
}{
\card{B} \leq \aleph_0
}.
  \]
$\mathcal{A} $ is not an algebra.\end{example}
\begin{example}
Let $A$ be an uncountable set.
Let $\mathcal{A} $ be the collection of all countable subsets of $A$.
$\mathcal{A} $ is not a sigma algebra.\end{example}