
%!name:inverses_of_composite_relations
%!need:relation_composites
%!need:converse_relations
%!need:equivalence_relations
%!refs:paul_halmost/naive_set_theory/section_10

\section*{Why}

How do inverse and converse relations interact.

\section*{Results}

Let $R$ be a relation between $X$ and $Y$ and let $S$ be a relation between $Y$ and $Z$.

\begin{proposition}
$(RS)^{-1} = S^{-1}R^{-1}$
\end{proposition}

\section*{Identity relations}

Recall that $I$ is the identity relation on $X$ if $x\,I\,y$ if and only if $x = y$.

\begin{proposition}
Let $R$ be a relation on $X$.
Let $I$ be the identity relation on $X$.
Then $RI = IR = R$.
\end{proposition}

One would like $RR^{-1} \supset I$, $R^{-1}R \supset I$.
The father of the son is the father and the son of the father is the son.
But the empty relation violates these claims.

\section*{Relation properties}

\begin{proposition}
$R$ is symmetric if and only if $R \subset R^{-1}$
\end{proposition}

\begin{proposition}
$R$ is reflextive if and only if $I \subset R$
\end{proposition}

\begin{proposition}
$R$ is transitive if and only if $RR \subset R$.
\end{proposition}

\blankpage