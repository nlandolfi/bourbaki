%!name:chordal_graphs_and_vertex_separators
%!need:vertex_separators

\section*{Why}

We characterize chordal graphs using vertex separators, and vice versa.
  \ifhmode\unskip\fi\footnote{
Future editions will expand and may include graphics.
  }

\section*{Main Result}

\begin{proposition}[Chordal Graphs and Vertex Separators]An undirected graph is chordal if and only if all minimal vertex separators are complete.\end{proposition}
\begin{proof}Let $G = (V, E)$ be an undirected graph.
First, suppose that all minimal vertex separators of $G$ are complete.
Let $c$ be a cycle of length greater than 3.
Let $v, w$ be nonconsecutive vertices in $c$.
If $v$ and $w$ are adjacent in $G$, then $\set{v, w} \in E$ is a chord.
If $v$ and $w$ are nonadjacent, then $vw$-separator exists.

The key insight is that there exists two non-consecutive vertices in the cycle that are also included in any $vw$-separator $T$.
Split the cycle into the path from $v$ to $w$, call it $p_1$ and the path from $w$ to $v$, call it $p_2$.
$T$ must include an interior point of $p_1$, call it $u_1$, otherwise $v$ and $w$ are connected.
Similarly, $T$ must include an interior point of $p_2$, call it $u_2$.
$u_1$ and $u_2$ are not consecutive in $c$, since they are distinct from $x$ and $y$.

Let $S$ be a minimal $vw$-separator.
Let $s, t \in S$ be two non-consecutive vertices in the cycle different from $v$ and $w$
By assumption $S$ is complete, so $s$ and $t$ are adjacent in $G$.

Second, let $G = (V, E)$ be a chordal graph.
Let $S$ be a minimal $vw$-separator.
Let $C_v$ and $C_w$ be the connected components containing $v$ and $w$ of the subgraph induced by $V \setminus S$.

If $\nu m{S} = 1$, then $S$ is complete.
Otherwise, let $x, y \in S$ by distinct.
We want to show $\set{x, y} \in E$.
The key insight is that $x$ is adjacent to vertices in $C_v$ and $C_w$.
If there were no such vertex, $S \setminus \set{x}$ would be a $vw$-separator and $S$ would not be minimal.
Similarly with $y$.
Also, $\nu m{C_v}, \nu m{C_w} \geq 1$.

With these observations, there exists a path from $x$ to $y$ through $C_v$.
Let $p_v = (x, v_1, \dots , v_k, y)$ be a path of shortest length with at least one interior vertex (so $k \geq 1$) from $x$ to $y$ using interior vertices in $v_1, \dots , v_k \in C_v$.
Let $p_w = (y, w_1, \dots , w_l, x)$ be a path of shortest length with at least one interior vertex (so $l \geq 1$) from $y$ to $x$ using interior vertices $w_1, \dots , w_k \in C_w$.
Use $p_v$ and $p_w$ to define the cycle $c = (x, v_1, \dots , v_k, y, w_1, \dots , w_l, x)$ which has length at least four.
$G$ is chordal, so $c$ has a chord.

We argue that the chord of $c$ is $\set{x, y}$.
Since $C_w$ and $C_V$ are different connected components (whose vertices are not included in $S$), there are no edges $\set{v_i, w_j}$ for $i = 1, \dots , k$ and $j = 1, \dots , l$.
Since $p_v$ and $p_w$ are paths of shortest length, they have no chords.
In particular, there is no edge $\set{v_i, v_j}$ for $\intabs{i - j} > 1$ or $\set{v_i, x}$ for $i = 1, \dots , l$.
Similarly, there is no edge $\set{w_i, w_j}$ for $\intabs{i - j} > 1$ or $\set{w_i, y}$ for $i = 1, \dots , l$.
The only remaining pair is $\set{x, y}$, and so it must be the chord.
\end{proof}