%!name:greatest_lower_bounds
%!need:real_order
%!need:total_orders

\ssection{Why}\footnote{To be given in future editions.}

\ssection{Definition}

Let $(A, \leq)$ be a chain.

A \t{lower bound} for $B \subset A$ is an element $a \in A$ so that $a \leq b$ for all $b \in B$.
A set is \t{bounded from below} if it has a least upper bound.
A \t{greatest lower bound} for $B$ is an element $c \in A$ so that $c$ is a lower bound and $c < a$ for all other lower bounds $a$.

\begin{proposition}
  If there is a greatest lower bound it is unique.\footnote{Proof in future editions.}
\end{proposition}

We call the unique greatest lower bound of a set (if it exists) the \t{infimum}.

\ssection{Notation}

We denote the infimum of a set $B \subset A$ by $\inf A$.

\blankpage
