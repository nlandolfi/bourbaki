
\section*{Definition}

Suppose $(A, \leq)$ is a partially ordered set.
A \t{lower bound} for $B \subset A$ is an element $a \in A$ satisfying
\[
a \leq b \quad \text{for all } b \in B
\]
In words, $a$ is a predecessor of every element of $B$.
A set is \t{bounded from below} if it has a lower bound.
A \t{greatest lower bound} for $B$ is an element $c \in A$ so that $c$ is a lower bound and $c < a$ for all other lower bounds $a$.

\begin{proposition}
If there is a greatest lower bound it is unique.\footnote{Proof in future editions.}
\end{proposition}

We call the unique greatest lower bound of a set (if it exists) the \t{infimum}.

\section*{Notation}

We denote the infimum of a set $B \subset A$ by $\inf A$.

\blankpage