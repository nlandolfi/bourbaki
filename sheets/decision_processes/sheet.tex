%!name:decision_processes
%!need:set_numbers
%!need:probability_distributions
%!need:sequences

\ssection{Why}

We want to talk about making a sequence of decisions.

\ssection{Definition}

Let $S$ and $A$ be finite sets.
Let $T: S \times A \to (S \to [0, 1])$ so that for each $s \in S$ and $a \in A$, $T_{sa}: S \to [0, 1]$ is a probability distribution over $S$.
We call the ordered triple $(S, A, T)$ a \t{finite state-action process}.



A \t{trajectory} in the \t{state set} $S$ and \t{action set} $A$ is a sequence in $S \times A$.
We interpret


Let $r: S \times A \times S \to \R$, $N  \in \N$.

A \t{decision process} is a sequence $(S, A, T, r, \gamma, $, consists of two sets, a function set, an action

\ssection{Other terminology}

Decision processes are commonly called \t{markov decision processes}.\footnote{As usual, we avoid this terminology in connection with the projects guidelines against using particular names.}

\blankpage
