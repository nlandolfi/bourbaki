\sinput{../sheet.tex}
\sbasic

\sinput{../sets/macros.tex}
\sinput{../cartesian_product/macros.tex}
\sinput{../relations/macros.tex}

\sinput{../order_relations/macros.tex}

\sstart

\stitle{Order Relations}

\ssection{Why}

We want to handle elements of a set in a particular order.

\ssection{Definition}

Let $R$ be a relation on a non-empty set $A$.
$R$ is a \ct{partial order}{partialorder} if it is
\rt{reflexive}{reflexive}, \rt{transitive}{transitive},
and \rt{anti-symmetric}{anti-symmetric}.

A \ct{partially ordered set}{partiallyorderedset} is a set
and a partial order.
The language partial is meant to suggest that two elements
need not be comparable.
For example, suppose $R$ is $\Set{(a,a)}{a \in A}$;
we may justifiably call this no order at all and
call $A$ totally unordered, but it is a partial order
by our definition.

Often we want all elements of the set $A$ to be comparable.
We call $R$ \ct{connexive}{connexive} if for all
$a, b \in A$, $(a, b) \in R$ or $(b, a) \in R$.
If $R$ is a partial order and connexive,
we call it a \ct{total order}{totalorder}.

A \ct{totally ordered set}{totallyorderedset} is a set
together with a total order.
The language is a faithful guide: we can compare any
two elements.
Still, we prefer one word to three, and so we will use
the shorter term \ct{chain}{chain} for a
totally ordered set; other terms include
\ct{simply ordered set}{simplyorderedset} and
\ct{linearly ordered set}{linearlyorderedset}.

\ssubsection{Notation}
We denote total and partial orders on a set $A$ by $\preceq$.
We read $\preceq$ aloud as \say{precedes or equal to} and so read $a\preceq b$ aloud as \say{a precedes or is equal to b.}
If $a \preceq b$ but $a \neq b$, we write $a \prec b$, read aloud as \say{a precedes b.}


\strats
