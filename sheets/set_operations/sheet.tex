%!name:set_operations
%!need:operations
%!need:pair_unions
%!need:pair_intersections
%!need:set_symmetric_differences
%!refs:paul_halmos/naive_set_theory/section_06

\ssection{Why}

We want to consider the elements of two sets
together at once, and other sets created
from two sets.

\ssection{Definitions}

We have already mentioned that set unions is an operation when considered on the powerset of some given set (see \sheetref{operations}{Operations}).
It is natural to expect the same for intersections (see \sheetref{pair_intersections}{Pair Intersections}) and symmetric differences (see \sheetref{symmetric_differences}{Symmetric Differences}).

We call the operation of \t{forming unions} the function $(A, B) \mapsto A \union B$.
We call the operation of \t{forming intersections} the function $(A, B) \mapsto A \intersect B$.
We call the operation of \t{forming symmetric differences} the function $(A, B) \mapsto A \symdiff B$.

We have seen that forming unions commutes and is associative and likewise with forming intersections.
As a result of the commutativity of unions and intersections, forming symmetric differences also commutes.
%The \tasdf{union} of $A$ with $B$ is the
%\rt{set}{set} whose \rt{elements}{element} are
%in either $A$ \textit{or} $B$ \textit{or} both.
%The key word in the definition is \textit{or}.
%
%The \ct{intersection}{intersection} of $A$ with $B$ is
%the set whose elements are in both $A$ \textit{and} $B$.
%The keyword in the definition is \textit{and}.

%Viewed as operations, both union and intersection commute;
%this property justifies the language \say{with.}
%The intersection is a subset of $A$, of $B$,
%and of the union of $A$ with $B$.
%
%The \ct{symmetric difference}{symmetricdifference}
%of $A$ and $B$ is the set whose elements are in the union
%but not in the intersection.
%The symmetric difference commutes because both union and
%intersection commute; this property justifies the
%language \say{and.}
%The symmetric difference is a subset of the union.


%Let $C$ be a set containing $A$.
%The \casdft{complement}{complement} of $A$ in $C$ is
%the symmetric difference of $A$ and $C$.
%Since $A \subset C$, the union is $C$ and the
%intersection is $A$.
%So the complement is the \say{left-over} elements of $B$ after removing the elements of $A$.

We call these three operations the \t{set operations}.

\blankpage

%\ssubsection{Notation}
%
%Let $A, B$ be sets.
%We denote the union of $A$ with $B$ by $A \union B$, read aloud as \say{A union B.}
%$\union$ is a stylized U.
%We denote the intersection of $A$ with $B$ by $A \intersect B$, read aloud as \say{A intersect B.}
%We denote the symmetric difference of $A$ and $B$ by $A \symdiff B$, read aloud as \say{A symdiff B.}
%\say{Delta} is a mnemonic for difference.
%
%Let $C$ be a set containing $A$.
%We denote the complement of $A$ in $C$ by $C - A$, read aloud as \say{C minus A.}
%
%\ssubsection{Results}
%
%\begin{proposition}
%  For all sets $A$ and $B$ the operations $\union$, $\intersect$, and $\symdiff$ commute.
%\end{proposition}
%
%\begin{proposition}
%  Let $S$ a set.
%  For all sets $A, B \subset S$,
%  \[
%    \begin{aligned}
%      \text{(1)} \quad & S - (A \union B) = (S - A) \intersect (S - B) \\
%      \text{(2)} \quad & S - (A \intersect B) = (S - A) \union (S - B).
%    \end{aligned}
%  \]
%\end{proposition}
%
%\begin{proposition}
%  Let $S$ a set. For all sets $A, B \subset S$,
%  \[
%    A \symdiff B = (A \union B) \intersect C_S(A \intersect B)
%  \]
%  \boxed{TODO: notation}
%\end{proposition}
%
%\ssection{Exercises}
%
%% TODO: this copied directly from halmos
%If $A$, $B$, $X$, and $Y$ are sets, then
%\begin{enumerate}
%  \item $(A \union B) \cross X) = (A \cross X) \union (B \cross X)$,
%  \item $(A \intersect B) \cross (A \intersect Y) = (A \cross X) \intersect (B \cross Y)$,
%  \item $(A - B) \times X = (A \times X) - (B \times X)$.
%\end{enumerate}
%	If either $A = \varnothing$ or $B = \varnothing$, then $A \times B = \varnothing$, and conversely.
%	If $A \subset X$ and $B \subset Y$, then $A \times B \subset X \times Y$, and (provided $A \times B \neq \varnothing)$ conversely.
