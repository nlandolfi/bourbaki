
%!name:entire_functions
%!need:complex_analytic_functions
%!refs:yellow/IX/4

\section*{Definition}

An \t{entire function} is a complex function $f: \C  \to \C $ which is analytic for all $z \in \C $.

\blankpage
\sbasic
%%%% MACROS %%%%%%%%%%%%%%%%%%%%%%%%%%%%%%%%%%%%%%%%%%%%%%%

\newcommand{\PM}{\mathbf{P}}

%%%%%%%%%%%%%%%%%%%%%%%%%%%%%%%%%%%%%%%%%%%%%%%%%%%%%%%%%%%

%%%% MACROS %%%%%%%%%%%%%%%%%%%%%%%%%%%%%%%%%%%%%%%%%%%%%%%

% use \set{stuff} for { stuff }
% use \set* for autosizing delimiters
\DeclarePairedDelimiter{\set}{\{}{\}}

% use \Set{a}{b} for {a | b}
% use \Set* for autosizing delimiters
\DeclarePairedDelimiterX{\Set}[2]{\{}{\}}{#1 \nonscript\;\delimsize\vert\nonscript\; #2}

% use \powerset{A} for power set of A
\newcommand{\powerset}[1]{2^{#1}}

\renewcommand{\emptyset}{\varnothing}

\newcommand{\SA}{\mathcal{A}}
\newcommand{\SB}{\mathcal{B}}
\newcommand{\SC}{\mathcal{C}}
\newcommand{\SD}{\mathcal{D}}
\newcommand{\SE}{\mathcal{E}}
\newcommand{\SF}{\mathcal{F}}
\newcommand{\SG}{\mathcal{G}}
\newcommand{\SH}{\mathcal{H}}
\newcommand{\SI}{\mathcal{I}}
\newcommand{\SJ}{\mathcal{J}}
\newcommand{\SK}{\mathcal{K}}
\newcommand{\SL}{\mathcal{L}}

%%%%%%%%%%%%%%%%%%%%%%%%%%%%%%%%%%%%%%%%%%%%%%%%%%%%%%%%%%%

%%%% MACROS %%%%%%%%%%%%%%%%%%%%%%%%%%%%%%%%%%%%%%%%%%%%%%%

\newcommand{\PM}{\mathbf{P}}

%%%%%%%%%%%%%%%%%%%%%%%%%%%%%%%%%%%%%%%%%%%%%%%%%%%%%%%%%%%

%%%% MACROS %%%%%%%%%%%%%%%%%%%%%%%%%%%%%%%%%%%%%%%%%%%%%%%

\newcommand{\PM}{\mathbf{P}}

%%%%%%%%%%%%%%%%%%%%%%%%%%%%%%%%%%%%%%%%%%%%%%%%%%%%%%%%%%%

%%%% MACROS %%%%%%%%%%%%%%%%%%%%%%%%%%%%%%%%%%%%%%%%%%%%%%%

\newcommand{\PM}{\mathbf{P}}

%%%%%%%%%%%%%%%%%%%%%%%%%%%%%%%%%%%%%%%%%%%%%%%%%%%%%%%%%%%

%%%% MACROS %%%%%%%%%%%%%%%%%%%%%%%%%%%%%%%%%%%%%%%%%%%%%%%

\newcommand{\PM}{\mathbf{P}}

%%%%%%%%%%%%%%%%%%%%%%%%%%%%%%%%%%%%%%%%%%%%%%%%%%%%%%%%%%%

%%%% MACROS %%%%%%%%%%%%%%%%%%%%%%%%%%%%%%%%%%%%%%%%%%%%%%%

\newcommand{\PM}{\mathbf{P}}

%%%%%%%%%%%%%%%%%%%%%%%%%%%%%%%%%%%%%%%%%%%%%%%%%%%%%%%%%%%

%%%% MACROS %%%%%%%%%%%%%%%%%%%%%%%%%%%%%%%%%%%%%%%%%%%%%%%

\newcommand{\PM}{\mathbf{P}}

%%%%%%%%%%%%%%%%%%%%%%%%%%%%%%%%%%%%%%%%%%%%%%%%%%%%%%%%%%%

%%%% MACROS %%%%%%%%%%%%%%%%%%%%%%%%%%%%%%%%%%%%%%%%%%%%%%%

\newcommand{\PM}{\mathbf{P}}

%%%%%%%%%%%%%%%%%%%%%%%%%%%%%%%%%%%%%%%%%%%%%%%%%%%%%%%%%%%

%%%% MACROS %%%%%%%%%%%%%%%%%%%%%%%%%%%%%%%%%%%%%%%%%%%%%%%

\newcommand{\PM}{\mathbf{P}}

%%%%%%%%%%%%%%%%%%%%%%%%%%%%%%%%%%%%%%%%%%%%%%%%%%%%%%%%%%%

%%%% MACROS %%%%%%%%%%%%%%%%%%%%%%%%%%%%%%%%%%%%%%%%%%%%%%%

\newcommand{\PM}{\mathbf{P}}

%%%%%%%%%%%%%%%%%%%%%%%%%%%%%%%%%%%%%%%%%%%%%%%%%%%%%%%%%%%

\sstart
\stitle{Set Operations}

\ssection{Why}

We want to consider the elements of two sets
together at once, and other sets created
from two sets.

\ssection{Definitions}

Let $A$ and $B$ be two \rt{sets}{set}.

The \ct{union}{union} of $A$ with $B$ is the
\rt{set}{set} whose \rt{elements}{element} are
in either $A$ \textit{or} $B$ \textit{or} both.
The key word in the definition is \textit{or}.

The \ct{intersection}{intersection} of $A$ with $B$ is
the set whose elements are in both $A$ \textit{and} $B$.
The keyword in the definition is \textit{and}.

Viewed as operations, both union and intersection commute;
this property justifies the language \say{with.}
The intersection is a subset of $A$, of $B$,
and of the union of $A$ with $B$.

The \ct{symmetric difference}{symmetricdifference}
of $A$ and $B$ is the set whose elements are in the union
but not in the intersection.
The symmetric difference commutes because both union and
intersection commute; this property justifies the
language \say{and.}
The symmetric difference is a subset of the union.


Let $C$ be a set containing $A$.
The \ct{complement}{complement} of $A$ in $C$ is
the symmetric difference of $A$ and $C$.
Since $A \subset C$, the union is $C$ and the
intersection is $A$.
So the complement is the \say{left-over} elements of $B$ after removing the elements of $A$.

We call these four operations
\ct{set-algebraic operations}{setalgebraicoperations}.

\ssubsection{Notation}

Let $A, B$ be sets.
We denote the union of $A$ with $B$ by $A \union B$, read aloud as \say{A union B.}
$\union$ is a stylized U.
We denote the intersection of $A$ with $B$ by $A \intersect B$, read aloud as \say{A intersect B.}
We denote the symmetric difference of $A$ and $B$ by $A \symdiff B$, read aloud as \say{A symdiff B.}
\say{Delta} is a mnemonic for difference.

Let $C$ be a set containing $A$.
We denote the complement of $A$ in $C$ by $C - A$, read aloud as \say{C minus A.}

\ssubsection{Results}

\begin{prop}
  For all sets $A$ and $B$ the operations $\union$, $\intersect$, and $\symdiff$ commute.
\end{prop}

\begin{prop}
  Let $S$ a set.
  For all sets $A, B \subset S$,
  \[
    \begin{aligned}
      \text{(1)} \quad & S - (A \union B) = (S - A) \intersect (S - B) \\
      \text{(2)} \quad & S - (A \intersect B) = (S - A) \union (S - B).
    \end{aligned}
  \]
\end{prop}

\begin{prop}
  Let $S$ a set. For all sets $A, B \subset S$,
  \[
    A \symdiff B = (A \union B) \intersect C_S(A \intersect B)
  \]
  \boxed{TODO: notation}
\end{prop}
\strats
