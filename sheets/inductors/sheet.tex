%!name:inductors
%!need:functions

\ssection{Why}

We want to talk about learning
associations between perceptions
in time or space.

\ssection{Definition}

Consider two sets
and a relation between them.
Consider a finite sequence of elements
from their product.
An \ct{inductor}{} is a function
from the set of finite sequences of
elements from the product of the two
sets to to the set of functions from
the first set to the second set.

We call the first set the
\ct{precepts}{} and the second
set the
\ct{postcepts}{}.
We call the finite sequence of
elements the \ct{record sequence}{}
and an element of it a \ct{record}{}.
We call a function from the precepts
to the postcepts a \ct{predictor}{}.
We call the result of a precept
under a predictor a \ct{prediction}{}.

We interpret the \ct{inductor}{} as
association to observations of the
association between the precepts
and postcepts a function encoding
their relation. There may be no
functional relation between these
sets.

We interpret the first set
as the \ct{precepts}{}
and the second set as the
\ct{postcepts}{}. The relation
is the prior knowledge about
which precepts may be related to
postcepts.
We interpret the function as a
predictor of the



We have two sets and a relation
between them.
We have a finite sequence of
elements from the product of
the two sets.
We will encounter a sequence of
elements from the first set
and want to produce elements of
the second set.
an element
of the first set
Using the records we want to
associate
We want to associate a func

We call the first set the
\ct{precepts}{} and the second the
\ct{postcepts}{}. We cal the relation
a \ct{prelation}{}.
We call the sequence of elements
the \ct{record sequence}{}.

We want to construct a
: the
first is called
Let $A$ be a set and $B$
be a set. Let $R$ be a relation
on $A$ and $B$
Let $\CU$ be a set
and $\CV$ be a set.
We also have a relation $R$
Our first perception is an
element of the precepts and
our second perception is an
element of the postcepts.
The prelation dictates which
postcepts can follow precepts.
We call $\CU$ the
\ct{precepts}{}
and $\CV$ the \ct{postcepts}{}.
A \ct{prelation}{} is
a relation between precepts
and postcepts.


The prelation may be
complete, in that any two precepts
and postcepts may be related.
Or the prelation may
be functional, in that any given
precept is related to a particular
postcept.
Or a precept may be related
to several postcepts.

If the prelation is complete, we say
call it \ct{unpresumptive}{}.
If the prelation is functional, we call
it \ct{presumptive}{}.
