
%!name:entire_functions
%!need:complex_analytic_functions
%!refs:yellow/IX/4

\section*{Definition}

An \t{entire function} is a complex function $f: \C  \to \C $ which is analytic for all $z \in \C $.

\blankpage
\sbasic
%%%% MACROS %%%%%%%%%%%%%%%%%%%%%%%%%%%%%%%%%%%%%%%%%%%%%%%

\newcommand{\PM}{\mathbf{P}}

%%%%%%%%%%%%%%%%%%%%%%%%%%%%%%%%%%%%%%%%%%%%%%%%%%%%%%%%%%%

%%%% MACROS %%%%%%%%%%%%%%%%%%%%%%%%%%%%%%%%%%%%%%%%%%%%%%%

\newcommand{\PM}{\mathbf{P}}

%%%%%%%%%%%%%%%%%%%%%%%%%%%%%%%%%%%%%%%%%%%%%%%%%%%%%%%%%%%

%%%% MACROS %%%%%%%%%%%%%%%%%%%%%%%%%%%%%%%%%%%%%%%%%%%%%%%

\newcommand{\PM}{\mathbf{P}}

%%%%%%%%%%%%%%%%%%%%%%%%%%%%%%%%%%%%%%%%%%%%%%%%%%%%%%%%%%%

%%%% MACROS %%%%%%%%%%%%%%%%%%%%%%%%%%%%%%%%%%%%%%%%%%%%%%%

% use \set{stuff} for { stuff }
% use \set* for autosizing delimiters
\DeclarePairedDelimiter{\set}{\{}{\}}

% use \Set{a}{b} for {a | b}
% use \Set* for autosizing delimiters
\DeclarePairedDelimiterX{\Set}[2]{\{}{\}}{#1 \nonscript\;\delimsize\vert\nonscript\; #2}

% use \powerset{A} for power set of A
\newcommand{\powerset}[1]{2^{#1}}

\renewcommand{\emptyset}{\varnothing}

\newcommand{\SA}{\mathcal{A}}
\newcommand{\SB}{\mathcal{B}}
\newcommand{\SC}{\mathcal{C}}
\newcommand{\SD}{\mathcal{D}}
\newcommand{\SE}{\mathcal{E}}
\newcommand{\SF}{\mathcal{F}}
\newcommand{\SG}{\mathcal{G}}
\newcommand{\SH}{\mathcal{H}}
\newcommand{\SI}{\mathcal{I}}
\newcommand{\SJ}{\mathcal{J}}
\newcommand{\SK}{\mathcal{K}}
\newcommand{\SL}{\mathcal{L}}

%%%%%%%%%%%%%%%%%%%%%%%%%%%%%%%%%%%%%%%%%%%%%%%%%%%%%%%%%%%

%%%% MACROS %%%%%%%%%%%%%%%%%%%%%%%%%%%%%%%%%%%%%%%%%%%%%%%

\newcommand{\PM}{\mathbf{P}}

%%%%%%%%%%%%%%%%%%%%%%%%%%%%%%%%%%%%%%%%%%%%%%%%%%%%%%%%%%%

%%%% MACROS %%%%%%%%%%%%%%%%%%%%%%%%%%%%%%%%%%%%%%%%%%%%%%%

\newcommand{\PM}{\mathbf{P}}

%%%%%%%%%%%%%%%%%%%%%%%%%%%%%%%%%%%%%%%%%%%%%%%%%%%%%%%%%%%

\sstart
\stitle{Inductors}

\ssection{Why}

We want to talk about learning
associations between perceptions
in time or space.

\ssection{Overview}

Consider two sets and
a finite sequence of elements from
their product.
An \ct{inductor}{} is a function
between
finite sequences in their product
and the functions between them.

We call the first set the
\ct{precepts}{}, the
second the \ct{postcepts}{},
and their product the \ct{percepts}{}.
We call the sequence the \ct{records}{}.
An inductor produces a function from
precepts to postcepts, which we call
a \ct{predictor}{}.

\subsection{Notation}

Let $\CU$ be precepts and $\CV$ be postcepts.
Then $\CU \times \CV$ are the percepts.
A record sequence is an element of
$(\CU \times \CV)^n$.
We often denote an element by
$((u^1,v^1), \dots, (u^n, v^n))$.

Consider a second sequence of records.
A predictor
One inductor
We judge an inductor on this sequence
We encou
We judge inductors by their

The \ct{inductor}{} associates
predictors to records.
We want to produce predictors
which

We want to produce predictors
to produce
association to observations of the
association between the precepts
and postcepts a function encoding
their relation. There may be no
functional relation between these
sets.

We interpret the first set
as the \ct{precepts}{}
and the second set as the
\ct{postcepts}{}. The relation
is the prior knowledge about
which precepts may be related to
postcepts.
We interpret the function as a
predictor of the



We have two sets and a relation
between them.
We have a finite sequence of
elements from the product of
the two sets.
We will encounter a sequence of
elements from the first set
and want to produce elements of
the second set.
an element
of the first set
Using the records we want to
associate
We want to associate a func

We call the first set the
\ct{precepts}{} and the second the
\ct{postcepts}{}. We cal the relation
a \ct{prelation}{}.
We call the sequence of elements
the \ct{record sequence}{}.

We want to construct a
: the
first is called
Let $A$ be a set and $B$
be a set. Let $R$ be a relation
on $A$ and $B$
Let $\CU$ be a set
and $\CV$ be a set.
We also have a relation $R$
Our first perception is an
element of the precepts and
our second perception is an
element of the postcepts.
The prelation dictates which
postcepts can follow precepts.
We call $\CU$ the
\ct{precepts}{}
and $\CV$ the \ct{postcepts}{}.
A \ct{prelation}{} is
a relation between precepts
and postcepts.


The prelation may be
complete, in that any two precepts
and postcepts may be related.
Or the prelation may
be functional, in that any given
precept is related to a particular
postcept.
Or a precept may be related
to several postcepts.

If the prelation is complete, we say
call it \ct{unpresumptive}{}.
If the prelation is functional, we call
it \ct{presumptive}{}.
\strats
