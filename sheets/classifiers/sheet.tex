%!name:classifiers
%!need:inductors
%!need:finite_sets

\ssection{Why}

We name a predictor whose set of postcepts is finite.

\ssection{Definition}

A \t{classifer} is a predictor whose codomain is a finite set.
In the case that we call the predictor a classifier, we call the postcepts \t{classes} or \t{labels}.
We call the prediction of a classifer on a precept the \t{classification} of the precept.

We call the classifier a \t{binary classifier} if the set of labels has 2 elements.
In the case that there are $k$ labels, we call the classifier a \t{$k$-way classifier} or \t{multi-class classifier}.
This second term is used, illogically but conventionally, in contrast to binary classification.

Following this terminology, but speaking of processes, some authors refer to the application of inductors for these special cases as \t{binary classification} and \t{multi-class classification}.

\ssection{Example: binary classification}

Let $A$ be a set of precepts (inputs) and let $B$ be a set of labels (postcepts, outputs).
Suppose $B = \set{0, 1}$, so that, in particular $B$ is finite.
Then $f: A \to B$ is a binary classifier with labels $0$ and $1$.

\ssection{Example: three-way classification}

Suppose instead that $B = \set{\text{\textsc{Yes}}, \textsc{No}, \textsc{Maybe}}$
In this case, we would call $f: A \to B$ a three-way classifier.
