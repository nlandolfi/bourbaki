%!name:classifiers
%!need:inductors
% needed for the horse example, can maybe remove eventually
%!need:factorials
%!refs:sanjay_lall/introduction_to_machine_learning/classifiers u

\ssection{Why}
We name a predictor whose set of outputs is finite.

\ssection{Definition}
A \t{classifer} is a predictor whose codomain is a finite set.
In this case, we call the outputs \t{classes} (or \t{labels}, \t{categories}, \t{label set}).
We call the prediction of a classifier on an input a \t{classification}.

If the set of labels has two elements, then we call the classifier a \t{binary classifier} (or \t{two-way classifier}, \t{two-class classifier}, \t{boolean classifier}).
In the case that there are $k$ labels, we call the classifier a \t{$k$-way classifier} (or \t{$k$-class classifier},\t{multi-class classifier}).
The second term is meant to indicate, not that the classifier assigns to each point several classes, but that the classification decision is made \textit{between} several classes.

\ssubsection{Basic Example}
Let $A$ be a set of inputs and let $B$ be a set of labels.
Define $B = \set{0, 1}$ (or $\set{-1,1 }$, $\set{\textsc{False}, \textsc{True}}$, $\set{\textsc{Negative}, \textsc{Positive}}$).
Then $B$ is finite with two elements and $f: A \to B$ is a binary classifier with labels $0$ and $1$.

If the case $B = \set{\text{\textsc{No}}, \textsc{Maybe}, \textsc{Yes}}$, we call $f: A \to B$ a three-way classifier.
Other examples for $B$ include a list of languages, the set of English words in some dictionary, or the set of $m!$ possible orders of $m$ horses in a race.
Often convenient to take $B = \set{1, \dots, k}$ for $k \in \mathbfsf{N}$.

\ssubsection{Other terminology}
Following our terminology, but speaking of processes, some authors refer to the application of inductors for these special cases as \t{binary classification} and \t{multi-class classification}.
Or they speak of \t{classification} and a \t{classification problems}.
Roughly speaking, a classifier \t{classifies} all inputs into categories.

Alternatively, some authors (especially in the statistics literature) refer to a classifier as a \t{discriminator} and reference \t{discrimination problems}.
