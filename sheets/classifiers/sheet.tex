%!name:classifiers
%!need:inductors
%!need:finite_sets

\ssection{Why}

We name a predictor whose set of outputs is finite.

\ssection{Definition}

A \t{classifer} is a predictor whose codomain is a finite set.
In the case that we call the predictor a classifier, we call the outputs \t{classes} or \t{labels} or \t{label set}.
We call the prediction of a classifier on an input the \t{classification} of the input.

We call the classifier a \t{binary classifier} (some authors: \t{two-class classifier}) if the set of labels has two elements.
In the case that there are $k$ labels, we call the classifier a \t{$k$-way classifier}, \t{$k$-class classifier} or \t{multi-class classifier}.
This second term is used, illogically but conventionally, in contrast to binary classification.

Let $A$ be a set of inputs and let $B$ be a set of labels (outputs).
Suppose $B = \set{0, 1}$, so that, in particular $B$ is finite.
Then $f: A \to B$ is a binary classifier with labels $0$ and $1$.
Suppose instead that $B = \set{\text{\textsc{Yes}}, \textsc{No}, \textsc{Maybe}}$
In this case, we would call $f: A \to B$ a three-way classifier.

\ssubsection{Other terminology}

Following our terminology, but speaking of processes, some authors refer to the application of inductors for these special cases as \t{binary classification} and \t{multi-class classification}.
Or they speak of \t{classification} or a \t{classification problem}.

Some authors refer to a classifier as a \t{discriminator} and reference \t{discrimination problems}.
Some authors refer to a classifier as a \t{point classifier} since it makes one guess.\footnote{Future editions may remove this. This term is used in contrast with list predictors, mentioned in subsequent sheets.}

