
\section*{Why}

Here are a few questions, which turn out to be related.

(1) What is the smallest convex set containing a given (possibly nonconvex) set $A \subset \R ^n$.

(2) Suppose we want to describe the set of points a rectangle in the plane.
It seems natural that we should be able to do so by specifying the four vertices $v_1, v_2, v_3, v_4 \in \R ^2$, and then saying, roughly that the points in of any of these vertices, or any point on a segment between these two points is in the rectangle.
Define a function $\kappa : \powerset{\R ^2} \to \powerset{\R ^2}$ by
\[
\kappa (A) = \cup_{x, y \in A} [a, b]
\]
Then
\[
\kappa (\set{v_1, v_2, v_3, v_4}) =
\cup [v_1, v_2] \cup [v_1, v_3] \cup [v_1, v_4]
\cup [v_2, v_3] \cup [v_2, v_4]
\cup [v_3, v_4]
\]
Then it is natural to say $S_1 \subset R$.
Are there points not in $S_1$ but in $R$?
Sure, all the points on the red line.
\begin{center}  \includegraphics[width=0.100\textwidth]{./graphics/convexhullrect.pdf}\end{center}
So consider a convex combination of two of them, say
\[
u_1 = v_1 + \alpha (v_2 - v_1) \quad \text{ and } \quad u_2 = v_3 + \beta (v_2 - v_3)
\]
Now pick a $\lambda  \in [0,1]$ and form
\[
\begin{aligned}
\lambda u_1 + (1-\lambda )u_2
&= \lambda (1-\alpha )v_1 + \lambda \alpha v_2 + (1-\lambda )(1-\beta )v_2 + (1-\lambda )\beta v_3 \\
&= \lambda (1-\alpha )v_1 + (\lambda \alpha  + (1-\lambda )(1-\beta ))v_2 + (1-\lambda )\beta v_3
\end{aligned}
\]
Notice that each of the coefficients on $v_1, v_2, v_3$ are nonnegative.
Moreover, notice that
\[
\lambda (1-\alpha ) + \lambda \alpha  + (1-\lambda )(1-\beta ) + (1-\lambda )\beta  = \lambda  + (1-\lambda ) = 1
\]
In other words, the points on the line segment between $u_1$ and $u_2$ are points which can be written as a sum of the vertices of the square, scaled by nonnegative coefficients which sum to one.
A natural question is: does $R = \kappa (\kappa (S))$?

\section*{Definitions}

A \t{convex combination} is a linear combination whose sequence of scalars is nonnegative and sums to one.
Given any set $A \subset \R ^n$, the convex hull of $A$ is
And
What is the smallest convex
Suppose we want to describe a rectangle in the plane.
We can give its four verices $v_1, v_2, v_3, v_4 \in \R ^2$ and then say that the set we speak of the set of all line
segments between the two points.
Suppose we take some points in
What is the smallest convex set containing a given (possibly nonconvex) set?
Alternatively, given some points $x_1, \dots , x_n \in \R ^d$, can
        \section*{Definition}

We have seen that convex sets are closed under intersection.
Clearly $\R ^n$ is a convex set.
So the set of convex sets in $\R ^n$ containing $A \subset \R ^n$ has at least one member.
The \t{convex hull} (or \t{real convex hull}) of a set $A \subset \R ^n$ is the intersection of all convex sets containing the set.
In other words, it is the smallest convex set containing $A$.

        \subsection*{Notation}

We denote the convex hull of $S \subset \R ^n$ by $\conv S$.

        \section*{Characterization}

\begin{proposition}
Let $S \subset \R ^n$. $\conv S$ is the set of all convex combinations of elements of $S$.
\end{proposition}

\begin{proposition}
The convex hull of $\set{b_1, \dots , b_m} \subset \R ^n$ consists of all vectors
\[
\lambda _1b_1 + \lambda _2b_2 + \cdots + \lambda _mb_m.
\]
where $\lambda _i \geq 0$ and $\sum_{i}\lambda _i = 1$.
\end{proposition}

\blankpage


%macros.tex
%\newcommand{\conv}{\mathword{conv}}
