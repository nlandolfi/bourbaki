
\section*{Why}

We want numbers to count with.\footnote{Future editions will expand on this sheet with a more justified why.}
%  TODO: better why.
%  We want to count, forever.
%  


\section*{Definition}

The \t{successor} of a set is the set which is the union of the set with the singleton of the set.
In other words, the successor of a set $A$ is $A \cup \set{A}$.
This definition is primarily of interest for the particular sets introduced here.

These sets are the following (and their successors):
We call the empty set \t{zero}.\footnote{In future editions, zero may be a separate sheet.}
We call the successor of the empty set \t{one}.
In other words, one is $\varnothing \cup \set{\varnothing} = \set{\varnothing}$.
We call the successor of one \t{two}.
In other words, two is $\set{\varnothing} \cup \set{\set{\varnothing}} = \set{\varnothing, \set{\varnothing}}$.
Likewise, the successor of two we call \t{three} and the successor of three we call \t{four}.
And we continue as usual,\footnote{Future editions will assume less in the introduction of natural numbers.}
using the English language in the typical way.

A set is a \t{successor set} if it contains zero and if it contains the successor of each of its elements.

\subsection*{Notation}

Let $x$ be a set.
We denote the successor of $x$ by $\ssuc{x}$.
We defined it by
\[
\ssuc{x} \coloneqq x \cup \set{x}
\]

We denote one by $1$.
We denote two by $2$.
We denote three by $3$.
We denote four by $4$.
So
\[
\begin{aligned}
0 &= \varnothing \\
1 &= \ssuc{0} = \set{0} \\
2 &= \ssuc{1} = \set{0, 1} \\
3 &= \ssuc{2} = \set{0, 1, 2} \\
4 &= \ssuc{3} = \set{0, 1, 2, 3} \\
\end{aligned}
\]
