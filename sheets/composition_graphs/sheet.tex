%!name:composition_graphs
%!need:function_composites
%!need:directed_graphs
%!refs:bert_mendelson/introduction_to_topology/theory_of_sets

\section*{Why}

We want to visualize function composition.

\section*{Definition}

A \t{composition graph} (or \t{composition diagram}) is a directed graph along with a map from vertices to the powerset of a set and a map from edges to functions between sets associated with incident vertices.

\section*{Example}

For example, let $A$ and $R$ be sets and let $i: A \to A$, $f: A \to R$ and $g: R \to A$ be functions.
We can consider the diagram whose graph is $(\set{1, 2, 3}, \set{(1, 2), (2, 3), (1, 3)})$, with vertices one and three corresponding to $A$, vertex 2 corresponding to $R$, edge $(1, 2)$ corresponding to $f$, edge $(2, 3)$ corresponding to $g$ and edge $(1, 3)$ corresponding to $i$.
    \ifhmode\unskip\fi\footnote{
Future editions will include the highly important figures associated with function diagrams.
    }

\section*{Path composition}

The function associated with a path (or \t{path composition}) is the composition of the functions corresponding to the edges along the path.
The digram is \t{commutative} (call a \t{commutative diagram}) if the composition of any two paths between any two vertices result in identical functions.

\blankpage