\sinput{../sheet.tex}
\sbasic

\sinput{../sets/macros.tex}
\sinput{../set_operations/macros.tex}
\sinput{../measure_space/macros.tex}
\sinput{../probability_measures/macros.tex}

\sstart

\stitle{Probability Measures}

\ssection{Why}

We use the language
of measure theory
to give a mathematical model
for uncertain outcomes.
TODO: probability
intuition sheet.

\ssection{Definition}

A
\ct{probability measure}
is a finite measure on
a measurable space
which assigns the
value one to the base set.
Since a finite measure
can always be scaled to
a probability measure,
these measures are
standard examples of
finite measures.

A
\ct{probability space}{probabilityspace}
is a measure space
whose measure is a
probability measure.
The \ct{outcomes}{}
of a probability space
are the elements
of the base set.
The \ct{set of outcomes}{}
is the base set.
The \ct{events}{} are the
elements of the sigma algebra.

\ssubsection{Notation}

We denote the
set of outcomes by
$\Omega$,
a mnemonic for
\say{outcomes.}
We denote the
sigma-algebra by
$\SA$,
as usual.
We denote a probability
measure
by $\pmeas$, a mnemonic
for \say{probability.}
Thus, we often say
\say{Let
$(\Omega, \SA, \pmeas)$
be a probability space.}

\ssubsection{Properties}

\strats
