%!name:probability_measures
%!need:measures
%!need:event_probabilities

\ssection{Why}

We use the language of measure theory to give a mathematical model for uncertain outcomes.
TODO: probability intuition sheet.

\ssection{Definition}

A \t{probability measure} is a finite measure on a measurable space which assigns the value one to the base set.
A finite measure can always be scaled to a probability measure, so these measures are standard examples of finite measures.

A \t{probability space} is a measure space whose measure is a probability measure.
The word \t{space} is natural, since we developed measure theory partly as a generalization of volume in three-dimensional space.
The \t{outcomes} of a probability space are the elements of the base set.
The \t{set of outcomes} is the base set.
The \t{events} are the elements of the sigma algebra.

The measure in a probability space corresponds to the event probability function.

\ssubsection{Notation}

Let $(A, \SA)$ be a measurable space.\footnote{Often, other authors will denote the set of outcomes (here denoted by $A$) by $\Omega$, a mnemonic for \say{outcomes.}}
We denote the sigma-algebra by $\SA$, as usual.
We denote a probability measure by $\PM$, a mnemonic for \say{probability,} and intended to remind of the event probabilty function.
Thus, we often say \say{Let $(A, \SA, \PM)$ be a probability space.}

\ssubsection{Properties}
