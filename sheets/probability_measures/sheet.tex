%!name:probability_measures
%!need:measures
%!need:event_probabilities

\section*{Why}

A probability event function is a measure on the set of outcomes.

\section*{Definition}

A \t{probability measure} is a finite measure on a measurable space which assigns the value one to the base set.
A finite measure can always be scaled to a probability measure, so these measures are standard examples of finite measures.

A \t{probability space} is a measure space whose measure is a probability measure.
The word \say{space} is natural, since the notion of a measure generalized the notion of volume in real space (see \sheetref{real_space}{Real Space} and \sheetref{n-dimensional_space}{N-Dimensional Space}).
The \t{outcomes} of a probability space are the elements of the base set.
The \t{set of outcomes} is the base set.
The \t{events} are the elements of the sigma algebra.
The measure in a probability space corresponds to the event probability function.

\subsection*{Notation}

Let $(A, \mathcal{A} )$ be a measurable space.
  \ifhmode\unskip\fi\footnote{
Often, other authors will denote the set of outcomes (here denoted by $A$) by $\Omega $, an apparent mnemonic for \say{outcomes}.
  }
We denote the sigma-algebra by $\mathcal{A} $, as usual.
We denote a probability measure by $\mathbfsf{P} $, a mnemonic for \say{probability,} and intended to remind of the event probabilty function.
Thus, we often say \say{Let $(A, \mathcal{A} , \mathbfsf{P} )$ be a probability space.}

Many authors associate an event $A \in \mathcal{A} $ with a function $\pi : \mathcal{X}  \to \set{0, 1}$ so that $A = \Set*{x \in \mathcal{X} }{\pi (x) = 1}$.
In this context, it is common to write $\mu [\pi (x)]$ for $\mu (A)$.

\blankpage