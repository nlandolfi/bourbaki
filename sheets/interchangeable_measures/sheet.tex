
%!name:interchangeable_measures
%!need:real_integral_monotone_convergence
% both these are through monotone, if we keep that in the why
%%!need:measures
%%!need:real_integrals

\section*{Why}

We want to compute integrals with respect to arbitrary measures.
If we find a special measure with which we can compute integrals, would there be some way to translate integrals against another measure into integrals against this special measure?

It would suffice to produce a special function
whose integral against the base measure
would over any measurable set gave
the measure of the set under the alternate
measure.
In this case, the monotone convergence
theorem would say that the integral of
any function under the alternate
measure was the integral of the function
against the special function under the
base measure.

No small miracle is that the condition
for this interchange is extremely simple.
In this sheet, we will define the
condition. In a later sheet we will
construct the special function.

\section*{Definition}

Consider two measures.
Suppose there exists a measurable set for which the first was positive but the second was zero.
Then we have no hope of interchanging these measures, because the integral of any function over this set against the second measure is zero.

So let us rule out this case.
Suppose that whenever the first was positive on a set, the second was positive on that set.
The contrapositive is that whenever the second is zero on a set, the first is zero on the same.
This condition is sufficient for interchangeability.

A first measure is \t{interchangeable} with respect to a second measure if the first measure of any measurable set is zero whenever the second measure of that set is zero.\footnote{The language ``interchangeable'' is not at all standard. We have coined it for the reasons outlined in the why of this sheet.}
As we said earlier, this condition means that the first measure is never nonzero on a set for which the second measure is zero.
It is common to say that the first measure is \t{absolutely continuous} with respect to the second measure.\footnote{This needs more justification, perhaps, as this naming seems to be an artifact of a notion of absolutely continuous functions.}

\subsection*{Notation}

Let $(X, \mathcal{A} )$ be a measurable space.
Let $\mu $ and $\nu $ be two measures on $\mathcal{A} $.
Then $\nu $ is interchangeable with respect to $\mu $ if for all measurable $A \in \mathcal{A} $, $\mu (A) = 0 \implies \nu (A) = 0$.
Equivalently, $\nu (A) > 0 \implies \mu (A) > 0$ for all $A \in \mathcal{A} $.
Notice that in this second statement, instead of the relation of equality ($=$) we are using the relation of inequality ($<$).
% We need not consider negative real numbers since $\mu$ is a non-negative real-valued function. 


There is often used notation for a measure $\mu $ being interchangeable (absolutely continuous) with respect to a second measure $\nu $.
It is $\mu  \ll \nu $.
The notation is a reminder of the fact that $\nu (A) = 0 \implies \mu (A) = 0$ for all $A \in \mathcal{A} $.
