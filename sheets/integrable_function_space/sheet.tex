
%!name:integrable_function_space
%!need:extended_real_numbers
%!need:real_integrals
%!need:complex_integrals

\section*{Why}

The integrable functions are a vector space.

\section*{Definition}

The \t{interable function space} corresponding to a measure space is the set of real-valued functions which are integrable with respect to the measure.
The term space is appropriate because this set is a real vector space.
If we scale an integrable function, it remains integrable.
If we add two integrable functions, the sum is integrable.
Thus, a linear combination of integrable functions is integrable.
The zero function is the zero element of the vector space.

% TODO 

The open question is: what elements of our geometric intuition can we bring to a space of functions.
Do functions have a size?
Are certain functions near each other?

\subsection*{Notation}

Let $(X, \mathcal{A} , \mu )$ be a measure space.

We denote set the real-valued integrable functions on $X$ by $\mathcal{I} (X, \mathcal{A} , \mu , \R )$, read aloud as ``the real integrable functions on the measure space X script A mu.''
We denote set the complex-valued integrable functions on $X$ by $\mathcal{I} (X, \mathcal{A} , \mu , \C )$, read aloud as ``the complex integrable functions on the measure space X script A mu.''
When the field is irrelevant, we denote them by $\mathcal{I} (X, \mathcal{A} , \mu )$, read aloud as ``integrable functions on the measure space X script A mu.''
The $\mathcal{I} $ is a mnemonic for ``integrable.''
