
\section*{Why}

Given an outcome variable $X: \Omega  \to V$, and some probabilities on a sample space $\Omega $, there is a natural set of probabilities to associate with outcomes in $V$.

\section*{Result}

Suppose $X: \Omega  \to V$ is a random variable on a finite sample space $\Omega $.
Given $p: \Omega  \to \R $ is a probability distribution inducing probability measure $P: \pow{\Omega } \to \R $, define the function $q: V \to \R $ by
\[
q(x) = P(X = x)
\]
It is easy to verify that $q$ is nonnegative and normalized.
The latter fact follows from the observation that the sets $\set{X^{-1}(x)}{x \in V}$ partition the set $\Omega $.

The function $q$ is sometimes called the \t{distribution of $X$} (the \t{induced distribution}, \t{induced probability distribution}) of the random variable $X$.

\subsection*{Notation}

Given a random variable $X: \Omega  \to V$, and some distribution $p: \Omega  \to \R $, it is common to denote the induced distribution by $p_X$.
Given a distribution $q: V \to \R $ and a random variable $X: \Omega  \to V$ it is common to see the notation $X \sim q$ as an abbreviation of the sentence ``The random variable $X$ has distribution $q$.

\subsection*{Computation}

Suppose $X: \Omega  \to V$ is a random variable and $f: V \to U$.
Define $Y: \Omega  \to V$ by $y(\omega ) \equiv f(x(\omega ))$ for every $\omega  \in \Omega $.
We frequently denote this by $Y = f(X)$.
$Y$ is a random variable with induced distribution $p_{Y}: \Omega  \to \R $ satisfying
\[
\textstyle
p_{Y}(y) = \sum_{\omega  \in \Omega  \mid  Y(\omega ) = y} p(\omega ) = \sum_{x \in V \mid  f(x) = b} p_X(x).
\]
Consequently, as a matter of practical computation, we can evaluate probabilities having to do with the outcome variable $X$ using $p_X$ instead of $p$ and same with $Y$.\footnote{Future editions will give an example.}

\blankpage