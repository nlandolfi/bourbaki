
%!name:peano_axioms
%!need:natural_induction
%!refs:bert_mendelson/introduction_to_topology/theory_of_sets/introduction

\section*{Why}

Historically considered a fountainhead for all of mathematics.

\section*{Discussion}

So far we know that $\omega $ is the unique smallest successor set.
In other words, we know that $0 \in \omega $, $n \in \omega \implies \ssuc{n} \in \omega $ and that if these two properties hold of some $S \subset \omega $, then $S = \omega $.
We can add two important statements to this list.
First, that 0 is the successor of no number.
In other words, $n^+ \neq 0$ for all $n \in \omega $.
Second, that if two numbers have the same successor, then they are the same number
In other words, $\ssuc{n} = \ssuc{m} \implies n = m$

These five properties were historically considered the fountainhead of all of mathematics.
One by the name of Peano used them to show the elementary properties of arithmetic.
They are:
    \begin{enumerate}
      \item $0 \in \omega $.
      \item $n \in \omega  \implies n^+ \in \omega $ for all $n \in \omega $.
      \item If $S$ is a successor set contained in $\omega $, then $S = \omega $.
      \item $\ssuc{n} \neq 0$ for all $n \in \omega $
      \item $\ssuc{n} = \ssuc{m} \implies n = m$ for all $n, m \in \omega $.
    \end{enumerate}

These are collectively known as the \t{Peano axioms}.
Recall that the third statement in this list is the \t{principle of mathematical induction}.

\section*{Statements}

Here are the statements.

\begin{proposition}[Peano’s First Axiom]
$0 \in \omega$.
\end{proposition}

\begin{proposition}[Peano’s Second Axiom]
$n \in \omega \implies n^+ \in \omega$.
\end{proposition}

\begin{proposition}[Peano’s Third Axiom]
Suppose $S \subset \omega$, $0 \in S$, and $(n \in S \implies n^+ \in S$.
Then $S = \omega$.
\end{proposition}

\begin{proposition}[Peano’s Fourth Axiom]
$n^+ \neq 0$ for all $n \in \omega$.
\end{proposition}

The last one uses the following two useful facts.

\begin{proposition}
$x \in n \implies n \not\subset x$.
\end{proposition}

\begin{proposition}
$(x \in y \land y \in n) \implies x \in n$
\end{proposition}

This latter proposition is sometimes described by saying that $n$ is a \t{transitive set}.
This notion of transitivity is not the same as that described in \sheetref{relations}{Relations}.
Using these one can show:

\begin{proposition}[Peano’s Fifth Axiom]
Suppose $n, m \in \omega$ with $n^+ = m^+$.
Then $n = m$.
\end{proposition}
