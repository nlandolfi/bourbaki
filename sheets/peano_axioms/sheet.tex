%!name:peano_axioms
%!need:natural_numbers

\ssection{Why}

Historically considered a fountainhead for all of mathematics.

\ssection{Discussion}

Our discussion of sets has suceeded in making Peano's axioms provable propositions.

\begin{enumerate}
  \item $0 \in \omega$.
  \item $n \in \omega \implies n^+ \in \omega$.
  \item Suppose $S \subset \omega$, $0 \in S$, and $(n \in S \implies n^+ \in S$.
  Then $S = \omega$.
\end{enumerate}

\begin{proposition}[Peano's First Axiom]
  $0 \in \omega$.
\end{proposition}

\begin{proposition}[Peano's Second Axiom]
  $n \in \omega \implies n^+ \in \omega$.
\end{proposition}

\begin{proposition}[Peano's Third Axiom]
  Suppose $S \subset \omega$, $0 \in S$, and $(n \in S \implies n^+ \in S$.
  Then $S = \omega$.
\end{proposition}


Peano's third axiom is called the \t{principle of mathematical induction}.

\begin{proposition}[Peano's Fourth Axiom]
  $n^+ \neq 0$ for all $n \in \omega$.
\end{proposition}

\begin{proposition}[Peano's Fifth Axiom]
  Suppose $n, m \in \omega$ with $n^+ = m^+$.
  Then $n = m$.
\end{proposition}
