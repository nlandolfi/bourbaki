%!name:matrices
%!need:vectors

\ssection{Why}


\ssection{Definition}



Consider two sets: the natural numbers from
$1$ to $n$ and those from $1$ to $m$.
Consider a third non-empty set.
A \ct{matrix of}{} elements of the third set is a function from the cartesian product of the first two sets of natural numbers to the third set.
We call such a function a \ct{matrix}{}

We think of the objects in the third set as arrayed in a grid or arrayed in a table.
We call $n$ and $m$ the \ct{dimensions}{} of the matrix.
We call $n$ the \ct{height}{} and $m$ the \ct{width}{}.
If the height of the matrix is the same as the width of the matrix then we call the matrix \ct{square}{}.
If the height is larger than the width, we call the matrix \ct{tall}{}.
If the width is larger than the height, we call the matrix \ct{wide}{}.

\ssubsection{Notation}

Let $S$ be non-empty set.
We denote the set $n \times m$ over
the set $S$ by -valued
matrices by $\Mat{S}{n}{m}$.
We often denote matrices by upper-case
latin numbers.
Let $A \in \Mat{S}{n}{m}$.
This means the same as
$A: \set{1, \dots, n} \times \set{1, \dots m} \to S$.
We denote $A(i, j)$ by $A_{ij}$.
