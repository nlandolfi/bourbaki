
% WHy? probably fractions.. 

\section*{Rational equivalence}

Consider $\Z  \times  (\Z  - \set{0_{\Z }})$.
We say that the elements $(a, b)$ and $(c, d)$ of this set are \t{rational equivalent} if $ad = bc$.
Briefly, the intuition is that $(a, b)$ represents $a$ over $b$
In the usual notation, $(a, b)$ represents ``$a/b$''.
So this equivalence relation says these two are the same if $a/b = c/d$ or else $ad= bc$.

\begin{proposition}
Rational equivalence is an equivalence relation on $\Z \times (\Z  - \set{0_{\Z }})$.\footnote{Future editions will include an account.}

\end{proposition}

\section*{Definition}

The \t{set of rational numbers} is the set of equivalence classes (see \sheetref{equivalence_classes}{Equivalence Classes}) of $\Z  \times (\Z  - \set{0_{\Z }})$ under ratioanl equivalence.
We call an element of the set of rational numbers a \t{rational number} or \t{rational}.
We call the set of rational numbers the \t{set of rationals} or \t{rationals} for short.

\section*{Notation}

We denote the set of rationals by $\Q $.\footnote{From what we can tell, $\Q $ is a mnemonic for ``quantity'', from the latin ``quantitas.''
It may also be a mnemonic for quotient.}
If we denote rational equivalence by $\sim$ then $\Q  = (\Z \times (\Z  - \set{0_{\Z }}))/\sim$.

\blankpage
%macros.tex
%\newcommand{\Q}{\mathbfsf{Q}}
