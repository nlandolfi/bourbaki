%!name:rational_numbers
%!need:integer_numbers
%!need:integer_arithmetic

\ssection{Why}

We want fractions.\footnote{This why will be expanded in future editions.}

\ssection{Definition}

Consider $\Z \times (\Z - \set{0_{\Z}})$.
We say that the elements $(a, b)$ and $(c, d)$ of this set are \t{rational equivalent} if $ad = bc$.
Briefly, the intuition is that $(a, b)$ represents a over b, or in the usual notation \say{$a/b$}.
So this equivalence relation says these two are the same if $a/b = c/d$ or else $ad= bc$.

\begin{proposition}
  Rational equivalence is an equivalence relation on $\Z \times (\Z - \set{0_{\Z}})$.
\end{proposition}

We define the \t{set of rational numbers} to be the set of equivalence classes (see \sheetref{equivalence_classes}{Equivalence Classes}) under ratioanl equivalence on $\Z \times (\Z - \set{0_{\Z}})$.
We call an element of the set of ratioanl numbers a \t{rational number} or \t{rational}.
We call the set of rational numbers the \t{set of rationals} or \t{rationals} for short.

\ssection{Notation}

We denote the set of rationals by $\Q$.\footnote{From what we can tell so far, $\Q$ is a mnemonic for \say{quantity}, from the latin \say{quantitas}.}
If we denote rational equivalence by $\sim$ then $\Q = (\Z \times (\Z - \set{0_{\Z}}))/\sim$.


\blankpage
