%!name:ordered_undirected_graphs
%!need:undirected_graphs
%!need:sequences
%!refs:vandenberghe2014chordal

\ssection{Why}\footnote{Future editions will include. This sheet is needed, for example, in discussing perfect elimination orderings.}

\ssection{Definition}

An \t{ordering} of an undirected graph is an ordering (see \sheetref{sequences}{Sequences}) of its vertices.
An \t{ordered undirected graph} is an ordered pair $((V, E), \sigma: \upto{\num{V}} \to V)$ where $(V, E)$ is an undirected graph (see \sheetref{undirected_graphs}{Undirected Graphs}) and $\sigma$ is an ordering of the vertex set $V$.

\ssection{Notation}

Let $((V, E), \sigma)$ be an ordered undirected graph.
We commonly associate it with $(V, E, \sigma)$ and call this ordered triple an undirected graph as well.
But, throughout these sheets, an ordered undirected graph is an ordered pair.

We denote that $\inv{\sigma}(v) < \inv{\sigma}(w)$ by $v \prec_{\sigma} w$ and $v \preceq_{\sigma} w$ by $\inv{\sigma}(v) \leq \inv{\sigma}(w)$.
We occasionally omit the subscripts in $\prec_{\sigma}$ and $\preceq_{\sigma}$ when clear from context.


\ssection{Visualization}

We visualize an ordered undirected graph by labeling its nodes with the indices of each vertex.
Let $(V, E)$ be an undirected graph with $V = \set{a,b,c,d,e}$ and
\[
  E = \set*{\set{a, b}, \set{a, c}, \set{a, e}, \set{b, d}, \set{b, e}, \set{c, d}, \set{c, e}, \set{d,e}}.
\]
Let $\sigma: \set{1, \dots, 5} \to V$ be an ordering with
\[
  \sigma(1) = a \quad  \sigma(2) = c \quad \sigma(3) = d \quad \sigma(4) = b \quad \sigma(e) = 5.
\]
We visualize the ordered graph in the figure.

\begin{figure}
  \centering
  \includegraphics[width=0.3\textwidth]{graphics_included/ordered_undirected_graph}
  \caption{Ordered undirected graph.}
\end{figure}
