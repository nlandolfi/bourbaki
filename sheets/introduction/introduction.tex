
%!name:entire_functions
%!need:complex_analytic_functions
%!refs:yellow/IX/4

\section*{Definition}

An \t{entire function} is a complex function $f: \C  \to \C $ which is analytic for all $z \in \C $.

\blankpage
\sbasic
%%%% MACROS %%%%%%%%%%%%%%%%%%%%%%%%%%%%%%%%%%%%%%%%%%%%%%%

\newcommand{\PM}{\mathbf{P}}

%%%%%%%%%%%%%%%%%%%%%%%%%%%%%%%%%%%%%%%%%%%%%%%%%%%%%%%%%%%

\sstart
\stitle{Introduction}

The Bourbaki Project is a directed acyclic graph of
mathematical concept sheets.
A \ct{sheet}{} is a two-page document of
concepts, terms, results and notation.
The graph is defined implicitly by
the dependencies between concepts, terms,
results and notation introduced in the sheets.

The graph is defined in such a way as to
order the sheets for an unacquainted reader.
Suppose the reader wants to understand the
mathematical concept of function, defined
in the sheet \say{Functions}.
The concept of function uses the concept of a
relation,
defined in the sheet \say{Relations}.
The concept of relation uses the
concept of ordered pairs, defined in
the seet \say{Ordered Pairs}.
An so on.
To understand functions, we must, therefore
understand ordered pairs.

to read sheet
$A$, but sheet $A$ uses
uses concepts, terms,
notation or results introduced in sheet $B$.
To understand $A$, the reader should first read $B$.
If, further, $B$ uses concepts from $C$, the reader
must first read $C$.
Then go on to read $B$, and finally read $A$.
The unacquainted reader must read all
sheets introduction
So, the sheets a use needs to read
The idea
Needs should be transitive: if $A$ needs
$B$ and $B$ needs $C$, then $A$ needs $C$.


The Bourbaki graph is the minimal graph
whose transitive closure gives all
needs: in other words, if $A$ needs $B$
and $B$ needs $C$, then the Bourbaki
graph include edges from $A$ to $B$ and
$B$ to $C$, and not from $A$ to $C$ since
this edge is implied by transitivity.

\begin{quote}
We must not forget that the
modern digital computer and
the screen are the \textit{envy} of
every scholar in every age
before ours. Let us not rely
so much
%more so
on our cleverness
than on our prudence in doing
what they would have in our shoes.
\end{quote}

\begin{quote}
Yes, there are several things
which different people call
least squares. And with good
reason. But let us decide on
what we mean by least squares
and so know what we are talking
about.
\end{quote}

These sheets contain nothing
but fiction. And yet, to the best
of my knowledge, everything
is true.

\strats
