%!name:introduction

The Bourbaki Project is a directed acyclic graph of
mathematical concept sheets.
A \ct{sheet}{} is a two-page document of
concepts, terms, results and notation.
The graph represents dependencies between concepts,
terms, results and notation introduced in sheets.

The graph is defined in such a way as to
order the sheets for an unacquainted reader.
Suppose the reader wants to understand the
mathematical concept of function, defined
in the sheet \textit{Functions}.
The concept of function uses the concept of a
relation,
defined in the sheet \textit{Relations}.
In turn, the concept of relation uses the
concept of ordered pairs, defined in
the sheet \textit{Ordered Pairs}.
And so on.
To understand functions, the reader must
first understand ordered pairs.

We say that the sheet \textit{Functions}
\textbf{needs} the sheet \textit{Relations}.
In this sense, it needs all concepts,
terms, notation or results in the
\textit{Relations} sheet.
Likewise the sheet \textit{Relations}
needs the sheet \textit{Ordered Pairs}.
Of course, in a second sense, the sheet
\textit{Functions} needs the sheet
\textit{Ordered Pairs},
since any term, concept,
result, or notation defined in
\textit{Ordered Pairs} may appear in
\textit{Functions}.

Naturally, needs are based not only
on terms, but also by results and notation,
and in the broadest sense, by things that are said in the
needed sheet.
\clearpage

To say more is worth less than the words it requires;
the project is best explored.

\begin{quote}
We must not forget that the
modern digital computer and
the screen are the \textit{envy} of
every scholar in every age
before ours. Let us not rely
so much
%more so
on our prudence in doing
as they would.
\end{quote}

\begin{quote}
Yes, there are several things
which different people call
least squares. And with good
reason. But let us decide on
what we mean by least squares
and so know what we are talking
about.
\end{quote}

These sheets contain nothing
but fiction. And yet, to the best
of my knowledge, everything
is true.

The Bourbaki Project is supported by
funds from the Department of
Defense of the United States of America
and by Stanford University.
