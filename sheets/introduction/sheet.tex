%!name:introduction

The Bourbaki Project is a collection
of documents describing mathematical
concepts, terminology, results and
notation. Each document is named and
labeled with the names of those documents
that should be read before it by the
unaquainted reader.

Our primary aim is to understand and
develop the mathematical concepts ourselves.
A natural (and old) endeavor.
Besides this, we have two auxiliary aims.
First, to provide writings to teach
the concepts to an unaquainted reader. In other
words, to provide useful exposition.
And second, to serve as a reference for
further work.

We call these documents \ct{sheets}{}.
They are only ever two-pages long, and
sometimes shorter.
They can be printed on the front
and back of a single
sheet of a paper, hence the name sheet.
The decision to cap at two pages is
arbitrary.
But our
experience suggests it is convenient.

We call the sheets that should
be read before a particular sheet
$X$ the \ct{needs}{} of $X$.
For example, the sheet \textit{Relations}
needs the sheet \textit{Ordered Pairs}.
The reason, in this case, is that the
concept of a relation is discussed using
the concept of an ordered pair of objects.
And since the phrase
\say{ordered pair of objects} makes sense
only if we know what is meant by object
(discussed in the sheet \textit{Objects}),
the sheet \textit{Relations} needs the sheet
\textit{Objects} also.
The reader unaquainted with
ordered pairs and
objects must read (at least) these two
sheets before the sheet on relations.

%For convenience, we only list
%the needs of a sheet which are not
%the needs of some other sheet
%already listed.
%This results in a unique list because
%we can always start with all sheets
%possibly needed, and then remove
%sheets needed by others until we can
%remove no more.
%The \ct{immediate needs}{} of
%Relations is only Ordered Pairs.
%To determine all needs of a sheet,
%we have look at the immediate needs of the
%sheet, and the immediate needs of those sheets
%and so on.


%
% \begin{quote}
% The modern digital computer and
% screen may well be cause for
% the \textit{envy} of
% scholars in ages past.
% Let us not disappoint.
% \end{quote}
%
% \begin{quote}
% Yes, there are several things
% which different people call
% least squares. And with good
% reason. But let us decide on
% what we mean by least squares
% and so know what we are talking
% about.
% \end{quote}

The sheets and their needs are probably
best explored;
the index is
a reasonable starting point.
Bourbaki is like a map: it gives
a route to different concepts.
The path is, of course, not physical.
But it is just one map, there are
undoubtedly many, resulting from
equivalent concepts, different proofs,
and so on.

We conclude with wink and add that these
sheets contain only fiction.
They contain only ideas.
That is not to say, however, that
(to the best of our knowledge)
everything is true.
But this caveat is more important
than the last.

% These sheets contain nothing
% but fiction. And yet, to the best
% of our knowledge, everything
% is true.

% To say more would be, in our
% estimation,
% worth less than the reader's
% time spent inspecting the sheets.
% So we will conclude.

\vfill

\begin{center}{\small The project is supported by
funds from the Department of
Defense of the United States of America
and Stanford University.}\end{center}
