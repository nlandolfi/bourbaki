%!name:introduction

The Bourbaki Project is a collection
of documents describing mathematical
concepts, terminology, results and
notation. Each document is named and
labeled with the names of those documents
that should be read before it by the
unaquainted reader.

\textbf{Aims.}
Our primary aim is to understand and
develop the mathematical concepts ourselves.
Besides this, we have two other goals.
First, to provide useful exposition to teach
the concepts to an unaquainted reader.
And second, to serve as a reference for
further work.

\textbf{Sheets.}
We call these documents \ct{sheets}{}.
They are only ever two-pages long, and
sometimes shorter.
They can be printed on the front
and back of a single
sheet of a paper, hence the name sheet.
The decision to cap at two pages is
arbitrary.
But our
experience suggests it is convenient.


\textbf{Needs.}
We call the sheets that should
be read before a particular sheet
$X$ the \ct{needs}{} of $X$.
For example, the sheet \textit{Relations}
needs the sheet \textit{Ordered Pairs}.
The reason, in this case, is that the
concept of a relation is discussed using
the concept of an ordered pair of objects.
And since the phrase
\say{ordered pair of objects} makes sense
only if we know what is meant by object
(discussed in the sheet \textit{Objects}),
the sheet \textit{Relations} needs the sheet
\textit{Objects} also.
The reader unaquainted with
ordered pairs and
objects must read (at least) these two
sheets before the sheet on relations.
So needs order the sheets to be read.

The needs of a sheet are naturally
ordered by their respective needs.
Suppose $X$ needs both $Y$ and $Z$,
and
 $Y$ in turn needs $Z$.
In this case,
$Z$ ought to
be read first, $Y$ second, and $X$ last.
We ensure that such an ordering
always exists by enforcing
the following constraint:
if a
sheet $X$ needs a sheet $Y$,
then $Y$ can not need $X$ or any sheet
that needs $X$.

For convenience, the needs listed on
a page are minimal. That is to say,
for sheet $X$ we only list the sheets
which are in the needs of $X$
and not needed by any other sheet
in the needs of $X$.
If $X$, $Y$ and $Z$ are as before,
then we only list $Y$ as $X$'s needs
because $Z$ is implicit (through $Y$).
The sheets and their needs are probably
best explored by browsing the project;
the index is
a reasonable starting point.

\textbf{Caveats.}
There are two caveats.
First, Bourbaki gives
only one path to concepts.
Bourbaki is like a map:
the landmarks are concepts.
Walking is reading.
And you must walk along the
trails specified by the needs.
The point is that the Bourbaki way
of structuring the concepts is just
one way, and there are many ways, since
there are equivalent concepts,
alternate proofs, and so on.
The second caveat is a wink:
these sheets are fiction.
They contain only ideas.
We have done our
best to eliminate all false statements.
But very little is said about
fitting these puzzle pieces to
reality.

%\vfill

%\begin{center}{\small The project is supported by
%funds from the Department of
%Defense of the United States of America
%and Stanford University.}\end{center}
