%!name:canonical_maps
%!need:functions
%!need:equivalence_relations

\ssection{Why}

How do equivalence classes and functions relate

\ssection{Definition}

We can associate to each element of a set its equivalence class under an equivalence relation.
Let $X$ denote a set and $R$ an equivalence relation.
We call the function $f: X \to X/R$ defined by $f(x) = x/R$ the \t{canonical map} from  $X$ to $X/R$.

Conversely, if $f$ is an arbitrary function from $X$ onto $Y$, we can naturally define an equivalence relation $R$ in $X$ so that for $a, b \in X$, $a\,R\,b \iff f(a) = f(b)$
$f$ was onto, so for each $y \in Y$, there exists an $x \in X$ with $f(x) = y$.
Now let $g: Y \to X/R$ be defined by $g(y) = x/R$.
The values of $g$ are the subset $X$ which are mapped to the same value under $f$.
Moreover, the function $g$ is one-to-one.

\blankpage
