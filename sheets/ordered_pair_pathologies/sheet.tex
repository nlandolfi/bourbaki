
\section*{Why}

Why define ordered pairs in terms of sets?
Why not make them their own intangible object?

\section*{Pathologies}

Notice that $a \not\in (a, b)$ and similarly $b \not\in (a, b)$.
These facts led us to use the terms first and second ``coordinate'' in \sheetref{ordered_pairs}{Ordered Pairs}rather than the term ``element'' (used in \sheetref{sets}{Sets}).
Neither $a$ nor $b$ is an element of the ordered pair $(a, b)$.
On the other hand, it is true that $\set{a} \in (a, b)$ and $(a, b) \in (a, b)$.
These facts are odd.
Should they bother us?

We chose to define ordered pairs in terms of sets so that we could reuse notions about a particular type of object (sets) that we had already developed.
We chose what we may call conceptual simplicty (reusing notions from sets) over defining a new type of object (the ordered pair) with its own primitive properties.
Taking the former path, rather than the latter is a matter of taste, really, and not a logical consequence of the nature of things.

The argument for our taste is as follows.
We already know about sets, so let's use them, and let's forget cases like $(a, b) \in (a, b)$ (called by some authors ``pathologies'').
It does not bother us that our construction admits many true (but irrelevant) statements.
Such is the case in life.

Suppose we did choose to make the object $(a, b)$ primitive.
Sure, we would avoid oddities like $\set{a} \in (a, b)$.
And we might even get statements like $a \in (a, b)$ to be true.
But to do so we would have to define the meaning of $\in$ for the case in which the right hand object is an ``ordered pair''.
Our current route avoids introducing any new concepts, and simply names a construction using already developed concepts.

\blankpage