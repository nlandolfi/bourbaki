%!name:sequences
%!need:direct_products
%%!need:function_composites
% get families thru direct products
%%!need:families
%!refs:paul_halmos/naive_set_theory/section_11
%!refs:bert_mendelson/introduction_to_topology/theory_of_sets/indexed_families_of_sets
%!need:set_numbers
%!need:family_unions_and_intersections

\ssection{Why}

The most important families are those indexed by (subsets of) the natural numbers.

\ssection{Definition}

A \t{finite sequence} is a family whose index set is a nonzero natural number.
In other words, a finite sequence is a function whose domain is a nonzero natural number.

The \t{length} of a finite sequence is the size of its index set (which is the same as the domain of the sequence, see \sheetref{set_numbers}{Set Numbers}).
Some authors call a finite sequence, in certain contexts, a \t{numbering} or, in other contexts, a \t{string}.
A sequence whose codomain is some nonempty $A$ is said to be a \t{sequence in} $A$.

\ssubsection{Notation}

Since the index set of a finite sequence is ordered (see \sheetref{natural_order}{Natural Order}), we write sequences in order from left to right.
For example, let $A$ be a set and let $a: 4 \to A$ a sequence.
We denote $a$ by listing its elements in between parentheses, as in $(a_1, a_2, a_3, a_4)$.

\ssubsection{Relation to Direct Products}

A \t{natural direct product} is a product of a sequence of sets.
We denote the direct product of a sequence of sets $A_1$, $\dots$, $A_n$ by $\prod_{i = 1}^{n} A_i$.
If each $A_i$ is the same set $A$, then we denote the product $\prod_{i = 1}^{n} A_i$ by $A^n$.
In this case, we call an element (the sequence $a = (a_1, a_2, \dots, a_n) \in A^n$) an \t{$n$-tuple} or \t{tuple}.
The set of sequences in a set $A$ is the direct product $A^n$.

\ssection{Infinite Sequences}

An \t{infinite sequence} is a family whose index set is $\N$ (the set of natural numbers without zero).
The \t{$n$th term} of a sequence is the result of the $n$th natural number, $n \in \N$.\footnote{Future editions may also comment that we are introducing language for the steps of an infinite process.}

\ssubsection{Notation}

Let $A$ be a non-empty set and $a: \N \to A$.
Then $a$ is a (infinite) sequence in $A$.
$a(n)$ is the $n$th term.
We also denote $a$ by $\seq{a}$ and $a(n)$ by $\seqt{a}$.
If $\set{A_n}_{n \in \N}$ is an infinite sequence of sets, then we denote the direct product of the sequence by $\prod_{i = 1}^{\infty} A_i$.

\ssection{Natural unions and intersections}

We denote the family union of the finite sequence of sets $A_1$, $\dots$, $A_n$ by $\union_{i = 1}^{n} A_i$.
We denote the family of the infinite sequence of sets $\seq{A}$ by $\union_{i = 1}^{\infty} A_i$.
Similarly, we denote the intersections of a finite and infinite sequence of sets $\set{A_i}$ by $\intersection_{i = 1}^{n} A_i$ and $\intersection_{i = 1}^{\infty} A_i$, respectively.

% \blankpage

% TODO: integrate, from direct products
%
% If $I$ is the set of natural numbers we denote the direct product by
% \[
%   \product_{i = 1}^{\infty} A_{i}.
% \]
% We denote an element of $\product_{i = 1}^{\infty} A_{i}$ by $(a_i)$ with the understanding that $a_i \in A_i$ for all $i = 1,2,3,\dots$.
% If $A_i = A$ for all $i = 1, 2, 3,\dots$, then $(a_i)$ is a sequence in $A$.


%!TODO(next edition): perhaps have natural families back and make sequences always be infinite?
%%!name:natural_families
%%!need:natural_numbers
%%!need:families

% \ssection{Why}
%
% It is extremely
% common to use
% the first so
% many natural numbers as
% an index set.
%
% \ssection{Definition}
%
% A \casdft{natural family}{}
% of sets is a indexedkj family of sets
% whose index set
% is an subset of the
% natural numbers smaller
% than or equal to some
% specified natural number.
%
% \ssubsection{Notation}
%
% Let $n$ be a natural
% number.
% Let $A$ be a non-empty
% set.
% Let $a: \set{1, \dots, n} \to \powerset{A}$
% We often denote the family of $a$
% by $A_1, \dots, A_n$.
% We often say, \say{let $A_1, \dots, A_n$
% be sets,} without any reference to
% the fact that we are thinking of a family.
