%!name:sequences
%%!need:direct_products
%%!need:function_composites
%!need:natural_numbers
%!need:families
%!refs:paul_halmos/naive_set_theory/section_11
%!refs:bert_mendelson/introduction_to_topology/theory_of_sets/indexed_families_of_sets

\ssection{Why}

We introduce language for the steps of an infinite process.

\ssection{Definition}

A \t{finite sequence} is a family whose index set is a natural number (excluding zero).
An \t{infinite sequence} is a family whose index set is the set of natural numbers (without zero).
The \t{$n$th term} of a sequence (finite or infinite) is the result of the $n$th natural number.
Let $A$ be a non-empty set.
A sequence \t{in} $A$ is a function from the natural numbers to the set.

\ssubsection{Notation}

Let $A$ be a non-empty set.
Let $a: \N \to A$
Then $a$ is a sequence in $A$.
$a(n)$ is the $n$th term.
We also denote $a$ by
$\seq{a}$ and $a(n)$ by $\seqt{a}$.


\ssection{Natural Unions and intersections}

If $\set{A_i}$ is a finite sequence of sets indexed by $\upto{n}$, then we denote the union of the family by
\[
  \union_{i = 1}^{n} A_i
\]
If $\set{A_i}$ is an infinite sequence of sets, then we denote the union of the family by
\[
  \union_{i = 1}^{\infty} A_i.
\]
Similarly, we denote the intersections of a finite and infinite sequence of sets $\set{A_i}$ by
\[
  \intersection_{i = 1}^{n} A_i \quad \text{ and } \quad \intersection_{i = 1}^{\infty} A_i.
\]
respectively.

% \blankpage

% TODO: integrate, from direct products
%
% If $I$ is the set of natural numbers we denote the direct product by
% \[
%   \product_{i = 1}^{\infty} A_{i}.
% \]
% We denote an element of $\product_{i = 1}^{\infty} A_{i}$ by $(a_i)$ with the understanding that $a_i \in A_i$ for all $i = 1,2,3,\dots$.
% If $A_i = A$ for all $i = 1, 2, 3,\dots$, then $(a_i)$ is a sequence in $A$.


%!TODO(next edition): perhaps have natural families back and make sequences always be infinite?
%%!name:natural_families
%%!need:natural_numbers
%%!need:families

% \ssection{Why}
%
% It is extremely
% common to use
% the first so
% many natural numbers as
% an index set.
%
% \ssection{Definition}
%
% A \casdft{natural family}{}
% of sets is a indexedkj family of sets
% whose index set
% is an subset of the
% natural numbers smaller
% than or equal to some
% specified natural number.
%
% \ssubsection{Notation}
%
% Let $n$ be a natural
% number.
% Let $A$ be a non-empty
% set.
% Let $a: \set{1, \dots, n} \to \powerset{A}$
% We often denote the family of $a$
% by $A_1, \dots, A_n$.
% We often say, \say{let $A_1, \dots, A_n$
% be sets,} without any reference to
% the fact that we are thinking of a family.
