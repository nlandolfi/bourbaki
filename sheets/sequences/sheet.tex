%!name:sequences
%!need:lists
%!refs:paul_halmos/naive_set_theory/section_11
%!refs:bert_mendelson/introduction_to_topology/theory_of_sets/indexed_families_of_sets

\ssection{Why}
We want to speak of infinite processes, and to do so we define sequences indexed by $\N$.
In other words, important families are those indexed by the natural numbers.

\ssection{Definition}

A \t{sequence} (\t{infinite sequence}) is a family whose index set is $\N $ (the set of natural numbers without zero).
The \t{$n$th term} or \t{coordinate} of a sequence is the result of the $n$th natural number, $n \in \N $.
  \ifhmode\unskip\fi\footnote{
Future editions may also comment that we are introducing language for the steps of an infinite process.
  }

\ssubsection{Notation}

Let $A$ be a non-empty set and $a: \N  \to A$.
Then $a$ is a (infinite) sequence in $A$.
$a(n)$ is the $n$th term.
We also denote $a$ by $\seq{a}$ and $a(n)$ by $\seqt{a}$.
If $\set{A_n}_{n \in \N }$ is an infinite sequence of sets, then we denote the direct product of the sequence by $\prod_{i = 1}^{\infty} A_i$.

\ssection{Natural unions and intersections}

We denote the family of the infinite sequence of sets $\seq{A}$ by $\union_{i = 1}^{\infty} A_i$.
Similarly, we denote the intersection of an infinite sequence of sets by $\intersection_{i = 1}^{\infty} A_i$, respectively.

\blankpage
