
\section*{Why}

We can think of a limit as existing when the limit of upper bounds and lower bounds on final parts of the sequence coincide.

\section*{Definition}

The \t{limit superior} of a sequence of real numbers is the limit of the sequence of suprema of final parts of the sequence.
Similarly, the \t{limit inferior} of a sequence of real numbers is the limit of the sequence of infima of final parts of the sequence.

The limit of the sequence exists if and only if the limit superior and the limit inferior coincide, then the sequence has a limit which is defined to be the limiting value of each of those two sequences.

\subsection*{Notation}

We denote the limit superior of a sequence $a_n$ by
\[
\limsup_{n \to \infty} x_n \quad \text{ or } \quad \overline{\lim}_{n \to \infty} x_n,
\]
Similarly, we denote the limit inferior by
\[
\liminf_{n \to \infty} x_n \quad \text{ or } \underline{\lim}_{n \to \infty} x_n
\]

\section*{Properties}

\begin{proposition}
Suppose $x: \N   \to \R $ is a sequence.
Then,
\[
\liminf x_n = - \limsup (-x_n)
\]
\end{proposition}

\begin{proposition}
Suppose $x: \N   \to \R $ is a sequence.
Then $\limsup x_n = \liminf x_n = \ell $ if and only if $\lim x_n = \ell $.
Here $\ell  \in \R $.
\end{proposition}

\blankpage