
%!name:filled_graphs
%!need:ordered_undirected_graphs
%!need:chordal_graphs

\section*{Why}

We want to talk about optimally eliminating variables in a system of linear equations.\footnote{Future editions will expand. For example, this sheet is needed for perfect elimination orderings.}

\section*{Definition}

An \textit{ordered} undirected graph is \t{filled} or \t{monotone transitive} if all higher neighborhoods induce complete subgraphs.
An ordering $\sigma $ of an undirected graph $(V, E)$ is a \t{perfect elimination ordering} if the ordered undirected graph $((V, E), \sigma )$ is filled.

Let $G = ((V, E), \sigma )$ be an ordered undirected graph.
$G$ is filled if, for all $v \in V$, $w, z \in \adjp(v) \implies \set{w, z} \in E$.
Equivalently stated, $G$ is filled if, for all $i < j < k$, $\set{\sigma (i), \sigma (j)} \in E$ and $\set{\sigma (i), \sigma (k)} \in E$ imply $\set{\sigma (j), \sigma (k)} \in E$.

\section*{Chordality}

\begin{proposition}
If $(V, E, \sigma )$ is a filled graph, then $(V, E)$ is chordal.\end{proposition}
\begin{proof}Take the vertex with the lowest index on a cycle of length greater than three.
Take\end{proof}
\blankpage