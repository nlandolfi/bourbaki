
%!name:family_unions_and_intersections
%!need:families
%!need:set_unions_and_intersections
%!need:generalized_set_dualities
%!refs:paul_halmos/naive_set_theory/section_09
%!refs:bert_mendelson/introduction_to_topology/theory_of_sets/indexed_families_of_sets

\section*{Why}

We can use families to think about unions and intersections.

\section*{Family unions}

Let $A: I \to \powerset{X}$ be a family of subsets.
We refer to the union (see \sheetref{set_unions}{Set Unions}) of the range (see \sheetref{relations}{Relations}) of the family the \t{family union}.
We denote it $\cup_{i \in I} A_i$.

\begin{proposition}
$(x \in \cup_{i \in I} A_i) \iff (\exists i)(x \in A_i)$\end{proposition}
If $I = \set{a, b}$ is a pair with $a \neq b$, then $\cup_{i \in I} = A_a \cup A_b$.

There is no loss of generality in considering family unions.
Every set of sets is a family: consider the identity function from the set of sets to itself.

We can also show generalized associative and commutative law\footnote{The commutative law will appear in future editions.}
for unions.

\begin{proposition}
Let $\set{I_j}$ be a family of sets and define $K = \union_{j} I_j$. Then $\union_{k \in K} A_k = \union_{j \in J}(\union_{i \in I_j} A_i)$.\footnote{An account will appear in future editions.}\end{proposition}
\section*{Family intersection}

If we have a nonempty family of subsets $A: I \to \powerset{X}$, we call the intersection (see \sheetref{set_intersections}{Set Intersections}) of the range of the family the \t{family intersection}.
We denote it $\cap _{i \in I} A_i$.

\begin{proposition}
$x \in \cap _{i \in I} A_i \iff (\forall i)(x \in A_i)$\end{proposition}
Similarly we can derive associative and commutative laws for intersection\footnote{Statements of these will be given in future editions.}
.
They can be derived as for unions, or from the facts of unions using generalized DeMorgan's laws (see \sheetref{generalized_set_dualities}{Generalized Set Dualities}).

\subsection*{Connections}

The following are easy\footnote{Accounts will appear in future editions.}

Let $\set{A_i}$ be a family of subsets of $X$ and let $B \subset X$.

\begin{proposition}
$B \cap \bigcup_{i} A_i = \bigcup_{i} (B \cap A_i)$\end{proposition}
\begin{proposition}
$B \cup \bigcap_{i} A_i = \bigcap_{i} (B \cup A_i)$\end{proposition}
Let $\set{A_i}$ and $\set{B_j}$ be families of sets.\footnote{An account of the notation used and the proofs will appear in future editions.}

\begin{proposition}
$(\bigcup_{i} A_i) \cap (\bigcup_{j} B_j) = \bigcup_{i,j}(A_i \cap B_j)$\end{proposition}
\begin{proposition}
$(\bigcap_{i} A_u) \cup (\bigcap_{j} B_j) = \bigcap_{i,j}(A_i \cup B_j).$\end{proposition}
\begin{proposition}
$\cap _i X_i \subset X_j \subset \cup_i X_i$ for each $j$.\end{proposition}
%%% TODO(next edition): characetrization of union and intersection as extreme of the previous propositon
%%!see exercise at end of paul_halmos/naive_set_theory/section_09
%% old family set operatiosn
%%%!name:family_set_operations
%%%!need:family_operations
%%%!need:set_operations
%% \ssection{Why}
%%
%% Family set operations
%% are common.
%% TODO: this works
%% for infinite stuff too
%%
%% \ssection{Definition}
%%
%% We define the set whose elements are the objects
%% which are contained in at least one family member
%% the \casdft{family union}{familyunion}.
%% We define the set whose elements are the objects
%% which are contained in all of the family members
%% the \casdft{family intersection}{familyintersection}.
%%
%% \ssubsection{Notation}
%%
%% We denote the family union by $\union_{\alpha \in I} A_{\alpha}$.
%% We read this notation as \say{union over alpha in I of A sub-alpha.}
%% We denote family intersection by $\intersect_{\alpha \in I} A_{\alpha}$.
%% We read this notation as \say{intersection over alpha in I of A sub-alpha.}
%%
%% \ssubsection{Results}
%%
%% \begin{proposition}
%%   For an indexed family $\set{A_{\alpha}}_{\alpha \in I}$ in $S$, if $I = \set{i, j}$ then
%%   \[
%%     \union_{\alpha \in I} A_{\alpha} = A_i \union A_j
%%   \]
%%   and
%%   \[
%%     \intersect_{\alpha \in I} A_{\alpha} = A_i \intersect A_j.
%%   \]
%% \end{proposition}
%%
%% \begin{proposition}
%%   For an indexed family $\set{A_{\alpha}}_{\alpha \in I}$ in $S$, if $I = \emptyset$, then
%%   \[
%%     \union_{\alpha \in I} A_{\alpha} = \emptyset
%%   \]
%%   and
%%   \[
%%     \intersect_{\alpha \in I} A_{\alpha} = S.
%%   \]
%% \end{proposition}
