%!name:real_matrices
%!need:linear_equations
% don’t need since made decision that to understand vector space you need n dimensional space and to understand that you need the real numbers
%%!need:real_numbers

\ssection{Why}
We compress the notation for linear equations.

\ssection{Definition}
A \t{real matrix} (\t{real-valued matrix}, \t{matrix of real numbers}, \t{matrix}) is a two-dimensional array of real numbers.
Recall that we are interested in solutions of the linear equations
  \[
\begin{aligned}
y_1 &= A_{11}x_1 + A_{12}x_2 + \cdots + A_{1n}x_n, \\
y_2 &= A_{21}x_1 + A_{22}x_2 + \cdots + A_{2n}x_n, \\
&\vdots \\
y_n &= A_{m1}x_1 + A_{m2}x_2 + \cdots + A_{mn}x_n. \\
\end{aligned}
  \]
We have suggestively used the notation $A_{ij}$ for the coefficients of the equations, so they are the entries of a matrix $A \in \R^{m \times n}$.

We call $n$ and $m$ the \t{dimensions} of the matrix.
We call $n$ the \t{height} and $m$ the \t{width}.
If the height of the matrix is the same as the width of the matrix then we call the matrix \t{square}.
If the height is larger than the width, we call the matrix \t{tall}.
If the width is larger than the height, we call the matrix \t{wide}.

\ssubsection{Matrix-vector products}
Using the notation $A \in \R^{m \times n}$ and $x \in \R^n$ we want a compressed way to write the above system of linear equations.
Define the \t{real matrix-vector product} $z$ of $A$ with $x$ by $z_{i} = \sum_{j = 1}^{n} A_{ij}x_j$
We denote the matrix vector product $z$ by $Ax$.

\ssubsection{Notation}
We express the above system of linear equations as
  \[
y = Ax,
  \]
where {\small
  \[
y = \bmat{y_1 \\ y_2 \\ \vdots \\ y_m}, \;
A = \bmat{
A_{11} & A_{12} & \cdots & A_{1n} \\
A_{21} & A_{22} & \cdots & A_{2n} \\
\vdots & & \ddots & \vdots \\
A_{m1} & A_{m2} & \cdots & A_{mn} \\
}, \text{ and }
x = \bmat{x_1 \\ x_2 \\ \vdots \\ x_m}.
  \]
}
The compact notation $y = Ax$ is sometimes called the \t{matrix form} of the $m$ linear equations and $A$ the \t{coefficient matrix}.

This notation suggests both algebraic and geometric interpretations of solving systems of linear equations.
The algebraic interpretation is that we are interested in the invertibility of the function $x \mapsto Ax$.
In other words, we are interested in the existence of an inverse element of $A$.
The geometric interpretation is that $A$ transforms the vector $x$.

\ssubsection{Note on terminology}
The etymology of the word \say{matrix} is from the Latin \say{mater,} meaning mother, and has an old sense in the English language similar to the sense of the English word \say{womb.}
The matrix is source of many determinants (discussed later).
  \ifhmode\unskip\fi\footnote{
Future editions may elide this discussion.
  }
