
\section*{Why}

A measurable
function from
a first measure space
to a second measurable space
induces a measure
on the latter.

\section*{Definition}

Consider two measurable spaces and a measurable function between them.
The \t{image measure} of a measure on the first space \t{under} the measurable function is the measure on the second space which assigns to each measurable set (of that space) the measure of the inverse image of that set.

We say that the function \t{induces} the image measure on the codomain.
Alternatively, we say that we \t{push forward} the measure to the codomain, and so call the image measure a \t{push forward measure}.

\subsection*{Notation}

Let
$(X, \mathcal{A} )$
and
$(Y, \mathcal{B} )$
be two measurable spaces.
Let $f: X \to Y$ be
a measurable function.
Let
$\mu : \mathcal{A}  \to \nneri$
be a measure.
We denote the
image measure of $\mu $
under $f$ by
$\pushforward{\mu }{f}$,
for the reason that it
\[
\pushforward{\mu }{f}(B) = \mu (f^{-1}(B))
\]
for every $B \in \mathcal{B} $.

\section*{Change of variables}

The main property
we would like to hold
is that integration
on the new measure space
is the same as integration
on the old.

\begin{proposition}
Let $(X, \mathcal{A} )$ and $(Y, \mathcal{B} )$ be measurable spaces.
Let $f: X \to Y$ be a measurable function and let $\mu : \mathcal{A}  \to \nneri$ be a measure.
Then $g: Y \to $ is integrable with respect to $\pushforward{\mu }{f}$ if and only if $g \circ f$ is integrable with respect to $\mu $.
In this case,
\[
\int  g d(\pushforward{\mu }{f})
=
\int  g \circ f d\mu .
\]

\end{proposition}
