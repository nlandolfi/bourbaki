
\section*{Why}

There is a natural orientation of an ordered undirected graph.

\section*{Motivating result}

An ordered undirected graph can be converted into a directed graph by orienting the edges from lower to higher index.
The \t{orientation} of an ordered undirected graph $((V, E),\sigma )$ is the directed graph $(V, F)$ where
\[
\set{v, w} \in V \implies (v, w) \in F \text{ and } \inv{\sigma }(v) < \inv{\sigma }(w).
\]
In other words, we can ``convert'' the ordered undirected graph by ``orienting'' the edges from lower to higher index.

\begin{proposition}
Let $G = ((V, E), \sigma )$ be an ordered undirected graph.
The orientation of $G$ is acyclic.
\begin{proof}Contradiction on the existence of a cycle.\footnote{Future editions will expand.}\end{proof}
\end{proposition}

Conversely, let $(V, F)$ be directed acyclic.
To each topological numbering $\sigma $ of $(V, F)$ (see \sheetref{directed_paths}{Directed Paths}) there exists an ordered undirected graph $((V, E), \sigma )$ where $(V, E)$ is the skeleton of $(V, F)$.

\section*{Example}

\begin{center}\includegraphics[width=0.70\textwidth]{./graphics/figuresidebyside.png}\end{center}
%\begin{figure}
%  \centering
%  \begin{subfigure}{0.4\textwidth}
%  \includegraphics[width=0.9\textwidth]{graphics_included/ordered_undirected_graph}
%    \label{figure:ordered_undirected_graph_orientations:ordered_undirected_graph}
%  \end{subfigure}
%  \begin{subfigure}{0.4\textwidth}
%    \includegraphics[width=0.9\textwidth]{graphics_included/directed_ordered_undirected_graph}
%    	\label{figure:ordered_undirected_graph_orientations:directed_ordered_undirected_graph}
%  \end{subfigure}
%  \caption{$G$ and its (directed acyclic) orientation.}
%  \label{figure:ordered_undirected_graph_orientations:main}
%\end{figure}


Let $G = ((V, E), \sigma )$ be an undirected graph with
\[
V = \set{a,b,c,d,e},
\]
\[
E = \set*{\set{a, b}, \set{a, c}, \set{a, e}, \set{b, d}, \set{b, e}, \set{c, d}, \set{c, e}, \set{d,e}},
\]
and
\[
\sigma (1) = a \quad \sigma (2) = c \quad \sigma (3) = d \quad \sigma (4) = b \quad \sigma (e) = 5.
\]
We visualize $G$ and its (directed acyclic) orientation above.
