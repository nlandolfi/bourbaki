%!name:ordered_undirected_graph_orientations
%!need:ordered_undirected_graphs
%!need:directed_paths

\ssection{Why}\footnote{Future editions will include.}

There is a natural orientation of an ordered undirected graph.

\ssection{Motivating Result}

An ordered undirected graph can be converted into a directed graph by orienting the eges from lower to higher index.
The \t{orientation} of an ordered undirected graph $((V, E),\sigma)$ is the directed graph $(V, F)$ where
\[
  \set{v, w} \in V \implies (v, w) \in F \text{ and } \inv{\sigma}(v) < \inv{\sigma}(w).
\]
In other words, we can \say{convert} the ordered undirected graph by \say{orienting} the edges from lower to higher index.

\begin{proposition}
  Let $G = ((V, E), \sigma)$ be an ordered undirected graph.
  The orientation of $G$ is acyclic.
  \begin{proof}
    Contradiction on the existence of a cycle.\footnote{Future editions will expand.}
  \end{proof}
\end{proposition}

Conversely, consider the directed acyclic graph $(V, F)$.
To each topological numbering of $\sigma$ $(V, F)$ (see \sheetref{directed_path}{Directed Paths}) there exists an ordered undirected graph $((V, E), \sigma)$ where $(V, E)$ is the skeleton of $(V, F)$.

\blankpage
