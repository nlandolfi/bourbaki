%!name:ordered_pair_projections
%!need:set_products
%!need:set_unions
%!refs:paul_halmos/naive_set_theory/section_06

\ssection{Why}

The product of two sets is a (sub)set of ordered pairs.
Is every set or ordered pairs a subset of a product of two sets?

\ssection{Result}

The answer is easily seen to be yes.
Let $R$ denote a set or ordered pairs.
So for $x \in R$, $x = \set{\set{a}, \set{a, b}}$.
First consider $\bigcup R$.
Then $\set{a} \in \bigcup R$ and $\set{a, b} \in \bigcup R$.
Next consder $\bigcup\bigcup R$.
Then $a, b \in \bigcup\bigcup R$.
So if we want to sets---denote them by $A$ and $B$---so that $R \subset A \times B$, we can take both $A$ and $B$ to be the set $\bigcup\bigcup R$.

We often want to shrink the sets $A$ and $B$ to only include the relevant members.
In other words, we specify the elements of $\bigcup\bigcup R$ which are actually a first coordinate or second coordinate for some ordered pair in the set $R$.
In other words, we define $A' = \Set{a \in A}{(\exists b)((a, b) \in R)}$ and likewise $B' = \Set{b \in B}{(\exists a)((a, b) \in R)}$.
We call $A'$ the \t{projection of $R$ onto the first coordinate} and $B'$ the \t{projection of $R$ onto the second coordinate.}


\blankpage
