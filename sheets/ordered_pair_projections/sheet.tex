
\section*{Why}

The product of two sets is a (sub)set of ordered pairs.
Is every set of ordered pairs a subset of a product of two sets?

\section*{Result}

The answer is easily seen to be yes.
Let $R$ denote a set of ordered pairs.
So for $x \in R$, $x = \set{\set{a}, \set{a, b}}$.
First consider $\bigcup R$.
Then $\set{a} \in \bigcup R$ and $\set{a, b} \in \bigcup R$.
Next consder $\bigcup\bigcup R$.
Then $a, b \in \bigcup\bigcup R$.
So if we want two sets---denote them by $A$ and $B$---so that $R \subset A \times  B$, we can take both $A$ and $B$ to be the set $\bigcup\bigcup R$.

\section*{Projections}

We often want to shrink the sets $A$ and $B$ to include only the \textit{relevant} members.
In other words, to include only those members which appear as either the first coordinate (for $A$) or second coordinate (for $B$) in an element of $R$.
We can do this by specifying the elements of $\bigcup\bigcup R$ which are actually a first coordinate or second coordinate for some ordered pair in the set $R$.

Define
\[
A' = \Set{a \in A}{(\exists b)((a, b) \in R)},
\]
and likewise
\[
B' = \Set{b \in B}{(\exists a)((a, b) \in R)}.
\]
We call $A'$ the \t{projection onto the first coordinate} and $B'$ the \t{projection onto the second coordinate.}

\blankpage