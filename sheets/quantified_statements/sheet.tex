
\section*{Why}

We want symbols for talking about the existence of objects and for making statements which hold for all objects.\footnote{This sheet does not explain quantifiers. In the next edition there will be several more sheets serving this function}

\section*{Definition}

If we say there exists an object that is blue, we mean the same as if we say that not every object is not blue.
If we say that every object is blue, we mean the same as if we say there does not exist an object that is not blue.
In other words, ``there exists an object so that \_'' is the same as ``not every object is not \_''.
Or, ``every object is \_'' is the same as ``there does not exist an object that is \_''.

When we assert something of every object we also assert the nonexistence of the contrary of that assertion.
And likewise when we assert that an object exists with some conditions, we assert that not every object exists without that condition.

The content of our assertions will be logical statements (see \sheetref{logical_statements}{Logical Statements}) and when we want to make them for all objects or for no object we will use the following symbols.
The symbols introduced here are \t{quantifier symbols} and statements using them are \t{quantified statements}.

\subsection*{Existential quantifier}

Consider the symbol $\exists $.
We agree that it means ``there exists an object''.
We write $(\exists x)(\_)$ and then substitute any logical statement which uses the name $x$ for $\_$ .
For example, we write $(\exists x)(x \in A)$ to mean ``there exists an object in the set denoted by $A$.''
%  % For \say{there exists an object and we denote that very object by $a$} by $∃a$.
%  % We want to say that there exists an object.
%  % Denote that object by $a$.
%  

We call $\exists $ the \t{existential quantifier} symbol.

\subsection*{Universal quantifier}

Consider the symbol $\forall$.
We agree that it means ``for every object''.
We write $(\forall x)(\_)$ and then substitute any logical statement which uses the name $x$ for $\_$.
For example, we write $(\forall x)((x \in A)\implies(x \in B)$ to mean, ``every object which is in the set denoted by $A$ is in the set denoted by $B$''.
We call $\forall$ the \t{universal quantifier} symbol.

\subsection*{Binding}

When we have a name following a $\forall$ or $\exists $ we say that the name is \t{bound}.
If a name is bound, then the statement uses it in one sense but not in another.
The name is only used in that single statement.
Regular names in statements we call \t{unbound} or \t{free}.

\subsection*{Negations}

The statement $\neg(\forall x)(\_)$ is the same as $(\exists x)(\neg(\_))$ and $\neg(\exists x)(\_)$ is the same as $(\forall x)(\neg(\_))$.
