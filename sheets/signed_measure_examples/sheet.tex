
\section*{Why}

Let us consider examples of signed measures.

\section*{Examples}

Consider an integrable
function defined on
some measurable space.
The extended-real-valued
funciton which assigns to
each distinguished set the
value of the integrating
the function over that set
is a signed measure.

\begin{example}
Suppose $(X, \mathcal{A} , \mu )$ a measure space and $f: X \to \R $ is an integrable function.
Define $\nu : \mathcal{A}  \to R$ by
\[
\nu (A) = \int_{A} f d\mu .
\]
Then $\nu $ is a signed measure.
%  <proof>
%   <div data-littype='run'> First, </div>
%    <div data-littype='displaymath'>
%     <div data-littype='run'> ν(∅) = ∈t fχ_{∅} dμ = ∈t 0 dμ = 0. </div>
%    </div>
%   <div data-littype='run'> Next, let $\seq{A}$ disjoint. </div>
%   <div data-littype='run'> Notice that, </div>
%    <div data-littype='displaymath'>
%     <div data-littype='run'> χ_{∪_{i = 1}^{n} A_k} = ∑_{i = k}^{n} χ_{A_k} </div>
%    </div>
%  </proof>
%  

\end{example}

%macros.tex
%%%%% MACROS %%%%%%%%%%%%%%%%%%%%%%%%%%%%%%%%%%%%%%%%%%%%%%%
%%%%%%%%%%%%%%%%%%%%%%%%%%%%%%%%%%%%%%%%%%%%%%%%%%%%%%%%%%%%
