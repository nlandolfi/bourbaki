
\section*{Definition}

A \t{finite automaton} (or \t{machine}, \t{deterministic finite automaton}) $M = (Q, \Sigma , \delta , q_0, F)$ is a list where $Q$ and $\Sigma $ is are finite sets (alphabets), $\delta : Q \times  \Sigma  \to Q$, $q_0 \in Q$ and $F \subset Q$.

We call $Q$ the \t{states}, $\Sigma $ the \t{alphabet}, $\delta $ the \t{transition function}, $q_0$ the \t{start state} (\t{initial state}), and $F$ the \t{accept states} (or \t{final states}).
An input $u \in \str(\Sigma )$ results in a state sequence $x \in \str(Q)$ with $x_1 = q_0$ and $x_{i+1} = \delta (x_i, u_i)$ for $i = 1, \dots , \num{u}$.
$M$ \t{accepts} $x$ if $x_{\num{x}+1} \in F$.
The set of all strings that $M$ accepts is the \t{language} of the machine $M$.
We say that $M$ \t{recognizes} or \t{accepts} this set.
Although a language may accept many different strings, it only ever accepts one language.
For example, if the machine accepts no strings, then it accepts the language $\varnothing$.

A $L \subset \str(\Sigma )$ is called \t{regular} if there exists a finite automaton that recognizes it.

\blankpage