%!name:logical_statements
%!need:statements

\s{Why}

We want symbols for \say{and}, \say{or}, \say{not}, and \say{implies}.\footnote{This sheet does not explain logic. In the next edition there will be several more sheets serving this function.}

\s{Definition}
%
%We have discussed already that nouns are names and that we will only use \sheetref{statements}{Statements} we discussed that nouns are names and that we will only use the present tense of the verbs \say{is} and \say{belongs}.
%We had statements like ${a}={b}$ (identity) and $a \in A$ (belonging).

We call $=$ and $\in$ \t{relational symbols}.
They say how the objects denoted by a pair of placeholder names relate to each other in the sense of being or belonging.
We call $\_=\_$ and $\_\in\_$ \t{simple statements}.
They denote simple sentences \say{the object denoted by \_ is the object denoted by \_} and \say{the object denoted by \_ belongs to the set denoted by \_}.
A \t{logical statement} is a statement using any of the four following symbols
% A \t{complicated} statement is formed by using the words \say{and}, \say{or}, \say{not} and \say{implies}.
% We want to assert that two primitive statements at once.
% Or we want to

%TODO: edition 2, these are each their own sheet

\ssection{Conjunction}

Consider the symbol $\land$.
We will agree that it means \say{and}.
If we want to make two simple statements like $a = b$ and $a \in A$ at once, we write write $(a = b) \land (a \in A)$.
The symbol $\land$ is symmetric, reflecting the fact that a statement like $(a \in A) \land (a = b)$ means the same as $(a = b) \land (a \in A)$.

\ssection{Disjunction}

Consider the symbol $\lor$.
We will agree that it means \say{or} in the sense of either one, the other, or both.
If we want to say that  at least one of the simple statements like $a = b$ and $a \in A$, we write write $(a = b) \lor (a \in A)$.
The symbol $\lor$ is symmetric, reflecting the fact that a statement like $(a \in A) \lor (a = b)$ means the same as $(a = b) \lor (a \in A)$.

\ssection{Negation}

Consider the symbol $\neg$.
We will agree that it means \say{not}.
We will use it to say that one object \say{is not} another object and one object \say{does not belong to} another object.
If we want to say the opposite of a simple statement like $a = b$ we will write $\neg(a = b)$.
We read it aloud as \say{not a is b} or (the more desirable) \say{a is not b}.
Similarly, $\neg(a \in A)$ we read as \say{not, the object denoted by $a$ belongs to the set denoted by $A$}.
Again, the more desirable english expression is something like \say{the object denoted by $a$ does not belong to the set $A$}
For these reasons, we introduce two new symbols $\neq$ and $\not\in$.
$a \neq b$ means $\neg(a = b)$ and $a \not\in A$ means $\neg(a \in A)$.

\s{Implication}

Consider the symbol $\implies$.
We will agree that it means \say{implies}.
% If we want to say that it is always that case either the former statement and the latter statement or the former statement and either the latter statement or not the latter statement we.
For example $(a \in A) \implies (a \in B)$ means \say{the object denoted by $a$ belongs to the object denoted by $A$ implies the object denoted by $a$ belongs to the set denoted by $B$}
It is the same as $(\neg(a \in A)) \lor (a \in B)$.
In other words, if $a \in A$, then always $a \in B$.
The symbol $\implies$ is not symmetric, since implication is not symmetric.
The symbol $\iff$ means \say{if and only if}.

% And for this reason, we introduce a
%
%
% We want to connect these statements with the words \say{and} and \say{or} so that we can say more complicated things.
% For example $\_ belongs to \_$
% We build
%
% A \t{logical statement} is a sequence of symbols, a subset of which are
%
% We want to predict assertions when  w q
% We want to
%
% \blankpage
