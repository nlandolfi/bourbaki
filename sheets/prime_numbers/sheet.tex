
\section*{Definition}

A number $n \in \N  $ larger than 1 $n > 1$ is \t{composite} if there exists numbers $m, k \in \N  $ (not necessarily distinct), both smaller than $n$, $m, k < n$, with $n = m\cdot k$.
A number which is not composite is called \t{prime}.

\section*{Examples}

\textit{The first few primes.}
Since the only number smaller than $2$ is $1$ and $2 \neq 1\cdot 1$, $2$ is the first and smallest prime.
Likewise, $3 \neq 1\cdot 2$, $3 \neq 1\cdot 1$, $3 \neq 2\cdot 2$.
So $3$ is the second smallest prime.

\textit{The first composite.}
Now consider $4$.
Since $4 = 2\cdot 2$ and $2 \leq 4$, 4 is the smallest composite number.

\blankpage