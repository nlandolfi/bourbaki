%!name:empty_set
%!need:set_equality

\ssection{Why}

Can a set have no elements?

\s{Definition}

Sure.
Take a set.
At least one exists by the principle of existence (see \sheetref{sets}{Sets}).
Now, take the universally false statement $x \neq x$ and apply the principle of specification (see \sheetref{set_specification}{Set Specification}) to that set.
The set we get has no elements, since no object satisfies $x \neq x$.
The principle of extension says that the set obtained is unique (contradiction).\footnote{This account will be expanded in the next edition.}
We call this set with no elements \t{the empty set.}


%\begin{account}
%  \namee{$A$}{$B$}
%  \have{empty_set:aempty}{$\neg((\exists a)(a \in A))$}
%  \have{empty_set:bempty}{$\neg((\exists b)(b \in B))$}
%  \thus{empty_set:ainb}{$(\forall x)(x \in A \implies b \in A)$}{\ref{empty_set:aempty}}
%  \thus{empty_set:bina}{$(\forall x)(x \in B \implies b \in B)$}{\ref{empty_set:bempty}}
%  \thus{empty_set:conclusion}{$A = B$}{\ref{empty_set:ainb},\ref{empty_set:bina}}
%\end{account}
% The English follows:
% Denote two sets which are both empty by $A$ and by $B$.
% Suppose toward contradiction that there exists an element $a \in A$ with $a \not\in B$ or $b \in B$ with $b \not\in B$.
% But then $A$ or $B$ would be nonempty.
% Thus there is a set which is empty and any other empty set is this set.
% In other words, this set is unique.
% We call it the \t{empty set}.

%\ssection{Definition}
%
%First, we assume there exists a set.
%As a consequence, there exists a set which contains no elements at all.
%We use the axiom of specification with a condition that is always false, and so selects no elements.
%
%As a result of the axiom of extension, this set with no elements is unique.

\ssection{Notation}

We denote the empty set by $\emptyset$.
In other words, in all future accounts (see \sheetref{accounts}{Accounts}), there are two implicit lines. First, \say{\texttt{name} $\emptyset$} and second \say{\texttt{have} (\forall x)(x \not\in \emptyset)$}.

\blankpage