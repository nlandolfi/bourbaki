
\section*{Why}

Can a set have no elements?

\section*{Definition}

Sure.
A set exists by the principle of existence (see \sheetref{sets}{Sets}); denote it by $A$.
Specify elements (see \sheetref{set_specification}{Set Specification}) of any set that exists using the universally false statement $x \neq x$.
We denote that set by $\Set{x \in A}{x \neq x}$.
It has no elements.
In other words, $(\forall x)(x \not \in A)$.
The principle of extension (see \sheetref{set_equality}{Set Equality}) says that the set obtained is unique (contradiction).\footnote{This account will be expanded in the next edition.}
We call the unique set with no elements the \t{empty set}.
If a set is not the empty set, we call it \t{nonempty}.

%<div data-littype='paragraph'>
%  <div data-littype='run'> </div>
%  <div data-littype='run'> ❲%\begin{account}❳ </div>
%  <div data-littype='run'> ❲%  \namee{$A$}{$B$}❳ </div>
%  <div data-littype='run'> ❲%  \have{empty_set:aempty}{$\neg((∃ a)(a ∈ A))$}❳ </div>
%  <div data-littype='run'> ❲%  \have{empty_set:bempty}{$\neg((∃ b)(b ∈ B))$}❳ </div>
%  <div data-littype='run'> ❲%  \thus{empty_set:ainb}{$(∀ x)(x ∈ A ⟹ b ∈ A)$}{\ref{empty_set:aempty}}❳ </div>
%  <div data-littype='run'> ❲%  \thus{empty_set:bina}{$(∀ x)(x ∈ B ⟹ b ∈ B)$}{\ref{empty_set:bempty}}❳ </div>
%  <div data-littype='run'> ❲%  \thus{empty_set:conclusion}{$A = B$}{\ref{empty_set:ainb},\ref{empty_set:bina}}❳ </div>
%  <div data-littype='run'> ❲%\end{account}❳ </div>
%  <div data-littype='run'> ❲% The English follows:❳ </div>
%  <div data-littype='run'> ❲% Denote two sets which are both empty by $A$ and by $B$.❳ </div>
%  <div data-littype='run'> ❲% Suppose toward contradiction that there exists an element $a ∈ A$ with $a ∉ B$ or $b ∈ B$ with $b ∉ B$.❳ </div>
%  <div data-littype='run'> ❲% But then $A$ or $B$ would be nonempty.❳ </div>
%  <div data-littype='run'> ❲% Thus there is a set which is empty and any other empty set is this set.❳ </div>
%  <div data-littype='run'> ❲% In other words, this set is unique.❳ </div>
%  <div data-littype='run'> ❲% We call it the ❬empty set❭.❳ </div>
%</div>
%<div data-littype='paragraph'>
%  <div data-littype='run'> ❲%\ssection{Definition}❳ </div>
%  <div data-littype='run'> ❲%❳ </div>
%  <div data-littype='run'> ❲%First, we assume there exists a set.❳ </div>
%  <div data-littype='run'> ❲%As a consequence, there exists a set which contains no elements at all.❳ </div>
%  <div data-littype='run'> ❲%We use the axiom of specification with a condition that is always false, and so selects no elements.❳ </div>
%  <div data-littype='run'> ❲%❳ </div>
%  <div data-littype='run'> ❲%As a result of the axiom of extension, this set with no elements is unique.❳ </div>
%</div>

\subsection*{Notation}

We denote the empty set by $\varnothing$.
% <div data-littype='run'> In other words, in all future accounts (see␣
%    <a href='/sheets/accounts.html'>
%     <div data-littype='run'> Accounts </div>
%    </a>
%    ), there are two implicit lines. First, “⁅name⁆ $∅$” and
%    second “⁅have⁆ $(∀ x)(x ∉ ∅)$”. </div>
%    


\section*{Properties}

It is immediate from our definition of the empty set and of the definition of inclusion (see \sheetref{set_inclusion}{Set Inclusion}) that the empty set is included in every set (including itself).

\begin{proposition}
$(\forall A)(\varnothing \subset A)$
\end{proposition}

\begin{proof}Suppose toward contradiction that $\varnothing \not\subset A$.
Then there exists $y \in \varnothing$ such that $y \not \in A$.
But this is impossible, since $(\forall x)(x \not \in \varnothing)$.\end{proof}
\blankpage