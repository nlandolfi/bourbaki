%!name:empty_set
% %!need:set_specification
%!need:set_equality

\ssection{Why}

If there is a set, there is an empty set.
Are there many such sets?
How do they (or it) relate to other sets?

\s{Empty Set}

An immediate consequence of the axiom of extension is that there is a unique set that is empty.

\begin{account}
  \namee{$A$}{$B$}
  \have{empty_set:aempty}{$\neg((\exists a)(a \in A))$}
  \have{empty_set:bempty}{$\neg((\exists b)(b \in B))$}
  \thus{empty_set:ainb}{$(\forall x)(x \in A \implies b \in A)$}{\ref{empty_set:aempty}}
  \thus{empty_set:bina}{$(\forall x)(x \in B \implies b \in B)$}{\ref{empty_set:bempty}}
  \thus{empty_set:conclusion}{$A = B$}{\ref{empty_set:ainb},\ref{empty_set:bina}}
\end{account}
% The English follows:
% Denote two sets which are both empty by $A$ and by $B$.
% Suppose toward contradiction that there exists an element $a \in A$ with $a \not\in B$ or $b \in B$ with $b \not\in B$.
% But then $A$ or $B$ would be nonempty.
% Thus there is a set which is empty and any other empty set is this set.
% In other words, this set is unique.
% We call it the \t{empty set}.

\ssection{Definition}

First, we assume there exists a set.
As a consequence, there exists a set which contains no elements at all.
We use the axiom of specification with a condition that is always false, and so selects no elements.

As a result of the axiom of extension, this set with no elements is unique.
We call this empty set \t{the empty set.}

\ssection{Notation}

We denote the empty set by $\emptyset$.
