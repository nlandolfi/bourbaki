%!name:cartesian_products
%!need:ordered_pairs
%!need:set_powers
%!refs:paul_halmos/naive_set_theory/section_06

\ssection{Why}

Does a set exist of all the ordered pairs of elements from an ordered pair of sets?

\ssection{Definition}

Indeed.
Let $A$ and $B$ denote sets and let $a$ and $b$ denote objects.
If $a \in A$ and $b \in B$, then $\set{a} \subset A$ and $\set{b} \subset B$ and so $\set{a},\set{b},\set{a, b} \in \powerset{\parens{A \union B}}$.



The \t{product} of the set denoted by $A$ and the set denoted by $B$ is the set of all ordered pairs.
This set is also called the \t{cartesian product}.
If $A \neq B$, the ordering causes the product of $A$ and $B$ to differ from the product of
$B$ with $A$.
If $A = B$, however, the symmetry holds.

\ssubsection{Notation}

We denote the product of $A$ with $B$ by $A \cross B$, read aloud as \say{A cross B.}
In this notation, if $A \neq B$, then $A \cross B \neq B \cross A$.

\blankpage
