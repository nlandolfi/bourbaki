%!name:set_inclusion
%!need:standardized_accounts
%!need:definitions

\ssection{Why}

We want language for all of the elements of a first set being the elements of a second set.

\ssection{Definition}

Denote a set by $A$ and a set by $B$.

\begin{definition}[Subsets]
If every element of the set denoted by $A$ is an element of the set denoted by $B$, then we say that the set denoted by $A$ is a \t{subset} of the set denoted by $B$.
\label{set_inclusion:subsets}
\end{definition}

We say that the set denoted by $A$ is \t{included} in the set denoted by $B$.
We say that the set denoted by $B$ is a \t{superset} of the set denoted by $A$ or that the set denoted by $B$ \t{includes} the set denoted by $A$.

Every set is included in and includes itself.
\begin{account}
	\name{$A$}
	\have{}{$(\forall x)(x \in A \implies x \in A)$}
	\thus{}{$A \subset A$}{Def~\ref{set_inclusion:subsets}}
\end{account}

%If the sets denoted by $A$ and $B$ are identical, then each contains the other.
% \begin{account}[]
%   \namee{$A$}{$B$}
%   \have{}{$A = B$}
%   \thus{}{$(\forall x)(x \in A \implies x \in B)$}
%   \thus{}{$(\forall x)(x \in B \implies x \in A)$}
% \end{account}
% are identical
%If $A = B$, then the set denoted by $A$ includes the set denoted by $B$ and the set denoted by $B$ includes the set denoted by $A$.
%The axiom of extension asserts the converse also holds.
%If the set denoted by $A$ includes the set denoted by $B$ and the set denoted by$B$ includes the set denoted by $A$, then $A$ and $B$ denote the same set.
%% I$A = B$.
%In other words, if the set denoted by $A$ is a subset of the set denoted by $B$ and the set denoted by $B$ a subset of the set denoted by $A$, then $A = B$.
%
%The empty set is a subset of every other set.
%
%\begin{account}[Empty Set Inclusion]
%\namee{$A$}{$\emptyset$}
%\have{set_inclusion:empty:definition}{$\neg((\exists x)(x \in \emptyset))$}
%\thus{set_inclusion:empty:implication}{$(\forall x)((x \in \varnothing) \implies (x \in A))$}{\ref{set_inclusion:empty:definition}}
%\ie{set_inclusion:empty:conclusion}{$\varnothing \subset A$}{\ref{set_inclusion:empty:implication}}
%\end{account}
%
%Suppose toward contradiction that $A$ were a set which did not include the empty set.
%Then there would exist an element in the empty set which is not in $A$.
%But then the empty set would not be empty.
%We call the empty set and $A$ \t{improper subsets} of $A$.
%All other subsets we call \t{proper subsets}.
%In other words, $B$ is an improper subset of $A$ if and only if $A$ includes $B$, $B \neq A$ and $B \neq \emptyset$.

\ssubsection{Notation}

Let $A$ denote a set and $B$ denote a set.
We denote that $A$ is included in $B$ by $A \subset B$.
In other words, $A \subset B$ means $(\forall x)((x \in A) \implies (x \in B))$.
We read the notation $A \subset B$ aloud as \say{A is included in B} or \say{A subset B}.
Or we write $B \supset A$, and read it aloud \say{B includes A} or \say{B superset $A$}.
$B \supset A$ also means $(\forall x)((x \in A) \implies (x \in B))$.

\ssubsection{Properties}

Given a set $A$, $A \subset A$.
Like equality, we say that inclusion is \t{reflexive}.
Given sets $A$ and $B$, if $A \subset B$ and $B \subset C$ then $A \subset C$.
Like equality, we say that inclusion is \t{transitive}.
If $A \subset B$ and $B \subset A$, then $A = B$ (by the axiom of extension).
Unlike equality, which is symmetric, we say that inclusion is \t{antisymmetric}.

\ssubsection{Comparison with belonging}

Given a set $A$ inclusion is reflexive.
$A \subset A$ is always true.
$A \in A$ may be true.
%Is $A \in A$ ever true?
Inclusion is transitive, whereas belonging is not.
