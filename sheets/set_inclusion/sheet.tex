%!name:set_inclusion
%!need:set_equality

\ssection{Why}

We want language for all of the elements of a first set being the elements of a second set.

\ssection{Definition}

Denote a set by $A$ and a set by $B$.
If every element of the set denoted by $A$ is an element of the set denoted by $B$, then we say that the set denoted by $A$ is a \t{subset} of the set denoted by $B$.
We say that the set denoted by $A$ is \t{included} in the set denoted by $B$.
We say that the set denoted by $B$ is a \t{superset} of the set denoted by $A$ or that the set denoted by $B$ \t{includes} the set denoted by $A$.
A set includes and is included in itself.

If the sets denoted by $A$ and $B$ are identical, then each contains the other.
% \begin{account}[]
%   \namee{$A$}{$B$}
%   \have{}{$A = B$}
%   \thus{}{$(\forall x)(x \in A \implies x \in B)$}
%   \thus{}{$(\forall x)(x \in B \implies x \in A)$}
% \end{account}
% are identical
If $A = B$, then $A$ includes $B$ and $B$ includes $A$.
The axiom of extension asserts the converse also holds.
If $A$ includes $B$ and $B$ includes $A$, then $A = B$.
In other words, if $A$ is a subset of $B$ and $B$ a subset of $A$, then $A = B$.

The empty set is a subset of every other set.
Suppose toward contradiction that $A$ were a set which did not include the empty set.
Then there would exist an element in the empty set which is not in $A$.
But then the empty set would not be empty.
We call the empty set and $A$ \t{improper subsets} of $A$.
All other subsets we call \t{proper subsets}.
In other words, $B$ is an improper subset of $A$ if and only if $A$ includes $B$, $B \neq A$ and $B \neq \emptyset$.

\ssubsection{Notation}
Given two sets $A$ and $B$, we denote that $A$ is included in $B$ by $A \subset B$.
We read the notation $A \subset B$ aloud as \say{A is included in B} or \say{A subset B}.
Or we write $B \supset A$, and read it aloud \say{B includes A} or \say{B superset $A$}.

In this notation, we express the axiom of extension
\[
  A = B \Leftrightarrow (A \supset B) \land (A \subset B).
\]
The notation $A \subset B$
is a concise symbolism for
the sentence
\say{every element of $A$ is an element of
$B$.} Or for the alternative notation
$a \in A \implies a \in B$.

\ssubsection{Properties}

Given a set $A$, $A \subset A$.
Like equality, we say that inclusion is \t{reflexive}.
Given sets $A$ and $B$, if $A \subset B$ and $B \subset C$ then $A \subset C$.
Like equality, we say that inclusion is \t{transitive}.
If $A \subset B$ and $B \subset A$, then $A = B$ (by the axiom of extension).
Unlike equality, which is symmetric, we say that inclusion is \t{antisymmetric}.

\ssubsection{Comparison with belonging}

Given a set $A$ inclusion is reflexive.
$A \subset A$ is always true.
Is $A \in A$ ever true?
Also, inclusion is transitive.
Whereas belonging is not.
