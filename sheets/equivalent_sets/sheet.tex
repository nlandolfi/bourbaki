%!name:equivalent_sets
%!need:natural_numbers
%!need:function_inverses
%!need:equivalence_relations
%!need:natural_order

\ssection{Why}

We want to talk about the size of a set.

\ssection{Definition}

Two sets are \t{equivalent} if there exists a bijection between them.
Let $X$ be a set.
Then set equivalence as a relation in $\powerset{X}$ is an equivalence relation (see \sheetref{equivalence_relations}{Equivalence Relations}).

\ssubsection{Notation}

If $A$ and $B$ are sets and they are equivalent, then we write $A \sim B$, read aloud as \say{$A$ is equivalent to $B$.}

\ssubsection{Basic Result}

Every set is equivalent to itself, whether two sets are equivalent does not depend on the order in which we consider them, and if two sets are equivalent to the same set then they are equivalent to each other.
These facts can be summarized by the following proposition.

\begin{proposition}
	Let $X$ a set. Then $\sim$ is an equivalence relation on $\powerset{X}$.\footnote{The proof is direct and will appear in future editions.}
\end{proposition}

\ssection{For natural numbers}

\begin{proposition}
	Every proper subset of a natural number is equivalent to some smaller natural number.\footnote{The proof, which uses induction, will appear in future editions.}
\end{proposition}

\ssection{Equivalence to subsets}

It is unusual that a set can be equivalents to a proper subset of itself.

\begin{proposition}
	A set may be equivalent to a proper subset of itself.
\end{proposition}
\begin{proof}
The example is the set of natural numbers and the function $f(n) = n^+$.
It is a bijection from $\omega$ onto $\N$.\footnote{The account will be expanded in future editions.}
\end{proof}
However, this never holds for natural numbers.
\begin{proposition}
	If $n \in \omega$ then $n \not\sim x$ for any $x \subset n$ and $x \neq n$.
\end{proposition}

