%!name:equivalent_sets
%!need:natural_numbers
%!need:function_inverses
%!need:equivalence_relations

\ssection{Why}

We want to talk about the size of a set.

\ssection{Definition}

Two sets are \t{equivalent} if there exists a bijection between them.

\begin{prop}
  Set equivalence in the sense defined above is an equivalence relation in the power set of a set.
\end{prop}


\begin{prop}
Every proper subset of a natural number is equivalent to some smaller natural number.
\begin{proof}
TODO induction
\end{proof}
\end{prop}
TODO: smaller defined?

\begin{prop}
A set can be equivalent to a proper subset of itself.
\end{prop}
Halmos' example here is not a bijection, though...

\begin{prop}
If $n$ is a natural number, then $n$ is not equivalent ot a proper subset of itself.
\end{prop}

\begin{prop}
A set can be equivalent to at most one natural number.
\end{prop}

\begin{prop}
The set of natural numbers is infinite.
\end{prop}

\begin{prop}
A finite set is never equivalent to a proper subset of itself.
\end{prop}

\begin{prop}
Every subset of a finite set is finite.
\end{prop}

\begin{prop}
Every subset of a natural number is equivalent to a natural number.
\end{prop}

