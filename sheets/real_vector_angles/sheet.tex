
%!name:real_vector_angles
%!need:space_vector_angles
%!need:real_norm

\section*{Why}

We generalize the notion of angle between vectors in $\R ^2$ and $\R ^3$ to vectors in $\R ^n$.

\section*{Definition}

The \t{angle} (\t{unsigned angle}) between the nonzero vectors $x, y \in \R ^n$, is the real knumber
\[
\theta  = \angle(x,y) = \cos^{-1} \frac{x^\top y}{\norm{x}\norm{y}}.
\]
In the case that one (or both) of the vectors is zero, we define the angle between them to be 0.
Thus, $x^\top  y = \norm{x}\norm{y}\cos\theta $, which is a convenient way to remember the inner product norm inequality.

\subsection*{Terminology}

$x$ and $y$ are \t{aligned} if $\theta  = 0$, in which case $x^\top  y = \norm{x}\norm{y}$.
In the case that $x \neq 0$, $x$ and $y$ are aligned if $x = \alpha y$ for some $\alpha  \geq 0$.
$x$ and $y$ are \t{opposed} if $\theta  = \pi $, in which case $x^\top  y = \norm{x}\norm{y}$.
In the case that $x \neq 0$, $x$ and $y$ are opposed if $x = -\alpha y$ for some $\alpha  \geq 0$.
Two nonnzero vectors $x$ and $y$ are \t{orthogonal} if $\theta = \pi /2$ or $-\pi /2$, in which case $x^\top  y = \norm{x}\norm{y}$.
The origin is orthogonal to every other vector.
We denote that two vectors $x$ and $y$ are orthogonal by $x \perp  y$.
\blankpage
