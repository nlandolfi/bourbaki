
\section*{Why}

We want to talk about the size of a displacement.\footnote{Future editions will complete.}

\section*{Definition}

The \t{norm} of a vector $x \in \R ^2$ is
\[
\sqrt{x_1^2 + x_2^2}.
\]

\subsection*{Notation}

We denote the norm of $x$ by $\norm{x}$.
In other words, $\norm{\cdot}: \R ^2 \to \R $ is a function from vectors in $\R ^2$ to real numbers.
The notation follows the notation of absolute value, the \textit{magnitude} of a real number, and the double verticals remind us that $x$ is a vector.
A warning: some authors write $\abs{x}$ for the norm of $x$ when it is understood that $x \in \R ^2$.

\subsection*{Visualization}

\begin{center}\includegraphics[width=0.70\textwidth]{./graphics/x-y.pdf}\end{center}
\blankpage
%macros.tex
%%%%% MACROS %%%%%%%%%%%%%%%%%%%%%%%%%%%%%%%%%%%%%%%%%%%%%%%
%% use \norm{x} for |x| 11/05/2021 I changed this back to ||x||, since that is alsmost always what I want.
%% use \norm* for autosizing delimiters
%\DeclarePairedDelimiter{\norm}{\lVert}{\rVert}
%% use \normm{x} for ||x||
%% use \normm* for autosizing delimiters
%\DeclarePairedDelimiter{\normm}{\lVert}{\rVert}
%%%%%%%%%%%%%%%%%%%%%%%%%%%%%%%%%%%%%%%%%%%%%%%%%%%%%%%%%%%%
