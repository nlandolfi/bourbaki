%!name:supremum_norm_complete
%!need:supremum_norm
%!need:real_limits
%!!need:complete_norms?

\ssection{Why}

We want a complete
norm on the vector
space of continuous
functions.

\ssection{Result}

\begin{prop}
The supremum norm
is complete.

\begin{proof}
Let $R$ denote the real numbers.
Let $\seq{f}$ be an egoprox sequence
in $C[a, b]$.

\textbf{Candidate.}
$\seq{f}$ is egoprox means
$\forall \epsilon > 0, \exists N$
so that
\[
  m, n > N \implies \supnorm{f_n - f_m} < \epsilon.
\]
Since $\supnorm{f_n - f_m} < \epsilon \implies
\abs{f_n(x) - f_m(x)} < \epsilon$ for
all $x \in [a, b]$, the
sequence of real numbers $\set{f_n(x)}_n$
is egoprox for each $x \in [a, b]$.
Since the metric space $(R, \abs{\cdot})$
is complete, there is a limit $l_x \in R$
such that $f_n(x) \goesto l_x$ as
$n \goesto \infty$, for each $x \in [a, b]$.
Define $f: [a, b] \to R$
by $f(x) = l_x$ for each $x \in [a, b]$.

\textbf{Candidate is Limit.}
First, we argue that
$\supnorm{f_n - f} \goesto 0$
as $n \goesto \infty$.
Since $\seq{f}$ is an egoprox sequence,
there exists $n_0$ so that
\[
  n,m \geq n_0
  \implies
  \supnorm{f_n - f_m} < \epsilon/2.
\]
So for all $x \in [a, b]$,
\[
  n,m \geq n_0
  \implies
  \abs{f_n(x) - f_m(x)} < \epsilon/2.
\]
For all $x \in [a, b]$,
and $n \geq n_0$,
\[
  \lim_{m \to \infty} \abs{f_n(x) - f_m(x)}
  \leq \epsilon/2 < \epsilon.
\]
The sequence
$\set{f_k(x)}_{k = m}^{\infty}$
is a final part of
$\set{f_k(x)}_{k = 1}^{\infty}$,
and so has the same limit, $f(x)$.
Therefore, using continuity
of subtraction and the absolute
value,
\[
  \lim_{m \to \infty}
    \abs{f_n(x) - f_m(x)}
  =
  \abs{f_n(x) - f(x)}.
\]
We conclude that for
$n \geq n_0$,
$x \in [a, b]$,
$\abs{f_n(x) - f(x)} < \epsilon$,
from which we deduce
$\supnorm{f_n - f} < \epsilon$.
Thus $f_n \goesto f$
as $n \goesto \infty$.

\textbf{Limit is Continuous.}
Next, we argue that $f$ is continuous.
Let $x_0 \in [a, b]$.
Let $\epsilon > 0$.
Since $f_n \goesto f$ there exists
$n_0$ so that
\[
  \supnorm{f_{n_0} - f} < \epsilon/3.
\]
By the triangle inequality,
\[
  \begin{aligned}
    \abs{f(x_0) - f(x)} \leq \abs{f(x_0) - f_{n_0}(x_0)} + \abs{f_{n_0}(x_0) - f(x)},
  \end{aligned}
\]
for all $x \in [a, b]$.
Using
$\abs{f(x_0) - f_{n_0}(x_0)} < \epsilon/3$,
\[
  \begin{aligned}
    \abs{f(x_0) - f(x)} &< \epsilon/3 + \abs{f_{n_0}(x_0) - f(x)},
  \end{aligned}
\]
for all $x \in [a, b]$.
Using the triangle inequality,
\[
   \abs{f(x_0) - f(x)} < \epsilon/3 + \abs{f_{n_0}(x_0) - f_{n_0}(x)} + \abs{f_{n_0}(x) - f(x)} \\
\]
for all $x \in [a, b]$.
Using
$\abs{f_{n_0}(x_0) - f(x)} < \epsilon/3$
\[
  \begin{aligned}
   \abs{f(x_0) - f(x)} < \epsilon/3 + \abs{f_{n_0}(x_0) - f_{n_0}(x)} + \epsilon/3 \\
  \end{aligned}
\]
for all $x \in [a, b]$.
Since $f_{n_0}$ is
continuous, there exists
$\delta > 0$ so that
\[
  \abs{x_0 - x} < \delta
  \implies
  \abs{f_{n_0}(x_0) - f_{n_0}(x)} < \epsilon/3,
\]
for $x \in [a, b]$.
In this case,
\[
  \begin{aligned}
    \abs{f(x_0) - f(x)} &< \epsilon/3 + \epsilon/3 + \epsilon/3 = \epsilon.
  \end{aligned}
\]
Since $\epsilon$ was arbitrary, $f$
is continuous at $x_0$.
Since $x_0$ was arbitrary,
$f$ is continuous everywhere.
Some call the above the
\ct{three epsilon argument}{threeepsilonargument}.
\end{proof}

\end{prop}
