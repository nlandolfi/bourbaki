%!name:complex_numbers
%!need:real_arithmetic
%!need:real_plane
%!need:absolute_value
%!refs:david_brillinger/time_series_data_analysis_and_theory/02_foundations

\ssection{Why}

We want to find the roots of negative numbers.\footnote{Future editions will modify this, and will discuss the existence of solutions of algebraic equations.}

\ssection{Definition}

A \t{complex number} is an ordered pair of real numbers.
The \t{real part} of a complex number is its first coordinate.
The \t{imaginary part} of a complex number is its second coordinate.

The \t{complex conjugate} (or \t{conjugate}) of a complex number $z$ is the complex number whose real part matches $z$ and whose imaginary part is the additive inverse of $z$.
The complex conjugate of a real number (imaginary part is zero) is the real number.
In other words, the complex conjugate of a complex number with no imaginary part is the same complex number.

\ssubsection{Notation}

When we think of $\R^2$ as the set of complex numbers, we denote it by $\C$.
Let $z \in \C$.
We denote the real part of $z$ by $\re(z)$, read \say{real of z,} and the imaginary part by $\im(z)$, read \say{imaginary of z.}
If $z = (a, b)$ for $a, b \in \R$, then $\re(z) = a$ and $\im(z) = b$.

We denote the complex conjugate of the complex number $z \in \C$ by $\Cconj{z} \in \C$.
Another common notation, not used in these sheets is $\overline{z}$ or $\bar{z}$.
If there exists $a, b \in \R$ so that $z = (a, b)$, then $\Cconj{z} = (a, -b)$.

\ssection{Modulus and argument}

The \t{modulus} of $z \in \C$ is the distance of $z$ to the origin.
If $z \in \C$, then the modulus of $z$ is
\[
  \sqrt{\re{z}^2 + \im{z}^2}.
\]
We denote the modulus of $z$ by $\Cmod{z}$.

The \t{argument} of $z \in \C$ is $\tan^{-1}(\im{z}/\re{z})$.
We denote the argument of $z$ by $\arg z$.\footnote{Future editions will include the geometric interpretations.}

\blankpage
