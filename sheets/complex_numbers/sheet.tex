
\section*{Why}

We want to find the roots of negative numbers.\footnote{Future editions will modify this, and will discuss the existence of solutions of algebraic equations.}

\section*{Definition}

A \t{complex number} is an ordered pair of real numbers.
The \t{real part} of a complex number is its first coordinate.
The \t{imaginary part} of a complex number is its second coordinate.

We can identify the imaginary numbers with no complex part (i.e., the set $\Set*{(a, b) \in \R ^2}{b = 0}$) with $\R $ in the obvious way.
For this reason, such a complex number is sometimes referred to as a \t{purely real number}.
On the other hand, a complex number with zero imaginary part (i.e., an element of the set $\Set*{(a, b) \in \R ^2}{a = 0}$) is said to be a \t{purely imaginary number}.

\subsection*{Notation}

When treating $\R ^2$ as the set of complex numbers, we denote it by $\C $.
Let $z \in \C $ with $z = (a, b)$.
The real part of $z$ is $a$ and its imaginary part is $b$.
It is universal to denote $z$ by $a + ib$, and to call $i$ an (or the) \t{imaginary number}.
Some authors use $j$, it is a matter of notation.

We denote the real part of $z$ by $\re(z)$, read ``real of z,'' and the imaginary part by $\im(z)$, read ``imaginary of z.''
So, in particular, $\re(z) = a$ and $\im(z) = b$.

\blankpage
%macros.tex
%%%%% MACROS %%%%%%%%%%%%%%%%%%%%%%%%%%%%%%%%%%%%%%%%%%%%%%%
%\newcommand{\C}{\mathbf{C}}
%\newcommand{\re}{\mathword{Re}}
%\newcommand{\im}{\mathword{Im}}
%\renewcommand{\Re}{\re}
%\renewcommand{\Im}{\im}
%\newcommand{\Cconj}[1]{#1^{*}}
%\newcommand{\Cconjp}[1]{\parens{#1}^{*}}
%%%%%%%%%%%%%%%%%%%%%%%%%%%%%%%%%%%%%%%%%%%%%%%%%%%%%%%%%%%%
