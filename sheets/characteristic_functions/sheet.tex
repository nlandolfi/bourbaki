%!name:characteristic_functions
%!need:natural_numbers
%!need:set_powers

\ssection{Why}

We represent rectangles by functions.

\ssection{Definition}

The \t{characteristic function} of a $A$ of a set $X$ is a function from $X$ to $2$ which is $1$ if the argument is in $A$ and 0 otherwise.

%!TODO: next edition, fix tthis
If the base set is the real
numbers and the subset is
an interval, then the
characteristic function
is a rectangle with height
one and the width of the interval.
% Put in indicator random variables
%We call
%the
%\ct{characteristic function}{characteristic}
%of a subset of some set
%the
%\ct{indicator function}{indicator function}.

The function which assigns to each subset $A$ of $X$ to characteristic function of $A$ is a one-to-one function from $\powerset{X}$ onto $2^{X}$.

\ssubsection{Notation}

Let $A$ be a non-empty
set and $B \subset A$.
We denote the characteristic
function of $B$ in $A$ by
$\chi_{B}: A \to R$.
The Greek letter $\chi$
is a mnemonic for
\say{characteristic}.

The subscript indicates
the set on which the function
is one.
In other words, for all $B \subset A$,
$\chi_{B}^{-1}(\set{1}) = B$.

If $B$ is an interval
and
$\alpha$ is a real number
then $\alpha \chi_{B}$ is
a rectangle with
height $\alpha$.

\blankpage
