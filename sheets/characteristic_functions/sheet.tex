%!name:characteristic_functions
%!need:natural_numbers
%!need:set_powers

\ssection{Why}

We want to indicate membership in a set by a function.\footnote{Future editions will elaborate, perhaps with forward-looking connections to \sheetref{rectangular_functions}{Rectangular Functions}.}

\ssection{Definition}

The \t{characteristic function} of a set $X$ is a function from $X$ to $2$ which is $1$ if the argument is in $A$ and 0 otherwise.

The function which assigns to each subset $A$ of $X$ to characteristic function of $A$ is a one-to-one function from $\powerset{X}$ onto $2^{X}$.

\ssubsection{Notation}

Let $A$ be a non-empty
set and $B \subset A$.
We denote the characteristic
function of $B$ in $A$ by
$\chi_{B}: A \to R$.
The Greek letter $\chi$
is a mnemonic for
\say{characteristic}.

The subscript indicates
the set on which the function
is one.
In other words, for all $B \subset A$,
$\chi_{B}^{-1}(\set{1}) = B$.\footnote{Another notation used, when referring to these as \say{indicator functions,} is $1_{B}: A \to \set{0, 1}$ or $\mathbf{1}_B$.}

\blankpage
