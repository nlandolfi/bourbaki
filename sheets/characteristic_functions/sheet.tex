%!name:characteristic_functions
%!need:real_functions

\ssection{Why}

We represent
rectangles by functions.

\ssection{Definition}

The
\ct{characteristic function}{}
of a subset of some base set
is the function from
the base set to the real
numbers which maps elements
contained in the subset to
value one and maps all other
elements to zero.
The range of the funtion is
the set consisting of the
real numbers one and zero.

If the base set is the real
numbers and the subset is
an interval, then the
characteristic function
is a rectangle with height
one and the width of the interval.
% Put in indicator random variables
%We call
%the
%\ct{characteristic function}{characteristic}
%of a subset of some set
%the
%\ct{indicator function}{indicator function}.

\ssubsection{Notation}

Let $A$ be a non-empty
set and $B \subset A$.
We denote the characteristic
function of $B$ in $A$ by
$\chi_{B}: A \to R$.
The Greek letter $\chi$
is a mnemonic for
\say{characteristic}.

The subscript indicates
the set on which the function
is one.
In other words, for all $B \subset A$,
$\chi_{B}^{-1}(\set{1}) = B$.

If $B$ is an interval
and
$\alpha$ is a real number
then $\alpha \chi_{B}$ is
a rectangle with
height $\alpha$.

\blankpage
