
\section*{Why}

We generalize the notion of sequence to index sets beyond the naturals.
\section*{Definition}

A sequence is a function on the natural numbers; this set has two important properties: (a) we can order the natural numbers and (b) we can always go ``further out.''

To elaborate on property (b):
if handed two natural numbers
$m$ and $n$,
we can always find another,
for example $\max\set{m,n}+1$,
larger than $m$ and $n$.
We might think of larger as
``further out'' from the
first natural number: 1.

Now combine these two observations.
A \t{directed set} is a set $D$ with a partial order $\preceq$ satisfying one additional property: for all $a, b \in D$, there exists $c \in D$ such that $a \preceq c$ and $b \preceq c$.

A \t{net} is a function on a directed set.

\subsection*{Examples}

A sequence, then, is a net.
The directed set is the set of natural numbers and the partial order is $m \preceq n$ if $m \leq n$.

Consider $\N   \times  \N  $, and write $(a,b) \preceq (c, d)$ if $a \leq c$ \textit{and} $b \leq d$.
Clearly, $(\N  ^2, \preceq)$ so defined is a partially ordered set.
Notice that given $a = (a_1, a_2)$ and $b = (b_1, b_2)$ the point $(\max(a_1, b_1), \max(a_2, b_2))$ succeeds or is equal to both $a$ and $b$.
Thus $(\N  ^2, \preceq)$ is a directed set on which we can define a net.

\subsection*{Notation}

Directed sets involve a set and a partial order.
We commonly assume the partial order, and just denote the set.
We use the letter $D$ as a mnemonic for directed.

For nets, we use function notation and generalize sequence notation.
We denote the net $x: D \to A$ by $\{a_{\alpha }\}$, emulating notation for sequences.
The use of $\alpha $ rather than $n$ reminds us that $D$ need not be the set of natural numbers.
