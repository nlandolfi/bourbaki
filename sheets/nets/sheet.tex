%!name:nets
%!need:orders
%!need:sequences

\stitle{Nets}

\ssection{Why}
We generalize the
notion of sequence to
index sets beyond the naturals.


\ssection{Definition}

A sequence is a function
on the natural numbers;
this set has two important
properties: (a) we can order
the natural numbers
and (b) we can always go
\say{further out.}

To elaborate on property (b):
if handed two natural numbers
$m$ and $n$,
we can always find another,
for example $\max\set{m,n}+1$,
larger than $m$ and $n$.
We might think of larger as
\say{further out} from the
first natural number: 1.

Combining these to observations,
we define a directed set:
\begin{defn}
A \ct{directed set}{directedset} is a set $D$ with a partial order $\preceq$ satisfying one additional property: for all $a, b \in D$, there exists $c \in D$ such that $a \preceq c$ and $b \preceq c$.
\end{defn}

\begin{defn}
A \ct{net}{net} is a function on a directed set.
\end{defn}

A sequence, then, is a net.
The directed set is the set of natural numbers and the partial order is $m \preceq n$ if $m \leq n$.

\ssubsection{Notation}

Directed sets involve a set and a partial order.
We commonly assume the partial order, and just denote the set.
We use the letter $D$ as a mnemonic for directed.

For nets, we use function notation and generalize sequence notation.
We denote the net $x: D \to A$ by $\net{a_{\alpha}}$, emulating notation for sequences.
The use of $\alpha$ rather than $n$ reminds us that $D$ need not be the set of natural numbers.
