%!name:pair_unions
%!need:set_unions
%!refs:paul_halmos/naive_set_theory/section_04

\s{Why}

We often unite the elements of one set with another.

\s{Discussion}

Let $A$ and $B$ denote sets.
We call $\union \set{A, B}$ the \t{pair union} of $A$ and $B$.
We denote the union of the pair $\set{A, B}$ by $A \union B$.
Clearly the pair union does not depend on the order of $A$ and $B$.
In other words, $A \union B = B \union A$.

\s{Facts}

Here are some basic facts about unions of a pair of sets.\footnote{Proofs will appear in the next edition.}
Let $A$ and $B$ denote sets.

\begin{proposition}[Identity Element]
  $A \union \emptyset = A$
  \label{pair_unions:identity_element}
\end{proposition}

\begin{proposition}[Commutativity]
  $A \union B = B \union A$
\end{proposition}

\begin{proposition}[Associativity]
  $(A \union B) \union C = A \union (B \union C)$
\end{proposition}

\begin{proposition}[Idempotence]
  $A \union A = A$.
\end{proposition}

\begin{proposition}
  $A \subset B \iff A \union B = B$
\end{proposition}

\blankpage
