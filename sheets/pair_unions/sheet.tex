%!name:pair_unions
%!need:set_unions
%!refs:paul_halmos/naive_set_theory/section_04
%!refs:bert_mendelson/introduction_to_topology/theory_of_sets/set_operations

\section*{Why}

We often unite the elements of one set with another.

\section*{Discussion}

Let $A$ and $B$ denote sets.
We call $\cup \set{A, B}$ the \t{pair union} of $A$ and $B$.
We denote the union of the pair $\set{A, B}$ by $A \cup B$.
Clearly the pair union does not depend on the order of $A$ and $B$.
In other words, $A \cup B = B \cup A$.

\section*{Facts}

Here are some basic facts about unions of a pair of sets.
    \ifhmode\unskip\fi\footnote{
Proofs will appear in the next edition.
    }
Let $A$ and $B$ denote sets.

\begin{proposition}[Identity Element]

\label{pair_unions:identity_element}$A \cup \varnothing = A$\end{proposition}
\begin{proposition}[Commutativity]
$A \cup B = B \cup A$\end{proposition}
\begin{proposition}[Commutativity]
$(A \cup B) \cup C = A \cup (B \cup C)$\end{proposition}
\begin{proposition}[Idempotence]
$A \cup A = A$.\end{proposition}
\begin{proposition}
$A \subset B \iff A \cup B = B$\end{proposition}
\blankpage