
\section*{Why}
\footnote{Future editions will include, and will likely continue to higher degree equations.}
\section*{Definition}

Suppose $f: \R  \to \R $ is a first-order polynomial.
In other words, there are coefficients $\alpha , \beta  \in \R $, $\alpha  \neq 0$, so that $f(x) = \alpha  x + \beta $ for every $x \in \R $.
Then $f(x) = 0$ is a \t{first degree equation}.

We write
\[
\alpha x + \beta  = 0
\]
Notice that we can divide through by $\alpha $ to obtain,
\[
x + (\beta /\alpha ) = 0
\]

\subsection*{Solutions}

Given a real number $a \in \R $, suppose we want to find $x \in \R $ so that
\[
x + a = 0
\]
Clearly, $x = -a$ is a solution.

\blankpage