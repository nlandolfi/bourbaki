%!name:function_restrictions_and_extensions
%!need:functions
%!refs:paul_halmos/naive_set_theory/section_08
%!refs:bert_mendelson/introduction_to_topology/theory_of_sets/inverse_functions_extensions_and_restrictions

\ss{Why}

The relationship between the inclusion map and the identity map is characteristic of making small functions out of large ones.

\ss{Definition}

Let $X \subset Y$ and $f: Y \to Z$.
There is a natural function $g: X \to Z$, namely the one defined by $g(x) = f(x)$ for all $x \in X$.
We call $g$ the \t{restriction} of $f$ to $X$.
We call $f$ an \t{extension} of $g$ to $Y$.
Clearly, there may be more than one extension of a function

\ss{Notation}

We denote the restriction of $f: Y \to Z$ to the set $X \subset Y$ by $f|X$.

\ss{Example}

A simple example is the that the inclusion mapping from $X$ to $Y$ with $X \subset Y$ is a restriction of the identity map on $X$

\blankpage
