
%!name:permutations
%!need:set_numbers
%!need:groups
%!refs:sourav/153/1

\section*{Why}

We want to discuss rearranging the order of some set.

\section*{Definition}

Let $X$ be a nonempty set.
A \t{permutation} is a bijection from $X$ to $X$.

\subsection*{As a group}

It happens that the set of permutations with the operation of composition is a \textit{group} (see \sheetref{groups}{Groups}).
To see this, suppose $\pi : X \to X$ and $\sigma : X \to X$ are two permutations.
Then $\pi  \circ \sigma $ is a permutation (the composition of two invertible functions is invertible).
Also, the identity function $\id_{X}: X \to X$ is a permutation (it has an inverse, \textit{itself}).
The identity \textit{function} is the \sheetref{groups}{identity element}, since $\pi \id = \id\pi $ for all permutations $\pi $.
Since each permutation is invertible, inverse elements exist.
And the associative law is valid for permutations since function composition is associative.
We the group consisting of the set of permutations on $X$ and the operation of composition the \t{symmetric group on $X$}.

\subsection*{Finite case}

If $X$ is finite and $\num{X} = n$, then we can associate each element of $X$ with a number $\upto{n}$ and in so doing consider permutations of the set $\set{1, \dots , n}$.
This special symmetric group is called the \t{symmetric group of degree $n$}.

\subsection*{Notation}

We denote the symmetric group on $X$ by $\Sym(X)$.
It is common to denote the symmetric group of degree $n$ by $S_n$.

\blankpage