
%!name:permutations
%!need:set_numbers
%!need:groups

\section*{Why}

We want to discuss rearranging the order of some set.

\section*{Definition}

Let $X$ be a nonempty set.
A \t{permutation} is a bijection from $X$ to $X$.

Let $\pi : X \to X$ and $\sigma : X \to X$ be two permutations.
Then $\pi  \circ \sigma $ is a permutation.
Also, the identity function $\id_{X}: X \to X$ is a permutation.
In particular $\pi \id = \id\pi $ for all permutations.
Since each permutation is invertible, inverse elements exist.
And the associative law is valid for permutations since function composition is associative.
Therefore the set of permutations with the operation of composition is a group (see \sheetref{groups}{Groups}).
We call it the \t{symmetric group on $X$}.

If $X$ is finite and $\nu m{X} = n$, then we can associate each element of $X$ with a number $\upto{n}$ and in so doing consider permutations of the set $\set{1, \dots , n}$.
This special symmetric group is called the \t{symmetric group of degree $n$}.

\subsection*{Notation}

We denote the symmetric group on $X$ by $\Sym(X)$.
It is common to denote the symmetric group of degree $n$ by $S_n$.

\blankpage