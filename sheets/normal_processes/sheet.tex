%!name:normal_processes
%!need:families
%!need:random_variables

\ssection{Why}

\ssection{Definition}
Let $I$ be a (indexing) set and let $(\Omega, \mathcal{A}, \mathbfsf{P})$ be a probability space.
A \t{normal process} (more commonly, \t{gaussian process}) $x$ on $I$ is a family of real-valued random variables with the property that there exists a $m: I \to \R$ and positive definite $k: I \times I \to \R$ with the property that if $J \subset I$, $\abs{J} = d$, then $x_J \sim \mathcal{N}(m(J), k(J \times J))$.
In other words, for each $i \in I$, $x_i: \Omega \to \R$ is a random variable
And $x_J: \Omega \to \R^d$ is a Gaussian random variable.
We call $m$ is the \t{mean function} and $k$ is the \t{covariance function}

\blankpage
