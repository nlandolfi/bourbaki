\sinput{../sheet.tex}
\sbasic

\sinput{../sets/macros.tex}
\sinput{../ordered_pairs/macros.tex}
\sinput{../relations/macros.tex}

\sinput{../function_inverses/macros.tex}

\sstart

\stitle{Function Inverses}

\ssection{Why}

We want a notion
of reversing functions.

\ssection{Definition}

An
\ct{identity function}{identityfunction}
is
a relation on a set
which is functional and
reflexive.
It associates
each element in the
set with itself.
There is only one
identity function
associated to each set.

Consider
two functions
for which the codomain
of the first function
is the domain of the
second function and
the codomain of the second
function is the domain
of the first function.
These functions are
\ct{inverse functions}{inversefunctions}
if the composition of the
second with the first
is the identity
function
on the first's domain
and the composition of
the first with the second
is the identity
function on the
second's domain.

In this case we say
that the second function
is an
\ct{inverse}{aninversefunction}
of the second, and vice versa.
When an inverse exists,
it is unique, so we refer to
the \ct{inverse}{inversefunction}
of a function.

\ssubsection{Notation}

Let $A$ a non-empty set.
We denote the identity
function on $A$ by $\idfn{A}$,
read aloud as
\say{identity on $A$.}
$\idfn{A}$ maps $A$ onto $A$.

Let $A, B$ be non-empty sets.
Let $f: A \to B$ and $g: B \to A$
be functions.
$f$ and $g$ are inverse functions
if $g \comp f = \idfn{A}$
and $f \comp g = \idfn{B}$.

\ssection{The Inverse}

We discuss existence and
uniqueness of an inverse.

\begin{prop}
  Let $f: A \to B$,
  $g: B \to A$,
  and $h: B \to A$.

  If $g$ and $h$
  are both inverse
  functions of $f$,
  then $g = h$.

  \begin{proof}
  \end{proof}
\end{prop}

\begin{prop}
  If a function is one-to-one
  and onto, it has an inverse.

  \begin{proof}
  \end{proof}
\end{prop}


\strats
