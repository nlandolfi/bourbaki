
\section*{Why}

We want a notion of reversing functions.

\section*{Definition}

Reversing functions does not make sense if the function is not one-to-one.
Let $f: X \to Y$.
If $x_1$ goes to $y$ and $x_2$ goes to $y$ (i.e., $f(x_1) = f(x_2) = y$), then what should $y$ go to.
One answer is that we should have a function which gives all the domain values which could lead to $y$.
This is the inverse image (see \sheetref{function_images}{Function Images}) $f^{-1}(\set{y})$.
Nor does reversing functions make sense if $f$ is not onto.
If there does not exist $x \in X$ so that $y = f(x)$, then $f^{-1}(\set{y}) = \varnothing$.

In the case, however, that the function is one-to-one and onto, then each element of the domain corresponds to one and only one element of the codomain and vice versa.
In this case, for all $y \in Y$, $f^{-1}(\set{y})$ is a singleton $\set{x}$ where $f(x) = y$.
In this case, we define a fucntion $g: Y \to X$ so that $g(y) = x$ if and only if $f(x) = y$.

%<div data-littype='paragraph'>
% <div data-littype='run'> ❲% An❳ </div>
% <div data-littype='run'> ❲% \casdft{identity function}{identityfunction}❳ </div>
% <div data-littype='run'> ❲% is❳ </div>
% <div data-littype='run'> ❲% a relation on a set❳ </div>
% <div data-littype='run'> ❲% which is functional and❳ </div>
% <div data-littype='run'> ❲% reflexive.❳ </div>
% <div data-littype='run'> ❲% It associates❳ </div>
% <div data-littype='run'> ❲% each element in the❳ </div>
% <div data-littype='run'> ❲% set with itself.❳ </div>
% <div data-littype='run'> ❲% There is only one❳ </div>
% <div data-littype='run'> ❲% identity function❳ </div>
% <div data-littype='run'> ❲% associated to each set.❳ </div>
%</div>
%<div data-littype='paragraph'>
% <div data-littype='run'> In general, if we have two functions, where the codomain of
%    the first is the domain of the second, and the codomain of
%    the second is the domain of the first, we call them
%    ❬inverse functions❭ if the composition of the second with the
%    first is the identity function on the first's domain and the
%    composition of the first with the second is the identity
%    function on the second's domain (see
%    \sheetref{functions}{Functions} and
%    \sheetref{function_composites}{Function Composites}). </div>
%</div>
%<div data-littype='paragraph'>
% <div data-littype='run'> In this case we say that the second function is an
%    ❬inverse❭ of the second, and vice versa. </div>
% <div data-littype='run'> When an inverse exists, it is unique,
%    <div data-littype='footnote'>
%     <div data-littype='run'> Future editions will prove this assertion and all
%        unproven propositions herein. </div>
%    </div>
%    so we refer to ❬the inverse❭ of a function. </div>
% <div data-littype='run'> We call the first function ❬invertible❭. </div>
% <div data-littype='run'> Other names for an invertible function include ❬bijection❭. </div>
%</div>
%<div data-littype='paragraph'>
% <div data-littype='run'>     <div data-littype='section' data-litsectionlevel='2' data-litsectionnumbered='false'> Notation </div></div>
%</div>
%<div data-littype='paragraph'>
% <div data-littype='run'> Let $A$ a non-empty set. </div>
% <div data-littype='run'> We denote the identity </div>
% <div data-littype='run'> function on $A$ by $\idfn{A}$, </div>
% <div data-littype='run'> read aloud as </div>
% <div data-littype='run'> “identity on $A$.} </div>
% <div data-littype='run'> $\idfn{A}$ maps $A$ onto $A$. </div>
% <div data-littype='run'> Let $A, B$ be non-empty sets. </div>
% <div data-littype='run'> Let $f: A → B$ and $g: B → A$ </div>
% <div data-littype='run'> be functions. </div>
% <div data-littype='run'> $f$ and $g$ are inverse functions </div>
% <div data-littype='run'> if $g \comp f = \idfn{A}$ </div>
% <div data-littype='run'> and $f \comp g = \idfn{B}$. </div>
%</div>
%<div data-littype='section' data-litsectionlevel='1' data-litsectionnumbered='false'> The inverse </div>
%<!--
% We discuss existence and
% uniqueness of an inverse.

\begin{proposition}[Uniqueness]
Let $f: A \to B$,
$g: B \to A$,
and $h: B \to A$.
If $g$ and $h$
are both inverse
functions of $f$,
then $g = h$.

\end{proposition}

\begin{proposition}[Existence]
If a function is one-to-one and onto, it has an inverse; and conversely.\footnote{A proof will appear in future editions.}

\end{proposition}

\subsection*{Composites and inverses}

Let $f: X \to Y$ and $g: Y \to Z$.
Then $g^{-1}$ maps $\powerset{Z}$ to $\powerset{Y}$ and $f^{-1}$ maps $\powerset{Y}$ to $\powerset{X}$.
Then the following is immediate
{\small
\begin{proposition}
$(gf)^{-1} = f^{-1}g^{-1}$
\end{proposition}

}

%%%!name:composites_and_inverses
%%%!need:function_composites
%%%!need:function_composites
%%%!refs:paul_halmos/naive_set_theory/section_10
%% \ssection{Why}
%%
%% How does inverses relation to com
%%! I moved this to funciton images
%% \ssection{Inverse Images}
%%
%% The \casdft{inverse image}{} of a codomain
%% set under a function is the set of
%% elements which map to elements of
%% the codomain set under the function.
%% We denote the inverse image of $D \subset B$ by $f^{-1}(D)$, read aloud as \say{f inverse D.}
