%!name:function_inverses
%!need:function_composites
%!need:function_images
%!refs:paul_halmos/naive_set_theory/section_10

\ssection{Why}

We want a notion of reversing functions.

\ssection{Definition}

Reversing functions does not make sense if the function is not one-to-one.
Let $f: X \to Y$.
If $x_1$ goes to $y$ and $x_2$ goes to $y$ (i.e., $f(x_1) = f(x_2) = y$), then what should $y$ go to.
One answer is that we should have a function which gives all the domain values which could lead to $y$.
This is the inverse image (see \sheetref{function_images}{Function Images}) $f^{-1}(\set{y})$.
Nor does reversing funcitons make sense if $f$ is not onto.
If there does not exist $x \in X$ so that $y = f(x)$, then $f^{-1}(\set{y}) = \emptyset$.

In the case, however, that the function is one-to-one and onto, then each element of the domain corresponds to one and only one element of the codomain and vice versa.
In this case, for all $y \in Y$, $f^{-1}(\set{y})$ is a singleton $\set{x}$ where $f(x) = y$.
In this case, we define a fucntion $g: Y \to X$ so that $g(y) = x$ if and only if $f(x) = y$.

% An
% \casdft{identity function}{identityfunction}
% is
% a relation on a set
% which is functional and
% reflexive.
% It associates
% each element in the
% set with itself.
% There is only one
% identity function
% associated to each set.

In general, if we have two functions, where the codomain of the first is the domain of the second, and the codomain of the second is the domain of the first, we call them \t{inverse functions} if the composition of the second with the first is the identity function on the first's domain and the composition of the first with the second is the identity function on the second's domain (see \sheetref{functions}{Functions} and \sheetref{function_composites}{Function Composites}).

In this case we say that the second function is an \t{inverse} of the second, and vice versa.
When an inverse exists, it is unique,\footnote{Future editions will prove this assertion.} so we refer to \t{the inverse} of a function.
We call the first function \t{invertible}.
Other names for an invertible function include \t{bijection}.

\ssubsection{Notation}

Let $A$ a non-empty set.
We denote the identity
function on $A$ by $\idfn{A}$,
read aloud as
\say{identity on $A$.}
$\idfn{A}$ maps $A$ onto $A$.

Let $A, B$ be non-empty sets.
Let $f: A \to B$ and $g: B \to A$
be functions.
$f$ and $g$ are inverse functions
if $g \comp f = \idfn{A}$
and $f \comp g = \idfn{B}$.

\ssection{The Inverse}

We discuss existence and
uniqueness of an inverse.

\begin{prop}[Uniqueness]
  Let $f: A \to B$,
  $g: B \to A$,
  and $h: B \to A$.
  If $g$ and $h$
  are both inverse
  functions of $f$,
  then $g = h$.\footnote{Future editions will include proof.}
\end{prop}

\begin{prop}[Existence]
  If a function is one-to-one
  and onto, it has an inverse.\footnote{Future editions will include proof.}
\end{prop}

\ss{Composites and Inveres}

  Let $f: X \to Y$ and $g: Y \to Z$.
  then $g^{-1}$ maps $\powerset{Z}$ to $\powerset{Y}$ and $f^{-1}$ maps $\powerset{Y}$ to $\powerset{X}$.

\begin{proposition}
  $(gf)^{-1} = f^{-1}g^{-1}$\footnote{A proof will appear in future editions.}
\end{proposition}

%!name:composites_and_inverses
%!need:function_composites
%!need:function_composites
%!refs:paul_halmos/naive_set_theory/section_10

% \ssection{Why}
%
% How does inverses relation to com

%! I moved this to funciton images
% \ssection{Inverse Images}
%
% The \casdft{inverse image}{} of a codomain
% set under a function is the set of
% elements which map to elements of
% the codomain set under the function.
% We denote the inverse image of $D \subset B$ by $f^{-1}(D)$, read aloud as \say{f inverse D.}
