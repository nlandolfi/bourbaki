%!name:set_decompositions
%!need:pair_unions
%!need:pair_intersections
%!need:set_complements

\s{Why}

Let $E$ denote a set and let $A$ denote a set with $A \subset E$.
$A$ and $\complement{A}$ as breaking $E$ into two pieces which do not overlap.

\s{Discussion for complements}

To make this precise, let us say that by \say{breaking $E$ into two pieces} we mean that these two pieces are all of $E$.
In other words, every element of $E$ is contained either in $A$ or $\complement{A}$.
We use the language of set unions (\sheetref{set_unions}{Pair Unions}).

\begin{proposition}[Breaking]
  $A \union \complement{A} = E$
\end{proposition}

Next, let us say that \say{do not overlap} means that no elemenet of $A$ is an element of $\complement{A}$ and vice versa.
We use the language of set interserctions (see \sheetref{pair_intersections}{Pair Intersections}).

\begin{proposition}[Non-overlapping]
  $A \cap \complement{A} = \varnothing$
\end{proposition}

\s{Definition}

We call a pair $\set{A, B}$ a \t{decomposition} of $E$ if $A \cap B = \emptyset$ and $A \cup B = E$.
If $A \cap B$ we say that $\set{A, B}$ are \t{disjoint}.
If we have a set of sets $\CA$ satisfying $(A \in \CA \land B \in \CA) \implies (A \intersect B = \emptyset)$ then we call $\CA$ \t{pairwise disjoint}.

\blankpage
