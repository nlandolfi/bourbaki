
\section*{Why}

We want to express how many parts (of equal size) of a whole there are.

\section*{Definition}

A \t{fraction} (or \t{natural fraction}, \t{ratio}) is an ordered pair $(a, b)$ of natural numbers where $b \neq 0$.
$a$ is called the \t{numerator} and $b$ is called the \t{denominator}.

If $a < b$ then the fraction is called \t{proper} (a \t{proper fractiont}).
Otherwise, the fraction is called \t{improper} (an \t{improper fraction}).

The fraction is a \t{unit fraction} if $a = 1$.

\subsection*{Notation}

We denote the fraction $(a, b)$ by $a/b$ or
\[
\frac{a}{b}
\]

\subsection*{Etymology}

The word fraction comes from the Latin \textit{fractus}, meaning \textit{broken}.

\blankpage