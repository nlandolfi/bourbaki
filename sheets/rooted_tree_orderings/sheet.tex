%!name:rooted_tree_orderings
%!need:rooted_trees
%!refs:vandenberghe2014chordal

\ssection{Why}

We want to simplify common questions using orderings of a rooted tree.\footnote{This sheet is incomplete and will be updated in future editions.}

\ssection{Definition}

A \t{topological ordering} of a rooted tree is an ordering $\sigma$ for which $v \prec_{\sigma} \pa(v)$.
If there are $n$ vertices, the root has index $n$ and every other vertex has an index less than its parent.

A \t{postordering} is a topological ordering in which descendents of a vertex are given consecutive numbers.
For the postordering $\sigma$, if $\inv{\sigma}(v) = j$ and $v$ has $k$ proper descendents then the proper descendents of $v$ have are numbered consecutively from $j - k$ through $j -1$.
Figure~\ref{figure:rooted_tree_orderings:postordering} shows an example.


One can generate a postordering by numbering vertices in decreasing sequence (starting at $n$) in the order they are visited.\footnote{Future editions will clarify.}

Given an ordering, the \t{first descendent} of $v$ (which we denote $\fdesc(v)$) is the descendent with the lowest index.
Given a postordering $\sigma$, one can check whether a vertex $w$ is a proper descendent of $v$, by $\fdesc(v) \preceq_{\sigma} w \prec_{\sigma} v$.

\begin{figure}
  \centering
  \includegraphics[width=0.5\textwidth]{graphics_included/postordering}
  \caption{A postordering of a rooted tree.}
  \label{figure:rooted_tree_orderings:postordering}
\end{figure}

\blankpage
