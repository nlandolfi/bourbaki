%!name:functions
%!need:relations
%!refs:paul_halmos/naive_set_theory/section_08
%!refs:bert_mendelson/introduction_to_topology/theory_of_sets/functions
% for inclusion mappings:
%!refs:bert_mendelson/introduction_to_topology/theory_of_sets/inverse_functions_extensions_and_restrictions

\ssection{Why}

We want a notion for a correspondence between two sets.

\ssection{Definition}

A \t{function}
(or \t{mapping} or \t{map}) $f$ \t{from} a set $X$ \t{to} a set $Y$ is a relation (see \sheetref{relations}{Relations}) whose domain is $X$ and whose range is a subset of $Y$ such that for each $x \in X$, there exists a unique $y \in Y$ so that $(x, y) \in f$.

We call the unique $y \in Y$ the \t{result} of the function \t{at} the \t{argument} $x$.
We call $Y$ the \t{codomain}.
If the range is $Y$ we say that $f$ is a function from $X$ \t{onto} $Y$ (or $f$ is \t{surjective}).
If distinct elements of $X$ are mapped to distinct elements of $Y$, we say that the function is \t{one-to-one} (or $f$ is \t{injective}).

% We know functions by how the associate elements of their codomain with elements of their domain.
We say that the function \t{maps} elements from the domain to the codomain.
Since the word \say{function} and the verb \say{maps} connote activity, some authors refer to the set of ordered pairs as the \t{graph} of a function and avoid defining \say{function} in terms of sets.
% ill-define.
% ---namely, the set of ordered pairs which that function produces---and leave the concept of function undefined.
% They do not define functions in terms of sets.

\ssubsection{Notation}

Let $X$ and $Y$ denote sets.
We denote a function named $f$ whose domain is $X$ and whose codomain is $Y$ by $f: X \to Y$.
We read the notation aloud as \say{$f$ from $X$ to $Y$}.
We denote the set of all functions from $X$ to $Y$ (which is a subset of $\powerset{(X \times Y)}$) by $Y^{X}$.
A less standard but equally good notation is $X \to Y$, read aloud as \say{$A$ to $B$}.
Using the notations introduced so far, we denote that $f \in (A \to B)$ by $f: A \to B$.
We tend to denote function by lower case latin letters, especially $f$, $g$, and $h$.
$f$ is a mnemonic for function and $g$ and $h$ are nearby.

Let $f: A \to B$.
For each element $a \in A$, we denote the result of applying $f$ to $a$ by $f(a)$, read aloud \say{f of a.}
We sometimes drop the parentheses, and write the result as $f_a$, read aloud as \say{f sub a.}
Let $g: A \times B \to C$.
We often write $g(a,b)$ or $g_{ab}$ instead of $g((a,b))$.
We read $g(a, b)$ aloud as \say{g of a and b}.
We read $g_{ab}$ aloud as \say{g sub a b.}
%TODO: maps to notation
% We sometimes write $x \mapsto f(x)$ read aloud as \say{$x$ maps to $f(x)$.

\s{Examples}

If $X \subset Y$, the function $\Set{(x, y) \in X \times Y}{x = y}$ is the \t{inclusion function} of $X$ into $Y$.
We often introduce such a function as \say{the function from $X$ to $Y$ defined by $f(x) = y$}.
We mean by this that $f$ is a function and that we are specifying the appropraite ordered pairs using the statement, called \t{argument-value notation}.
The inclusion function of $X$ into $X$ is called the \t{identity function} of $X$.
If we view the identity function as a relation on $X$, it is the relation of equality on $X$.

The functions $f: (X \times Y) \to X$ defined by $f(x, y) = x$ is the \t{pair projection} of $X \times Y$ ono $X$.
Similarly $g: (X \times Y) \to Y$ defined by $g(x, y) = y$ is the pair projection of $X \times Y$ onto $Y$.
% The range of $f$ is the first coordinate projection of $f$ and the range of $g$ is the second coordinate projection of $f$ (see \sheetref{ordered_pair_projections}{Ordered Pair Projections}).
The identity function is one-to-one and onto, the inclusion functions are one-to-one but not always onto, and the pair projections are usually not one-to-one.

% What follows are the old extra sheets.
%%!name:injective_functions
%%!need:functions

% \ssection{Why}
%
% An element of the codomain may be the result
% of several elements of the domain.
% This overlapping, using an element of the
% codomain more than once, is a regular occurrence.
%
% \ssection{Definition}
%
% If a function is a unique correspondence in that
% every domain element has a different result,
% we call it \casdft{one-to-one}{one-to-one}.
% This language is meant to suggest that each
% element of the domain corresponds to one and
% exactly one element of the codomain, and vice versa.
%
% We also call such functions \casdft{injective}{}.

%%!name:surjective_functions
%%!need:functions

% \ssection{Why}
%
% The range need not equal the codomain; though it,
% like every other image, is a subset of the codomain.
%
%
% \ssection{Definition}
%
% The function \rasdft{maps}{functionmaps} to domain
% \casdft{on}{functionon} to the codomain if the
% range and codomain are equal;
% in this case we call the function \casdft{onto}{onto}.
% This language suggests that every element of
% the codomain is used by the function.
% It means that for each element of the codomain,
% we can find an element of the domain whose result is that first element of the codomain.
%
% We also call the function \casdft{surjective}{}.
%
% \ssubsection{Notation}
%
% Let $f: A \to B$.
% Using prior notation, we can
% state that $f$ is onto by
% writing $f(A) = B$.

%%!name:identity_functions
%%!need:functions

% \ssection{Why}
%
% What is an example
% of a function?
%
% \ssection{Definition}
%
% Consider the relation on
% a non-empy set which includes
% ordered pairs of elements
% from the set if the first
% element is the second element.
% In other words, two elements
% are related if and only if
% they are the same object.
%
% This relation is functional:
% Each element corresponds to
% only one element: itself.
% We call this functional relation
% the
% \casdft{identity function}{}
% of the set.
%
% \ssubsection{Notation}
%
% Let $A$ be a non-empty set.
% If $f: A \to A$ satisfies
% $f(a) = a$ for each $a \in A$,
%   then $f$ is the identity function.
%
% We denote the identity
% function on $A$ by $\id_A$.
% So $\id_A: A \to A$ satisfies
% $\id_A(a) = a$ for each $a \in A$.
