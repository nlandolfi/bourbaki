
\section*{Why}

We want a notion for a correspondence between two sets.

\section*{Definition}

A \t{function} $f$ (or \t{correspondence}, \t{mapping}, \t{map}) \t{from} a set $X$ \t{to} a set $Y$ is a \sheetref{relations}{relation}whose domain is $X$ and whose range is a subset of $Y$, such that for each $x \in X$,
    \begin{enumerate}
      \item there exists $y \in Y$ so that $(x, y) \in f$
      \item if $(x,y) \in f$ and $(x,z) \in f$, then $y = z$; where $y$ and $z$ are in $Y$
    \end{enumerate}
We often summarize these two conditions by saying: to every element $x \in X$ there corresponds a \t{unique} element $y \in Y$ so that $(x, y) \in f$.

We call this unique element $y \in Y$ the \t{result} of the function \t{at} the \t{argument} $x$.
We call $Y$ \textit{a} \t{codomain}---notice our use of the word ``a'', since the codomain is not a property of the function.
If the range is $Y$ we say that $f$ is a function from $X$ \t{onto} $Y$ (or call $f$ \t{onto}, \t{surjective}).
If distinct elements of $X$ are mapped to distinct elements of $Y$, we say that the function is \t{one-to-one} (or \t{injective}).


% We know functions by how the associate elements of their codomain with elements of their domain.  

We say that the function \t{maps} (or \t{takes}) elements from the domain to the codomain.
Since the word ``function'' and the verb ``maps'' connote activity, some authors refer to the set of ordered pairs as the \t{graph} of a function and avoid defining the term ``function'' as we have, in terms of sets.

\subsection*{Notation}

Given sets $X$ and $Y$, we abbreviate the statement that the object denoted by $f$ is a function whose domain is a $X$ and whose codomain is a set $Y$ by
\[
f: X \to Y
\]
We read the notation aloud as ``$f$ from $X$ to $Y$.''
We emphasize again that the \textit{range} of $f$ need not be $Y$, but must necessarily be a subset.

We denote by $Y^X$ the set of functions from $X$ to $Y$.
This set is contained in the power set $\powerset{(X \times Y)}$.
A reasonable but nonstandard notation is $X \to Y$, read as ``$A$ to $B$.''
All the following three statements have the same meaning:
\[
f: X \to Y, \quad f \in Y^X, \quad f \in (X \to Y).
\]

We tend to denote functions by lower case latin letters; especially $f$, $g$, and $h$.
$f$ is a mnemonic for function and $g$ and $h$ are nearby in the usual ordering of the \sheetref{letters}{Latin letters}.

Suppose $f: A \to B$.
For each element $a \in A$, we denote the result of applying $f$ to $a$ by $f(a)$, read aloud ``f of a.''
We sometimes drop the parentheses, and write the result as $f_a$, read aloud as ``f sub a.''
Let $g: A \times  B \to C$.
We often write $g(a,b)$ or $g_{ab}$ instead of $g((a,b))$.
We read $g(a, b)$ aloud as ``g of a and b''.
We read $g_{ab}$ aloud as ``g sub a b.''
%  <div data-littype='run'> ❲%TODO: maps to notation❳ </div>
%  <div data-littype='run'> ❲% We sometimes write $x ↦ f(x)$ read aloud as \say{$x$ maps to $f(x)$.❳ </div>
%    


\subsection*{Examples}

If $X \subset Y$, the function $\Set{(x, y) \in X \times  Y}{x = y}$ is the \t{inclusion function} of $X$ into $Y$.
We often introduce such a function as ``the function from $X$ to $Y$ defined by $f(x) = y$''.
We mean by this that $f$ is a function and that we are specifying the appropraite ordered pairs using the statement, called \t{argument-value notation}.
The inclusion function of $X$ into $X$ is called the \t{identity function} of $X$.
If we view the identity function as a relation on $X$, it is the relation of equality on $X$.

The functions $f: (X \times  Y) \to X$ defined by $f(x, y) = x$ is the \t{pair projection} of $X \times  Y$ ono $X$.
Similarly $g: (X \times  Y) \to Y$ defined by $g(x, y) = y$ is the pair projection of $X \times  Y$ onto $Y$.
% The range of $f$ is the first coordinate projection of $f$ and the range of $g$ is the second coordinate projection of $f$ (see \sheetref{ordered_pair_projections}{Ordered Pair Projections}). 

The identity function is one-to-one and onto, the inclusion functions are one-to-one but not always onto, and the pair projections are usually not one-to-one.

%<div data-littype='paragraph'>
%  <div data-littype='run'> ❲% What follows are the old extra sheets.❳ </div>
%  <div data-littype='run'> ❲%%!name:injective_functions❳ </div>
%  <div data-littype='run'> ❲%%!need:functions❳ </div>
%</div>
%<div data-littype='paragraph'>
%  <div data-littype='run'> ❲% \ssection{Why}❳ </div>
%  <div data-littype='run'> ❲%❳ </div>
%  <div data-littype='run'> ❲% An element of the codomain may be the result❳ </div>
%  <div data-littype='run'> ❲% of several elements of the domain.❳ </div>
%  <div data-littype='run'> ❲% This overlapping, using an element of the❳ </div>
%  <div data-littype='run'> ❲% codomain more than once, is a regular occurrence.❳ </div>
%  <div data-littype='run'> ❲%❳ </div>
%  <div data-littype='run'> ❲% \ssection{Definition}❳ </div>
%  <div data-littype='run'> ❲%❳ </div>
%  <div data-littype='run'> ❲% If a function is a unique correspondence in that❳ </div>
%  <div data-littype='run'> ❲% every domain element has a different result,❳ </div>
%  <div data-littype='run'> ❲% we call it \casdft{one-to-one}{one-to-one}.❳ </div>
%  <div data-littype='run'> ❲% This language is meant to suggest that each❳ </div>
%  <div data-littype='run'> ❲% element of the domain corresponds to one and❳ </div>
%  <div data-littype='run'> ❲% exactly one element of the codomain, and vice versa.❳ </div>
%  <div data-littype='run'> ❲%❳ </div>
%  <div data-littype='run'> ❲% We also call such functions \casdft{injective}{}.❳ </div>
%</div>
%<div data-littype='paragraph'>
%  <div data-littype='run'> ❲%%!name:surjective_functions❳ </div>
%  <div data-littype='run'> ❲%%!need:functions❳ </div>
%</div>
%<div data-littype='paragraph'>
%  <div data-littype='run'> ❲% \ssection{Why}❳ </div>
%  <div data-littype='run'> ❲%❳ </div>
%  <div data-littype='run'> ❲% The range need not equal the codomain; though it,❳ </div>
%  <div data-littype='run'> ❲% like every other image, is a subset of the codomain.❳ </div>
%  <div data-littype='run'> ❲%❳ </div>
%  <div data-littype='run'> ❲%❳ </div>
%  <div data-littype='run'> ❲% \ssection{Definition}❳ </div>
%  <div data-littype='run'> ❲%❳ </div>
%  <div data-littype='run'> ❲% The function \rasdft{maps}{functionmaps} to domain❳ </div>
%  <div data-littype='run'> ❲% \casdft{on}{functionon} to the codomain if the❳ </div>
%  <div data-littype='run'> ❲% range and codomain are equal;❳ </div>
%  <div data-littype='run'> ❲% in this case we call the function \casdft{onto}{onto}.❳ </div>
%  <div data-littype='run'> ❲% This language suggests that every element of❳ </div>
%  <div data-littype='run'> ❲% the codomain is used by the function.❳ </div>
%  <div data-littype='run'> ❲% It means that for each element of the codomain,❳ </div>
%  <div data-littype='run'> ❲% we can find an element of the domain whose result is that first element of the codomain.❳ </div>
%  <div data-littype='run'> ❲%❳ </div>
%  <div data-littype='run'> ❲% We also call the function \casdft{surjective}{}.❳ </div>
%  <div data-littype='run'> ❲%❳ </div>
%  <div data-littype='run'> ❲% \ssubsection{Notation}❳ </div>
%  <div data-littype='run'> ❲%❳ </div>
%  <div data-littype='run'> ❲% Let $f: A → B$.❳ </div>
%  <div data-littype='run'> ❲% Using prior notation, we can❳ </div>
%  <div data-littype='run'> ❲% state that $f$ is onto by❳ </div>
%  <div data-littype='run'> ❲% writing $f(A) = B$.❳ </div>
%</div>
%<div data-littype='paragraph'>
%  <div data-littype='run'> ❲%%!name:identity_functions❳ </div>
%  <div data-littype='run'> ❲%%!need:functions❳ </div>
%</div>
%<div data-littype='paragraph'>
%  <div data-littype='run'> ❲% \ssection{Why}❳ </div>
%  <div data-littype='run'> ❲%❳ </div>
%  <div data-littype='run'> ❲% What is an example❳ </div>
%  <div data-littype='run'> ❲% of a function?❳ </div>
%  <div data-littype='run'> ❲%❳ </div>
%  <div data-littype='run'> ❲% \ssection{Definition}❳ </div>
%  <div data-littype='run'> ❲%❳ </div>
%  <div data-littype='run'> ❲% Consider the relation on❳ </div>
%  <div data-littype='run'> ❲% a non-empy set which includes❳ </div>
%  <div data-littype='run'> ❲% ordered pairs of elements❳ </div>
%  <div data-littype='run'> ❲% from the set if the first❳ </div>
%  <div data-littype='run'> ❲% element is the second element.❳ </div>
%  <div data-littype='run'> ❲% In other words, two elements❳ </div>
%  <div data-littype='run'> ❲% are related if and only if❳ </div>
%  <div data-littype='run'> ❲% they are the same object.❳ </div>
%  <div data-littype='run'> ❲%❳ </div>
%  <div data-littype='run'> ❲% This relation is functional:❳ </div>
%  <div data-littype='run'> ❲% Each element corresponds to❳ </div>
%  <div data-littype='run'> ❲% only one element: itself.❳ </div>
%  <div data-littype='run'> ❲% We call this functional relation❳ </div>
%  <div data-littype='run'> ❲% the❳ </div>
%  <div data-littype='run'> ❲% \casdft{identity function}{}❳ </div>
%  <div data-littype='run'> ❲% of the set.❳ </div>
%  <div data-littype='run'> ❲%❳ </div>
%  <div data-littype='run'> ❲% \ssubsection{Notation”❳ </div>
%  <div data-littype='run'> ❲%❳ </div>
%  <div data-littype='run'> ❲% Let $A$ be a non-empty set.❳ </div>
%  <div data-littype='run'> ❲% If $f: A → A$ satisfies❳ </div>
%  <div data-littype='run'> ❲% $f(a) = a$ for each $a ∈ A$,❳ </div>
%  <div data-littype='run'> ❲%   then $f$ is the identity function.❳ </div>
%  <div data-littype='run'> ❲%❳ </div>
%  <div data-littype='run'> ❲% We denote the identity❳ </div>
%  <div data-littype='run'> ❲% function on $A$ by $\id_A$.❳ </div>
%  <div data-littype='run'> ❲% So $\id_A: A → A$ satisfies❳ </div>
%  <div data-littype='run'> ❲% $\id_A(a) = a$ for each $a ∈ A$.❳ </div>
%  <div data-littype='run'> </div>
%</div>

%macros.tex
%%%%% MACROS %%%%%%%%%%%%%%%%%%%%%%%%%%%%%%%%%%%%%%%%%%%%%%%
%\newcommand{\id}{\mathword{id}}
%%%%%%%%%%%%%%%%%%%%%%%%%%%%%%%%%%%%%%%%%%%%%%%%%%%%%%%%%%%%
