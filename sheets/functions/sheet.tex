%!name:functions
%!need:relations
%!refs:paul_halmos/naive_set_theory/section_08

\ssection{Why}

We want a notion for a correspondence between two sets.

\ssection{Definition}

A \t{function} $f$ \t{from} a set $X$ \t{to} a set $Y$ is a relation (see \sheetref{relations}{Relations}) whose domain is $X$ and whose range is a subset of $Y$ such that for each $x \in X$, there exists a unique $y \in Y$ so that $(x, y) \in f$.
We call the unique $y \in Y$ the \t{result} of the function \t{at} the \t{argument} $x$.
We call $Y$ the \t{codomain}.
If the range is $Y$ we say that $f$ is a function from $X$ \t{onto} $Y$.

We know functions by how the associate elements of their codomain with elements of their domain.
We say that the function \t{maps} elements from the domain to the codomain.
Since the word function and our language of using the verb \say{maps} connote activity, some authors refer to the concept that we have defined as a function as the \t{graph} of a function---namely, the set of ordered pairs which that function produces---and leave the concept of function undefined.

\ssubsection{Notation}

Let $X$ and $Y$ denote sets.
We denote a function named $f$ whose domain is $X$ and whose codomain is $Y$ by $f: X \to Y$.
We read the notation aloud as \say{$f$ from $X$ to $Y$}.
We denote the set of all functions from $X$ to $Y$ (which is a subset of $\powerset{(X \times Y)}$) by $Y^{X}$.
A less standard but equally good notation is $X \to Y$, read aloud as \say{$A$ to $B$}.
Using the notations introduced so far, we denote that $f \in (A \to B)$ by $f: A \to B$.

We tend to denote function by lower case latin letters, especially $f$, $g$, and $h$.
The letter $f$ is a mnemonic for function and $g$ and $h$ follow $f$ in the latin alphabet.

Let $f: A \to B$.
For each element $a \in A$, we denote the result of applying $f$ to $a$ by $f(a)$, read aloud \say{f of a.}
We sometimes drop the parentheses, and write the result as $f_a$, read aloud as \say{f sub a.}

Let $g: A \times B \to C$.
We often write $g(a,b)$ or $g_{ab}$ instead of $g((a,b))$.
We read $g(a, b)$ aloud as \say{g of a and b}.
We read $g_{ab}$ aloud as \say{g sub a b.}

\s{A first example: inclusion map}

If $X$ is a subset of $Y$, the function
\[
  \Set{(x, y) \in X \times Y}{x = y}
\]
is the \t{inclusion map} of $X$ into $Y$.
We often introduce such a function as \say{the function from $X$ to $Y$ defined by $f(x) = y$}.
We mean by this that $f$ is a function and that we are specifying the appropraite ordered pairs using the statement.
Some authors call this the \t{argument-value notation}.

The inclusion map of $X$ into $X$ is called the \t{identity map} of $X$.
If we view the identity map as a relation on $X$, it is the relation of equality on $X$.
