%!name:simple_functions
%!need:characteristic_functions
%!need:real_functions

\ssection{Why}

We want to define area under a real function.
We start with defining functions for which this notion is obvious.

\ssection{Definition}

A \t{simple function} (or \t{step function}) is a real-valued function whose range is a finite set.
We can write simple function as the sum of the characteristic functions fo the inverse image elements.

We can partition the range of the function into a finite family of one-elements sets.
Then the family whose members are the inverse images of these sets partitions the domain.
We call this the \t{simple partition} of the function.

\ssubsection{Notation}

We denote the set of simple functions on $A$ by $\SimpleF(A)$.
We denote subset of non-negative simple real functions with domain $A$ by by $SimpleF_+(A)$.

Let $f \in \SimpleF(A)$.
Let $\set{a_1, \dots, a_n} = f(A)$.
Define $A_i = f^{-1}(\set{a_i})$.
Then $f = \sum_{i = 1}^{n} a_i \chi_{A_i}$.

\blankpage
