%!name:set_products
%!need:ordered_pairs
%!need:set_powers
%!refs:paul_halmos/naive_set_theory/section_06
%!refs:bert_mendelson/introduction_to_topology/theory_of_sets/products_of_sets

\ssection{Why}

Does a set exist of all the ordered pairs of elements from an ordered pair of sets?

\ssection{Definition}

Let $A$ and $B$ denote sets.
Ordered pairs are sets of singletons and pairs.
So to construct the set of all ordered pairs taken from two sets, we want to specify the elements of a set which contains all singletons $\set{a}$ and pairs $\set{a, b}$ for $a \in A$, $b \in B$.

Notice that $a \in A$ and $b \in A$ mean $a, b \in (A \union B)$.
In other words, $\set{a} \subset A$ and $\set{b} \subset B$ and $\set{a}, \set{b} \subset (A \union B)$.
In particular, $\set{a} \in \powerset{(A \union B)}$.
Similarly, $\set{a, b} \in \powerset{(A \union B)}$.
And so $\set{\set{a}, \set{a, b}} \in \powerset{\powerset{(A \union B)}}$.

We define the set of \say{all ordered pairs} from $A$ and $B$ by specifying the appropriate pairs of this set.\footnote{The specific statement used here requires some translation. A discussion of this and the full statement will appear in a future edition.}
\[
  \Set{(a, b) \in \powerset{\powerset{(A \union B)}}}{a \in A \land b \in B}
\]
We name this set the \t{product} of the set denoted by $A$ and the set denoted by $B$ is the set of all ordered pairs.
This set is also called the \t{set product} (or \t{cartesian product}\footnote{This second term is universal, but avoided in accordance with the project policy on naming.}).
If $A \neq B$, the ordering causes the product of $A$ and $B$ to differ from the product of
$B$ with $A$.
If $A = B$, however, the symmetry holds.

\ssubsection{Notation}

We denote the product of $A$ with $B$ by $A \cross B$, read aloud as \say{A cross B.}
In this notation, if $A \neq B$, then $A \cross B \neq B \cross A$.\footnote{Future editions may include a table figure visualizing the product.}
