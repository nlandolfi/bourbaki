
\section*{Definition}

An optimization problem is $(X, f)$ is an \t{inequality constrained space optimization problem} if $X \subset \R ^n$, $f: \R ^n \to \R $, and there exists $g: \R ^n \to \R ^m$ so that
\[
X = \Set{x \in \R ^n}{g(x) \leq 0}
\]
For this reason, $(f, g)$ is sometimes called the \t{problem data} (\t{abstract problem data}) of the problem.

\subsection*{Notation}

We often write such problems as: given $f: \R ^n \to \R $ and $g: \R ^n \to \R ^m$, find $x \in \R ^n$ to
\[
\begin{aligned}
\text{minimize} & \quad f(x) \\
\text{subject to} & \quad g(x) \leq 0 \\
\end{aligned}
\]
Some authors abbreviate inequality constrained space optimization problem as ICP.

\subsection*{Handles equality constraints}

Suppose $f: \R ^n \to \R $ and $h: \R ^n \to \R ^m$ are the (abstract) problem data for an \textit{equality} constrained space optimization problem.
Define $g: \R ^n \to \R ^{2m}$ so that
\[
g(x) = (h(x), -h(x)) \quad \text{for all } x \in \R ^n
\]
Then the ECP $(f, h)$ and ICP $(f, g)$ have the same feasible set and optimal solutions.
In other words, given an equality constrained problem we can always write it as an inequality constrained problem.

\blankpage