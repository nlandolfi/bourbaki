
%!name:entire_functions
%!need:complex_analytic_functions
%!refs:yellow/IX/4

\section*{Definition}

An \t{entire function} is a complex function $f: \C  \to \C $ which is analytic for all $z \in \C $.

\blankpage
\sbasic
%%%% MACROS %%%%%%%%%%%%%%%%%%%%%%%%%%%%%%%%%%%%%%%%%%%%%%%

\newcommand{\PM}{\mathbf{P}}

%%%%%%%%%%%%%%%%%%%%%%%%%%%%%%%%%%%%%%%%%%%%%%%%%%%%%%%%%%%

%%%% MACROS %%%%%%%%%%%%%%%%%%%%%%%%%%%%%%%%%%%%%%%%%%%%%%%

\newcommand{\PM}{\mathbf{P}}

%%%%%%%%%%%%%%%%%%%%%%%%%%%%%%%%%%%%%%%%%%%%%%%%%%%%%%%%%%%

%%%% MACROS %%%%%%%%%%%%%%%%%%%%%%%%%%%%%%%%%%%%%%%%%%%%%%%

\newcommand{\PM}{\mathbf{P}}

%%%%%%%%%%%%%%%%%%%%%%%%%%%%%%%%%%%%%%%%%%%%%%%%%%%%%%%%%%%

%%%% MACROS %%%%%%%%%%%%%%%%%%%%%%%%%%%%%%%%%%%%%%%%%%%%%%%

% use \set{stuff} for { stuff }
% use \set* for autosizing delimiters
\DeclarePairedDelimiter{\set}{\{}{\}}

% use \Set{a}{b} for {a | b}
% use \Set* for autosizing delimiters
\DeclarePairedDelimiterX{\Set}[2]{\{}{\}}{#1 \nonscript\;\delimsize\vert\nonscript\; #2}

% use \powerset{A} for power set of A
\newcommand{\powerset}[1]{2^{#1}}

\renewcommand{\emptyset}{\varnothing}

\newcommand{\SA}{\mathcal{A}}
\newcommand{\SB}{\mathcal{B}}
\newcommand{\SC}{\mathcal{C}}
\newcommand{\SD}{\mathcal{D}}
\newcommand{\SE}{\mathcal{E}}
\newcommand{\SF}{\mathcal{F}}
\newcommand{\SG}{\mathcal{G}}
\newcommand{\SH}{\mathcal{H}}
\newcommand{\SI}{\mathcal{I}}
\newcommand{\SJ}{\mathcal{J}}
\newcommand{\SK}{\mathcal{K}}
\newcommand{\SL}{\mathcal{L}}

%%%%%%%%%%%%%%%%%%%%%%%%%%%%%%%%%%%%%%%%%%%%%%%%%%%%%%%%%%%

%%%% MACROS %%%%%%%%%%%%%%%%%%%%%%%%%%%%%%%%%%%%%%%%%%%%%%%

\newcommand{\PM}{\mathbf{P}}

%%%%%%%%%%%%%%%%%%%%%%%%%%%%%%%%%%%%%%%%%%%%%%%%%%%%%%%%%%%

%%%% MACROS %%%%%%%%%%%%%%%%%%%%%%%%%%%%%%%%%%%%%%%%%%%%%%%

\newcommand{\PM}{\mathbf{P}}

%%%%%%%%%%%%%%%%%%%%%%%%%%%%%%%%%%%%%%%%%%%%%%%%%%%%%%%%%%%

%%%% MACROS %%%%%%%%%%%%%%%%%%%%%%%%%%%%%%%%%%%%%%%%%%%%%%%

\newcommand{\PM}{\mathbf{P}}

%%%%%%%%%%%%%%%%%%%%%%%%%%%%%%%%%%%%%%%%%%%%%%%%%%%%%%%%%%%

%%%% MACROS %%%%%%%%%%%%%%%%%%%%%%%%%%%%%%%%%%%%%%%%%%%%%%%

\newcommand{\PM}{\mathbf{P}}

%%%%%%%%%%%%%%%%%%%%%%%%%%%%%%%%%%%%%%%%%%%%%%%%%%%%%%%%%%%

%%%% MACROS %%%%%%%%%%%%%%%%%%%%%%%%%%%%%%%%%%%%%%%%%%%%%%%

\newcommand{\PM}{\mathbf{P}}

%%%%%%%%%%%%%%%%%%%%%%%%%%%%%%%%%%%%%%%%%%%%%%%%%%%%%%%%%%%

%%%% MACROS %%%%%%%%%%%%%%%%%%%%%%%%%%%%%%%%%%%%%%%%%%%%%%%

\newcommand{\PM}{\mathbf{P}}

%%%%%%%%%%%%%%%%%%%%%%%%%%%%%%%%%%%%%%%%%%%%%%%%%%%%%%%%%%%

%%%% MACROS %%%%%%%%%%%%%%%%%%%%%%%%%%%%%%%%%%%%%%%%%%%%%%%

\newcommand{\PM}{\mathbf{P}}

%%%%%%%%%%%%%%%%%%%%%%%%%%%%%%%%%%%%%%%%%%%%%%%%%%%%%%%%%%%

%%%% MACROS %%%%%%%%%%%%%%%%%%%%%%%%%%%%%%%%%%%%%%%%%%%%%%%

\newcommand{\PM}{\mathbf{P}}

%%%%%%%%%%%%%%%%%%%%%%%%%%%%%%%%%%%%%%%%%%%%%%%%%%%%%%%%%%%

%%%% MACROS %%%%%%%%%%%%%%%%%%%%%%%%%%%%%%%%%%%%%%%%%%%%%%%

\newcommand{\PM}{\mathbf{P}}

%%%%%%%%%%%%%%%%%%%%%%%%%%%%%%%%%%%%%%%%%%%%%%%%%%%%%%%%%%%

\newcommand{\union}{\,\cup\,}
\newcommand{\bunion}{\bigcup}
\newcommand{\intersect}{\cap}
\newcommand{\intersection}{\cap}
\newcommand{\bintersection}{\bigcap}
\newcommand{\symdiff}{\Delta}

%%%% MACROS %%%%%%%%%%%%%%%%%%%%%%%%%%%%%%%%%%%%%%%%%%%%%%%

\newcommand{\PM}{\mathbf{P}}

%%%%%%%%%%%%%%%%%%%%%%%%%%%%%%%%%%%%%%%%%%%%%%%%%%%%%%%%%%%

%%%% MACROS %%%%%%%%%%%%%%%%%%%%%%%%%%%%%%%%%%%%%%%%%%%%%%%

\newcommand{\PM}{\mathbf{P}}

%%%%%%%%%%%%%%%%%%%%%%%%%%%%%%%%%%%%%%%%%%%%%%%%%%%%%%%%%%%

%%%% MACROS %%%%%%%%%%%%%%%%%%%%%%%%%%%%%%%%%%%%%%%%%%%%%%%

\newcommand{\PM}{\mathbf{P}}

%%%%%%%%%%%%%%%%%%%%%%%%%%%%%%%%%%%%%%%%%%%%%%%%%%%%%%%%%%%

%%%% MACROS %%%%%%%%%%%%%%%%%%%%%%%%%%%%%%%%%%%%%%%%%%%%%%%

\newcommand{\PM}{\mathbf{P}}

%%%%%%%%%%%%%%%%%%%%%%%%%%%%%%%%%%%%%%%%%%%%%%%%%%%%%%%%%%%

%%%% MACROS %%%%%%%%%%%%%%%%%%%%%%%%%%%%%%%%%%%%%%%%%%%%%%%

\newcommand{\PM}{\mathbf{P}}

%%%%%%%%%%%%%%%%%%%%%%%%%%%%%%%%%%%%%%%%%%%%%%%%%%%%%%%%%%%

%%%% MACROS %%%%%%%%%%%%%%%%%%%%%%%%%%%%%%%%%%%%%%%%%%%%%%%

\newcommand{\PM}{\mathbf{P}}

%%%%%%%%%%%%%%%%%%%%%%%%%%%%%%%%%%%%%%%%%%%%%%%%%%%%%%%%%%%

\sstart
\stitle{Measure Properties}

\ssection{Why}

We expect measure to have the
common sense properties we stated
when trying to define a notion of
length for the real line.

\ssection{Monotonicity}

An extended-real-valued function on an
alebra is
\ct{monotone}{}
if, given a first distinguished set contained in
a distinguished second set, the result of the
first is no greater than the result of the second.

\begin{prop}
  All measures are monotone.
  \begin{proof}
    Let $(A, \mathcal{A}, \mu)$ be
    a measure space.
    Let $A, B \in \mathcal{A}$
    and $A \subset B$.
    Then $B = A \union (B - A)$,
    a disjoint union.
    So
    \[
      \mu(B) = \mu(A \union (B - A))= \mu(A) + \mu(B - A),
    \]
    by the additivity of $\mu$.
    Since $\mu(B - A) \geq 0$,
    we conclude $\mu(A) \leq \mu(B)$.
  \end{proof}
\end{prop}

\begin{prop}
  $A \subset B$ and $B$ finite
  means $\mu(B - A) = \mu(B) - \mu(A)$.
  \boxed{TODO}
\end{prop}

\ssection{Subadditivity}


Monotonicity along with additivity of
measures give us one other convenient
property: subadditivity.

An extended-real-valued function on an algebra is
\ct{subadditive}{}
if, given a sequence of distinguished sets, the
result of union of the sequence is no greater than
the limit of the partial sums of the results on
each element of the sequence.

\begin{prop}
  All measures are subadditive.
  \begin{proof}
    Let $(A, \mathcal{A}, \mu)$ be a measure space.

    Let $\set{A_n} \subset \mathcal{A}$.
    Define $\set{B_n} \subset \mathcal{A}$
    with $B_n := A_n - \union_{i = 1}^{n-1} A_i$.
    Then
    $\union_n A_n = \union_n B_n$,
    $\set{B_n}$ is a disjoint sequence, and
    $B_n \subset A_n$ for each $n$.
    So
    \[
      \mu(\union_{n} A_n) = \mu(\union_{n} B_n) = \sum_{i = 1}^{\infty} \mu(B_n) \leq \sum_{i = 1}^{\infty} \mu(A_n),
    \]
    by additivity and then montonicity of measure.
  \end{proof}

\end{prop}

\ssection{Limits}
Measures also behave well under limits.

An extended-real-valued function on an algebra
\ct{resolves under increasing limits}{}
if the result of the union of an increasing sequence
of distinguished sets coincides with the
limit of the sequence of results on the individual
sets.
An extended-real-valued function on an algebra
\ct{resolves under decreasing limits}{}
if the result of the intersection of a decreasing sequence
of distinguished sets coincides with the
limit of the sequence of results on the individual
sets.


\begin{prop}
  Measures resolve under increasing limits.
  \begin{proof}
    Let $(A, \mathcal{A}, \mu)$ be
    a measure space.
    Let $\set{A_n}$
    be an increasing sequence in $\mathcal{A}$.
    Then we want to show:
    $\mu(\union_{n} A_n) = \lim_{n \to \infty} \mu(A_n)$.

    Define $\set{B_n}$ such that
    $B_n := A_n - \union_{i = 1}^{n-1} A_i$.
    Then
    $\set{B_n}$ is disjoint,
    $A_n = \union_{i = 1}^{n} B_i$ for
    each $n$,
    $\union_n A_n = \union_n B_n$, and
    $\mu(\union_{i = 1}^{n} B_i)
    = \sum_{i = 1}^{n} \mu(B_i)$,
    by additivity.
    So
    \[
      \begin{aligned}
        \mu(\union_n A_n)
        &= \mu(\union_n B_n) \\
        &= \lim_{n \to \infty} \sum_{i = 1}^{n} \mu(B_i) \\
        &= \lim_{n \to \infty} \mu(\union_{i = 1}^{n} B_i) \\
        &= \lim_{n \to \infty} \mu(A_n).
      \end{aligned}
    \]
  \end{proof}
\end{prop}

\begin{prop}
  Measures resolve under decreasing
  limits if there is a finite set in
  the decreasing sequence.
  \begin{proof}
    Let $(A, \mathcal{A}, \mu)$ be
    a measure space.
    Let $\set{A_n}$
    be a decreasing sequence in $\mathcal{A}$
    with one element finite.
    Then we want to show:
    $\mu(\intersection_n A_n) = \lim_{n \to \infty} \mu(A_n)$.

    On one hand,
    let $n_0$ be the index of the first
    finite element of the sequence.
    Then for all $n \geq n_0$,
    the sequence is finite because
    of the monotonicity of measure.
    Denote this decreasing finite subsequence
    of sets by $\set{B_n}$.
    Then $\intersect_{n} A_n = \intersect_n B_n$
    and $\lim_n A_n = \lim_n B_n$.

    On the other hand,
    the sequence $\set{B_1 - B_n}$ is an
    increasing sequence in $\mathcal{A}$.
    Also $\intersect_n B_n = B_1 - \union_n (B_1 - B_n)$.
    So
    \[
    \begin{aligned}
      \mu(\intersect_n B_n)
      &= \mu(B_1 - \union_n (B_1 - B_n)) \\
      &= \mu(B_1) - \mu(\union_n (B_1 - B_n)) \\
      &= \mu(B_1) - \lim_{n} \mu(B_1 - B_n) \\
      &= \mu(B_1) - \left(\lim_{n} \mu(B_1) - \mu(B_n)\right) \\
      &= \lim_{n} B_n.
    \end{aligned}
    \]
  \end{proof}
\end{prop}
\strats
