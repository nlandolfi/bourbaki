
%!name:measure_properties
%!need:measures

\section*{Why}

We expect measure to have the common sense properties we stated when trying to define a notion of length for the real line.

\section*{Monotonicity}

An extended-real-valued function on an alebra is \t{monotone} if, given a first distinguished set contained in a distinguished second set, the result of the first is no greater than the result of the second.

\begin{proposition}
All measures are monotone.\end{proposition}
\begin{proof}Let $(A, \mathcal{A} , \mu )$ be a measure space.
Let $A, B \in \mathcal{A} $ and $A \subset B$.
Then $B = A \cup (B - A)$, a disjoint union.
So
  \[
\mu (B) = \mu (A \cup (B - A))= \mu (A) + \mu (B - A),
  \]
by the additivity of $\mu $.
Since $\mu (B - A) \geq 0$,
we conclude $\mu (A) \leq \mu (B)$.\end{proof}
\begin{proposition}
If $A \subset B$ and $B$ finite, then $\mu (B - A) = \mu (B) - \mu (A)$.\end{proposition}
\section*{Subadditivity}

Monotonicity along with additivity of measures give us one other convenient property: subadditivity.

An extended-real-valued function on an algebra is \t{subadditive} if, given a sequence of distinguished sets, the result of union of the sequence is no greater than the limit of the partial sums of the results on each element of the sequence.

\begin{proposition}
All measures are subadditive.\end{proposition}
\begin{proof}Let $(A, \mathcal{A} , \mu )$ be a measure space.
Let $\set{A_n} \subset \mathcal{A} $.
Define $\set{B_n} \subset \mathcal{A} $ with $B_n := A_n - \cup_{i = 1}^{n-1} A_i$.
Then $\cup_n A_n = \cup_n B_n$, $\set{B_n}$ is a disjoint sequence, and $B_n \subset A_n$ for each $n$.
So
    \[
\mu (\cup_{n} A_n) = \mu (\cup_{n} B_n) = \sum_{i = 1}^{\infty} \mu (B_n) \leq \sum_{i = 1}^{\infty} \mu (A_n),
    \]
by additivity and then montonicity of measure.
\end{proof}
The inequality involved in subadditivity is sometimes called \t{Boole's inequality} or \t{Bonferroni's inequalities} or the \t{union bound}; each of these terms is most common when discussing \sheetref{probability_measures}{Probability Measures}.

\section*{Limits}

Measures also behave well under limits.

An extended-real-valued function on an algebra \t{resolves under increasing limits} if the result of the union of an increasing sequence of distinguished sets coincides with the limit of the sequence of results on the individual sets.
An extended-real-valued function on an algebra \t{resolves under decreasing limits} if the result of the intersection of a decreasing sequence of distinguished sets coincides with the limit of the sequence of results on the individual sets.

\begin{proposition}
All measures resolve under increasing limits.\end{proposition}
\begin{proof}Let $(A, \mathcal{A} , \mu )$ be a measure space.
Let $\set{A_n}$ be an increasing sequence in $\mathcal{A} $.
Then we want to show:
$\mu (\union_{n} A_n) = \lim_{n \to \infty} \mu (A_n)$.
Define $\set{B_n}$ such that
$B_n := A_n - \union_{i = 1}^{n-1} A_i$.
Then $\set{B_n}$ is disjoint, $A_n = \cup_{i = 1}^{n} B_i$ for each $n$, $\cup_n A_n = \cup_n B_n$, and $\mu (\cup_{i = 1}^{n} B_i) = \sum_{i = 1}^{n} \mu (B_i)$, by additivity.
So
    \[
\begin{aligned}
\mu (\cup_n A_n)
&= \mu (\cup_n B_n) \\
&= \lim_{n \to \infty} \sum_{i = 1}^{n} \mu (B_i) \\
&= \lim_{n \to \infty} \mu (\cup_{i = 1}^{n} B_i) \\
&= \lim_{n \to \infty} \mu (A_n).
\end{aligned}
    \]
\end{proof}
\begin{proposition}
Measures resolve under decreasing
limits if there is a finite set in
the decreasing sequence.\end{proposition}
\begin{proof}Let $(A, \mathcal{A} , \mu )$ be a measure space.
Let $\set{A_n}$ be a decreasing sequence in $\mathcal{A} $ with one element finite.
Then we want to show:
$\mu (\cap _n A_n) = \lim_{n \to \infty} \mu (A_n)$.
On one hand,
let $n_0$ be the index of the first
finite element of the sequence.
Then for all $n \geq n_0$,
the sequence is finite because
of the monotonicity of measure.
Denote this decreasing finite subsequence
of sets by $\set{B_n}$.
Then $\cap _{n} A_n = \cap _n B_n$
and $\lim_n A_n = \lim_n B_n$.

On the other hand,
the sequence $\set{B_1 - B_n}$ is an
increasing sequence in $\mathcal{A} $.
Also $\cap _n B_n = B_1 - \cup_n (B_1 - B_n)$.
So
    \[
\begin{aligned}
\mu (\cap _n B_n)
&= \mu (B_1 - \cup_n (B_1 - B_n)) \\
&= \mu (B_1) - \mu (\cup_n (B_1 - B_n)) \\
&= \mu (B_1) - \lim_{n} \mu (B_1 - B_n) \\
&= \mu (B_1) - \left(\lim_{n} \mu (B_1) - \mu (B_n)\right) \\
&= \lim_{n} B_n.
\end{aligned}
    \]
\end{proof}