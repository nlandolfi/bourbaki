
%!name:real_convex_functions
%!need:real_convex_sets
%!need:real_functions

\section*{Why}

We speak of functions which always bends up.\footnote{Future editions may expand.}

\section*{Definition}

Suppose $X \subset \R $ is a convex set. A function $f: A \to \R $ is \t{convex} if
    \[
f(tx + (1-t)y) \leq tf(x) + (1-t)f(y)
    \]
for all $y \in [0,1]$ and $x, y \in X$
In other words, a real-valued function is a function defined on a convex set of real numbers for which the result of the function on a convex combination of any two points in the domain is smaller than the convex combination of the same length of the value of the function on the endpoints.\footnote{Future editions will include figures}
A function $f: \R  \to \R $ is \t{concave} if the function $-f$ is convex.

\blankpage