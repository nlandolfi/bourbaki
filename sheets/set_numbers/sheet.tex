%!name:set_numbers
%!need:finite_sets

\section*{Why}

We want to count the number of elements in a set.

\section*{Defining result}

\begin{proposition}
A set can be equivalent to at most one natural number.
  \ifhmode\unskip\fi\footnote{
A proof will appear in future editions.
  }\end{proposition}
The \t{number} of a finite set is the unique natural number equivalent to it.
We also call this the \t{size} of the set.


%%TODO: include basic facts about unions and numbers0

\subsection*{Notation}

We denote the number of a set by $\num{A}$.

\section*{Restriction to a finite set}

If we restrict $E \indent \num{E}$ to the domain $\powerset{X}$ of some set $X$ then $\num{\cdot }: \powerset{X} \to \omega $ is a function.
    \ifhmode\unskip\fi\footnote{
Future editions will clarify this point.
    }

\section*{Properties}

\begin{proposition}
$A \subset B \implies \num{A} \leq \num{B}$\end{proposition}
%% \begin{proposition}
%%   $A \intersect B = \emptyset \implies \num{A \union B} = \num{A} + \num{B}$.
%% \end{proposition}

\blankpage