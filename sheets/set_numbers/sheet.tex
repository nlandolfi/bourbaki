%!name:set_numbers
%!need:finite_sets

\ssection{Why}

We want to count the number of elements in a set.

\ssection{Defining Result}

\begin{proposition}
  A set can be equivalent to at most one natural number.\footnote{A proof will appear in future editions.}
\end{proposition}

The \t{number} of a finite set is the unique natural number equivalent to it.
We also call this the \t{size} of the set.

%TODO: include basic facts about unions and numbers

\ssubsection{Notation}

We denote the number of a set by $\num{A}$.

\ssection{Restriction to a finite set}

If we restrict $E \mapsto \num{E}$ to the domain $\powerset{X}$ of some set $X$ then $\num{\cdot}: \powerset{X} \to \omega$ is a function.\footnote{Future editions will clarify this point.}

\ssection{Properties}

\begin{proposition}
	$A \subset B \implies \num{A} \leq \num{B}$
\end{proposition}


% \begin{proposition}
%   $A \intersect B = \emptyset \implies \num{A \union B} = \num{A} + \num{B}$.
% \end{proposition}



\blankpage
