
%!name:entire_functions
%!need:complex_analytic_functions
%!refs:yellow/IX/4

\section*{Definition}

An \t{entire function} is a complex function $f: \C  \to \C $ which is analytic for all $z \in \C $.

\blankpage
\sbasic
%%%% MACROS %%%%%%%%%%%%%%%%%%%%%%%%%%%%%%%%%%%%%%%%%%%%%%%

\newcommand{\PM}{\mathbf{P}}

%%%%%%%%%%%%%%%%%%%%%%%%%%%%%%%%%%%%%%%%%%%%%%%%%%%%%%%%%%%

%%%% MACROS %%%%%%%%%%%%%%%%%%%%%%%%%%%%%%%%%%%%%%%%%%%%%%%

\newcommand{\PM}{\mathbf{P}}

%%%%%%%%%%%%%%%%%%%%%%%%%%%%%%%%%%%%%%%%%%%%%%%%%%%%%%%%%%%

\sstart
\stitle{Distance}

\ssection{Why}

We want to talk about the \say{distance} between objects in a set.

\ssection{Common Notions}

Our inspiration is the notion of distance in the plane of geometry.
The objects are points and the distance between them is the length of the line segment joining them.
We note a few properties of this notion of distance:

\begin{enumerate}
  \item
    The distance between any two
    distinct objects is not zero.

  \item
    The distance between any
    two objects does not depend
    on the order in which we
    consider them.

  \item
    The distance between
    two objects is no larger
    than the sum of the distances
    of each with any third object
\end{enumerate}

The first observation
is natural: if two points are not
the same, then they are some
distance apart.
In other words, the line
segment between
them has length.

The second observation is natural:
the line segment connecting
two points does not depend on the
order specifying the points.
This observation
justifies the word
\say{between.}
If it were not the case,
then we should use different words,
and be careful to speak
of the distance \say{from} a
first point \say{to} a second
point.

The third property is
a non-obvious property of
distance in the plane.
It says, in other words,
that the length of any side
of a triangle is no larger than
the sum of the lengths of the
two other sides.
With experience in geometry,
the observation may become
natural. But it does
not seem to be superficially so.

A more muddled but superficially
natural justification for our
concern with third observation
is that it says something
about the transitivity of
closeness.
Two objects are close if
their distance is small.
Small is a relative concept,
and needs some standard of
comparison.
Let us fix two points, take
the distance between them,
and call it a unit.
We call two objects close
with respect to our unit
if their distance is less than a unit.

In this language, the third
observation says that if we
know two objects are each half of
a unit distance
from a third object, then
the two objects are close (their
distance is less than a unit).
We might call this third object
the reference object.
Here, then, is the usefulness of
the third property:
we can infer closeness of two objects
if we know their distance to a reference
object.

%\subsection{Paradoxes}
%This point can be emphasized
%by an example when the intuition
%fails: walking up a hill
%may be the same distance as walking
%down it.
\strats
