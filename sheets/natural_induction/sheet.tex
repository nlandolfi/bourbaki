%!name:natural_induction
%!need:natural_numbers

\ssection{Why}

We want to show something holds for every natural number.\footnote{Future editions will modify this superficial why.}

\ssection{Definition}

The most important property of the set of natural numbers is that it is the unique smallest successor set.
In other words, if $S$ is a successor set contained in $\omega$ (see \sheetref{natural_numbers}{Natural Numbers}), then $S = \omega$.
This is useful for proving that a particular property holds for the set of natural numbers.

To do so we follow standard routine.
First, we define the set $S$ to be the set of natural numbers for which the property holds.
This step uses the principle of selection (see \sheetref{set_selection}{Set Selection}) and ensures that $S \subset \omega$.
Next we show that this set $S$ is indeed a successor set.
The first part of this step is to show that $0 \in S$.
The second part is to show that $n \in S \implies \ssuc{n} \in S$.
These two together mean that $S$ is a successor set, and since $S \subset \omega$ by definition, that $S = \omega$.
In other words, the set of natural numbers for which the property holds is the entire set of natural numbers.
We call this the \t{principle of mathematical induction.}

% TODO read and decide whetehr to include
% We assert two additonal self-evident and indispensable properties of these natural numbers.
% First, one is the successor of no other element.
% Second, if we have a subset of the naturals containing one with the property that it contains successors of its elements, then that set is equal to the natural numbers.

% These two properties, along with the existence
% and uniqueness of successors are together called
% \ct{Peano's axioms}{peanosaxioms} for the natural numbers.
% When in familiar company, we freely assume Peano's axioms.
%
% \ssection{Notation}
%
% As an exercise in the notation assumed so far, we can write Peano's axioms: $N$ is a set along with a function $s: N \to N$ such that
% \begin{enumerate}
%   \item $s(n)$ is the successor of $n$ for all $n \in N$.
%   \item s is one-to-one; $s(n) = s(m) \implies m = n$  for all $m, n \in N$.
%   \item There does not exist $n \in N$ such that $s(n) = 1$.
%   \item If $T \subset N$, $1 \in T$, and $s(n) \in T$ for all $n \in T$, then $T = N$.
% \end{enumerate}

\blankpage
