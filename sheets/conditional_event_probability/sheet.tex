%!name:conditional_event_probability
%!need:event_probabilities

\ssection{Why}

Given that we know that one event has occured, we want language for what the new probabilities should be.\footnote{Future editions will improve.}

\ssection{Definition}

Consider two events, the second of which has non-zero probability.
The \t{conditional probability} of the first event \t{conditioned} on a second event is the result of dividing the probability of the second event into the probability of the intersection of the two events.

\ssubsection{Notation}

Let $\PE$ be the event probability function.
Let $A$ and $B$ be two events with $\PE(B) \neq 0$.
We denote the conditional probability of $A$ conditioned on $B$ by $\PE(A \mid B)$; defined by
\[
  \PE(A \mid B) = \frac{\PE(A \intersect B)}{\PE(B)}.
\]

\blankpage
