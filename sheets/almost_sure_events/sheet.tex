%!name:almost_sure_events
%!need:negligible_sets
%!need:probability_measures

\ssection{Why}

We discuss negligible sets in the langauge of probability theory.\footnote{Future editions may modify this explanation.}

\ssection{Definition}

Let $(\Omega, \CA, \PM)$ be a probability space.
An event $A \in \CA$ happens \t{almost surely} (or \t{almost certainly} or \t{almost always}) if $\PM(A) = 1$ (equivalent, if $\PM(\Omega \setminus A) = 0$.
Conversely, an event $B \subset \Omega$ happens \t{almost never} if $\PM(B) = 0$.

\blankpage
