%!name:subset_systems
%!need:ordered_pairs
%!need:power_set

\ssection{Why}

We speak of a \t{set} and a subset of its power set which satisfies certain properties.
The utility of this abstract concept is proved by its examples, in future sheets.

\ssection{Definition}

A \t{subset system} is a a pair of sets: the second set contains subsets of the first.

We call the first \t{set}{set} the \t{base set}.
We call elements of the second set \t{distinguished subsets}.
An \t{undistingished subset} is a subset of the first set which is not \t{distinguished}.

%Useful
%\rt{subset spaces}{subsetspace}
%are those for which the
%\rt{distinguished subsets}{distinguishedsubset}
%satisfy some
%\rt{set-algebraic}{setalgebraicoperations}
%properties.
%For one example, the distinguished sets
%may be closed under set union or
%set intersection.
%As another example, the distinguished sets
%may be closed under complements or under subsets.

\ssubsection{Notation}

Let $A$ be a set and $\mathcal{A} \subset \powerset{A}$.
We denote the subset system of $A$ and $\mathcal{A}$
by $\tuple{A, \mathcal{A}}$, read aloud as \say{A, script
  A.}

\ssection{Example}

\begin{expl}
Let $A$ be a nonempty set.
Let $\mathcal{A}$ be $\powerset{A}$.
Then $(A, \mathcal{A})$ is a subset system.
\end{expl}
