
\section*{Why}

We are often interested in a set of subsets of a given set.

\section*{Definition}

Let $A$ be a non-empty set.
A \t{subset system} is a pair $(A, \mathcal{A} )$ in which $\mathcal{A}  \subset \powerset{A}$.
In this common case we call the first set the \t{base set} and the second set the \t{distinguished subsets}.
A subset of $B \subset A$ which is not \t{distinguished} (i.e., $B \not\in \mathcal{A} $) is called \t{undistinguished}.

\begin{example}
Let $A$ be a nonempty set.
Let $\mathcal{A} $ be $\powerset{A}$.
Then $(A, \mathcal{A} )$ is a subset system.
\end{example}

\subsection*{Other terminology}

Other terminology refers to $(U, \mathcal{F} )$ as a \t{set system} when $U$ is a nonempty finite set and $\mathcal{F} $ is a family of subsets of $U$.
Set systems are also known as \t{hypergraphs}.
% TODO: this need set numbers!! 


\blankpage
%macros.tex
%%%%% MACROS %%%%%%%%%%%%%%%%%%%%%%%%%%%%%%%%%%%%%%%%%%%%%%%
%%%%%%%%%%%%%%%%%%%%%%%%%%%%%%%%%%%%%%%%%%%%%%%%%%%%%%%%%%%%
