%!name:subset_systems
%!need:set_inclusion
%!need:ordered_pairs

\ssection{Why}

We speak of a
\rt{set}{set}
and a
\rt{set}{set}
of its
\rt{subsets}{subset}
satisfying properties.
The utility of this
abstract concept
is proved by its examples,
in future sheets.

\ssection{Definition}

A
\ct{subset system}{subsetsystem}
is a a pair of sets:
the second set contains
subsets of the first.

We call the first
\rt{set}{set}
the
\ct{base set}{baseset}.
A
\ct{distinguished subset}{distinguishedsubset}
is an element of the second set.
An
\ct{undistingished subset}{undistinguishedsubset}
is a subset of the first set which is not
\rt{distinguished}{distinguishedsubset}.

%Useful
%\rt{subset spaces}{subsetspace}
%are those for which the
%\rt{distinguished subsets}{distinguishedsubset}
%satisfy some
%\rt{set-algebraic}{setalgebraicoperations}
%properties.
%For one example, the distinguished sets
%may be closed under set union or
%set intersection.
%As another example, the distinguished sets
%may be closed under complements or under subsets.

\ssubsection{Notation}

Let $A$ be a set and $\mathcal{A} \subset \powerset{A}$.
We denote the subset system of $A$ and $\mathcal{A}$
by $\tuple{A, \mathcal{A}}$, read aloud as \say{A, script
  A.}

\ssection{Example}

\begin{expl}
Let $A$ be a nonempty set.
Let $\mathcal{A}$ be $\powerset{A}$.
Then $(A, \mathcal{A})$ is a subset system.
\end{expl}
