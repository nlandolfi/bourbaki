
%!name:entire_functions
%!need:complex_analytic_functions
%!refs:yellow/IX/4

\section*{Definition}

An \t{entire function} is a complex function $f: \C  \to \C $ which is analytic for all $z \in \C $.

\blankpage
\sbasic
%%%% MACROS %%%%%%%%%%%%%%%%%%%%%%%%%%%%%%%%%%%%%%%%%%%%%%%

\newcommand{\PM}{\mathbf{P}}

%%%%%%%%%%%%%%%%%%%%%%%%%%%%%%%%%%%%%%%%%%%%%%%%%%%%%%%%%%%

%%%% MACROS %%%%%%%%%%%%%%%%%%%%%%%%%%%%%%%%%%%%%%%%%%%%%%%

\newcommand{\PM}{\mathbf{P}}

%%%%%%%%%%%%%%%%%%%%%%%%%%%%%%%%%%%%%%%%%%%%%%%%%%%%%%%%%%%

%%%% MACROS %%%%%%%%%%%%%%%%%%%%%%%%%%%%%%%%%%%%%%%%%%%%%%%

% use \set{stuff} for { stuff }
% use \set* for autosizing delimiters
\DeclarePairedDelimiter{\set}{\{}{\}}

% use \Set{a}{b} for {a | b}
% use \Set* for autosizing delimiters
\DeclarePairedDelimiterX{\Set}[2]{\{}{\}}{#1 \nonscript\;\delimsize\vert\nonscript\; #2}

% use \powerset{A} for power set of A
\newcommand{\powerset}[1]{2^{#1}}

\renewcommand{\emptyset}{\varnothing}

\newcommand{\SA}{\mathcal{A}}
\newcommand{\SB}{\mathcal{B}}
\newcommand{\SC}{\mathcal{C}}
\newcommand{\SD}{\mathcal{D}}
\newcommand{\SE}{\mathcal{E}}
\newcommand{\SF}{\mathcal{F}}
\newcommand{\SG}{\mathcal{G}}
\newcommand{\SH}{\mathcal{H}}
\newcommand{\SI}{\mathcal{I}}
\newcommand{\SJ}{\mathcal{J}}
\newcommand{\SK}{\mathcal{K}}
\newcommand{\SL}{\mathcal{L}}

%%%%%%%%%%%%%%%%%%%%%%%%%%%%%%%%%%%%%%%%%%%%%%%%%%%%%%%%%%%

%%%% MACROS %%%%%%%%%%%%%%%%%%%%%%%%%%%%%%%%%%%%%%%%%%%%%%%

\newcommand{\PM}{\mathbf{P}}

%%%%%%%%%%%%%%%%%%%%%%%%%%%%%%%%%%%%%%%%%%%%%%%%%%%%%%%%%%%

%%%% MACROS %%%%%%%%%%%%%%%%%%%%%%%%%%%%%%%%%%%%%%%%%%%%%%%

\newcommand{\PM}{\mathbf{P}}

%%%%%%%%%%%%%%%%%%%%%%%%%%%%%%%%%%%%%%%%%%%%%%%%%%%%%%%%%%%

%%%% MACROS %%%%%%%%%%%%%%%%%%%%%%%%%%%%%%%%%%%%%%%%%%%%%%%

\newcommand{\PM}{\mathbf{P}}

%%%%%%%%%%%%%%%%%%%%%%%%%%%%%%%%%%%%%%%%%%%%%%%%%%%%%%%%%%%

%%%% MACROS %%%%%%%%%%%%%%%%%%%%%%%%%%%%%%%%%%%%%%%%%%%%%%%

\newcommand{\PM}{\mathbf{P}}

%%%%%%%%%%%%%%%%%%%%%%%%%%%%%%%%%%%%%%%%%%%%%%%%%%%%%%%%%%%

\sstart
\stitle{Subset Systems}

\ssection{Why}

We speak of a
\rt{set}{set}
and a
\rt{set}{set}
of its
\rt{subsets}{subset}
satisfying properties.
The utility of this
abstract concept
is proved by its examples,
in future sheets.

\ssection{Definition}

A
\ct{subset system}{subsetsystem}
is a a pair of sets:
the second set contains
subsets of the first.

We call the first
\rt{set}{set}
the
\ct{base set}{baseset}.
If the
\rt{base set}{baseset}
is finite,
we call the subset system a
\ct{finite subset system}{finitesubsetsystem}.
A
\ct{distinguished subset}{distinguishedsubset}
is an element of the second set.
An
\ct{undistingished subset}{undistinguishedsubset}
is a subset of the first set which is not
\rt{distinguished}{distinguishedsubset}.

%Useful
%\rt{subset spaces}{subsetspace}
%are those for which the
%\rt{distinguished subsets}{distinguishedsubset}
%satisfy some
%\rt{set-algebraic}{setalgebraicoperations}
%properties.
%For one example, the distinguished sets
%may be closed under set union or
%set intersection.
%As another example, the distinguished sets
%may be closed under complements or under subsets.

\ssubsection{Notation}

Let $A$ be a set and $\mathcal{A} \subset \powerset{A}$.
We denote the subset system of $A$ and $\mathcal{A}$
by $\tuple{A, \mathcal{A}}$, read aloud as \say{A, script
  A.}

\ssection{Example}

\begin{expl}
Let $A$ be a nonempty set.
Let $\mathcal{A}$ be $\powerset{A}$.
Then $(A, \mathcal{A})$ is a subset system.
\end{expl}
\strats
