%!name:maximum_conditional_estimates
%!need:random_vectors
%!need:estimators
%!need:conditional_densities

\ssection{Why}

We want to estimate a random vector $x: \Omega \to \R^d$ from a random vector $y: \Omega \to \R^n$.

\ssection{Definition}

Denote by $g: \R^d \times \R^n \to \R$ the joint density for $(x, y)$.\footnote{Future editions will comment on the existence of such a density.}
Denote the conditional density for $x$ given $y$ by $g_{x \mid y} \R^d \times \R^n \to \R$
A \t{maximum conditional estimate} for $x: \Omega \to \R^n$ given that $y$ has taken the value $\gamma \in \R^n$ is a maximizer $\xi \in \R^d$ of $g_{x \mid y}(\xi, \gamma)$.
It is also called the \t{maximum a posteriori estimate} or \t{MAP estimate}.

\ssection{Also maximizes joint}

Notice that since $g(\xi, \gamma) = g_y(\gamma) g_{x \mid y}(\xi, \gamma)$, the MAP estimate also maximizes the joint pdf.

\begin{proposition}
  The MAP estimate maximizes the joint pdf.\footnote{Future editions will include an account.}
\end{proposition}


\blankpage
