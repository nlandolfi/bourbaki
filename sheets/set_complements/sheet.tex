%!name:set_complements
%!need:set_differences
%!need:empty_set
%!refs:paul_halmos/naive_set_theory/section_05

\ssection{Why}

It is often the case in considering set differences that all sets considered are subsets of one set.

\ssection{Definition}

Let $A$ and $B$ denote sets.
In many cases, we take the difference between a set and one contained in it.
In other words, we assume that $B \subset A$.
In this case, we often take complements relative to the same set $A$.
So we do not refer to it, and instead refer to the relative complement of $B$ in $A$ as the \t{complement} of $B$.

\ss{Notation}

Let $A$ denote a set, and let $B$ denote a set for which $B \subset A$.
We denote the relative complement of $B$ in $A$ by $\relcomplement{B}{A}$.
When we need not mention the set $A$, and instead speak of the complement of $B$ without qualification, we denote this complement by $\complement{B}$.

\s{Complement of a complement}

One nice property of a complement when $B \subset A$ is:
\begin{proposition}
  $(B \subset A) \iff (\relcomplement{\relcomplement{B}{A}}{A} = B)$
\end{proposition}

\s{Basic Facts}

Let $E$ denote a set and let $A$ and $B$ denote sets satisfying $A,B \subset E$.
Then take all complements with respect to $E$.
Here are some immediate consequences of the definition of complements.\footnote{Proofs will appear in future editions.}

\begin{proposition}
  $\complement{\complement{A}} = A$
\end{proposition}

\begin{proposition}
  $\complement{\emptyset} = E$
\end{proposition}

\begin{proposition}
  $\complement{E} = \varnothing$
\end{proposition}

\begin{proposition}
  $A \subset B \iff \complement{B} \subset \complement{A}$
\end{proposition}
