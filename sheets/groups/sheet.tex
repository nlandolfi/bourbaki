
\section*{Why}

% TODO: the genetic approach: polynomial roots and Galois? 

We further drop conditions on the structure of the binary operations, and study only the algebraic structure of addition over the integers.

\section*{Definition}

A \t{group} is an \sheetref{operations}{algebra}$(G, \circ)$ for which $\circ: G \times  G \to G$ is associative, has an identity element in $G$, and has inverse elements.
A group is a \t{commutative group} (or \t{abelian group}) if $\circ$ is commutative.
A group is a \t{finite group} if $G$ is a finite set.

\section*{Additive groups}

Suppose that $(R, +, \cdot )$ is ring.
Then $(R, +)$ is a commutative group.
Conversely, suppose $(G, +)$ is a commutative group.
Define multiplication on $S$ by $a\cdot b = 0$ for all $a, b \in R$.
Then $(S, +, \cdot )$ is a ring, called the \t{zero ring} of $(G, +)$.
For this reason, it is customary to write $+$ for the operation $\circ$ when handling commutative groups.

\section*{Group Operations}

Along with the group operation, we call the function which maps an element to its inverse element the \t{group operations}.

\blankpage
%macros.tex
%%%%% MACROS %%%%%%%%%%%%%%%%%%%%%%%%%%%%%%%%%%%%%%%%%%%%%%%
%%%%%%%%%%%%%%%%%%%%%%%%%%%%%%%%%%%%%%%%%%%%%%%%%%%%%%%%%%%%
