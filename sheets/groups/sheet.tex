%!name:groups
%!need:rings
%!refs:peter_cameron/introduction_to_algebra

\section{Why}

We further drop conditions on the structure of the binary operations, and study only the algebraic structure of addition over the integers.

\section{Definition}

A \t{group} is an \t{algebra} $(G, \circ)$ for which $\circ: G \times G \to G$ is associative, has an identity element in $G$, and has inverse elements.
It is a \t{commutative group} (or \t{abelian group}) if $\circ$ is commutative.

\section{Additive groups}

Suppose that $(R, +, \cdot )$ is ring.
Then $(R, +)$ is a commutative group.
Conversely, suppose $(G, +)$ is a commutative group.
Define multiplication on $S$ by $a\cdot b = 0$ for all $a, b \in R$.
Then $(S, +, \cdot )$ is a ring, called the \t{zero ring} of $(G, +)$.
For this reason, it is customary to write $+$ for the operations $\circ$ when handling commutative groups.

\blankpage