
\section*{Why}

Are all signed measures
the difference of two
positive measures?

Suppose we could partition
the base set into two
sets, one containing
all the sets with positive
measure and one containing
all sets with negative measure.
We could restrict the measure to the former and it would be positive, and we could restrict it to the latter and it would be negative.

Any measurable set could be partitioned
into a piece in the former and a
piece in the latter, and so
its signed measure could be
written as a sum of measures
of these pieces.

\section*{Definition}

By ``positive'' and ``negative'' we mean ``non-negative'' and ``non-positive.''
Let $(X, \mathcal{A} )$ be a measurable space.
Let $\mu : \mathcal{A}  \to \eri$
be a signed measure.

A \t{positive set}
is a measurable set
with the property that
each of its subsets have
non-negative measure
under $\mu $.
A \t{negative set}
is a measurable set
with the property that
each of its subsets have
non-positive measure
under $\mu $.

A \t{signed-set decomposition}
of $X$ under $\mu $
is a partition of $X$
into a positive and a negative set.
Some authors call it a
\t{Hahn decomposition}.

\subsection*{Notation}

Denote by $P$ a positive
and by $N$ a negative set.
When we say ``let $(P, N)$
be a signed-set decomposition
of $X$ under $\mu $'',
we mean that $P$ is
the positive set and $N$
is the negative set.

\section*{Motivating implication}

Does such a decomposition
always exist? Is it unique?
We are motivated to find
answers
by the following observation.

Suppose there was a signed-set
decomposition of $(X, \mathcal{A} )$
under $\mu $; denote it by $(P, N)$.
Then
$\mu (A \cap  P) \geq 0$
and $\mu (A \cap  N) \leq 0$
for all $A \in \mathcal{A} $.

Define
$\mu _1 = \mu (A \cap  P)$
and $\mu _2 = -\mu (A \cap  P)$,
then $\mu _1$ and $\mu _2$ are finite measures.
Moreover, $\mu  = \mu _1 - \mu _2$.
Thus, if we had a signed-set decomposition
we could write $\mu $ as the difference
of two measures.
