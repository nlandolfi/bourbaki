%!name:games
%!need:set_numbers
%!need:arrays
%!need:decisions

\ssection{Why}

We want to discuss interactive decision making.

\ssection{Discussion}

We are interested in talking about situations in which there are several \t{decision makers}, \t{agents} or \t{players}, each of which are making decisions that will affect the outcome for all involved.

\ssubsection{Example: rock paper scissors}

Consider the game \say{rock-paper-scissors} in which there are two players $A$ and $B$.
Each player may choose one of the three actions \textsc{Rock}, \textsc{Paper}, \textsc{Scissors}.
To play, each player simultaneously selects an action, and these are compared.
% So far we have a set of \t{players} or \t{agents} $P = \set{A, B}$ and a set of actions $\set{\textsc{Rock}, \textsc{Paper}, \textsc{Scissors}}$.
Here, both agents have the same actions.
% In this case, both agents have the same set of actions, but they need not.

\ssection{Definition}

In both these games there is a finite set of \t{players}, or \t{agents}, or \t{controllers}.
Let $\CI$ be a finite set with $\num{\CI} = n$, the players.

In rock-paper-scissors, for example, $\CI = \set{A, B}$.
There, each player could pick one of the three actions.
Define $\CA_A = \CA_B = \set{\textsc{Rock}, \textsc{Paper}, \textsc{Scissors}}$.
We call $\CA_A$ the actions of $A$ and $\CA_B$ the actions of $B$.

We have a set of outcomes $\CO = \set{\textsc{A Wins}, \textsc{B Wins}, \textsc{Tie}}$.
Let $f: \CA_A \times C_B \to \CO$ defined by $f(\textsc{Rock}, \textsc{Scissors}) = \textsc{A Wins}$

 the set of \t{players}.
Let $S$ be a finite set, the set of \t{states}.
For $i = 1, \dots, n$, let $\set{A^p_s}_{s \in S}$ be a family of sets, the \t{action sets by state}.
Define $\CA^i = \union_s A^p_s$ the set of \t{actions} for player $i = 1, \dots, n$.

Let $f: S \times \prod_{i} \CA^i \to S$, the \t{game dynamics} or \t{transition function}.

{\tiny
\begin{tabular}{c|c|c}
	 & $\;\,$ & $\;\,$ \\
	\hline
	& & \\
	\hline
	& &
\end{tabular}
}





\ssection{Definition}

The first thing to discuss is the set of players.
Let $P$ be a finite set with $\num{P} = n$.
The set $P$ is the set of players, and we

We begin with a single-player game.

Let $S$ be a set.

\blankpage
