
\section*{Why}

We want to discuss interactive decision making.

\section*{Discussion}

We are interested in talking about situations in which there are several \t{decision makers}, \t{agents} or \t{players}, each of which are making decisions that will affect the outcome for all involved.

\subsection*{Example: rock paper scissors}

Consider the game ``rock-paper-scissors'' in which there are two players $A$ and $B$.
Each player may choose one of the three actions \textsc{Rock}, \textsc{Paper}, \textsc{Scissors}.
To play, each player simultaneously selects an action, and these are compared.
% So far we have a set of ❬players❭ or ❬agents❭ $P = \set{A, B}$ and a set of actions $\set{⸤Rock⸥, ⸤Paper⸥, ⸤Scissors⸥}$.
Here, both agents have the same actions.
% In this case, both agents have the same set of actions, but they need not.

\section*{Definition}

There is a finite set of \t{players}, or \t{agents}, or \t{controllers}.
Let $\mathcal{I} $ be a finite set with $\num{\mathcal{I} } = n$, the players.

In rock-paper-scissors, for example, $\mathcal{I}  = \set{A, B}$.
There, each player could pick one of the three actions.
Define $\mathcal{A} _A = \mathcal{A} _B = \{$\textsc{Rock}, \textsc{Paper}, \textsc{Scissors}$\}$.
We call $\mathcal{A} _A$ the actions of $A$ and $\mathcal{A} _B$ the actions of $B$.

We have a set of outcomes $\mathcal{O} = \{$\textsc{A Wins}, \textsc{B Wins}, \textsc{Tie}$\}$.
Let $f: \mathcal{A} _A \times C_B \to \mathcal{O}$ defined by $f($\textsc{Rock}, \textsc{Scissors}$) = $\textsc{A Wins}

The set of \t{players}.
Let $S$ be a finite set, the set of \t{states}.
For $i = 1, \dots , n$, let $\set{A^p_s}_{s \in S}$ be a family of sets, the \t{action sets by state}.
Define $\mathcal{A} ^i = \cup_s A^p_s$ the set of \t{actions} for player $i = 1, \dots , n$.

Let $f: S \times  \prod_{i} \mathcal{A} ^i \to S$, the \t{game dynamics} or \t{transition function}.

\blankpage