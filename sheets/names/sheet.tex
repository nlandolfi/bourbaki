%!name:names
%!need:objects
%!need:letters

\s{Why}

We (still) want to talk and write about things.

\s{Names}

% TODO(next edition) define subscripts and superscripts
As we use sounds to speak about objects, we use symbols to write about objects.
In these sheets, we will mostly use the upper and lower case latin letters to denote objects.
We sometimes also use an \t{accent} $'$ or subscripts or superscripts.
When we write the symbols we say that the composite symbol formed \t{denotes} the object.
We call it the \t{name} of the object.

Since we use these same symbols for spoken words of the English language, we want to distinguish names from words.
One idea is to box our names, and agree that everything in a box is a name, and that a name always denotes the object.
For example, \fbox{A} or \fbox{A'} or \fbox{A$_0$}.
The box works well to group the symbols and clarifies that \fbox{A}\fbox{A} is different from \fbox{AA}.
But experience shows that we need not use boxes.

We indicate a name for an object with italics.
Instead of \fbox{A'} we use $A'$, instead of \fbox{A$_0$} we use $A_0$.
Experience shows that this subtlety is enough for clarity and it agrees with traditional and modern practice.
Other examples include $A''$, $A'''$, $A''''$, $B$, $C$, $D$, $E$, $F$, $f$, $f'$ $f_a$.


\s{No repetitions}

We never use the same name to refer to two different objects.
Using the same name for two different objects causes confusion.
We make clear when we reuse symbols to mean different objects.
We tend to introduce the names used at the beginning of a paragraph or section.

\s{Names are objects}

There is an odd aspect in these considerations.
The symbol $A$ may denote itself, that particular mark on the page.
There is no helping it.
As soon as we use some symbols to identify any object, these symbols can reference themselves.

An interpretation of this peculiarity is that names are objects.
In other words, the name is an abstract object, it is that which we use to refer to another object.
It is the thing pointing to another object.
And the marks on the page which are meant to look similar are the several uses of a name.

% \ssection{Assertions}
%
% We use the word \t{assertion} in the usual sense of the English language.
%
% \ssection{Names}
%
% To discuss objects we give them \t{names}.
% For example, \say{the pebble} or \say{the pebble's color}.
%
% \ssection{Formal Languages}
%
% We often will say things which are true about objects with particular properties.
% These words do not officially have any meaning yet for us.
% We will write our statements in terms.
% A term will be a placeholder of sorts.
% We will say things like "let $A$ be a term".
% We are interested in deducing belonging relationships a
%
% So we will have some signs.
% An \t{accent} is $'$.
% A \t{letter} is an upper or lower case Latin letter with or without accent.
% The lower case latin letters are
%
% We will print the placeholder in italics.
%
% We will use a formal language to
% Our formal language will consist of \tasdf{terms} and \tasdf{relations}.
% A term is a letter,
% To talk about o
%
% To help our development, we use a formal language.
% The language consists of a few symbols and has enough complexity to let us express English-language sentences like \say{every object in this set is in this other set} and \say{every object in this set has this property} and \say{there exists an object in this set with this property}.
%
% \ssubsection{Terms}
%
% \ssubsection{Symbols}
%
% A letter
%
% Test
%
% This is a sentence.
%
% \begin{construction}
%   \normalfont
%   Test
%
%   \begin{tabular}{rl}
%     \texttt{name} & $A$ \\
%     \texttt{name} & $B$ \\
%     \texttt{name} & $C$ \\
%     \texttt{have} & $A \subset B$ \\
%     \texttt{have} & $A \subset C$ \\
%     \texttt{thus} & $ A \subset C$
%   \end{tabular}
% \end{construction}

% \ssubsection{Names}
%
%
% \say{Variables}
%
% A single Latin letter regularly suffices.
% To aid our memory, we tend to choose the letter mnemonically.
%
% The use of letters to name objects is convenient, since they are short.
% But we must take care when speaking of objects by their names that we know which object is referred to.
%
%
% \ssubsection{Notation}
%
% We use italics when writing the name.
% We introduce a name by the word \say{let,} followed by the name in italics and then the word \say{be} followed by a description of the object the name refers to.
% For example: let $a$ be an object.
% Here the description is \say{an object}.

\s{Names as placeholders}

We frequently use a name as a \t{placeholder}.
In this case, we will say \say{let $A$ denote an object}.
By this we mean that $A$ is a name for an object, but we do not know what that object is.
This is frequently useful when the arguments we will make do not depend upon the particular object considered.
This practice is also old.
Experience shows it is effective.
As usual, it is best understood by example.
