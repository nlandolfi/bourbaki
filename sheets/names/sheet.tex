%!name:names
%!need:objects

\ssection{Why}

We (still) want to talk and write about things.

\ssection{Names}

We must use sounds to speak about objects.
Likewise we must use symbols to write about objects.
If we take some symbols like those in \sheetref{letters}{Letters}, and we write them we say that they symbols \t{denote} the object.
We call the collection of the symbols the \t{name} of the object.
In these sheets, we will mostly tend to use the upper and lower case latin letters to denote objects.
Sometimes, however, we will use the Arabic numerals, or add a mark like $'$ to latin letters, or we may use both letters and numerals to denote objects.
% will not always be so.
% For example, we will sometiems use the Arabic numerals.
% Or we may use a letter an accent or several accents after it.

We are, however, using these same symbols on these pages for spoken words of the English language.
% When we are using them in particular as a name for an object, we will put the symbol in italics.
So we need to distinguish when a symbol or group of symbols is meant to denote an object.
We could box our symbols, and agree that everything in the box denotes the object.
For example, \fbox{A}.
Or we could underline our symbols, like \underline{A}.
Either would work.
The box would work particularly well for using two symbols to denote an object.
For example, \fbox{AA}.% or \underline{AA}.
And \fbox{A}\fbox{A} is clearly different from \fbox{AA}.
% In the first case, we have denoted the same object \fbox{A} twice whereas in the second we have denoted the object \fbox{AA} once.

Experience shows that using two letters twice is often confusing, and if accents are used, not needed.
Rather than \fbox{AA} why not use \fbox{A'}.
Instead of $\fbox{AAA}$, use \fbox{A''}.
Then experience also shows that the complications like boxes around symbols are unecessary.
In other words, we agree never to use \fbox{A'B'}.
If we have \fbox{A'}\fbox{B'} there is really no confusion in dropping the boxes and writing A'B'.
But we still want some way of distinguishing that we are talking about objects.

In these sheets, then, we will indicate that we are denoting an object by using italics.
Instead of \fbox{A}, we will use $A$.
Instead of \fbox{A'}, we will use $A'$.
Experience shows that this practice is subtle, but easy enough to distinguish.
This choice has the added benefit of agreeing with the traditional and modern practice.
And the practice is several millenia old---so it ought to suit us in these sheets.
% until the next edition.

% For historical reasons and reasons of space, we will never use two letters
% There is a historical note that

There is an odd aspect in these considerations.
$A$ may denote itself, that particular mark on the page.
There is no helping it.
As soon as we use some symbols to identify any object, pathological things like this may happen.

% So these symbols can reference
% For example, if we want to use the upper case latin letter A as a symbol for an object, we when referencing the object we will write $A$.
% In all cases, we will help to distinguish that the symbols
%
% A second way of distinguishing this may be to box the symbol in question.
% For example, we would denote some object by \fbox{A}.
% or
% \underline{A}.
% It might help to point out another
%
% In all cases, when we use a particular symbol to denote an object, we are mean that the symbol
%
% Whenever we use a particular symbol to denote
%
% \ssection{Assertions}
%
% We use the word \t{assertion} in the usual sense of the English language.
%
% \ssection{Names}
%
% To discuss objects we give them \t{names}.
% For example, \say{the pebble} or \say{the pebble's color}.
% To make statements about objects we use verbs.
% It will turn out that in the sequel we will only use 'is' and 'belongs.'
% All things are stated to hold in the present.
% For example, the pebble is gray.
%
% \ssection{Formal Languages}
%
% We often will say things which are true about objects with particular properties.
% These words do not officially have any meaning yet for us.
% We will write our statements in terms.
% A term will be a placeholder of sorts.
% We will say things like "let $A$ be a term".
% We are interested in deducing belonging relationships a
%
% So we will have some signs.
% An \t{accent} is $'$.
% A \t{letter} is an upper or lower case Latin letter with or without accent.
% The lower case latin letters are
%
%  \t{placeholder} for us will mean any upper or lower case latin letter with or without one or more $'$ marks.
% We will print the placeholder in italics.
% For example, $A$, $A'$, $A''$, $A'''$, $A''''$, $B$, $C$, $D$, $E$, $F$, $f$, $f'$ are each placeholders.
%
% We will use a formal language to
% Our formal language will consist of \t{terms} and \t{relations}.
% A term is a letter,
% To talk about o
%
% To help our development, we use a formal language.
% The language consists of a few symbols and has enough complexity to let us express English-language sentences like \say{every object in this set is in this other set} and \say{every object in this set has this property} and \say{there exists an object in this set with this property}.
%
% \ssubsection{Terms}
%
% \ssubsection{Symbols}
%
% A letter
%
% Test
%
% This is a sentence.
%
% \begin{construction}
%   \normalfont
%   Test
%
%   \begin{tabular}{rl}
%     \texttt{name} & $A$ \\
%     \texttt{name} & $B$ \\
%     \texttt{name} & $C$ \\
%     \texttt{have} & $A \subset B$ \\
%     \texttt{have} & $A \subset C$ \\
%     \texttt{thus} & $ A \subset C$
%   \end{tabular}
% \end{construction}

% \ssubsection{Names}
%
%
% \say{Variables}
%
% A single Latin letter regularly suffices.
% To aid our memory, we tend to choose the letter mnemonically.
%
% The use of letters to name objects is convenient, since they are short.
% But we must take care when speaking of objects by their names that we know which object is referred to.
%
%
% \ssubsection{Notation}
%
% We use italics when writing the name.
% We introduce a name by the word \say{let,} followed by the name in italics and then the word \say{be} followed by a description of the object the name refers to.
% For example: let $a$ be an object.
% Here the description is \say{an object}.
