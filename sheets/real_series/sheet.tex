%!name:real_series
%!need:real_limits
%!refs:marina_iliopoulou/mathematical_tools_for_the_physical_sciences/series_power_series_taylors_theorem

\ssection{Why}

We want to sum infinitely many real numbers.

\ssection{Definition}

Let $(a_k)_{k \in \N}$ be a sequence in $\R$.
Define $(s_n)_{n \in \N}$ by \[s_n = \sum_{k = 1}^{n} a_k.\]
We call $s_n$ the \t{$n$th partial sum} of $(x_k)$.
In other words, the first partial sum $s_1$ is $a_1$, the second partial sum $s_2$ is $a_1 + a_2$, the third partial sum $s_3$ is $a_1 + a_2 + a_3$ and so on.

We call $(s_n)$ the \t{sequence of partial sums} or \t{series} of $(a_k)$.
If the \t{series} converges, then we say that $(a_k)$ is \t{summable}.
Clearly not every series is summable: consider, for example, $a_k = 1$ for all $k$. It has the divergent series $(1, 2, 3, 4, 5, \dots)$.


\ssubsection{Notation}

If the sequence is summable,
then there exists $s \in \R$ (the limit), which we denote
\begin{equation}
	  s = \lim_{n\to\infty} s_n = \lim_{n\to\infty} \sum_{k = 1}^{n} a_k.
	  \label{real_series:equation:real_series}
\end{equation}
We read these relations aloud as
\say{s is the limit as n goes to infinity
of s n} and
\say{s is the limit as n goes to infinity
of the sum of a k from k equals 1 to n.}
We often avoid referencing $s_n$ by abbreviating Equation~\eqref{real_series:equation:real_series} by
\[
  \sum_{k = 1}^{\infty} a_k = s.
\]
We read this notation aloud as \say{the sum from 1 to infinity of a k is s.}
The notation is subtle, and requires justification by the algebra of series.\footnote{Future editions will include such justification.}

\ssection{Existence and uniqueness}

Since there exist sequences which do not converge, there exist sequences which are not summable.
Consider the sequence which alternates between $+1$ and $-1$, and starts with $+1$.
Its series alternates between $+1$ and $0$, and so does not converge.
Since limits are unique, so is $\sum_{k=1}^{\infty}a_k$.

