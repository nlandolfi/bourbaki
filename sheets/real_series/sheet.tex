
\section*{Why}

We want to sum infinitely many real numbers.

\section*{Definition}

Let $(a_k)_{k \in \N  }$ be a sequence in $\R $.
Define $(s_n)_{n \in \N  }$ by
\[
s_n = \sum_{k = 1}^{n} a_k.
\]
We call $s_n$ the \t{$n$th partial sum} of $(x_k)$.
In other words, the first partial sum $s_1$ is $a_1$, the second partial sum $s_2$ is $a_1 + a_2$, the third partial sum $s_3$ is $a_1 + a_2 + a_3$ and so on.
We call $(s_n)$ the \t{sequence of partial sums} or \t{series} of $(a_k)$.
If the \t{series} converges, then we say that $(a_k)$ is \t{summable}.
Clearly not every series is summable: consider, for example, $a_k = 1$ for all $k$. It has the divergent series $(1, 2, 3, 4, 5, \dots )$.


\subsection*{Notation}

If the sequence is summable, then there exists a unique $s \in \R $ (the limit), which we denote
\[
s = \lim_{n\to\infty} s_n = \lim_{n\to\infty} \sum_{k = 1}^{n} a_k.
\]
We read these relations aloud as ``s is the limit as n goes to infinity of s n'' and ``s is the limit as n goes to infinity of the sum of a k from k equals 1 to n.''
We often avoid referencing $s_n$ by abbreviating the above with
\[
\sum_{k = 1}^{\infty} a_k = s.
\]
We read this notation aloud as ``the sum from 1 to infinity of a k is s.''
The notation is subtle, and requires justification by the algebra of series.\footnote{Future editions will include such justification.}

\section*{Convergence}

For a series to converge, intuition suggests that the additional terms added should be getting smaller and smaller. Indeed:

\begin{proposition}
Let $(a_k)_{k \in \N  }$ be a sequence of real numbers.
If $(a_k)$ is summable then $a_k$ converges to $0$.\footnote{Future editions will include an account.}
\end{proposition}

The converse of this theorem has immediate relevance as a preliminary test for determining whether a series converges.

\begin{proposition}
If $(a_k)$ does not converge or converges to $a_0 \neq 0$, then it is not summable.
\end{proposition}

%macros.tex
%%%%% MACROS %%%%%%%%%%%%%%%%%%%%%%%%%%%%%%%%%%%%%%%%%%%%%%%
%%%%%%%%%%%%%%%%%%%%%%%%%%%%%%%%%%%%%%%%%%%%%%%%%%%%%%%%%%%%
