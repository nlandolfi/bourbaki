%!name:total_orders
%!need:partial_orders

\ssection{Why}

Often we want all elements of the set $A$ to be comparable.

\ssection{Definition}

We call $R$ \t{connexive} if for all $a, b \in A$, $(a, b) \in R$ or $(b, a) \in R$.
If $R$ is a partial order and connexive, we call it a \t{total order}.

A \t{totally ordered set} is a set together with a total order.
The language is a faithful guide: we can compare any two elements.
Still, we prefer one word to three, and so we will use the shorter term \t{chain} for a totally ordered set; other terms include \t{simply ordered set} and
\t{linearly ordered set}.

Let $C = (A, R)$ be a chain.
A \t{minimal element} of $C$ is an element which precedes all other elements.
A \t{maximial element} of $C$ is an element which is preceded by all other elements.

\blankpage
