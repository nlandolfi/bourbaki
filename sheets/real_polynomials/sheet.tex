%!name:real_polynomials
%!need:real_arithmetic

\ssection{Why}

What are some simple functions?
Here's one answer: those that only involve addition and multiplication.\footnote{Future editions will modify and expand.}

\ssection{Definition}

A \t{real polynomial} (or \t{polynomial}) of degree $d$ is a function $p: \R \to \R$ for which there exists a finite sequence $a = (a_0, a_1, \dots, a_d)$ so that
\[
  p(x) = a_0 + a_1 x + a_2 x^2 + \cdots + a_nx^d.
\]
In particular, $q(x) = ax + b$ for $a, b \in \R$ is a polynomial of the first degree and $r(x) = ax^2 + bx + c$ for $a, b, c \in \R$ is a polynomial of the second degree.

In a sense, these are \say{simple} functions.
We require addition (and substraction) and multiplication; but no division.

\ssection{Properties}

\begin{proposition}
  Let $p: \R \to \R$ be a polynomial of degree $d$.
  Then $p$ is continuous.
\end{proposition}

\begin{proposition}
  Let $p: \R \to \R$ be a polynomial of degree $d$.
  Then $p$ has derivatives of all orders.
  Every derivative of $p$ is a polynomial.
\end{proposition}


\blankpage
