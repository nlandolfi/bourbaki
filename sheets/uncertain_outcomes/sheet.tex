%!name:uncertain_outcomes
%!need:natural_numbers

\ssection{Why}

We want to talk about an object in some set---and the properties of this object---without knowing the precise identity of the object.
When talking about an unknown object, we first enumerate the set of possibilities.

\ssection{Definition}

We construct a set which includes the objects.
The set is called the \t{set of possibilities} or \t{set of outcomes}.
We call an element of the set of outcomes a \t{possibility} or an \t{outcome}.

Some authors call the set of outcomces the \t{set of samples} and call an outcome a \t{sample}.
They also will speak of the set of outcomes as the \t{sample space}.
We avoid this terminology for various reasons.\footnote{To be clarified in future editions.}

The language outcome is regularly preferred because so often the unknown object is associated somehow with the future.
For example, we will observe the identity of the object in the future.

\ssection{Example: coin}

We want to talk about the result of flipping a coin.
The coin has two sides.
When we flip the coin, it lands heads or tails.

Let us use the set of outcomes $\set{0, 1}$.
If the coin lands tails, we will say that outcome 0 has occurred.
If the coin lands heads, we will say that outcome 1 has occurred.

\ssection{Example: die}

We want to talk about the result of rolling a die.
The die has six sides.
When we roll the die, one
of the six sides is facing
up.
Let us model this uncertain
outcome with the set of outcomes
$\set{1, 2, 3, 4, 5, 6}$.

\ssection{Why}

We will use our intuition of proportion to do so.
Such language is useful in modeling the future, in modeling degrees of belief, and in modeling populations.

\blankpage
