
\section*{Why}

We would like to speak about an object, which is a member of some set, and some attributes of this object---without knowing the precise identity of the object.
Such language would be especially useful in discussing games of chance.

\section*{Definition}

Suppose $\Omega $ is a set which includes all possible objects.
We call it the \t{sample space}.
We call an element of this set an \t{outcome}.
Other terms for outcome include \t{possibility}, \t{sample}, \t{elementary event}, \t{simple event}, and \t{sample point}.

Often, one imagines an \t{experiment} as having many possible outcomes.
We leave this term undefined.

\section*{Examples of modeling}

\textit{A model for flipping a coin}.
We want to talk about the result of flipping a coin.
The coin has two sides.
When we flip the coin, it lands heads or tails.\footnote{Of course, it may land on it's side, or roll away...but we will not model those situations.}
We may model these outcomes with the set $\set{0, 1}$.
If the coin lands tails, we say that outcome 0 has occurred.
If the coin lands heads, we say that outcome 1 has occurred.

\textit{A model for rolling a die}.
We want to talk about the result of rolling a die.
The die has six sides.
When we roll the die, it lands with one of its six sides facing up.
We may model this uncertain outcome as an element of the set $\set{1, 2, 3, 4, 5, 6}$.
Here we have used the first six natural numbers.
We say that event 1 occurs if there is one pip showing, that event 2 occurs if there are two pips showing, and so on.

\textit{A model for rolling two dice at once}.
We want to talk about the result of a simultaneous throw of two dice.
When we throw the dice, each one lands with one of six sides facing up.
To model this uncertain outcome, we first number the dice: 1 and 2.
We may model the sample space with the set $\Omega  = \set{1, \dots , 6}^2$.
We interpret $(\omega _1, \omega _2) \in \Omega $ as follows: $\omega _1$ is the number of pips showing on the first die---the die numbered 1---and $\omega _2$ is the number of pips showing on the second die---the die labeled 2.

\textit{A model for rolling one die twice}.
We want to talk about the result of rolling one die twice.
To model this uncertain outcome, we agree to speak of the \textit{first} and \textit{second} rolls.
We are interested in the number of pips showing face up on the first and second rolls.

As before, we may model this uncertain outcome as an element of the set $\Omega  = \set{1, \dots , 6}^2$.
We interpret $(\omega _1, \omega _2) \in \Omega $ so that $\omega _1$ is the number of pips showing on the first roll and $\omega _2$ is the number of pips showing on the second roll.
It would be natural to refer to $\omega _1$ as the \textit{first} outcome, and $\omega _2$ as the \textit{second} outcome.

We emphasize that we have used the same set of outcomes as in the previous case.
In other words, we can use the same set of outcomes to model two different situations.
% <div data-littype='run'> It is natural here to speak of ‹outcomes›, plural. </div>
%  

