
%!name:uncertain_outcomes
%!need:natural_numbers
%!need:pair_intersections
%!need:pair_unions

\section*{Why}

We would like to speak about an object, which is a member of some set, and some attributes of this objects---without knowing the precise identity of the object.
Such language would be especially useful in discussing games of chance.

\section*{Definition}

We have a set $\Omega $ which includes all possible objects.
We call it the \t{set of outcomes} (or \t{sample space}).
We call an element of the set of outcomes an \t{outcome} (or \t{possibility}, \t{sample}, \t{elementary event}, \t{simple event}, \t{sample point}).

An \t{event} (\t{compound event}, \t{random event}) is a subset of outcomes.
For events $A, B \subset \Omega $, we interpret $A \cup B$ as the event that either $A$ \textit{or} $B$ occurs.
Similarly we interpret $A \cap  B$ as the event that \textit{both} $A$ \textit{and} $B$ occur.
We interpret $\Omega  - A$, the \textit{complement} of $A$ in $\Omega $, as the event that $A$ \textit{does not} occur.

An \t{outcome variable} (or \t{random variable}) is a function from $\Omega $ to $V$, where $V$ is a set.
In this context, $V$ is called the set of \t{values} of the random variable.

\section*{Example: coin}

We want to talk about the result of flipping a coin.
The coin has two sides.
When we flip the coin, it lands heads or tails.
We model these outcomes with the set $\set{0, 1}$.
If the coin lands tails, we say that outcome 0 has occurred.
If the coin lands heads, we say that outcome 1 has occurred.

\section*{Example: die}

We want to talk about the result of rolling a die.
The die has six sides.
When we roll the die, one of the six sides is facing up.
We model this uncertain outcome with $\set{1, 2, 3, 4, 5, 6}$, whose elements represent the number of pips facing up.

Define $O = \set{1, 3, 5}$ and $E = \set{2, 4, 6}$.
We interpret $O$ as the event that the number of pips is odd, and $E$ as the event that the number of pips is even.

\section*{Example: two dice}

We want to talk about the sum of the pips shown facing up after rolling two dice.
We may take as our set of outcome $\set{1, \dots , 12}$, whose elements correspond to the sum.
We interpret $\Set{x \in \Omega }{x \geq 10}$ as the event that the sum of the two dice is greater than or equal to 10.

Alternatively, we may take the outcomes $\set{1, \dots , 6}^2$ and define an outcome variable $s: \set{1, \dots , 6}^2 \to \set{1, \dots , 12}$ by
    \[
s((d_1, d_2)) = d_1 + d_2.
    \]
We interpret this natural-number-valued outcome variable $s$ as sum of the two dice.
The event that the sum of the two dice is greater than or equal to to 10 corresponds to the set $\Set{(d_1, d_2) \in \set{1, \dots , 6}}{s((d_1, d_2)) \geq 10}$.
