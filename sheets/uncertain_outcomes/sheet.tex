%!name:uncertain_outcomes
%!need:natural_numbers
%!need:set_operations

\ssection{Why}

We want to talk about an object in some set---and the properties of this object---without knowing the precise identity of the object.
Such language is useful in discussing the future, degrees of belief, and arbitrary objects from large populations.
% When talking about an unknown object, we first enumerate the set of possibilities.

\ssection{Definition}

We have a set which includes all possible objects.
We call it the set of \t{outcomes} (\t{possibilities}, \t{samples} or \t{sample space}, \t{elementary events}).
We call an element of the set of outcomes an \t{outcome} (\t{possibility}, \t{sample}, \t{elementary event}).

We prefer the language of outcome because the unknown object is often associated somehow with the future, and its identity will be known in the future.
Still other authors refer to the outcomes as \t{elementary events}.

We often handle several outcomes at once.
An \t{event} (\t{compound event}, \t{random event}) is a subset of outcomes.
The set of events is the power set (see \sheetref{set_powers}{Set Powers}) of the set of outcomes.

\ssubsection{Event unions, intersections, complements}
For events $A, B \subset \Omega $, we interpret $A \cup B$ as the event that either $A$ \textit{or} $B$ occurs.
Similarly we interpret $A \cap B$ as the event that \textit{both} $A$ \textit{and} $B$ occur.
We interpret $\Omega  - A$, the \textit{complement} of $A$ in $\Omega $, as the event that $A$ \textit{does not} happen.

\ssection{Example: coin}

We want to talk about the result of flipping a coin.
The coin has two sides.
When we flip the coin, it lands heads or tails.
We model these outcomes with the set $\set{0, 1}$.
If the coin lands tails, we say that outcome 0 has occurred.
If the coin lands heads, we say that outcome 1 has occurred.

\ssection{Example: die}

We want to talk about the result of rolling a die.
The die has six sides.
When we roll the die, one of the six sides is facing up.
We model this uncertain outcome with the set of outcomes $\set{1, 2, 3, 4, 5, 6}$.
These elements correspond to the number of pips on the side which lands facing up.

Two events for this set of outcomes are $O = \set{1, 3, 5}$ and $E \set{2, 4, 6}$
We interpret $O$ as the event that the number of pips is odd, and $E$ as the event that the number of pips is even.

\ssection{Example: sum of two dice}

We want to talk about the sum of the pips shown after rolling two dice.
Each die has six sides.
We model this uncertain outcome with the set of outcomes $\Omega  = \set{1, 2, 3, 4, 5, 6, \dots, 12}$.
These elements correspond to the sum of the number of pips on the side which lands facing up.

We interpret $\Set{x \in \Omega }{x \geq 10}$ as the event that the sum of the two dice is greater than or equal to 10.
