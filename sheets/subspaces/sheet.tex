%!name:subspaces
%!need:vectors

\ssection{Why}

TODO

\ssection{Definition}

A \t{subspace} of a vector space is a subset of vectors that is itself a vector space.
In other words, a subspace is a subset of a vector space which is closed under vector addition and scalar multiplication.

\ssection{Notation}

Let $(V, \F)$ be a vector space.
Let $U \subset V$ with
$$
  \alpha u + \beta v \in U
$$
for all $\alpha, \beta \in \F$ and $u, v \in U$.
Then $U$ is a subspace of $(V, \F)$.

\ssection{Examples}

The entire set of vectors is a subspace.
The set consisting only of the zero vector is a subspace; we call this the \t{zero subspace}.
These two subspaces are called \t{trival subspaces}.
A \t{nontrivial subspace} is a subspace that is not trivial.

\ssection{Properties}

\begin{prop}
  The intersection of a family of subspaces is a subspace.
\end{prop}

\begin{prop}
  There exists a family of subspaces whose union is not a subspace;

  \begin{remark}
  In other words: the union of a family subspaces need not be a subspace.
  \end{remark}

\end{prop}

\begin{prop}
  A subspace must contain the zero vector; in other words, every subspace is nonempty.
\end{prop}
