
%!name:subspaces
%!need:vectors
%!need:real_subspaces

\section*{Definition}

A \t{subspace} of a vector space is a subset of vectors that is a vector space when restriction vector addition and scalar multiplication to the set.
In other words, a subspace is a subset of a vector space which is closed under vector addition and scalar multiplication.

For example, the entire set of vectors is a subspace.
As a second example, the set consisting only of the zero vector is a subspace; we call this the \t{zero subspace}.
These two subspaces are the \t{trivial subspaces}.
A \t{nontrivial subspace} is a subspace that is not trivial.

\subsection*{Notation}

Let $(V, \F)$ be a vector space.
Let $U \subset V$ with
\[
\alpha  u + \beta  v \in U
\]
for all $\alpha , \beta  \in \F$ and $u, v \in U$.
Then $U$ is a subspace of $(V, \F )$.

\section*{Properties}

\begin{proposition}
The intersection of a family of subspaces is a subspace.
\end{proposition}

\begin{proposition}
There exists a family of subspaces whose union is not a subspace;

\begin{remark}
In other words: the union of a family subspaces need not be a subspace.
\end{remark}
\end{proposition}

\begin{proposition}
A subspace must contain the zero vector; in other words, every subspace is nonempty.
\end{proposition}

\blankpage