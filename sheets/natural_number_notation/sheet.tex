
\section*{Why}

We want to record natural numbers.
Recall that we denote $\varnothing$ by 0, the successor of 0 by 1 and the sucessor of 1 by 2.
1 is the success of 0 and 2 is the successor of 1.
We agree to continue 3, 4, 5, 6, 7, 9, each the successor if its predecessor in the list.
Given these numbers, how should we denote the successor to 9?

\section*{Discussion}

We could of course have a new symbol, and denote the successor of 9 by this symbol.
But then we would need as many symbols are there are natural numbers.
And the natural numbers are \textit{infinite} (not finite!).
The following is a trick around this, re-using our ten symbols.
Why ten?...
How many fingers do you have?

\section*{Definition}

The \t{natural notation} (or \t{base-ten notation}, \t{base-10 notation}) for a natural number is a list of elements of the set
\[
\set{0, 1, 2, 3, 4, 5, 6, 7, 8, 9}
\]
To a notation $\eta $ corresponds a natural number, $n$ as follows.
Suppose $\eta $ has length $k$, so that $\eta  = (\eta _1, \dots , \eta _k)$.
The natural corresponding to $\eta $ is
\[
\eta _1 + \eta _2\cdot \ssuc{9} + \eta _3\cdot (\ssuc{9})^2 + \cdots + \eta _k\cdot  (\ssuc{9})^{k-1}
\]
We can use summation notation and the fact that $(\ssuc{9})^0 = 1$ to write
\[
\sum_{i = 1}^{k} \eta _i\cdot (\ssuc{9})^{i-1}
\]

\subsection*{Notation}

In a slightly different but universally standard way, we denote the natural notation $\eta  = (\eta _1, \dots , \eta _k)$ by
\[
\eta _k\eta _{k-1} \cdots \eta _2\eta _1
\]

\subsection*{Examples}

Some examples will help, and illustrate the notation.
Here is a natural notation for some number: $(0, 1)$.
Which number?
Well, we have agreed that it is the number
\[
0 + 1\cdot \ssuc{9} = \ssuc{9}
\]
We denote the natural notation for $10$.
This is the natural notation for the sucessor of 9.
The upshot is clear, we have denoted the sucessor to 9 using only ten symbols.

Here is a natural notation for some number: $(3, 2, 1)$.
Which number?
Well, we have agreed that it is the number
\[
1 + 2\cdot \ssuc{9} + 3\cdot (\ssuc{9})^2
\]
Recall that we have notation for the successor to 9 now.
We can write this number as
\[
1 + 2\cdot 10 + 3\cdot (10)^2
\]

\blankpage