
%!name:simple_integral_homogeneity
%!need:simple_integrals

\section*{Why}

If we stack a rectangle on top of itself we have a rectangle twice the height.
The additivity principle says that the area of the so-formed rectangle is the sum of the areas of the stacked rectangles.
Our definition of integral for simple functions has this property.

\section*{Result}

\begin{proposition}
The simple non-negative integral operator is homogenous over non-negative real values.
\end{proposition}

\begin{proof}Let $(X, \mathcal{A} , \mu )$ be a measure space.
Let $\SimpleF_+(X)$ denote the non-negative real-valued simple functions on $X$.
Define $s: \SimpleF_+(X) \to [0, \infty]$ by $s(f) = \int f d\mu $ for $f \in \SimpleF_+(X)$.
In this notation, we want to show that $s(\alpha  f) = \alpha s(f)$ for all $\alpha  \in [0, \infty)$ and $f \in \SimpleF_+(X)$.
Toward this end, let $f \in \SimpleF_+(X)$ with the simple partition $\set{A_n} \subset \mathcal{A} $ and $\set{a_n} \subset [0, \infty]$.

%%\begin{enumerate}
%%  \item
%%
%%  \item
%%  $s(f + g) = s(f) + s(g)$
%%  for all $f, g \in \Simple_+(X)$.
%%\end{enumerate}
%%From (a) and (b) we obtain that
%%\[
%%  s\left(\alpha fd\mu + \beta g\right) = \alpha s \left(f\right) + \beta s \left(g\right).
%%\]
%%for all $\alpha, \beta \in R$
%%$f, g \in \SimpleF_(X)$.


First, let
$\alpha  \in (0, \infty)$.
Then $\alpha  f \in \SimpleF_+(X)$,
with the simple partition
$\set{A_n} \subset \mathcal{A} $
and $\set{\alpha  a_n} \subset [0, \infty]$.
\[
s(\alpha  f) = \sum_{i = 1}^{n} \alpha  a_n \mu (A_i)
= \alpha  \sum_{i = 1}^{n} a_n \mu (A_i)
= \alpha  s(f).
\]
If $\alpha  = 0$, then $\alpha  f$ is uniformly zero; it is the non-negative simple with partition $\set{X}$ and $\set{0}$.
Regardless of the measure of $X$, this non-negative simple function is zero.
Recall that we define $0 \cdot  \infty = \infty \cdot  0 = 0$.
\end{proof}
\blankpage