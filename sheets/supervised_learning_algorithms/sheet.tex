%!name:supervised_learning_algorithms
%!need:inductors

\ssection{Why}

We sometimes use special language for a function inductor, which alludes to its similarities with \say{learning.}

\ssection{Definition}

Let $X$ and $Y$ be sets and let $\set{G_n: (X \times Y)^n \to \powerset{X \times Y}}_{n \in \N}$ be a family of functional inductors.
A predictor can be used to \say{guess} inputs which do not necessarily appear in the dataset.
For this reason, some authors call an inductor (or family of inductors) a \t{learner} or \t{learning algorithm}.
In accordance with this usage, they refer to the argument of an inductor as the \t{training data} or \t{training dataset} or \t{training set}.
As with our terminology dataset, the word \say{set}, however, may mislead since since we are speaking of a sequence.

It is common to refer to the construction a predictor from a dataset a \t{learning problem}.
In this case, the learning problem is said to be \t{supervised learning}.
By supervision, we mean to indicate that we have the outputs corresponding to the inputs.
In line with this usage, the outputs are often called \t{labels} and the labels are said \say{to provide supervision.}

\blankpage
