
\section*{Why}

We want to consider the non-overlapping elements of a pair of sets.

\section*{Definition}

In other words, we want to consider the set of elements which is one or the other but not in both.
The \t{symmetric difference} (or \t{Boolean sum}) of a set with another set is the union of the difference between the latter set and the former set and the difference between the former and the latter.

\subsection*{Notation}

Let $A$ and $B$ denote sets.
We denote the symmetric difference by $A + B$, so that
\[
A + B = (A - B) \cup (B - A)
\]

\section*{Properties}

Here are some immediate properties of symmetric differences.\footnote{Future editions will have more detailed (but obvious) hypotheses stated.}

\begin{proposition}[Commutative]
$A + B = B + A$.
\end{proposition}

\begin{proposition}[Associative]
$(A + B) + C = A + (B + C)$.
\end{proposition}

\begin{proposition}[Identity]
$(A + \varnothing) = A$
\end{proposition}

\begin{proposition}[Inverse]
$(A + A) = \varnothing$
\end{proposition}

\blankpage
%macros.tex
%\newcommand{\symdiff}{+}
