%!name:set_symmetric_differences
%!need:set_complements
%!need:set_unions
%!refs:paul_halmos/naive_set_theory/section_05

\ssection{Why}

We want to consider the no-overlapping elements of a pair of sets.

\ssection{Definition}

In other words, we want to consider the set of elements which is one or the other but not in both.
The \t{symmetric difference} of a set with another set is the union of the difference between the latter set and the former set and the difference between the former and the latter.
The symmetric differences is also called the \t{Boolean sum} of $A$ and $B$\footnote{Future editions will likely remove or modify this term in accordance with the project's policy on using names.}

\ss{Notation}

Let $A$ and $B$ denote sets.
We denote the symmetric difference by $A + B$.
\[
  A + B = (A - B) \union (B - A)
\]

\s{Properties}

Here are some immediate properties of symmetric differences.\footnote{Future editions will have more detailed (but obvious) hypotheses stated.}

\begin{proposition}[Commutative]
  $A + B = B + A$.
\end{proposition}

\begin{proposition}[Associative]
  $(A + B) + C = A + (B + C)$.
\end{proposition}

\begin{proposition}[Identity]
  $(A + \varnothing) = A$
\end{proposition}

\begin{proposition}[Inverse]
  $(A + A) = \varnothing$
\end{proposition}