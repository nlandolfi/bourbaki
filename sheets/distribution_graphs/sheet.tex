
%!name:distribution_graphs
%!need:typed_graphs
%!need:directed_graph_distributions
%!refs:

\section*{Why}

How can we construct distributions which factor according to a directed graph?\footnote{Future editions might flip the order of this sheet with that of directed graph distributions since, in the genetic approach, in may be more natural to think of constructing such distributions before analyzing them. This is partially motivated by the acyclic constraint here, which restricts the graphs according to which a distribution can factor.}
Related: how can we compactly represent complex distributions over high-dimensional spaces?

\section*{Definition}

Let $\bar{G} = (G, A)$ be a typed graph (see \sheetref{typed_graphs}{Typed Graphs}) with directed and acyclic $G$.
For source vertices $i$, let $g_i: A_i \to [0, 1]$ be a distribution and otherwise let $g_{i}: A \times  A_{\pa_i} \to [0, 1]$ denote a function satisfying $g_i(\cdot , x)$ is a distribution for every $x \in X_{\pa_i}$.

We call the ordered pair $(\bar{G}, g)$ a \t{distribution graph} .

The \t{distribution} of $(\bar{G}, g)$ is the function $p: \prod_{i} A_i \to [0, 1]$ defined by
\[
p(a)
=
\prod_{\pa_i = \varnothing} g_i(a_i)
\prod_{\pa_i \neq \varnothing} g_i(a_i, a_{\pa_i}).
\]
It is, of course, a distribution.
And it factors according to the directed and acyclic graph $G$.
Also, the $g_i$, $i = 1, \dots , n$, are the conditionals.\footnote{Future editions will elaborate and give a proof.}

In other words, a distribution graph represents a probability distributions via products of smaller, ``local'', conditional probability distributions.

\subsection*{Other terminology}

Other terminology includes \t{distribution network}, \t{conditional distribution network}, \t{conditional distribution graph}, \t{bayesian network},\footnote{Indeed, this term is near universal in certain literatures. We avoid it in these sheets as a result of the Bourbaki project's policy on naming.}
\t{bayes net}, \t{directed probabilistic graphical model}, \t{directed graphical model}.

\subsection*{Necessity of acyclicity}

If $G$ above is not taken to be acyclic, then the ``distribution'' of the distribution graph need not be a proper probability distribution (the condition which will fail is normalization).

\blankpage