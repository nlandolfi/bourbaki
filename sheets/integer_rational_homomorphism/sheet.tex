
\section*{Why}

Do the integer numbers correspond (in the sense of \sheetref{homomorphisms}{Homomorphisms}) to elements of the rationals.

\section*{Main result}

Indeed, roughly speaking the integers correspond to rationals whose denominator is 1.
Define
\[
\tilde{Q} := \Set{\eqc{(a, b)} \in \Q }{b = 1_{\Z }}.
\]

\begin{proposition}
The rings $(\tilde{\Q }, +_{\Q } \mid \tilde{\Q }, \cdot _{\Q } \mid \tilde{\Q })$ and $(Z, +_{\Z }, \cdot _{\Z })$ are homomorphic.\footnote{Indeed, more is true and will be included in future editions. There is an \textit{order perserving} ring homomorphism.}
\end{proposition}

\begin{proof}The function is $f: \Z  \to \Q $ with $f(z) = \eqc{(z, 1)}$.\footnote{The full account will appear in future editions.}\end{proof}
\blankpage