%!name:datasets
%!need:sequences

\ssection{Why}

We want language and notation for recording elements of a set.

\ssection{Definition}

A \t{dataset} of size $n$ \t{in} (or \t{from}) a non-empty set $A$ is a finite sequence in $A$ (in other words, an element of the direct product $A^n$).
In other words, it is an $n$-tuple.
In the case that we speak of a dataset, we call its coordinates the records.
So, for example, we call the first coordinate the first record and the second coordinate the second record.
The $i$th coordinate is the $i$th record, for $i = 1, \dots, n$.

Although the terminology dataset is standard, datasets are not sets.
Datasets may contain an element multiple times.
Thus we use the single word \say{dataset,} in contrast with the two words \say{data set,} which may together suggest that the object involved was a set.

In the case that the dataset is from a set $A$ which is a product of two sets (say, $\CU$ and $\CV$, so that $A = \CU \times \CV$) we say that it is a dataset of \t{record pairs}.
A record pair is sometimes called an \t{observation}.

\ssubsection{Notation}

Let $A$ be a non-empty set.
Although we usually denote sequences by $(a_1, \dots, a_n)$, we will denote datasets by $(a^1, \dots, a^n)$; notably, using a superscript.
Often $A$ is itself a set of sequences, and so we will want to refer to the $i$th component of $a^1$ by $a^1_i$.
