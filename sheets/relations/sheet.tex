%!name:relations
%!need:ordered_pairs

\ssection{Why}

How can we relate the elements of two sets?

\ssection{Definition}

A \t{relation} between two nonempty sets is a subset of their cross product.
A relation on a single set is a subset of the cross product of it with itself.

The \t{domain} of a relation is the set of all elements which appear as the first coordinate of some ordered pair of the relation.
The \t{range} of a relation is the set of all elements which appear as the second coordinate of some ordered pair of the relation.

\ssubsection{Notation}

Let $A$ and $B$ be two nonempty sets.
A relation on $A$ and $B$ is a subset of $A \cross B$.
Let $C$ be a nonempty set.
A relation on a $C$ is a subset of $C \cross C$.

Let $a \in A$ and $b \in B$.
The ordered pair $(a, b)$ may or may not be in a relation on $A$ and $B$.
Also notice that if $A \neq B$, then $(b, a)$ is not a member of the product $A \cross B$, and therefore not in any relation on $A$ and $B$.
If $A = B$, however, it may be that $(b, a)$ is in the relation.


\ssubsection{Notation}
Let $A$ and $B$ be nonempty sets
with $a \in A$ and $b \in B$.
Since relations are sets,
we can use upper case Latin letters.
Let $R$ be a relation on $A$ and $B$.
We denote that $(a, b) \in R$ by
$a R b$, read aloud as
\say{a in relation $R$ to b.}

When $A = B$, we tend to use other symbols instead of letters.
For example,
$\sim$, $=$, $<$,
$\leq$, $\prec$, and $\preceq$.

\ssection{Properties}

Often relations are defined over a single set, and there are a few useful properties to distinguish.

A relation is \t{reflexive} if every element is related to itself.
A relation is \t{symmetric} if two objects are related regardless of their order.
A relation is \t{antisymmetric}{antisymmetric} if two different objects are related only in one order, and never both.
A relation is \t{transitive} if a first element is related to a second element and the second element is related to the third element, then the first and third element are related.

\ssubsection{Notation}

Let $R$ be a relation on
a non-empty set $A$.
$R$ is reflexive if
$$(a, a) \in R$$
for all $a \in A$.
$R$ is transitive if
$$(a, b) \in R \land (b, c) \in R \implies (a, c) \in R$$
for all $a, b, c \in A$.
$R$ is symmetric if
$$(a, b) \in R \implies (b, a) \in R$$
for all $a, b \in A$.
$R$ is anti-symmetric if
$$(a, b) \in R \implies (b, a) \not\in R$$
for all $a, b \in A$.
