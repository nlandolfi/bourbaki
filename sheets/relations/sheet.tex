%!name:relations
%!need:cartesian_products
%!refs:paul_halmos/naive_set_theory/section_07
%!refs:bert_mendelson/introduction_to_topology/theory_of_sets/relations

\ssection{Why}

How can we relate the elements of two sets?

\ssection{Definition}

A \t{relation} is a set of ordered pairs (see \sheetref{ordered_pairs}{Ordered Pairs}).
So if an object $z$ is an element of a relation, there exists two other objects $x, y$ so that $z = (x, y)$.

The \t{domain} of a relation is the set of all elements which appear as the first coordinate of some ordered pair of the relation (the projection onto the first coordinate, see \sheetref{ordered_pair_projections}{Ordered Pair Projections})
The \t{range} of a relation is the set of all elements which appear as the second coordinate of some ordered pair of the relation (the projection onto the second coordinate).

When the domain of a relation $R$ is a subset of $X$ and the range is a subset of $Y$, we say $R$ is \t{from $X$ to $Y$} or \t{between} $X$ and $Y$.
If $X = Y$, then $R$ speak of a relation \t{in} or \t{on} $X$.

% Let $A$ and $B$ be two nonempty sets.
% A relation on $A$ and $B$ is a subset of $A \cross B$.
% Let $C$ be a nonempty set.
% A relation on a $C$ is a subset of $C \cross C$.
%
% Let $a \in A$ and $b \in B$.
% The ordered pair $(a, b)$ may or may not be in a relation on $A$ and $B$.
% Also notice that if $A \neq B$, then $(b, a)$ is not a member of the product $A \cross B$, and therefore not in any relation on $A$ and $B$.
% If $A = B$, however, it may be that $(b, a)$ is in the relation.

\ssubsection{Notation}

If $R$ is a relation, we express that $(x, y) \in R$ by writing $x\,R\,y$, which we read as \say{$x$ is in relation $R$ to $y$}.
We denote the domain of $R$ by $\dom R$ and the range of $R$ by $\ran R$.

\ssection{Examples}

For an uninteresting relation, consider the empty set.
In the empty (set) relation, no object is related to any other.
Both the domain and range of $\emptyset$ are $\emptyset$.
For another simple relation, consider the product of any two sets $X$ and $Y$.
In $X \times Y$, all objects are related.
The domain is $X$ and the range is $Y$.

For a more interesting example, consider the set
\[
  R := \Set{(x, y) \in X \times X}{x = y}.
\]
This relation is the relation of equality (see \sheetref{identities}{Identities}) between two objects.
Here $x\,R\,y \iff x = y$.
$\dom R = \ran R = X$.
Another similar example is if we consider the set $X$ and $\powerset{X}$, and the relation
\[
  R := \Set{(x, y) \in X \times \powerset{X}}{x \in y}.
\]
This relation is the relation of belonging (see \sheetref{sets}{Sets}).
Here $x\,R\,y \iff x \in y$.
Here $\dom R = X$ and $\ran R = \powerset{X}$.

% \ssubsection{Notation}
% Let $A$ and $B$ be nonempty sets
% with $a \in A$ and $b \in B$.
% Since relations are sets,
% we can use upper case Latin letters.
% Let $R$ be a relation on $A$ and $B$.
% We denote that $(a, b) \in R$ by
% $a R b$, read aloud as
% \say{a in relation $R$ to b.}
%
% When $A = B$, we tend to use other symbols instead of letters.
% For example,
% $\sim$, $=$, $<$,
% $\leq$, $\prec$, and $\preceq$.

\ssection{Properties}

Often relations are defined over a single set, and there are a few useful properties to distinguish.

A relation is \t{reflexive} if every element is related to itself.
A relation is \t{symmetric} if two objects are related regardless of their order.
% A relation is \t{antisymmetric}{antisymmetric} if two different objects are related only in one order, and never both.
A relation is \t{transitive} if a first element is related to a second element and the second element is related to the third element, then the first and third element are related.
Equality is reflexive, symmetric and transitive whereas belonging is neither.
Exercise: what is inclusion?


% \ssubsection{Notation}
%
% Let $R$ be a relation on
% a non-empty set $A$.
% $R$ is reflexive if
% $$(a, a) \in R$$
% for all $a \in A$.
% $R$ is transitive if
% $$(a, b) \in R \land (b, c) \in R \implies (a, c) \in R$$
% for all $a, b, c \in A$.
% $R$ is symmetric if
% $$(a, b) \in R \implies (b, a) \in R$$
% for all $a, b \in A$.
% $R$ is anti-symmetric if
% $$(a, b) \in R \implies (b, a) \not\in R$$
% for all $a, b \in A$.
%
% \blankpage
