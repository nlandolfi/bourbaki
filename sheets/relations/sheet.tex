%!name:relations
%!need:set_products
%refs:paul_halmos∕naive_set_theory∕section_07
%!refs:bert_mendelson∕introduction_to_topology∕theory_of_sets∕relations

\section*{Why}

How can we relate the elements of two sets?

\section*{Definition}

A \t{relation} is a set of ordered pairs (see \sheetref{ordered_pairs}{Ordered Pairs}).
So if an object $z$ is an element of a relation, there exists two other objects $x, y$ so that $z = (x, y)$.

The \t{domain} of a relation is the set of all elements which appear as the first coordinate of some ordered pair of the relation (the projection onto the first coordinate, see \sheetref{ordered_pair_projections}{Ordered Pair Projections})
The \t{range} of a relation is the set of all elements which appear as the second coordinate of some ordered pair of the relation (the projection onto the second coordinate).

When the domain of a relation $R$ is a subset of $X$ and the range is a subset of $Y$, we say $R$ is \t{from $X$ to $Y$} or \t{between} $X$ and $Y$.
If $X = Y$, then $R$ speak of a relation \t{in} or \t{on} $X$.


%<div data-littype='paragraph'>
%  <div data-littype='run'> ❲% Let $A$ and $B$ be two nonempty sets.❳ </div>
%  <div data-littype='run'> ❲% A relation on $A$ and $B$ is a subset of $A \cross B$.❳ </div>
%  <div data-littype='run'> ❲% Let $C$ be a nonempty set.❳ </div>
%  <div data-littype='run'> ❲% A relation on a $C$ is a subset of $C \cross C$.❳ </div>
%  <div data-littype='run'> ❲%❳ </div>
%  <div data-littype='run'> ❲% Let $a ∈ A$ and $b ∈ B$.❳ </div>
%  <div data-littype='run'> ❲% The ordered pair $(a, b)$ may or may not be in a relation on $A$ and $B$.❳ </div>
%  <div data-littype='run'> ❲% Also notice that if $A ≠ B$, then $(b, a)$ is not a member of the product $A \cross B$, and therefore not in any relation on $A$ and $B$.❳ </div>
%  <div data-littype='run'> ❲% If $A = B$, however, it may be that $(b, a)$ is in the relation.❳ </div>
%</div>

\subsection*{Notation}

If $R$ is a relation, we express that $(x, y) \in R$ by writing $x\,R\,y$, which we read as \say{$x$ is in relation $R$ to $y$}.
We denote the domain of $R$ by $\dom R$ and the range of $R$ by $\ran R$.

\section*{Examples}

For an uninteresting relation, consider the empty set.
In the empty (set) relation, no object is related to any other.
Both the domain and range of $\varnothing$ are $\varnothing$.
For another simple relation, consider the product of any two sets $X$ and $Y$.
In $X \times  Y$, all objects are related.
The domain is $X$ and the range is $Y$.

For a more interesting example, define $R \subset X \times  X$ by
  \[
R = \Set{(x, y) \in X \times  X}{x = y}.
  \]
This relation is the relation of equality (see \sheetref{identities}{Identities}) between two objects.
Here $x\,R\,y \iff x = y$.
$\dom R = \ran R = X$.
Another similar example is if we consider the set $X$ and $\powerset{X}$, and the relation
  \[
R := \Set{(x, y) \in X \times  \powerset{X}}{x \in y}.
  \]
This relation is the relation of belonging (see \sheetref{sets}{Sets}).
Here $x\,R\,y \iff x \in y$.
Here $\dom R = X$ and $\ran R = \powerset{X}$.


%<div data-littype='paragraph'>
%  <div data-littype='run'> ❲% \ssubsection{Notation}❳ </div>
%  <div data-littype='run'> ❲% Let $A$ and $B$ be nonempty sets❳ </div>
%  <div data-littype='run'> ❲% with $a ∈ A$ and $b ∈ B$.❳ </div>
%  <div data-littype='run'> ❲% Since relations are sets,❳ </div>
%  <div data-littype='run'> ❲% we can use upper case Latin letters.❳ </div>
%  <div data-littype='run'> ❲% Let $R$ be a relation on $A$ and $B$.❳ </div>
%  <div data-littype='run'> ❲% We denote that $(a, b) ∈ R$ by❳ </div>
%  <div data-littype='run'> ❲% $a R b$, read aloud as❳ </div>
%  <div data-littype='run'> ❲% \say{a in relation $R$ to b.}❳ </div>
%  <div data-littype='run'> ❲%❳ </div>
%  <div data-littype='run'> ❲% When $A = B$, we tend to use other symbols instead of letters.❳ </div>
%  <div data-littype='run'> ❲% For example,❳ </div>
%  <div data-littype='run'> ❲% $∼$, $=$, $<$,❳ </div>
%  <div data-littype='run'> ❲% $≤$, $≺$, and $≼$.❳ </div>
%</div>

\section*{Properties}

Often relations are defined over a single set, and there are a few useful properties to distinguish.

A relation is \t{reflexive} if every element is related to itself.
A relation is \t{symmetric} if two objects are related regardless of their order.
  %  <div data-littype='run'> ❲% A relation is ❬antisymmetric❭{antisymmetric} if two different objects are related only in one order, and never both.❳ </div>

A relation is \t{transitive} if a first element is related to a second element and the second element is related to the third element, then the first and third element are related.
Equality is reflexive, symmetric and transitive whereas belonging is neither.
Exercise: what is inclusion?


%<div data-littype='paragraph'>
%  <div data-littype='run'> </div>
%  <div data-littype='run'> ❲% \ssubsection{Notation”❳ </div>
%  <div data-littype='run'> ❲%❳ </div>
%  <div data-littype='run'> ❲% Let $R$ be a relation on❳ </div>
%  <div data-littype='run'> ❲% a non-empty set $A$.❳ </div>
%  <div data-littype='run'> ❲% $R$ is reflexive if❳ </div>
%  <div data-littype='run'> ❲% $$(a, a) ∈ R$$❳ </div>
%  <div data-littype='run'> ❲% for all $a ∈ A$.❳ </div>
%  <div data-littype='run'> ❲% $R$ is transitive if❳ </div>
%  <div data-littype='run'> ❲% $$(a, b) ∈ R ∧ (b, c) ∈ R ⟹ (a, c) ∈ R$$❳ </div>
%  <div data-littype='run'> ❲% for all $a, b, c ∈ A$.❳ </div>
%  <div data-littype='run'> ❲% $R$ is symmetric if❳ </div>
%  <div data-littype='run'> ❲% $$(a, b) ∈ R ⟹ (b, a) ∈ R$$❳ </div>
%  <div data-littype='run'> ❲% for all $a, b ∈ A$.❳ </div>
%  <div data-littype='run'> ❲% $R$ is anti-symmetric if❳ </div>
%  <div data-littype='run'> ❲% $$(a, b) ∈ R ⟹ (b, a) \not∈ R$$❳ </div>
%  <div data-littype='run'> ❲% for all $a, b ∈ A$.❳ </div>
%  <div data-littype='run'> ❲%❳ </div>
%  <div data-littype='run'> ❲% \blankpage❳ </div>
%  <div data-littype='run'> </div>
%</div>
