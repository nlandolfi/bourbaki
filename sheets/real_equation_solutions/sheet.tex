
\section*{Why}

Given a real valued function (e.g., a polynomial) how do we compute its root.

\section*{Definition}

Given $f: \R  \to \R $, we call $f(x) = 0$ a \t{nonlinear equation} (a \t{nonlinear homogenous equation}).
If $x \in \R $ with $f(x) = 0$ we call $x$ a \t{root} or \t{solution} of $f$.

\subsection*{Examples}

For a classic example, suppose $s \in \R $ is given and consider the function $f: \R  \to \R $ defined by
\[
f(t) = t - \sqrt{s}
\]
Then the solutions $r \in \R $ for which $f(r) = 0$ are those points for which
\[
0 = r - \sqrt{s} \Rightarrow r = \sqrt{s}
\]
In other words, the \textit{roots} of $s$.

\blankpage