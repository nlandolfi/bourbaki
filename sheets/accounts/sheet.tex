%!name:accounts
%!need:letters
%!need:objects

\s{Why}

We want to succinctly and unambiguously record statements about names and the objects and sets of objects which the names refer to.

\s{Definition}

So far we have three kinds of sentences which we would like to record.

\begin{enumerate}

  \item
  \textbf{Names}
  We want to declare that we are using a particular name.
  When we start with a blank slate, once we have declared a name we do not use the same name twice.
  So we will want to keep track of which names are in use.

  \item
  \textbf{Identities}
  We want to declare that two names are equal.
  When we have names, we want to say when the identities of those names are the same.

  \item
  \textbf{Belongings}
  We want to declare when the object represented by one name is an element of the set represented by a second name.

\end{enumerate}

Our main purpose is to keep a list of the statements we have made.
Experience suggests that we start with an example.

Suppose we want to summarize the following english language description of some names and objects.
Denote an object by $a$.
Denote the same object by $a'$
In other words, $a$ equals $a'$.
Denote a set by $A$.
The object denoted by $a$ is an element of the set denoted by $A$.

\begin{account}[Example]
\name{$a$}
\name{$a'$}
\have{}{$a = a'$}
\name{$A$}
\have{}{$a \in A'$}
\end{account}

We have three primary

In the English language we have nouns and verbs.
The nouns reference objects and the verbs reflect the relations of these objects to each other.
In these sheets the nouns are names (introduced in \sheetref{}{Objects}) and we speak only in the present tense.

\ssection{Why}

We want to succinctly and unambiguously record statements about objects and sets of objects, keeping track of which names we are using.

\ssection{Definition}

We want to say the necessary and not the superfluous.
It is common in the history of mathematics to describe the development in English.
There are minor pitfalls to this.

%TODO: Ed. 2
We will not describe these here, but rather will give one example.
The first is that people of say \say{let $A$ be such and such}.
What they mean is \say{denote such and such by $A$} with a tacit assumption that you know that such and such means.
% Such and such may be some tangible object in the real world, or it may be some idea.
The practice is similar to using a pronoun.
If I say \say{Ben walks to the store} and then \say{He bought some brocolli,} we understand that \say{He} refers to Ben.
If I say \say{John walks to the store} and then \say{He bought some tofu,} we undersatnd that \say{He} refers to John.

The importance of using pronouns is relevant when we do not have proper names like Ben or John.
For example, suppose I say \say{The tall brown-haired thin man walks to the store} it saves quite a bit of sound to say \say{He} in the next thought, or even, \say{the man}.
% In these sheets we have used the pronouns it in many places, and will continue to do so.
These sheets include many other examples of using pronouns

Instead of \say{Let $A$ be such and such}, we will just write $\texttt{name} \; A$.
The idea is that we are introducing a name, $A$, that is some symbols to denote some object.
The use of a different font and the symbol \texttt{name} before $A$ is an abbreviation of the following: \say{Look at this symbol here $A$. This symbol is a reference for some object. It will always reference that object.}
% We will procede to say things about these objects.

For the statement that the object denoted by $A$ \textit{is} or \textit{is the same} or \textit{is the very same object} as the object denoted by $B$, we will use $=$.
So $=$we will write $A = B$.

If all we want to say is that two objects

We also need some way of identifying that two names


\begin{account}
  \name{$x$}
  \name{$y$}
  \have{a}{$x = y$}
\end{account}

\begin{account}
  \name{$x$}
  \name{$y$}
  \name{$z$}
  \have{a}{$x \subset y$}
  \have{b}{$y \subset z$}
  \thus{c}{$(\forall a)(a \in x \implies a \in y)$}{\ref{a}}
  \thus{d}{$(\forall a)(a \in y \implies a \in z)$}{\ref{b}}
  \thus{asdf}{$x \subset z$}{\ref{c},\ref{d}}
\end{account}

\begin{account}
  \nameee{$x$}{$y$}{$z$}
  \have{a}{$x \subset y$}
  \have{b}{$y \subset z$}
  \thus{c}{$(\forall a)(a \in x \implies a \in y)$}{\ref{a}}
  \thus{d}{$(\forall a)(a \in y \implies a \in z)$}{\ref{b}}
  \thus{asdf}{$x \subset z$}{\ref{c},\ref{d}}
\end{account}

\blankpage
