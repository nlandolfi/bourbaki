%!name:accounts
%!need:deductions

\s{Why}

We want to succinctly and clearly make several statements about objects and sets. We want to track the names we use, taking care to avoid using the same name twice.

\s{Definition}

An \t{account}\footnote{This sheet will be expanded in future editions.} is several, sequential naming, logical, and quantified statements.
% , identity and belonging statements
We say \say{let \_ denote a \_} to introduce a name as a placeholder for a object, and we use $\_=\_$ and $\_\in\_$ to denote statements of identity and belonging.
In other words, we have three types of sentences to record.

\begin{enumerate}

  \item
  \textbf{Names.}
  State we are using a name.
%   When we start with a blank slate, once we have declared a name we do not use the same name twice.
%   So we will want to keep track of which names are in use.

  \item
  \textbf{Identity.}
  We want to make statements of identity.
%     that two names are equal.
%   When we have names, we want to say when the identities of those names are the same.

  \item
  \textbf{Belonging.}
  We want to make statements of belonging.
%     when the object represented by one name is an element of the set represented by a second name.

\end{enumerate}

Our main purpose is to keep a list of names, of quantified, logical and simple statments about them, and then statements we can deduce from these.
% of the statements we have made.
In particular we want to group our name usage.
% And we want to group declaration of names as placeholders with the statements that use the placeholders.
In the English language we use paragraphs or sections to do so.
In these sheets, we will use accounts.
We will list the statements and label each with Arabic numerals (see \sheetref{letters}{Letters}).
% which will be a list of statements, each of which is labeled by an Arabic numeral (see \sheetref{letters}{Letters}).

Experience suggests that we start with an example.
Suppose we want to summarize the following english language description of some names and objects.

\begin{quote}
Denote an object by $a$.
Also, denote the same object by $b$.
% Also, In other words, $a$ equals $a'$ or $a = a'$.
Also, denote a set by $A$.
Also, the object denoted by $a$ is an element of the set denoted by $A$.
Also denote an object by $c$.
Also $c$ is the same object as $b$.
\end{quote}

In our usual manner of speaking, we drop the word \say{also}.
In these sheets, we translate each of the sentences into our symbols.
For names we use, we write \texttt{name} in that font followed by the name.
For logical statements we assume or take as premisses (in other words, which we alrady \say{have}), we write \texttt{have} followed by the logical statement.
For deductions we write \texttt{thus} followed by the conclusion and then \texttt{by} followed by the Arabic numerals of the premisses.
So we write:

\begin{account}[First Example]
\name{$a$}
\name{$b$}
\have{accounts:firstexample:equality}{$a = b$}
\name{$A$}
\have{accounts:firstexample:inclusion}{$a \in A$}
\name{$c$}
\have{accounts:firstexample:secondequality}{$c = b$}
\thus{accounts:firstexample:thirdequality}{$a = c$}{\ref{accounts:firstexample:equality},\ref{accounts:firstexample:secondequality}}
\end{account}

% We will abbreviate like so, and bring all names to the beginning.

% \begin{account}[Example]
% \nameee{$a$}{$a'$}{$A$}
% \have{}{$a = a'$}
% \have{}{$a \in A'$}
% \end{account}

% \newcommand{\satisfies}{\triangleleft}
% \newcommand{\Set}{\mathsf{Set}}
% \newcommand{\Obj}{\mathsf{Obj}}
% \newcommand{\Group}{\mathsf{group}}
% \begin{account}[Example]
% \nameee{$a$}{$a'$}{$A$}
%   \have{}{a \dashv \texttt{Groups/Axioms}}
% \name{$a:\Obj$}
% \name{$a':\Obj$}
% \have{}{$a \satisfies \Set}
% \have{}{$a = a'$}
% \name{$A:\Set$}
% \have{}{$a \in A'$}
% \end{account}

% We have three primary
%
% In the English language we have nouns and verbs.
% The nouns reference objects and the verbs reflect the relations of these objects to each other.
% In these sheets the nouns are names (introduced in \sheetref{}{Objects}) and we speak only in the present tense.
%
% \ssection{Why}
%
% We want to succinctly and unambiguously record statements about objects and sets of objects, keeping track of which names we are using.
%
% \ssection{Definition}
%
% We want to say the necessary and not the superfluous.
% It is common in the history of mathematics to describe the development in English.
% There are minor pitfalls to this.
%
% %TODO: Ed. 2
% We will not describe these here, but rather will give one example.
% The first is that people of say \say{let $A$ be such and such}.
% What they mean is \say{denote such and such by $A$} with a tacit assumption that you know that such and such means.
% Such and such may be some tangible object in the real world, or it may be some idea.
% The practice is similar to using a pronoun.
% If I say \say{Ben walks to the store} and then \say{He bought some brocolli,} we understand that \say{He} refers to Ben.
% If I say \say{John walks to the store} and then \say{He bought some tofu,} we undersatnd that \say{He} refers to John.
%
% The importance of using pronouns is relevant when we do not have proper names like Ben or John.
% For example, suppose I say \say{The tall brown-haired thin man walks to the store} it saves quite a bit of sound to say \say{He} in the next thought, or even, \say{the man}.
% In these sheets we have used the pronouns it in many places, and will continue to do so.
% These sheets include many other examples of using pronouns
%
% Instead of \say{Let $A$ be such and such}, we will just write $\texttt{name} \; A$.
% The idea is that we are introducing a name, $A$, that is some symbols to denote some object.
% The use of a different font and the symbol \texttt{name} before $A$ is an abbreviation of the following: \say{Look at this symbol here $A$. This symbol is a reference for some object. It will always reference that object.}
% We will procede to say things about these objects.
%
% For the statement that the object denoted by $A$ \textit{is} or \textit{is the same} or \textit{is the very same object} as the object denoted by $B$, we will use $=$.
% So $=$we will write $A = B$.
%
% If all we want to say is that two objects
%
% We also need some way of identifying that two names
%
%
% \begin{account}
%   \name{$x$}
%   \name{$y$}
%   \have{a}{$x = y$}
% \end{account}
%
% \begin{account}
%   \name{$x$}
%   \name{$y$}
%   \name{$z$}
%   \have{a}{$x \subset y$}
%   \have{b}{$y \subset z$}
%   \thus{c}{$(\forall a)(a \in x \implies a \in y)$}{\ref{a}}
%   \thus{d}{$(\forall a)(a \in y \implies a \in z)$}{\ref{b}}
%   \thus{asdf}{$x \subset z$}{\ref{c},\ref{d}}
% \end{account}
%
% \begin{account}
%   \nameee{$x$}{$y$}{$z$}
%   \have{a}{$x \subset y$}
%   \have{b}{$y \subset z$}
%   \thus{c}{$(\forall a)(a \in x \implies a \in y)$}{\ref{a}}
%   \thus{d}{$(\forall a)(a \in y \implies a \in z)$}{\ref{b}}
%   \thus{asdf}{$x \subset z$}{\ref{c},\ref{d}}
% \end{account}

% \blankpage
