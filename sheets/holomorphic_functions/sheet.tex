
\section*{Definition}

Let $\Omega $ be an open set in $\C $ and let $f: \Omega  \to \C $.
The function $f$ is \t{holomorphic at the point} $z_0 \in \C $ if the complex quotient
\[
\frac{f(z_0 + h) - f(z_0)}{h}
\]
has a limit when $h \to \infty$, where $h \in \C $, $h \not= 0$ and $z_0 + h \in \Omega $ so that the quotient is well-defined.

This condition is similar to saying that a function is differentiable, except that the $h$ is complex and so the condition above encomposes all limits approaching $z$ (all angles) in the complex plane.\footnote{Future editions will clarify.}
But we emphasize that $h$ is a complex number approaching the complex number $(0, 0)$ from any direction.
If the limit exists, then we call its value the \t{derivative of $f$ at $z_0$}.

The function $f$ is \t{holomorphic} on $\Omega $ if $f$ is holomorphic at every point of $\Omega $.
If $C$ is a closed subset of $\C $, we say that $f$ is holomorphic on $C$ if $f$ is holomorphic on some open set containing $c$.
If $f$ is holomorphic on all of $C$ then we call $f$ \t{entire}.
A holomorphic function is sometimes called \t{regular} or \t{complex differentiable}.
The latter term is used in view of the similarities with the definition of a real derivative.

\subsection*{Notation}

In the case that $f: \Omega  \to \C $ is holomorphic at $z_0$ we denote the derivative at $z_0$ by $f'(z_0)$.
We have defined $f'(z_0)$ by
\[
f'(z_0) = \lim_{h \to 0} \frac{f(z_0 + h) - f(z_0)}{h}.
\]

\blankpage