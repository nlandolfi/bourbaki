
%!name:monotone_sequences
%!need:sequences

\section*{Why}

If the base set of a sequence has a partial order, then we can discuss its relation to the order of sequence.

\section*{Definition}

A sequence on a partially ordered set is \t{non-decreasing} if whenever a first index precedes a second index the element associated with the first index precedes the element associated with the second element.
A sequence on a partially ordered set is \t{increasing} if it is non-decreasing and no two elements are the same.

A sequence on a partially ordered set is \t{non-increasing} if whenever a first index precedes a second index the element associated with the first index succedes the element associated with the second element.
A sequence on a partially
ordered set is
\t{decreasing}
if it is non-increasing and
no two elements are the same.

A sequence on a partially
ordered set is
\t{monotone}
if it is non-decreasing,
or non-increasing.
An increasing sequence is
non-decreasing.
A decreasing sequences is
non-increasing.
A sequence on a partially
ordered set is
\t{strictly monotone}
if it is decreasing,
or increasing.

\subsection*{Notation}

Let $A$ a non-empty set with partial order $\preceq$.
Let $\seq{a}$ a sequence in $A$.

The sequence is non-decreasing if $n \leq m \implies a_n \preceq a_m$, and increasing if $n < m \implies a_n \prec a_m$.
The sequence is non-increasing if $n \leq m \implies a_n \succeq a_m$, and decreasing if $n < m \implies a_n > a_m$.

\section*{Examples}

\begin{example}
Let $A$ a non-empty set and $\seq{A}$ a sequence of sets in $\powerset{A}$.
Partially order elements of $\powerset{A}$ by the relation contained in.
\end{example}
