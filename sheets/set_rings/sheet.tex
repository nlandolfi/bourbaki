
%!name:set_rings
%᜶!need:rings
%!need:set_unions
%!need:set_differences
%!refs:paul_halmos/measure_theory/chapter_I_sets_and_classes/4_rings_and_algebra

\section*{Definition}

A \t{set ring} (or \t{Boolean set ring}, \t{ring of sets}, \t{Boolean ring of sets}) is a nonempty set of sets $R$ such that if
    \[
E \in R \quad \text{and} \quad F \in R
    \]
then
    \[
E \cup F \in R \quad \text{and} \quad E - F \in R.
    \]
In other words, a ring is a \sheetref{empty_set}{nonempty}set of sets which is closed under \sheetref{set_unions}{unions}and \sheetref{set_differences}{differences}.

Every ring contains the empty set, for if $E \in R$, then $E - E = \varnothing \in R$.

Also, since
    \[
E - F = (E \cup F) - F,
    \]
every nonempty set that is closed under unions and \textit{proper} differences is a ring.

\blankpage