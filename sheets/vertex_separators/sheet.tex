
\section*{Why}

We characterize chordal graphs.\footnote{Future editions will expand.}

\section*{Definition}

Let $(V, E)$ be an undirected graph.
A set $S \subset V$ is a \t{vertex separator} for two vertices $v, w$ (or a \t{$vw$-separator}) if $v$ and $w$ are disconnected in the subgraph induced by $V \setminus S$.
There always exists a vertex separator for two nonadjacent vertices.

A vertex separator is a \t{minimal vertex separator} for two vertices if no proper subset of it is a vertex separator for those vertices.
Another term for vertex separator is \t{cutset}.
Similarly, one for minimal vertex separator is \t{relatively minimal cutset}.

\subsection*{Example}

For example, for the graph in Figure below
%~\ref{figure:vertex_separators:cutsets}

, $\set{c,e}$ is a minimal $ag$-separator and $\set{b,c,e}$ is a minimal $ad$-separator.
A minimal vertex separator may contain another minimal vertex separator if they are minimal for different pairs of vertices.
%    <div data-littype='footnote'>
%      <div data-littype='run'> See Vandenberghe and Andersen, 2014. </div>
%    </div>


%  <div data-littype='run'> \begin{figure} </div>
%  <div data-littype='run'> \centering </div>
%  

\begin{center}\includegraphics[width=0.70\textwidth]{./graphics/cutsets.png}\end{center}
%  <div data-littype='run'> \caption{A chordal graph and its five minimal cutsets.} </div>
%  <div data-littype='run'> \label{figure:vertex_separators:cutsets} </div>
%  <div data-littype='run'> \end{figure} </div>
%</div>
