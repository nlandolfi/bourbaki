%!name:vertex_separators
%!need:set_differences
%!need:chordal_graphs

\ssection{Why}

We characterize chordal graphs.\footnote{Future editions will expand.}

\ssection{Definition}

Let $(V, E)$ be an undirected graph.
A set $S \subset V$ is a \t{vertex separator} for two vertices $v, w$ (or a \t{$vw$-separator}) if $v$ and $w$ are disconnected in the subgraph induced by $V \setminus S$.
There always exists a vertex separator for two nonadjacent vertices.

A vertex separator is a \t{minimal vertex separator} for two vertices if no proper subset of it is a vertex separator for those vertices.
Another term for vertex separator is \t{cutset}.
Similarly, one for minimal vertex separator is \t{relatively minimal cutset}.

\ssubsection{Example}

For example, for the graph in Figure~\ref{figure:vertex_separators:cutsets}, $\set{c,e}$ is a minimal $ag$-separator and $\set{b,c,e}$ is a minimal $ad$-separator.
A minimal vertex separator may contain another minimal vertex separator if they are minimal for different pairs of vertices.\footnote{See Vandenberghe and Andersen, 2014.}

\begin{figure}
  \centering
  \includegraphics[width=0.7\textwidth]{graphics_included/cutsets}
  \caption{A chordal graph and its five minimal cutsets.}
  \label{figure:vertex_separators:cutsets}
\end{figure}


