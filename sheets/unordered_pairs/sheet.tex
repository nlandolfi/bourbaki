%!name:unordered_pairs
%!need:set_specification

\ssection{Why}

Are there enough sets to ensure that every set is an element of some set?
What of one set and another set --- is there a set that they both belong to?

\ssection{Definition}

We will say that there is.
For one set and another set, there exists a set that they both belong to.
We refer to this as the \t{axiom of pairing}.

If there exists a set that contains both the sets we began with, then there exists a set which contains them and nothing else.
First, use the axiom of pairing to obtain a set containing both sets, and then use the axiom of specification with a sentence that is true only if the element considered is one of the sets we began with.
As a result of the axiom of extension, there can be only one set with this property.
We call this set a \t{pair} or an \t{unordered pair}.

\ssection{Notation}

Let $a$ and $b$ be objects.
We denote the set which contains $a$ and $b$ as elements and nothing else by $\set{a, b}$.

The pair of $a$ with itself is the set $\set{a, a}$ which we will denote by $\set{a}$.
