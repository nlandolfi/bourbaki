%!name:unordered_pairs
%!need:set_specification

\ssection{Why}

Are all objects contained in some set?
What of two objects---is there a set containing them?

\ssection{Definition}

We will say that there is.
For one set and another set, there exists a set that they both belong to.
We refer to this as the \t{principle of pairing}.

\begin{principle}[Pairing]
	Given two objects, there exists a set containing those two object.
\end{principle}

If there exists a set that contains both the objects we began with, then there exists a set which contains them and nothing else.
The principle of pairing dictates existence, and the principle of specification (see \sheetref{set_specification}{Set Specification}) gives a set which contains only those two elements.
We use the statement that asserts equality with on or the other of the two objects.
The principle of extension (see \sheetref{set_equality}{Set Equality}) As a result of the axiom of extension, there can be only one set with this property.
We call this set a \t{pair} or an \t{unordered pair}.

\ssection{Notation}

Let $a$ and $b$ be objects.
We denote the set which contains $a$ and $b$ as elements and nothing else by $\set{a, b}$.
The pair of $a$ with itself is the set $\set{a, a}$ which we will denote by $\set{a}$.

\blankpage
