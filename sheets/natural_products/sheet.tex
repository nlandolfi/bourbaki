
\section*{Why}

We want to add repeatedly.

\section*{Definitiong result}

\begin{proposition}
For each natural number $m$, there exists a function $p_m: \omega  \to \omega $ which satisfies
\[
p_m(0) = 0 \quad \text{ and } \quad p_m(\ssuc{n}) = \ssuc{(p_m(n))} + m
\]
for every natural number $n$.
\end{proposition}

\begin{proof}The proof uses the recursion theorem (see \sheetref{recursion_theorem}{Recursion Theorem}).\footnote{Future editions will give the entire account.}\end{proof}
Let $m$ and $n$ be natural numbers.
The value $p_m(n)$ is the \t{product} of $m$ with $n$.

% Let $m$ and $n$ be two natural numbers.\footnote{Future editions will explicitly use the recursion theorem.}
% If we apply the successor function to $m$ $n$
% times we obtain a number.
% If we apply the successor function to $n$ $m$
% times we obtain a number.
% Indeed, we obtain the same number in both cases.
% We call this number the \t{sum}
% of $m$ and $n$.
% We say we \t{add} $m$ to $n$,
% or vice versa.
% We call this correspondence, between
% $(m, n)$ and the sum, \t{addition}.
% 

\subsection*{Notation}

We denote the product $p_m(n)$ by $m \cdot  n$.
We often drop the $\cdot $ and write $m \cdot  n$ as $mn$.

% two numbers by $m + n$.
% We denote the function addition by $+$
% and so denote the sum of the naturals
% $m$ and $n$ by $m + n$.

\section*{Properties}

The properties of products are direct applications of the principle of mathematical induction (see \sheetref{natural_induction}{Natural Induction}).\footnote{Future editions will include the accounts.}

\begin{proposition}[Associativity]
Let $k$, $m$, and $n$ be natural numbers. Then
\[
(k \cdot  m) \cdot  n = k \cdot  (m \cdot  n).
\]
\end{proposition}

\begin{proposition}
Let $m$ and $n$ be natural numbers. Then
\[
m \cdot  n = n \cdot  m.
\]
\end{proposition}

%Let $m$ and $n$ naturals.\footnote{Future editions will explicitly use the recursion theorem.}
%If we add $n$ copies of $m$ we obtain a number.
%If we add $m$ copies of $n$ we obtain a number.
%Indeed, we obtain the same number in both cases.
%We call this number the \t{product} of $m$ and $n$.
%We say we \t{multiply} $m$ to $n$, or vice versa.
%We call this symmetric operation mapping $(m, n)$ to their product
%\t{multiplication}.
% \footnote{To invoke this we
%\ssubsection{Notation}
%We denote the product of $m$ and $n$ by $m \cdot n$.
