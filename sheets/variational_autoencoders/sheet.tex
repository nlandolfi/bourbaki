%!name:variational_autoencoders
%!need:autoencoders
%!need:probabilistic_models
%!need:parametric_distribution_network_families

\ssection{Why}\footnote{Future editions will include. Future editions may also change the name of this sheet. It is also likely that there will be added prerequisite sheets on variational inference.}

\ssection{Parameterizing distributions}

\ssection{Definition}

A \t{variational autoencoder} is an ordered pair whose first coordinate with \t{latent distribution} (density) $p_z: Z \to \R$ and \t{observation distribution} (density) $p_x: X \to \R$ is a ordered pair $()$ discrete (continuous) latent set $Z$ and discrete (continuous) observation set $X$ is a tuple



where
(a) $\nu$ is an autoencoder (which need not be regular, see \sheetref{autoencoders}{Autoencoders}),
(b) $q_{z \mid x}: Z \times X \to \R$ is a conditional distribution (density) called the \t{recognition distribution} (\t{recognition density}),
(c) $p_{z}: Z \to \R$ is a distribution (density) called the \t{latent prior model}, and
(d) $p_{x \mid z}: X \times Z \to \R$ is a conditional distribution (density) called the \t{generating model}.

In other words, for (a) $q_{z \mid x}(\cdot, \xi): Z \to \R$ is a distribution (density) for each $\xi \in X$ and for (d) $p_{x \mid z}(\cdot, \zeta): X \to \R$ is a distribution (density) on $X$ for each $\zeta \in Z$.

If the model has discrete latent set and discrete observation set (or continuous latent set and continuous observation set), the \t{joint distribution} (\t{joint density}) $p_{zx}: Z \times X \to \R$ is defined by $p_{zx} = p_{z}p_{x \mid z}$.
The \t{observation distribution}

A \t{continuous-continuous variational autoencoder family} (\t{discrete-discrete}, \t{discrete-continuous}, \t{continuous-discrete}) is a tuple
\[
  (\nu, \set{(q^{(\theta)}, p_z^{(\theta)}, p_{x \mid z}^{(\theta))})}_{\theta \in \Theta}),
\]
where:
\begin{itemize}

  \item
    $\nu$ is an autoencoder with encoder $f: \R^d \to \R^k$ and decoder $g: \R^\ell \to \R^m$.
    The autoencoder need not be regular, see \sheetref{autoencoders}{Autoencoders}).
%     We call it the \t{autoencoder}.

  \item
    $\Theta \subset \R^p$.
%     We call it
    The \t{parameter set} (or \t{parameter space}).

  \item
    $q^{(\theta)}: \R^h \to \R$ is a density (distribution, density, distribution), for each $\theta \in \Theta$.
    We call $\set{q^{(\theta)}}_{\theta \in \Theta}$ the \t{recognition model family}.

  \item
    $p_{z}^{\theta}: \R^h \to \R$ is a density (distribution, distribution, density), for each $\theta \in \Theta$.
    We call $\set{p_{z}^{(\theta)}}_{\theta}$ the \t{latent prior model family}.

  \item
    $p_{x \mid z}^{(\theta)}: \R^d \times \R^h \to \R$ is a conditional density (distribution, density, distribution).
    In other words,  $p_{x \mid z}^{(\theta)}(\cdot, \zeta): \R^d \to \R$ is a density (distribution, density, distribution) for every $\zeta \in \R^d$.
    We call $\set{p_{x \mid z}^{(\theta)}}_{\theta \in \Theta}$ the \t{observation model family}.
\end{itemize}

A \t{variational autoencoder} (or \t{VAE}) may refer to any of the above.
The convention we have adopted is \say{latent type}-\say{observation type}.

\blankpage
