%!name:set_powers
%!need:set_inclusion

\s{Why}

We want to consider all the subsets of a given set.

\begin{principle}[Powers]
  For every set, there exists a set of its subsets.
\end{principle}

We call the existence of this set the \t{principles of powers} and we call the set the \t{power set}.\footnote{This terminology is standard, but unfortunate. Future editions may change these terms.}
As usual, the principle of extension gives uniqueness (see \sheetref{set_equality}{Set Equality}).
The power set of a set includes the set itself and the empty set.

\ssubsection{Notation}

Let $A$ denote a set.
We denote the power set of $A$ by $\powerset{A}$, read aloud as \say{powerset of A.}
$A \in \powerset{A}$ and $\emptyset \in \powerset{A}$.
However, $A \subset \powerset{A}$ is false.

\ssubsection{Examples}

Let $a, b, c$ denote distinct objects. Let $A = \set{a, b ,c}$
and $B = \set{a, b}$. Then
$B \subset A$.
In other notation,
$B \in \powerset{A}$.
We can walk through examples of power sets.

\ss{Empty Set}

\begin{proposition}
  $\powerset{\emptyset} = \set{\emptyset}$
\end{proposition}

\ss{Singletons}

\begin{proposition}
  $\powerset{\set{a}} = \set{\emptyset, \set{a}}$
\end{proposition}

\ss{Pairs}

\begin{proposition}
  $\powerset{\set{a, b}} = \set{\emptyset, \set{a}, \set{b}, \set{a, b}}$
\end{proposition}

\ss{Triples}

\begin{proposition}
  $\powerset{\set{a, b, c}}
  \set{
    \emptyset,
    \set{a},
    \set{b},
    \set{c},
    \set{a, b},
    \set{b, c},
    \set{a, c},
    \set{a, b, c}
  }$
\end{proposition}

\ssection{Properties}

We can guess the following easy properties.\footnote{Future editions will expand this account.}

\begin{proposition}
  $\emptyset \in \powerset{A}$
\end{proposition}

\begin{proposition}
  $A \in \powerset{A}$
\end{proposition}

We call $A$ and $\emptyset$ the \t{improper} subsets of $A$.
All other subset we call $\t{proper}$.

\blankpage
