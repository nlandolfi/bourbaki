
\section*{Why}

We want to consider all the subsets of a given set.

\section*{Definition}

We do not yet have a principle stating that such a set exists, but our intuition suggests that it does.

\begin{principle}[powers]
For every set, there exists a set of its subsets.
\end{principle}

We call the existence of this set the \t{principle of powers} and we call the set the \t{power set}.\footnote{This terminology is standard, but unfortunate. Future editions may change these terms.}
As usual, the principle of extension gives uniqueness (see \sheetref{set_equality}{Set Equality}).
The power set of a set includes the set itself and the empty set.

\subsection*{Notation}

Let $A$ denote a set.
We denote the power set of $A$ by $\powerset{A}$, read aloud as ``powerset of A.''
$A \in \powerset{A}$ and $\varnothing \in \powerset{A}$.
However, $A \subset \powerset{A}$ is false.

\subsection*{Examples}

Let $a, b, c$ denote distinct objects. Let $A = \set{a, b ,c}$
and $B = \set{a, b}$. Then
$B \subset A$.
In other notation,
$B \in \powerset{A}$.
Showing each of the following is straightforward.
  \begin{enumerate}
    \item The empty set: $\powerset{\varnothing} = \set{\varnothing}$
    \item Singletons: $\powerset{\set{a}} = \set{\varnothing, \set{a}}$
    \item Pairs: $\powerset{\set{a, b}} = \set{\varnothing, \set{a}, \set{b}, \set{a, b}}$
    \item Triples:
\[
\powerset{\set{a, b, c}} =
\{
\varnothing,
\set{a},
\set{b},
\set{c},
\set{a, b},
\set{b, c},
\set{a, c},
\set{a, b, c}
\}
\]
  \end{enumerate}

\section*{Properties}

We can guess the following easy properties.\footnote{Future editions will expand this account.}

\begin{proposition}
$\varnothing \in \powerset{A}$
\end{proposition}

\begin{proposition}
$A \in \powerset{A}$
\end{proposition}

We call $A$ and $\varnothing$ the \t{improper} subsets of $A$.
All other subsets we call \t{proper}.

\section*{Basic fact}

\begin{proposition}
$E \subset F \implies \powerset{E} \subset \powerset{F}$
\end{proposition}
