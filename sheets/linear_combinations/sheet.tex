%!name:linear_combinations
%!need:vectors
%!need:real_linear_combinations

\section*{Why}

We want to build vectors out of other vectors using scalar multiplication and vector addition.

\section*{Definition}

A \t{linear combination} \t{from} a vector space is an ordered pair: the first coordinate is a sequence of vectors and the second is a sequence of scalars.
The \t{result} of a linear combination is the sum of the results of scaling each vector by the corresponding scalar in the sequence; itself a vector in the space.

A \t{trivial linear combination} is one whose sequence of scalars is zero at each coordinate.
The result of any trivial linear combination is the zero vector.
A \t{nontrivial linear combination} is one which is not trivial.
In other words, to be nontrivial, there must exist at least one index of the scalar sequence whose corresponding value is nonzero.

We say that a given vector \t{can be written as a linear combination of} a sequence of vectors if there exists a sequence of scalars such that the result of the linear combination of that sequence of vectors and scalars is the given vector.
In other words, a vector can be written as a linear combination of some other vectors if there exists scalars for those other vectors such that scaling them and adding the results gives the vector.

\subsection*{Notation}

Let $(V, \F )$ be a vector space.
Let $v = (v_1, \dots , v_n)$ be a sequence of vectors in $V$ and
$a = (a_1, \dots , a_n)$ be a sequence of scalars in $\F $.
Then $(v, a)$ is a linear combination and we can express its result by
  \[
a_1v_1 + a_2v_2 + \cdots + a_n v_n.
  \]
If $(v,a)$ is trivial, then $a_i = 0$ for $i \in \upto{n}$ and the result of $(v, a)$ is $0$ (the zero vector).
Otherwise, there exists $i \in \upto{n}$ such that $a_i \neq 0$; of course, the result of $(v, a)$ may still be $0$.
A vector $u$ can be written as a linear combination of the vectors $v_1, v_2, \dots , v_n$ if there exists scalars $a_1, a_2, \dots , a_n$ so that
  \[
u = a_1v_1 + a_2v_2 + \cdots + a_nv_n.
  \]
We often number (see \sheetref{lists}{Lists}) a set of finite vectors.
In this case, we say \say{Let $\set{u_1, \dots , u_n}$ be a (finite) set of vectors.}
