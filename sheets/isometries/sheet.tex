
\section*{Why}

A metric space is a set with a prescribed notion of distance.
Though the sets may be distinct, if elements can be made to correspond to each other with distances preserved then one space is as good a model as the other space.

\section*{Definition}

Two metric spaces are \t{isometric} if there exists an invertible function such that the distance of two elements in one space is the same as the distance of their result (under the function) in the second space (and vice versa).

We call either of the inverse functions an \t{isometry}.
An isometry maps each element of one space to an element of the second space and preserves distances.
Since inverse functions exist and are unique, either isometry is sufficient for demonstrating two spaces are isometric.

\subsection*{Notation}

Let $(A, d)$ and $(B, d')$ be metric spaces.
Let $f: A \to B$ and $g: B \to A$ such that
\[
d(a, b) = d'(f(a), f(b))
\]
for all $a, b \in A$ and
\[
d'(c, d) = d(g(c), g(d))
\]
for all $c, d \in B$.
Then $(A, d)$ and $(B, d')$ are isometric.
We call either of $f$ or $g$ an isometry.
