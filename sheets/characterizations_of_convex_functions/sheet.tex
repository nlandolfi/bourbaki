
\section*{Why}

Convex functions satisfy an interpolation property.
% TODO: who cares an where is it useful 


\section*{Discussion}

By definition, given $S \subset \R ^n$, $f: S \to \R $ is convex means that the point
\[
(1 - \lambda )(x \mu ) + \lambda (x, \nu ) = ((1-\lambda )x + \lambda y, (1-\lambda )\mu  + \lambda \nu
\]
belongs to $\epi f$ whenever $(x, \mu )$ and $(y, \nu )$ belong to $\epi f$ and $0 \leq \lambda  \leq 1$.
Said differently, we have $(1-\lambda )x + \lambda y \in S$, and
\[
f((1-\lambda )x + \lambda y) \leq (1-\lambda )\mu  + \lambda \nu
\]
whenever $x \in S$, $y \in S$, $f(x) \leq \mu  \in \R $, and $f(y) \leq \nu  \in \R $.

Concave functions have a similar property, with the inequalities flipped, and affine functions satisfy with qualities.
This shows that the functions $f: \R ^n \to \R $ we are calling affine coincide exactly with the affine transformations from $\R ^n$ to $\R $.

% TODO: add theorem 4.1 and 4.2 4.3 of Rockafeller 

\subsection*{Visualization}

\begin{center}\includegraphics[width=0.80\textwidth]{./graphics/cvxepi_interpolation.pdf}\end{center}
\blankpage