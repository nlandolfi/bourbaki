
\section*{Why}

We are constantly thinking of the real numbers as the points of a line.\footnote{Future editions will modify this sheet.}

\section*{Discussion}

We commonly associate elements of the real numbers (see \sheetref{real_numbers}{Real Numbers}) with points on a line (see \sheetref{geometry}{Geometry}).

\begin{principle}[Point Sets]
Given a line, there exists a set of its (infinite) points.
\end{principle}

\begin{principle}[Real Line Correspondence]
Let $P$ be the set of points for a line.
There exists a one-to-one correspondence mapping elements of $P$ onto elements of $\R $.
\end{principle}

For this reason, we sometimes call elements of the real numbers \t{points}.
We call the point associated with 0 the \t{origin}.

\subsection*{Visualization}

To visualize the correspondence we draw a line.
We then associate a point of the line with the $0 \in \R $.
We can label it so.
We then pick a unit length.
We associate the points a unit length away from zero with $1 \in \R $ (on the right) and $-1 \in \R $ (on the left).
We do the same for two and $2$ and $-2$, $3$ and $-3$, and then we say that we could continue the process indefinitely.
% todo: fix this; make it a figure 

We can visualize the image below
\begin{center}    \includegraphics[width=0.90\textwidth]{./graphics/real_line.pdf}\end{center}

%<div data-littype='paragraph'>
%  <div data-littype='run'> ❲% \par↤\rule{0.9\textwidth}{0.4pt}❳ </div>
%</div>
%<div data-littype='paragraph'>
% <div data-littype='run'> \begin{figure}[h] </div>
% <div data-littype='run'> \centering </div>
% <div data-littype='run'> \vspace{0.5cm} </div>
% <div data-littype='run'> \includegraphics[width=0.9\textwidth]{\figpath{real_line}{real_line}} </div>
% <div data-littype='run'> \caption{The real line} </div>
% <div data-littype='run'> \label{real_line:figure:real_line} </div>
% <div data-littype='run'> \end{figure} </div>
%</div>
%<tex>
% <div data-littype='run'> \blankpage </div>
%</tex>

\blankpage