%!name:real_line
%!need:real_order
%!need:integral_line

\ssection{Why}

We are constantly thinking of the real numbers as the points of a line.\footnote{Future editions will modify this sheet.}

\ssection{Discussion}

We commonly associate elements of the real numbers (see \sheetref{real_numbers}{Real Numbers}) with points on a line (see \sheetref{geometry}{Geometry}).

\begin{principle}[Point Sets]
  Given a line, there exists a set of its (infinite) points.
\end{principle}

\begin{principle}[Real Line Correspondence]
  Let $P$ be the set of points for a line.
  There exists a one-to-one correspondence mapping elements of $P$ onto elements of $\R$.
\end{principle}
For this reason, we sometimes call elements of the real numbers \t{points}.
We call the point associated with 0 the \t{origin}.

\ssection{Visualization}

To visualize the correspondence we draw a line.
We then associate a point of the line with the $0 \in \R$.
We can label it so.
We then pick a unit length.
We associate the points a unit length away from zero with $1 \in \R$ (on the right) and $-1 \in \R$ (on the left).
We do the same for two and $2$ and $-2$, $3$ and $-3$, and then we say that we could continue the process indefinitely.
We can visualize the image in Figure~\ref{real_line:figure:real_line}.

% \par\noindent\rule{0.9\textwidth}{0.4pt}

\begin{figure}[h]
\centering
\vspace{0.5cm}
  \includegraphics[width=0.9\textwidth]{\figpath{real_line}{real_line}}
\caption{The real line}
\label{real_line:figure:real_line}
\end{figure}


\blankpage
