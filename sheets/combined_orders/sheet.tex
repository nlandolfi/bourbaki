
%!name:combined_orders
%!need:orders

\section*{Why}

Given multiple orders, can we combine them?

\section*{Discussion}

Suppose we have two orders $\prec_1$ and $\prec_2$ on $A$.
Define $\prec$ by $a \prec b$ if and only if $a \prec_1 b$ and $b \prec_2 a$.
Notice that $\prec$ is reflexive, transitive, and antisymmetric, and so it is a partial order.
Call it the \t{combined order}.

Here's the rub.
Even if $\prec_1$ and $\prec_2$ are total, $\prec$ may not be total.\footnote{Future editions will include and expand.}
Consider the basic case $A = \set{a, b}$ and $\prec_1 \; = \set{(a,a), (a, b), (b,b)}$ and $\prec_2 \; = \set{(a,a), (b,a), (b,b)}$
Then $\prec \; = \set{(a,a), (b,b)}$, a partial order, to be sure, but not really any order at all.

There is not anything to be done about it, it is a fact.
Total orders do not (necessarily) induce total combined orders.

\blankpage