
\section*{Why}

Since every affine set is a translate of some (unique) subspace, it is natural to define the dimension of an affine set as the dimension of this subspace.

\section*{Definition}

The \t{dimension} of a nonempty affine set is the dimension of the subspace parallel to it.
By convention, $\varnothing$ has dimension $-1$.
Naturally, the \t{points}, \t{lines} and \t{planes} are affine sets of dimension 0, 1, and 2 respectively.

If an affine set has dimension $r$, then we often call it an \t{$r$-flat}.

For any $S \subset \R ^n$, we define the dimension of $A$ to be the dimension of the affine hull of $A$.

\subsection*{Notation}

We denote the dimension of the set $S \subset \R ^n$ by $\dim S$
We have defined it so that
\[
\dim S = \dim \aff S
\]
This makes sense if $S$ is affine, since in this case $\aff S = S$.

\section*{Result}

\begin{proposition}
Any $r$-flat has $r+1$ affinely independent points.
Each of its sets of size $r+2$ are affinely dependent.
\end{proposition}

\blankpage