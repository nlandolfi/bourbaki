%!name:prediction_evaluators
%!need:inductors
%!need:real_numbers

\ssection{Why}

We want to compare inductors using
their predictions.

\ssection{Definition}

A \ct{prediction evaluator}{}
is a real-valued function on
pairs of postcepts.
We call result under the evaluator
the \ct{error}{}.
Given a predictor and a precept,postcept
pair, the error fo the predictor on
that pair is the result of the prediciton
evaluator on the or
\ct{error of the predictor}{}.

An \ct{induction evaluator}{}
is a real-valued function
on inductors.
We call the result under the
evaluator of an inductor the
\ct{induction error}{} or
\ct{error of the inductor}.

We can use a prediction evaluator
and record sequences to construct
natural induction evaluators.
Fix a record sequence and the predictor
it induces.
Consider a second record sequence
and the sum the prediction errors
for the predictor
Consider


\ssubsection{Dual Record Prediction Evaluators}

Let $i$ be an inductor and let
$r$ and $s$ be two record sequences.
Denote the predictor $i(r)$ associated
with $r$ by $f$.
Let $g$ be a prediction evaluator.
Let $s = ((u^1, v^n), \dots, (u^n, v^n))$.
Then consider
\[
  \sum_{i = 1}^{n} g(f(u^n), v^n)).
\]



Consider an inductor, two record
sequences, and a prediction evaluator.
Consider
Consider the predictor
associated with first record sequences.
Consider the evaluator which
sums the prediction errors on the
second record sequence of the
the predictor induced on the first record
sequence.

The natural evaluator associated with
this inductor is the
We consider the pred
The first record sequencerecord sequence

Commonly evaluators have
structure.
We fix a record sequence
and consider the predictor
induced by it.
We consider a second
record sequence
and compare the predictor's
result for a precept with
the postcept paired with it
in the second record sequence.


A \ct{data-set evaluator}{} is
the evaluator for which
