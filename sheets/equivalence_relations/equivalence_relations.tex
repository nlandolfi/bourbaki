\sinput{../sheet.tex}
\sbasic

\sinput{../sets/macros.tex}

\sinput{../equivalence_relations/macros.tex}


\sstart

\stitle{Equivalence Relations}

\ssection{Why}

We want to handle at once
all elements
which are indistinguishable
or equivalent in some aspect.

\ssection{Definition}

A relation $R$ on a set $A$ is
an \ct{equivalence relation}{equivalencerelation}
if it is
\rt{reflexive}{reflexive},
\rt{symmetric}{symmetric},
and
\rt{transitive}{transitive}.

For an element $a \in A$,
we call the set of elements
in relation $R$ to $a$ the
\ct{equivalence class}{equivalenceclass}
of $a$.
The key observation,
recorded and proven below,
is that the equivalence classes
partition the set $A$.
A frequent technique is to define an appropriate equivalence
relation on a large set $A$ and then to work with the set of
equivalence classes of $A$.

We call the set of equivalence classes the \ct{quotient set}{quotientset}
of $A$ under $R$.
An equally good name is the divided set of $A$ under $R$, but this
terminology is not standard.
The language in both cases reminds us that $\sim$ partitions the set
$A$ into equivalence classes.

\ssubsection{Notation}
If $R$ is an equivalence relation on a set $A$,
we use the symbol $\sim$.
When alone, $\sim$ is read aloud as \say{sim,} but we still
read $a \sim b$ aloud as \say{a equivalent to b.}
We denote the quotient set of $A$ under $\sim$ by $A/\sim$,
read aloud as \say{A quotient sim}.

\ssubsection{Results}
\boxed{TODO}

\strats
