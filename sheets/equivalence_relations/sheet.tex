
\section*{Why}

We want to handle at once all the objects of a set which are indistinguishable or equivalent in some aspect.

\section*{Definition}

An \t{equivalence relation} on a set $X$ is a reflexive, symmetric, and transitive relation on $X$ (see \sheetref{relations}{Relations}).
The smallest equivalence relation in a set $X$ is the relation of equality in $X$.
The largest equivalence relation in a set $X$ is $X \times X$.

Equivalence relations are useful because they partition (see \sheetref{partitions}{Partitions}) the set.
If $R$ is an equivalence relation on $X$, the \t{equivalence class} of an object $x \in X$ is the set $\Set{y \in X}{x\;R\;y}$.
We call the set of equivalence classes the \t{quotient set} of the set under the relation (or the \t{quotient} of the set \t{by the relation}).
An equally good name is the divided set of the set under the relation, but this terminology is not standard.
The language in both cases reminds us that the relation partitions the set into equivalence classes.

If $\mathcal{C} $ is a partition of $X$, we can define a relation $R$ on $X$ for which $x\,R\,y \iff (\exists A \in \mathcal{C} )(x \in A \land y \in A)$.
In other words, if $x$ and $y$ are in the same piece (see \sheetref{partitions}{Partitions}) of $\mathcal{C} $.

The key result is that every equivalence relation partitions the set and every partition of the set is an equivalence relation.
Moreover, if we start with an equivalence relation, look for the partition, and then get the relation defined by the partition, we end up with the relation we started with.
Likewise, if we start with a partition relation, get the equivalence relation, and then get the partition defined by the relation, we end up with the partition we started with.
Before stating and proving this result, we give some notation.

\subsection*{Notation}

Let $R$ denote an equivalence relation on a set denoted by $X$.
We denote the equivalence class of $x \in X$ by $x / R$.
We denote the set of equivalence classes of $R$ by $X/R$, read aloud as ``$X$ modulo $R$'' or ``$x$ mod $R$''.
We denote the equivalence class of an element $x \in X$ by $\eqc{x}$.

\section*{Main Results}

The proofs of these results are straightforward.\footnote{Nonetheless, the full accounts will appear in future editions.}

\begin{proposition}
$X/\mathcal{C} $ is an equivalence relation.
\end{proposition}

\begin{proposition}
$X/R$ is a a partition.
\end{proposition}

\begin{proposition}
If $R$ is an equivalence relation on $X$, then $X/(X/R) = R$
\end{proposition}

\begin{proposition}
If $\mathcal{C} $ is a partition of $X$, then $X/(X/\mathcal{C} ) = \mathcal{C} $.
\end{proposition}

These last two propositions make clear the rationale for the notation.
The function mapping an element to its equivalence class is onto and is sometimes called the \t{projection}.
%  Of course, we have not yet had occasion to speak of numbers, much less of division.
%  

