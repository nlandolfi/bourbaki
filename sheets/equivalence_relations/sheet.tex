%!name:equivalence_relations
%!need:relations
%!need:partitions
%!refs:paul_halmos/naive_set_theory/section_07

\ssection{Why}

We want to handle at once all the objects of a set which are indistinguishable or equivalent in some aspect.

\ssection{Definition}

An \t{equivalence relation} on a set $X$ is a reflexive, symmetric, and transitive relation on $X$ (see \sheetref{relations}{Relations}).
The smallest equivalence relation in a set $X$ is the relation of equality in $X$.
The largest equivalence relation in a set $X$ is $X \times X$.

Equivalence relations are useful because they partition (see \sheetref{partitions}{Partitions}) the set.
If $R$ is an equivalence relation on $X$, the \t{equivalence class} of an object $x \in X$ is the set $\Set{y \in X}{x\;R\;y}$.
We call the set of equivalence classes the \ct{quotient set}{quotientset} of the set under the relation.
An equally good name is the divided set of the set under the relation, but this terminology is not standard.
The language in both cases reminds us that the relation partitions the set into equivalence classes.

If $\CC$ is a partition of $X$, we can define a relation $R$ on $X$ for which $x\,R\,y \iff (\exists A \in \CC)(x \in A \land y \in A)$.
In other words, if $x$ and $y$ are in the same piece (see \sheetref{partitions}{Partitions}) of $\CC$.

The key result is that every equivalence relation partitions the set and every partition of the set is an equivalence relation.
Moreover, if we start with an equivalence relation, look for the partition, and then get the relation defined by the partition, we end up with the relation we started with.
Likewise, if we start with a partition relation, get the equivalence relation, and then get the partition defined by the relation, we end up with the partition we started with.
Before stating and proving this result, we give some notation.

\ssubsection{Notation}

Let $R$ denote an equivalence relation on a set denoted by $X$.
We denote the equivalence class of $x \in X$ by $x / R$.
We denote the set of equivalence classes of $R$ by $X/R$, read aloud as \say{$X$ modulo $R$} or \say{$x$ mod $R$}.
We denote the equivalence class of an element $x \in X$ by $\eqc{x}$.

\ssection{Main Results}

The proofs of these results are straightforward.\footnote{Nonetheless, the full accounts will appear in future editions.}

\begin{proposition}
  $X/\CC$ is an equivalence relation
\end{proposition}
\begin{proposition}
  $X/R$ is a a partition.
\end{proposition}

\begin{proposition}
  If $R$ is an equivalence relation on $X$, then $X/(X/R) = R$
\end{proposition}

\begin{proposition}
  If $\CC$ is a partition of $X$, then $X/(X/\CC) = \CC$.
\end{proposition}

These last two propositions make clear the rationale for the notation.
Of course, we have not yet had occasion to speak of numbers, much less of division.

% old equivalence classes
% \ssubsection{Notation}

% We often use the symbol $\sim$ for equivalence relations.
% We read $\sim$ is read aloud as \say{sim,} but we still
% read $a \sim b$ aloud as \say{a equivalent to b.}

% \ssection{Why}
%
% Equivalence relations
% partition the base
% set into sets of \say{equivalent}
% elements.
% We can define an appropriate
% equivalence relation on
% a set and then work with
% the \textit{set of sets}
% of equivalent objects.
%
% \ssection{Definition}
%
% We call all elements related to a
% particular element under
% an equivalence relation the
% \ct{equivalence class}{equivalenceclass}
% of the element.
% The key observation,
% recorded and proven below,
% is that the equivalence classes
% partition the base set.
% This will allow us to define
% appropriate equivalence relations
% on a set and then work with the
% set of equivalence classes insteaded.
%
%
% \ssubsection{Notation}
% Let $A$ be a non-empty set
% and $\sim$ be an equivalence
% relation on $A$.
% We denote the quotient set of $A$ under $\sim$ by $A/\sim$,
% read aloud as \say{A quotient sim}.
%
% \ssubsection{Results}
%
% TODO
%
