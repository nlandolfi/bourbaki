
%!name:n-dimensional_space
%!need:distance
%!wiki:https://en.wikipedia.org/wiki/Real_coordinate_space

\section*{Why}

If $\R $ corresponds to a line, and $\R ^2$ to a plane, and $\R ^3$ to space, does $\R ^4$ correspond to anything? What of $\R ^5$?

\section*{Definition}

Let $n$ be a natural number.
We call the set $\R ^n$ \t{$n$-dimensional space} (or \t{Euclidean $n$-space}, or \t{real coordinate space}).
We call elements of $\R ^n$ \t{points}.
We identify $\R ^1$ with $\R $ in the obvious way.

We call the point associated with $x = (x_1, x_2, \dots , x_n) \in \R ^n$ with $x_i = 0$ for $1 \leqq i \leqq n$ the \t{origin}.
We denote the origin by $0$.
Similarly, we denote the point $x$ with $x_i = 1$ for all $i = 1, \dots , n$ by $1$.

\section*{Visualization}

We can not visualize $n$-dimensional space.
Thus, our intuition for it comes from real space (see \sheetref{real_space}{Real Space}).

\section*{Distance}

A natural notion of distance for $\R ^n$ generalizes that in $\R ^2$ and $\R ^3$.
We define the \t{distance} (\t{Euclidean distance}) between $(x_1, x_2, \dots , x_n)$, $(y_1, y_2, \dots , y_n) \in \R ^n$ as
    \[
\sqrt{(x_1 - y_1)^2 + (x_2 - y_2)^2 + \cdots + (x_n - y_n)^2}.
    \]
Does this have the properties that distance has in the plane and in space?
We discussed these properties
It does.
Denote the function which associates to $x, y \in \R ^n$ their distance $d: \R ^n \times  \R ^n \to \R $.
So $d(x, y)$ is the distance between the points corresponding to $x$ and $y$.

\begin{proposition}
$d$ is non-negative, symmetric, and the distance between two points is no larger than the sum of the distances with any third object.\footnote{Future editions will include an account.}\end{proposition}
\section*{Order}

Let $x, y \in \R ^n$.
If $x_i < y_i$ for all $i = 1, \dots , n$ then we say $x$ is \t{less than} $y$.
Likewise, if $x_i \leq y_i$ for all $i = 1, \dots , n$ then we say $x \leq y$.
Likewise for $>$ and $\geq$.

\subsection*{Notation}

If $x \in \R ^n$ is less than $y \in \R ^n$ then we write $x < y$.
Similarly for $x \leq y$, $x > y$ and $x \geq y$.
Other notation in the literature for $\R ^n$ includes $E^n$, which is a mnemonic for \say{euclidean.}
