
%!name:set_intersections
%!need:pair_intersections
%!need:empty_set
%!refs:paul_halmos∕naive_set_theory∕section_04

\section*{Why}

We can consider intersections of more than two sets.

\section*{Definition}

Let $\mathcal{A} $ denote a set of sets.
In other words, every element of $\mathcal{A} $ is a set.
And suppose that $\mathcal{A} $ has at least one set (i.e., $\mathcal{A}  \neq \varnothing$).
Let $C$ denote a set such that $C \in \mathcal{A} $.
Then consider the set,
\[
\Set{x \in C}{(\forall A)(A \in \mathcal{A}  \implies x \in A)}.
\]
This set exists by the principle of specification (see \href{sheets/set_specification.html}{Set Specification}).
Moreover, the set does not depend on which set we picked.
So the dependence on $C$ does not matter.
It is unique by the axiom of extension (see \sheetref{set_equation}{Set Equality}).
This set is called the \t{intersection} of $\mathcal{A} $.

\subsection*{Notation}

We denote the intersection of $\mathcal{A} $ by $\bigcap \mathcal{A} $.

\section*{Equivalence with pair intersections}

As desired, the the set denoted by $\mathcal{A} $ is a pair (see \sheetref{unordered_pairs}{Unordered Pairs}) of sets, the pair intersection (see \sheetref{pair_intersections}{Pair Intersections}) coincides with intersection as we have defined it in this sheet.\footnote{A full account of the proof will appear in future editions.}

\begin{proposition}
$\bigcap \set{A, B} = A \cap  B$
\end{proposition}

\blankpage