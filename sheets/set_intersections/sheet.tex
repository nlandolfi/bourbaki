%!name:set_intersections
%!need:set_specification

\ssection{Why}

Does a set exist containing the elements shared between two sets?
How might we construct such a set?


\ssection{Definition}

Let $A$ and $B$ denote sets.
Consider the set $\Set{x \in A}{x \in B}$.
This set exists by the principle of specification (see \sheetref{set_specification}{Set Specification}).
Moreover $(y \in \Set{x \in A}{x \in B}) \iff (y \in A \land y \in B)$.
In other words, $\Set{x \in A}{x \in B}$ contains all the elements of $A$ that are also elements of $B$.

We can also consider $\Set{x \in B}{x \in A}$, in which we have swapped the positions of $A$ and $B$.
Similarly, the set exists by the principle of specification (see \sheetref{set_specification}{Set Specification}) and again $y \in \Set{x \in B}{x \in A} \iff (y \in B \land y \in B)$.
Of course, $y \in A \land y \in B$ means the same as $y \in B \land y \in A$\footnote{Future editions will name and cite this rule.} and so by the principle of extension (see \sheetref{set_equality}{Set Equality})
\[
	\Set{x \in A}{x \in B} = \Set{x \in B}{x \in A}.
\]
We call this set the \t{pair intersection} of the set denoted by $A$ with the set denoted by $B$.

\ss{Notation}

We denote the intersection fo the set denoted by $A$ with the set denoted by $B$ by $A \intersect B$.
We read this notation aloud as \say{A intersect B}.

\ssection{A set of sets}


Using the principle of 
Yes.
The \t{intersection} of one set with another is the set obtained by specifying the elements of the former set which are members of the latter set.
The intersection is symmetric.
The intersection of one set with another is the same as the latter set with the former.

\blankpage