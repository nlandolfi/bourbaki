%!name:set_intersections
%!need:pair_intersections
%!refs:paul_halmos/naive_set_theory/section_04

\ssection{Why}

We can consider intersections of more than two sets.

% Surely we can do a similar with pair intersections for several sets.

\ssection{Definition}

Let $\CA$ denote a set of sets.
In other words, every element of $\CA$ is a set.
And suppose that $\CA$ has at least one set (i.e., $\CA \neq \emptyset$).
Let $C$ denote a set such that $C \in \CA$.
Then consider the set,
\[
	\Set{x \in C}{(\forall A)(A \in \CA \implies x \in A)}.
\]
This set exists by the principle of specification (see \sheetref{set_specification}{Set Specification}).
Moreover, the set does not depend on which set we picked.
So the dependence on $C$ does not matter.
It is unique by the axiom of extension (see \sheetref{set_equality}{Set Equality}).
This set is called the \t{intersection} of $\CA$.

\ss{Notation}

We denote the intersection of $\CA$ by $\bigcap \CA$.

\ss{Equivalence with pair intersections}

As desired, the the set denoted by $\CA$ is a pair (see \sheetref{unordered_pairs}{Unordered Pairs}) of sets, the pair intersection (see \sheetref{pair_intersections}{Pair Intersections}) coincides with intersection as we have defined it in this sheet.\footnote{A full account of the proof will appear in future editions.}

\begin{proposition}
  $\bigcap \set{A, B} = A \cap B$
\end{proposition}

\blankpage
