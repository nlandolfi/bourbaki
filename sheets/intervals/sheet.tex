%!name:intervals
%!need:real_line
%!refs:bert_mendelson/introduction_to_topology/theory_of_sets/sets_and_subsets

\ssection{Why}

We name and denote subsets of the set of real numbers which correspond to segments of a line.

\ssection{Definition}

Take two real numbers, with the
first less than the second.

An
\t{interval}
is one of four sets:
\begin{enumerate}
  \item
  the set of real numbers larger
  than the first number and smaller
  than the second; we call the
  interval \t{open}.

  \item
  the set of real numbers larger than
  or equal to the first number and
  smaller than or equal to the second
  number; we call the interval
  \t{closed}.

  \item
  the set of real numbers larger
  than the first number and smaller
  than or equal to the second;
  we call the interval
  \t{open on the left}
  and
  \t{closed on the right}

  \item
  the set of real numbers larger
  than or equal to the first number
  and smaller than the second;
  we call the interval
  \t{closed on the left}
  and
  \t{open on the right}.
\end{enumerate}
If an interval is neither open
nor closed we call it
\t{half-open}
or
\t{half-closed}

We call the two numbers
the
\t{endpoints}
of the interval.
An open interval does not
contain its endpoints.
A closed interval contains
its endpoints.
A half-open/half-closed
interval contains only one
of its endpoints.
We say that the
endpoints
\t{delimit}
the interval.

\ssubsection{Notation}

Let $a, b$ be two real numbers
which satisfy the relation $a < b$.

We denote the open interval from
$a$ to $b$ by $\oi{a, b}$.
This notation, although standard,
is the same as that for ordered pairs;
no confusion arises with adequate context.\footnote{In future editions, we may use $\oleft(a, b\oright)$ or even $\oleft[\,a, b\,\oright]$.}

We denote the closed interval from
$a$ to $b$ by $\ci{a, b}$.
We record the fact
$\oi{a, b} \subset \ci{a, b}$
in our new notation.

We denote the half-open interval
from $a$ to $b$,
closed on the right, by $\oci{a, b}$
and the half-open interval
from $a$ to $b$,
closed on the left, by $\coi{a, b}$.\footnote{Some authors use ]a, b], [a, b[ and ]a, b[.}

The \t{unit interval} is the set $\ci{0_{\R}, 1_{\R}}$ and we sometimes denote it by $\I$.
