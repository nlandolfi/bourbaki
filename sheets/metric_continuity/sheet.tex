
\section*{Why}

We define continuity for functions between metric spaces.

\section*{Definition}

Our inspiration is continuity of functions from the set of real numbers to the set of real numbers.
There we decided on a definition which codified our intuition that numbers which are sufficiently close to each other are mapped to numbers that are close to each other.

A function from a first metric space to a second metric space is \t{continuous at} an object of its domain if, for every positive real number (no matter how small), there is a second positive real number (possibly, though not necessarily, smaller) so that every element in the domain whose distance to the fixed object is less than the second positive number has a result under the function whose distance to the result of the fixed object is less than the first positive number.

A function between metric spaces is continuous if it is \t{continuous} at every object of its domain.

\subsection*{Notation}

Let $(A, d)$ and $(B, d')$ be metric spaces.
Let $f: (A, d) \to (B, d')$.
Then $f$ is continuous at $\bar{a} \in A$, if for all real numbers $\epsilon  > 0$, there exists a real number $\delta  > 0$ such that for all $a \in A$,
\[
d(\bar{a}, a) < \delta  \implies d'(f(\bar{a}), f(a)) < \epsilon .
\]

\blankpage