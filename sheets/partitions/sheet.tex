%!name:partitions
%!need:set_unions
%!need:set_intersections
%!refs:paul_halmos/naive_set_theory/section_07

\ssection{Why}
We divide a set into disjoint subsets whose union is the whole set.
In this way we can handle each subset of the main set individually, and so handle the entire set piece by piece.

\ssection{Definition}
A \t{partition} of a set $X$ is a set of pairwise disjoint (see \sheetref{set_decompositions}{Set Decompositions}) nonempty subsets of $X$ whose union is $X$.
We call the elements of a partition the \t{pieces} of the partition.

When speaking of a partition, we commonly call the set of sets \t{mutually exclusive} (by which we mean that they are pairwise disjoint) and \t{collectively exhaustive} (by which we mean that their union is full set).

\ssubsection{Notation}
Let $X$ be a set and $\CS$ be a set of subsets of $X$.
If $\CS$ is a partition of $X$, then
\begin{enumerate}
  \item $(\forall A \in \CS)(\forall B \in \CS)(A \neq B \implies A \cap B = \emptyset)$,
  \item $\bigcup \CS = X$.
\end{enumerate}

\blankpage
