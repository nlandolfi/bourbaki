
%!name:partitions
%!need:set_unions
%!need:set_intersections
%!refs:paul_halmos∕naive_set_theory∕section_07

\section*{Why}

We divide a set into disjoint subsets whose union is the whole set.
In this way we can handle each subset of the main set individually, and so handle the entire set piece by piece.

\section*{Decomposing a set}

Two sets $A$ and $B$ \t{divide} a set $X$ if $A \cup B = X$ and $A \cap  B = \varnothing$.
Although every element is in either $A$ or $B$, no element is in both.

If $\mathcal{A} $ is a set of sets, and $A, B \in \mathcal{A} $, then $\mathcal{A} $ is \t{pairwise disjoint} if $A \cap  B = \varnothing$ whenever $A \neq B$.

\section*{Definition}

A \t{partition} (or \t{decomposition}) of a set $X$ is a set of \textit{nonempty}, \textit{pairwise disjoint}, subsets of $X$ whose union is $X$.
We call the elements of a partition the \t{parts} (or \t{pieces}) of the partition.

When speaking of a partition, we commonly call the set of sets \t{mutually exclusive} (or \t{non-overlapping}), by which we mean that they are pairwise disjoint, and \t{collectively exhaustive}, by which we mean that their union is full set.\footnote{Future editions will include diagrams.}

\blankpage