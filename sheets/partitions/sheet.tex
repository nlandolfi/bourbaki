%!name:partitions
%!need:subsets
%!need:set_operations

\ssection{Why}

We divide a set into
disjoint subsets whose
union is the whole set.
In this way we can handle
each subset of the main set
individually, and so handle
the entire set piece by piece.

\ssection{Definition}

A set of sets is
\ct{disjoint}{}
if the intersection of any two
member sets is empty.
A \ct{partition}{} of
a set is a disjoint family of
subsets of the set
whose union is the set.
A
\ct{piece}{partitionpiece}
of a partition is an
element of the partition.

We say that the pieces
of the partition
are
\ct{mutually exclusive}{} (pairwise disjoint)
and
\ct{collectively exhaustive}{} (union is full set).

\ssubsection{Notation}

We avoid introducing new notation
for partitions.
Instead,
we here record the properties
of partitions in prior notation.

Let $A$ be a set.
Let $\CA$ be a set of
subsets of $A$
We denote the condition that
the $\CA$ is disjoint by
$B \intersect C = \emptyset$ for all
$B, C \in \CA$.
We denote the condition that the union
of the members of $\CA$ is $A$ by
$\union_{\alpha \in I} A_{\alpha} = A$.

\strats
