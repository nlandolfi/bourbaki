%!name:partitions
%!need:set_operations
%!refs:paul_halmos/naive_set_theory/section_07

\ssection{Why}

We divide a set into disjoint subsets whose union is the whole set.
In this way we can handle each subset of the main set individually, and so handle the entire set piece by piece.

\ssection{Definition}

A set of sets is \t{disjoint} if the intersection of any two
member sets is empty.
A \t{partition} of a set is a disjoint family of subsets of the set
whose union is the set.
A \t{piece} of a partition is an element of the partition.

We say that the pieces of the partition are \t{mutually exclusive} (they are pairwise disjoint) and \t{collectively exhaustive} (their union is full set).

\ssubsection{Notation}

Let $A$ be a set and $\CA$ be a set of subsets of $A$.
We denote the condition that $\CA$ is disjoint by
$B \intersect C = \emptyset$ for all
$B, C \in \CA$.
We denote the condition that the union
of the members of $\CA$ is $A$ by
$\union_{\alpha \in I} A_{\alpha} = A$.

\blankpage
