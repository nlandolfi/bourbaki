\sinput{../sheet.tex}
\sbasic

\sinput{../partitions/macros.tex}

\sstart

\stitle{Partitions}

\ssection{Why}

We divide a set into
disjoint subsets whose
union is the whole set.
In this way we can handle
each subset of the main set
individually, and so handle
the entire set piece by piece.

\ssection{Definition}

A \ct{disjoint family}{}
of sets is a family for which
the intersection of any two
member sets is empty.
A \ct{partition}{} of
a set is a disjoint family of
subsets of the set
whose union is the set.
A
\ct{piece}{partitionpiece}
of a partition is an
element of the family.

\ssubsection{Notation}

No new notation for partitions.
Instead,
we record the properties
of partitions in previously
introduced notation.

Let $A$ be a set and
$\set{A_{\alpha}}_{\alpha \in I}$
a family of subsets of $A$.
We denote the condition that
the family is disjoint by
$A_{\alpha} \intersect A_{\beta} = \emptyset$,
for all $\alpha,\beta \in I$,
We denote the condition that the family
union is $A$ by
$\union_{\alpha \in I} A_{\alpha} = A$.

\strats
