
\section*{Why}

We are constantly thinking of the integers as the endpoints of equal length segments of a line.

\section*{Discussion}

We commonly associate elements of the integers with the endpoints of equal-length segments of a real line.
Take segment $S_0$ of $L$ with endpoints $p$ and $q$.
Associate the point $p$ with $0$.
Associate the point $q$ with $1$.
Take a segment $S_1$ of equal length, non-overlapping with $S_0$, who shares the endpoint $q$.
Associate the second endpoint of this segment $2$.
Continue with the rest.
We call the line so formed the \t{integral line} of unit $S_0$.

\begin{center}\includegraphics[width=0.70\textwidth]{./graphics/integral_line.pdf}\end{center}
\section*{Integral Distance}

Let $f: \Z  \to \Z $ be defined by $f(a, b) = a - b$ if $a > b$ and $f(a, b) = b - a$ if $b > a$.
Notice that $f$ is symmetric: $f(a,b) = f(b, a)$.
The (geometric) interpretation of $f$ is the distance between the points associated with the two integers $a, b \in \Z $ in some integral line.
We call $f$ the \t{integral distance}.
Notice that $f(a, b) > 0$ for all $a, b \in \Z $.

\section*{Notation}

We denote the distance between $a, b \in \Z $ by $\abs{a - b}$.

\blankpage
%macros.tex
%% absolute_value macros
%% \ifdefined\abs
%% \else
%%  \newcommand{\abs}[1]{\left|#1\right|}
%% \fi
%% the above was creating issues - NCL 2/24/2022
%% following (roughly) the example of set_numbers/macros
%% \newcommand{\intabs}[1]{\left|#1\right|}
%\DeclarePairedDelimiter{\abs}{\lvert}{\rvert}
