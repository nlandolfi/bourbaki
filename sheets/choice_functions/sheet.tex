
%!name:choice_functions
%!need:direct_products
%!refs:bert_mendelson/introduction_to_topology/theory_of_sets/arbitrary_products

\section*{Why}

We want to choose a distinct sequence of elements from an infinite set.\footnote{Future editions will likely modify this why.}

\section*{What}

\begin{principle}[Choice]
The \sheetref{direct_products}{product}of a nonempty family of nonempty sets is nonempty.\footnote{Future editions will better motivate the axiom, and explain how it is not needed for finite sets or for sets with distinguishing features, but rather for infinitely many sets for which there is no selection criterion.}
\end{principle}

This sometimes called the \t{axiom of choice}.
It is equivalent to saying that if for each $\alpha  \in I$ we can choose a point $x_{\alpha } \in X_{\alpha }$ then we may construct a function $x \in \prod_{\alpha  \in I}X_{\alpha }$ by setting $x(\alpha ) = x_{\alpha }$

\blankpage