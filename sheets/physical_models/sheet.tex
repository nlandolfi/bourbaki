%!name:physical_models
%!need:differential_equations
%!need:probability

\ssection{Why}

We want to talk and think about physical phenomena by analogy with mathematical objects.
In these sheets, the objects are sets.\footnote{At present, this sheet deviates from the analytical nature of the Bourbaki project. This may change in future editions. In particular, we may use physical models as \textit{motivation} for much of the mathematics. This sheet borrows from the notes of S. Lall.}

\ssection{Models}

We call the mathematical objects which we use to reason (by analogy) about the physical phenomenon the \t{model} of the phenomenon.
One often has a choice of model.

\ssection{Two broad areas}

There are roughly two broad areas to construction models of physical phenomena.
The first is \t{deterministic}.
One constructs differential equations using physical principles and experiments.
The second is \t{probabilistic}.
One specifies the probability of events using physical principles (e.g. the symmetry of noise) and experiments (e.g. the observed frequency of particular events).

\blankpage
