%!name:measure_derivatives
%!need:real_limits
%!need:measures

\section*{Defining result}

\begin{proposition}
Let $(X, \SA)$ be a measurable space.
Let $\mu $ and $\nu $ be finite measures with $\nu  \ll \mu $.

There exists $g: X \to [0, \infty)$ such that
  \[
\nu (A) = \int_A g \, d\mu
  \]
for all $A \in \SA$.
The function $g$ is $\mu $-almost everywhere unique.
\begin{proof}
Define
  \[
\SF = \Set*{
f: X \to [0, \infty)
}{
f \text{ measurable and }
\int_A f d\mu  \leq \nu (A)
}.
  \]
The function $f \equiv 0$ is in $\mathcal{F} $, since it is a measurable simple function whose integral over every measureable set is zero.

If $f_1$ and $f_2$ are in $\SF$, then $f_1 \vee f_2$ is in $\SF$.
To check, let $A \in \SA$, and define the sets $A_1 = \Set{x \in A}{f_1(x) > f_2(x)}$ and $A_2 = \Set{x \in A}{f_1(x) \leq f_2(x)}$.
$A_1$ and $A_2$ partition $A$, so
  \[
\begin{aligned}
\int_A f_1 \vee f_2
&= \int_{A_1} f_1 \vee f_2 + \int_{A_2} f_1 \vee f_2 \\
&= \int_{A_1} f_1 + \int_{A_2} f_2 \\
&\leq \nu (A_1) + \nu (A_2)
\end{aligned}
  \]
Since $A_1$ and $A_2$ partition $A$,
    \[
\nu (A_1) + \nu (A_2) =
\nu (A_1 \cup A_2) =
\nu (A).
    \]

Select a sequence of
functions $\seq{f}$ in $\SF$
so that
\[
\lim_{n} \int \seqt{f} =
\sup\Set*{\int f}{f \in \SF}.
\]
Toward ensuring the sequence
is increasing, define
$g_1 = f_1$, $g_2 = g_1 \vee f_2$,
and $g_n = g_{n-1} \vee f_n$
for $n \geq 3$.
Using the observation in the previous
paragraph, $g_n \in \SF$ for each $n$.

Let $g$ be the pointwise limit
of the $\seq{g}$.
The monotone convergence of integrals shows
  \[
\int_A g = \lim_n \int_A g_n.
  \]
for each $A \in \SA$.
Since $\int_A g_n \leq \nu (A)$,
so too is the limit and thus
so too is $\int_A g$.
Thus, $g \in \SF$.
By construction,
for $A = X$,
$\int g = \sup\Set*{\int f}{f \in \SF}$.
We have constructed an element of $\SF$
attaining the supremum.

We know that the integral of
$g$ on $A$ with respect to $\mu $
is bounded above by $\nu (A)$.
We want the gap to be zero.
Regardless of the gap, the function $\nu _0: \SA \to [0, \infty)$ defined by
  \[
\nu _0(A) = \nu (A) - \int(g, A, \mu ),
  \]
for each $A \in \SA$ is a positive measure.
If $\nu _0$ is identically zero, then there is no gap.

Suppose there is a gap: then there exists a measurable set with strictly positive measure under $\nu _0$.
Since the base set contains this set, and measures are monotone, the base set must have stricty positive measure.
Since $\mu $ is finite, there exists a natural number $n$ so that
  \[
\nu _0(X) > \frac{1}{n}\mu (X).
  \]

Define a new measure $\nu _1 = \nu _0 - \frac{1}{n}\mu $.
Denote a signed-set decomposition of $\nu _1$
by $(P, N)$.
Then $\nu _1(A \cap P) \geq 0$, or equivalently,
  \[
\nu _0(A \cap P) - \frac{1}{n}\mu (A \cap P) \geq 0,
  \]
for all $A$, and so
  \[
\begin{aligned}
\nu (A) &= \nu _0(A) + \int(g, A, \mu ) \\
&\geq \nu _0(A \cap P) + \int (g, A, \mu ).
\end{aligned}
  \]
\end{proof}
\end{proposition}

Many authorities refer to this result as the \t{Radon-Nikodym theorem}.

\subsection*{Notation}
