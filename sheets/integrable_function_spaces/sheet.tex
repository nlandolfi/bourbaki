%!name:integrable_function_spaces
%!need:real_integrals
%!need:absolute_value
%!need:integrable_function_space
%!need:complex_numbers

\ssection{Why}

We have seen that the integrable functions form a vector space.
How about the square integrable functions?
And so on.\footnote{Future sheets are likely to being with $L^2$.}

\ssection{Definition}

The \t{integrable function spaces} are a collection of function spaces, one for each real number $p \geq 1$, for which the $p$th power of the absolute value of the function is integrable.\footnote{Future editions will include the case where $p = \infty$.}

\ssubsection{Notation}

Let $(X, \SA, \mu)$ be a measure space.
Let $p \geq 1$.
We denote the integrable function space corresponding to $p$ by $\SL^p(X, \SA, \mu, \R)$.
We have defined it by
\[
  \SL^p(X, \SA, \mu, \R) = \Set*{
    \text{ measurable } f: X \to R
  }{
    \int \abs{f}^p d\mu < \infty
  }
\]

Let $\C$ denote the set of complex numbers.
Similarly for complex-valued functions, we denote the $p$th space by $\SL^p(X, \SA, \mu, \C)$.

\blankpage
