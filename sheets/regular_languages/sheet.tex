
\section*{Definition}

Let $\Sigma $ be an alphabet.
A language $L \subset \str(\Sigma )$ is called \t{regular} if there exists a finite automaton that recognizes it.

\section*{Regular operations}

Let $A, B \subset \str(\Sigma )$ be languages in $\Sigma $.
  \subsection*{Union}

The \t{union} (\t{alternation}) of $A$ and $B$ is, as usual, the set $A \cup B$.
  \subsection*{Concatenation}

The \t{concatenation} of $A$ and $B$ is the set $\Set{xy}{x \in A \text{ and } y \in B}$, where $xy$ denotes length $\num{x}+\num{y}$ string which is the concatenation of $x$ and $y$
  \subsection*{Multi-concatenation}

The \t{star} (\t{Kleene star}, \t{multi-concatenation}) of $A$ is the set
\[
\Set{x \in \str(\Sigma )}{\exists k \geq 0, x = y_1y_2 \cdots y_k, y_i \in A}.
\]
By this definition we do mean to include the empty string $\varnothing$ in $A^*$, regardless of $A$.

\subsection*{Notation}

We denote the alternation of $A$ and $B$ by $A \cup B$ as usual, but other notations include $A + B$, $A\mid  B$, and $A \lor B$.
We denote the concatenation of $A$ and $B$ by $AB$, but other notations include $A \circ B$.
We denote the star of $A$ by $A^*$.

\blankpage