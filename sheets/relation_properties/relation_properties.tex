\sinput{../sheet.tex}
\sbasic

\sinput{../sets/macros.tex}
\sinput{../ordered_pairs/macros.tex}
\sinput{../relations/macros.tex}

\sinput{../relation_properties/macros.tex}

\sstart

\stitle{Relation Properties}

\ssection{Why}

Often relations
are defined over a single
set.
These have a few simple
properties.

\ssection{Definitions}

\begin{itemize}

\item
A relation is
\ct{reflexive}{reflexive}
if every element
is related to itself.

\item
A relation is
\ct{symmetric}{symmetric}
if two objects are related
regardless of their order.

\item
A relation is
\ct{antisymmetric}{antisymmetric}
if two objects are related only in
one order, and never both.

\item
A relation is
\ct{transitive}{transitive}
if a first element is
related to a second element
and the second element
is related to the third element,
then the first and third
element are related
\end{itemize}

\ssubsection{Notation}

Let $R$ be a relation on
a non-empty set $A$.
$R$ is \definition{reflexive}{reflexive} if
$$(a, a) \in R$$
for all $a \in A$.
$R$ is \definition{transitive}{transitive} if
$$(a, b) \in R \land (b, c) \in R \implies (a, c) \in R$$
for all $a, b, c \in A$.
$R$ is \definition{symmetric}{symmetric} if
$$(a, b) \in R \implies (b, a) \in R$$
for all $a, b \in A$.
$R$ is \definition{anti-symmetric}{anti-symmetric} if
$$(a, b) \in R \implies (b, a) \not\in R$$
for all $a, b \in A$.

\strats
