
%!name:tree_approximators_of_a_normal
%!need:optimal_tree_density_approximators
%!need:tree_normals
%!need:normal_conditionals

\section*{Why}

What is the optimal tree approximator of a multivariate normal density?

\section*{Result}

% TODO: remove language of Chow-Liu tree. 

% TODO: define precision matrix 

\begin{proposition}
Let $g: \R ^n \to \R $ be a normal density with mean $\mu \in \R ^d$ and covariance $\Sigma  \in \mathbf{S} ^d_{++}$.
The normal density $f^*_T: \R ^d \to \R $ with mean $\mu $ and precision matrix $P$ defined by
    \begin{itemize}
      \item $P_{11} = \Sigma _{11}^{-1} + \sum_{\pa{j} = 1} \Sigma _{j1}^2\Sigma _{11}^{-2}\Sigma _{j\mid 1}^{-1}$
      \item for $i = 2, \dots , d$, $P_{ii} = \Sigma _{i\mid\pa{i}}^{-1} + \sum_{\pa{j} = i} \Sigma _{ji}^2\Sigma _{ii}^{-2}\Sigma _{j\mid i}^{-1}$
      \item $i, j = 1, \dots  d$ and $i = \pa{j}$, $P_{ij} = P_{ji} = -\Sigma _{ji}\Sigma _{jj}^{-1}\Sigma _{j \mid i}^{-1}$
    \end{itemize}
where $\pa{i}$ is the parent of $i$ in an optimal approximator tree $T$ ($i = 2, \dots , n)$ is an optimal tree approximator of $g$.

\end{proposition}

\begin{proof}Using Proposition 1 of Best Tree Density Approximators, express an optimal tree approximator of $g$ by
\[
(1/c)
\exp\left(
-\frac{1}{2}
\left(
\Sigma _{11}^{-1}\bar{x}_1^2 +
\sum_{i \neq 1}
\left(\bar{x}_i - \Sigma _{i,\pa{i}}\Sigma _{\pa{i},\pa{i}}^{-1}\bar{x}_{\pa{i}}\right)^2\Sigma _{i\mid\pa{i}}^{-1}
\right)
\right)
\]
where $\bar{x}_i = x_i - \mu _i$ and $c = \sqrt{(2\pi )^d\Sigma _{11} \prod_{i \neq 1} \Sigma _{i \mid \pa{i}}}$.

Second, express the quadratic in the exponential as
\[
\Sigma _{11}^{-1}\bar{x}_1^2 +
\sum_{i \neq 1}
\left[
\Sigma _{i\mid\pa{i}}^{-1}
\bar{x}_i^2
-
2
\Sigma _{i,\pa{i}}\Sigma _{\pa{i},\pa{i}}^{-1}\Sigma _{i\mid\pa{i}}^{-1}
\bar{x}_i\bar{x}_{\pa{i}}
+
\Sigma _{i,\pa{i}}^2\Sigma _{\pa{i},\pa{i}}^{-2}\Sigma _{i\mid\pa{i}}^{-1}
\bar{x}_{\pa{i}}^2
\right]
\]
With $P$ defined as earlier, we can express the above as $\bar{x}^{\tp} P \bar{x}$.


%TODO: proof for exponential of quadratic integrating to one 

Third, note that $c$ is $\sqrt{(2\pi )^d\det P^{-1}}$ since $f^*_T$ is a density and so integrates to one.
\end{proof}
Notice that $f^*_T$ is a tree normal density.

\section*{Empirical normal}

In particular, notice that we can approximate the empirical normal density of a dataset with a density that factors according to a tree.
