
\section*{Why}

We look at a particular subset of vertices and the edges involved between them.

\section*{Definition}

Suppose $\mathcal{G}  = (V, E)$ is an undirected graph.
An undirected graph $(V', E')$ is a \t{subgraph} of $\mathcal{G} $ if $V' \subset V$ and $E \subset E'$.
The \t{vertex-induced subgraph} of an undirected graph $(V, E)$ \t{induced by} a subset of vertices $W \subset V$ is the undirected graph with vertices $W$ and all edges between vertices in $W$.
The \t{edge-induced subgraph} of an undirected graph $(V, E)$ induced by vertices $F \subset E$ whose edge set is $F$ and whose vertices are those vertices adjacent to the edges of $F$.

\subsection*{Notation}

Let $W \subset V$ and define $F$ by
\[
F = \Set*{\set{v, w} \in E}{v, w \in W}.
\]
The subgraph induced by $W$ is the undirected graph $(W, F)$.

Some authors denote the subgraph induced by $W$ by $G(W)$ or $(W, E(W))$ or $G[W]$.
We avoid this notation, as it abuses $G$, which is no longer an ordered pair, but (in our standard function notation) now indicates a function on subsets of $V$ with a complicated codomain.
Other authors occasionally refer to the ``subgraph $W$'', instead of ``the subgraph $G(W)$''.
Again, we avoid this practice.

For $D \subset E$, the subgraph induced by $D$ is the undirected graph $(U, D)$ where
\[
U = \Set{v \in V}{\exists e \in D: x \in e}
\]
Simiarly, people write $G(D)$ or $(V(D), D)$.
We avoid this.

\subsection*{Connected components}

A set of vertices $W$ in $G$ is \t{connected} if there is a path between any two vertices $v, w \in W$.
A set of vertices $W$ in $G$ is \t{maximimally connected} if there is no other vertex $v \not \in W$ connected to a vertex in $W$.
A \t{connected component} of $G$ is the subgraph induced by a maximally connected set of vertices.

Since the vertex set of a graph can always be partitioned into sets maximally connected vertices, and a connected components is connect, we think of the connected components of $G$ as the connected ``pieces'' of $G$.

\subsection*{Cliques}

A set of vertices is \t{complete} if the subgraph induced is complete.
A set of vertices $W$ is \t{maximally complete} if the subgraph induced is complete and there is no vertex $v \not\in W$ which is connected to every vertex in $W$.
In other words, there is no other vertex which we can add to $W$ so that the induced subgraph is still complete.

We call a \t{maximally complete} set of vertes a \t{clique}.
Some authors define a clique in the way we have defined a complete set of vertices, without reference to the maximality.

%% FROM THE OLD graph_cliques
%% %%%!name:graph_cliques
%% %%%!need:graphs
%% \ssection{Why}
%%
%% TODO
%% %We speak of the maximal complete subgraphs of a graph.
%%
%% \ssection{Definition}
%%
%% A
%% \ctasdf{complete}{completegraph}
%% graph is one
%% for which an edge exists
%% between any two nodes.
%%
%% A
%% \ctasdf{subgraph}{subgraph}
%% of a given graph
%% is a graph whose
%% vertex set is a subset
%% of the given vertex set
%% and whose edge set is the
%% subset of given edges connecting
%% vertices in the vertex subset.
%% With reference to the underling
%% graph, then, a subgraph can be
%% specified completely by its
%% vertex set.
%%
%% A \ctasdf{clique}{clique}
%% of a given graph
%% is a complete subgraph
%% of that graph.
%% When speaking of the cliques
%% of a given graph, we identify
%% the cliques with their vertex
%% set.
%% The relation contained in
%% gives a partial order on
%% cliques.
%% A clique is
%% \ctasdf{maximal}{maximal}
%% if it maximal with
%% respect to this relation;
%% i.e., it is contained
%% in no other clique.
%% As a convention, we include
%% $\emptyset$ as a clique.
%%
%% \ssubsection{Notation}
%%
%% Let $(V, E)$ a graph.
%% We denote a clique
%% by $C \subset V$,
%% a mnemonic for clique.
%  
