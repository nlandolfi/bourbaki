
\section*{Why}

We want a density that is symmetric about some center value with some spread.

\section*{Definition}

% A normal density is one which can be written as a normalized 

Let $f: \R  \to \R $ be a density.
If there exists $\mu  \in \R $ and $\sigma  \in \R $ with $\sigma  > 0$ so that for each $x \in \R $
\[
f(x) = \normaldensity{x}{\mu }{\sigma }
\]
then $f$ is a \t{normal density}.
A normal density is often called a \t{Gaussian density}.\footnote{We do not use this term in accordance with the Bourbaki project's policy on historical names.}
We often drop the word density and use refer to these as \t{normals} or \t{Gaussians}, using these words as substantives.

We call the special case when $\mu  = 0$ and $\sigma  = 1$ the \t{standard normal density} or \t{standard gaussian density}.

\section*{Maximum}

The maximum of a normal density is $\mu $.

\blankpage
%macros.tex
%\newcommand{\normaldensity}[3]{
%  \frac{1}{\sqrt{2\pi}#3}\exp\left(-\frac{1}{2}\left(\frac{#1-#2}{#3}\right)^2\right)
%}
%\newcommand{\normal}[2]{
%  \mathcal{N}(#1, #2)
%}
