%!name:ordinary_row_reductions
%!need:linear_system_row_reductions

\ssection{Why}

When does the technique of row reductions prevail?

\ssection{Multivariable row reductions}

Let $S = (A \in \R^{m \times m}, b \in \R^{m})$ be a linear system with $A_{kk} \neq 0$.
The \t{$k$th row reduction} of $S$ is the linear system $(C, d)$ with $C_{st} = A_{st} - (A_{sk}/A_{kk})A_{kt}$ if $i < s \leq m$ and $C_{st} = A_{st}$ otherwise.

The idea, as in the example in \sheetref{linear_system_row_reductions}{Linear System Row Reductions}, is to eliminate variable $k$ from equations $k+1, \dots, m$.
We are taking the $k$th column of $A$ from
\[
  \barray{A_{1k} \\ \vdots \\ A_{kk} \\ A_{k+1,k} \\ \vdots \\ A_{mk}} \quad \text{to} \quad \barray{A_{1k} \\ \vdots \\ A_{kk} \\ 0 \\ \vdots \\ 0}.
\]

We interpret the $i$th row reduction as \t{subtracting equations} of the system or \t{reducing rows} of the array $A$.
If $a^i, c^i \in \R^{n}$ denote the $i$th rows of $A$ and $C$, $c^i = a^i - (A_{ik}/A_{kk})a^k$ for $k < i \leq m$,
In other words, we obtain the $i$th row of matrix $C$ by subtracting a multiple of the $k$th row of matrix $A$ from the $i$th row of matrix $A$, for $k < i \leq m$.
The following is an immediate consequence of real arithmetic.
\begin{proposition}
  Let $(A \in \R^{m \times n}, b \in \R^{n})$ be a linear system which row reduces to $(C, d)$.
  Then $x \in \R^{n}$ is a solution of $(A, b)$ if and only if it is a solution of $(C, d)$.  \label{proposition:ordinary_row_reductions:basic}
\end{proposition}

\ssection{Ordinary reductions}

We call the system $S$ \t{ordinarily reducible} if there exists a sequence of systems $S_1, \dots, S_{m-1}$ so that $S_1$ is the $11$-reduction of $S$ and $S_{i}$ is the $ii$-reduction of $S_{i-1}$ for $i = 1, \dots, n-1$.
In this case, we call $S_{n-1}$ the \t{final ordinary reduction} (or just \t{ordinary reduction}) of $S$.
The following is an immediate consequence of Proposition~\ref{proposition:ordinary_row_reductions:basic}.
\begin{proposition}
	Let $S'$ be the (final) ordinary reduction of $S$. Then $S$ and $S'$ have equivalent solution sets.
	\label{proposition:ordinary_row_reductions:main}
\end{proposition}
This process of constructing the ordinary reduction is called \t{Gauss elimination} or \t{Gaussian elimination}.
We call the $kk$th entry of system $S_{k-1}$ the \t{pivot}.
In an ordinarily reducible system, the pivots are nonzero.

The idea is that a system is ordinarily reducible if we can take row reductions in sequence an end up with a system that is easy to back-substitute and solve.
The difficulty is that this need not be the case.
For example, consider the following obvious difficulty.
The system $(A, b)$ in which
\[
	A = \barray{0 & 1\\1 & 2} \text{ and } \barray{1 \\ 2}
\]
is not ordinarily reducible, but clearly solvable.
