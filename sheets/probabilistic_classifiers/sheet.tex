%!name:probabilistic_classifiers
%!need:classifiers
%!need:probability_distributions

\ssection{Why}

We use the language of probability distributions to characterize predictions.\footnote{Future editions will improve this.}

\ssection{Definition}

A \t{probabilistic classifier} is a function from precepts to probability distributions on the set of classes.
Throughout this sheet, let $A$ be a set of precepts and $B$ a set of postcepts.

\ssubsection{Point classifier as probabilistic classifier}

Let $f: A \to B$ be a point classifier.
One can always obtain a probabilistic classifier from $f$ in the following way.
Define the probabilistic classifier at a precept $a \in A$ to be the distribution that takes $1$ on the predicted value of the point classifier at $a$ and $0$ otherwise.

\ssubsection{Probabilistic classifer from point classifier}
Now let $g: A \to (B \to [0, 1])$ be a probabilistic classifier.
One can always obtain a point classifier from $g$ in the following way.
Assign to a precept $a$ the value of the distribution $g(a)$ with the most probability mass.
If there are ties, order the (finite) set $B$ arbitrarily, and break ties accordingly.
We call this the \t{maximum likelihood classifier} corresponding to the probabilisstic classifier $g$.

\blankpage
