%!name:probabilistic_classifiers
%!need:classifiers
%!need:probability_distributions
%!need:list_classifiers

\ssection{Why}

Since our predictions are often uncertain, we can use the language of probability distributions to characterize them.
  \ifhmode\unskip\fi\footnote{
Future editions will improve this.
  }

\ssection{Definition}

Denote the set of probability distributions on a set $X$ by $\Delta(X)$.

A \t{probabilistic classifier} $G: A \to \Delta(B)$ is a function from inputs $A$ to probability distributions over the classes $B$.

Given an input $a$, the \t{prediction} of $G$ on $a$ is a probability distribution $\hat{p}_a = G(a)$ on $B$.

\ssubsection{Point classifier as probabilistic classifier}

Given a point classifier $f: A \to B$, we can define a probabilistic classifier $G: A \to \Delta(B)$ corresponding to $f$ by
  \[
\hat{p}_a(b) =
\begin{cases}
1 & \text{ if } f(a) = b \\
0 & \text{ otherwise.} \\
\end{cases}
  \]
where $\hat{p}_a = G(a).$

\ssubsection{Probabilistic classifier from point classifier}

On the other hand, given probabilistic classifier $G: A \to \Delta(B)$, we can define a point classifier $f: A \to B$ by
  \[
f(a) = \underset{b \in B}{\text{argmax}} \; \hat{p}_a(v)
  \]
We call $f$ the \t{maximum likelihood classifier} corresponding to $G$.
If there are ties, we can order the (finite) set $B$ arbitrarily, and break ties accordingly.

We can extend this idea, and define a list classifier by sorting the outputs by their probability, from largest to smallest.

\ssection{Judging probabilistic classifiers}

\blankpage
