%!name:complex_products
%!need:complex_sums

\ssection{Why}\footnote{Future editions will include.}

\ssection{Definition}

Let $z_1, z_2 \in \C$ with $z_1 = (x_1, y_1)$ and $z_2 = (x_2, y_2)$.
The \t{complex product} of $z_1$ and $z_2$ is the complex number $(x_1x_2 - y_1y_2, x_1y_2 + y_1x_2)$.

\ssubsection{Notation}

We denote the complex product of $z_1$ and $z_2$ by $z_1 \cdot z_2$ or $z_1z_2$.
The notation is justified because the complex product of two purely real complex numbers corresponds to the purely real complex number whose real part is the real product of the real parts of the first two numbers.

Recall that we denote $z_1 = x_1 + iy_1$ and $z_2 = x_2 + iy_2$.
This notation is a mnemonic for the definition of a complex product if we treat $i^2 = -1$.
\[
\begin{aligned}
	z_1z_2 &= (x_1 + iy_1)(x_2 + iy_2) \\
		   &= x_1x_2 + ix_1y_2 + iy_1x_2 + i^2 y_1y_2 \\
		   &= (x_1x_2 - y_1y_2) + i(x_1y_2 + y_1x_2).
\end{aligned}
\]

\ssection{Properties}

\begin{proposition}[Commutativity]
For all $z_1, z_2 \in \C$,
  we have
  $z_1z_2 = z_2z_2$.
\end{proposition}

\begin{proposition}[Associativity]
For all $z_1, z_2, z_3 \in \C$, we have
  and $z_1(z_2z_3) = (z_1z_2)z_3$.
\end{proposition}

\ssection{Complex multiplication}

We call the operation that associates a pair of complex numbers with their product \t{complex multiplication}.
The operation is symmetric (commutative).

\ssection{Multiplicative identity and inverse}

Notice that the complex number $(1, 0)$ is the multiplicative identity.
It is unique,\footnote{Future editions will include an account} and so we call it the \t{complex multiplicative identity.}


\blankpage
