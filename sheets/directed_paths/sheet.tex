
\section*{Definition}

Let $(V, E)$ be a directed graph.
A \t{directed path} between vertex $v$ and vertex $w \neq v$ is a finite sequence of distinct vertices, whose first coordinate is $v$ and whose last coordinate is $w$, and whose consecutive coordinates (as ordered pairs) are edges in the graph.
We say that a path between $v$ and $w$ is from $v$ to $w$.
The \t{length} of the path is one less than the number of vertices: namely, the number of edges.

Two vertices are \t{connected} in a graph if there exists at least one path between them.
A directed graph is \t{connected} if there is a path between every pair of vertices.
A graph is \t{acyclic} if none of its paths cycle.

\subsection*{Other terminology}

Some authors allow paths to contain repeated vertices, and call a path with distinct vertices a \t{simple path}.
Similarly, some authors allow a cycle to contain repeated vertices, and call a path with distinct vertices a \t{simple cycle} or \t{circuit}.
Some authors use the term \t{loop} instead of \t{cycle}.

\blankpage