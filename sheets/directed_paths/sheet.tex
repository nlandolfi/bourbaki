%!name:directed_paths
%!need:directed_graphs

\ssection{Why}\footnote{Future editions will include.}

\ssection{Definition}

Let $(V, E)$ be a directed graph.
A \t{directed path} between vertex $v$ and vertex $w \neq v$ is a finite sequence of distinct vertices, whose first coordinate is $v$ and whose last coordinate is $w$, and whose consecutive coordinates (as ordered pairs) are edges in the graph.
We say that a path between $v$ and $w$ is from $v$ to $w$.
The \t{length} of the path is one less than the number of vertices: namely, the number of edges.

Two vertices are \t{connected} in a graph if there exists at least one path between them.
A directed graphis \t{connected} if there is a path between every pair of vertices.
A graph is \t{acyclic} if none of its paths cycle.

\ssubsection{Other Terminology}

Some authors allow paths to contain repeated vertices, and call a path with distinct vertices a \t{simple path}.
Similarly, some authors allow a cycle to contain repeated vertices, and call a path with distinct vertices a \t{simeple cycle} or \t{circuit}.
Some authors use the term \t{loop} instead of \t{cycle}.

\ssection{Directed Acyclic Graph}

Directed and acyclic graphs (sometimes \t{DAGs}) have some useful properties.
Clearly, every subgraph induced on a directed acyclic graph is a directed acyclic graph.

\begin{proposition}
  Let $(V, E)$ be a directed acyclic graph. Then there exists a vertex $v \in V$ which is a source and a vertex $w \in V$ which is a sink.
  \begin{proof}
  There exists a directed path of maximum length. It must start at a source and end at a sink.\footnote{Future editions will expand.}
  \end{proof}
\end{proposition}

A \t{topological numbering}, \t{topological sort} or \t{topological ordering} of a directed graph $(V, E)$ is a numbering $\sigma: \set{1, \dots, \num{V}} \to V$ satisfying
\[
  (v, w) \in E \implies \inv{\sigma}(v) < \inv{\sigma}(w).\footnote{Future editions will further explain this concept.}
\]

\begin{proposition}
  There exists a topological sort for every acyclic graph.
  \begin{proof}
    Let $(V, F)$ be a directed acyclic graph.
    There exists a source vertex, $v_1$.
    Set $\sigma(1) = v_1$.
    Take the subgraph induced by $V - \set{v_1}$.
    It is directed acyclic, and so has a source vertex, $v_2$.
    Set $\sigma(2) = v_2$.
    Continue in this way.\footnote{Future editions will clarify and expand.}
  \end{proof}
\end{proposition}
