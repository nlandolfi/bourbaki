
%!name:empirical_distribution
%!need:probability_distributions
%!need:set_numbers

\section*{Why}

A natural distribution to associate with a dataset is to assign to each outcome a probability which reflects the number of times it appears in the dataset.

\section*{Definition}

The \t{empirical distribution} of a dataset is the function which associates to each outcome the proportion times it appears in the dataset.
Since the proporitions are nonnegative and sum to one, the function is a probability distribution.

\subsection*{Notation}

Let $A$ be a non-empty set and $(a^1, \dots , a^n)$ be a dataset in $A$.
The empirical distribution $q: A \to \R $ of $(a^1, \dots , a^n)$ is
\[
q(a) = \frac{1}{n} \num{\Set*{k \in \set{1, \dots , n}}{a^k = a}}.
\]
In other words, to give the probability of outcome $a \in A$, we count the number of times it appeared in the dataset of size $n$, and then divide by $n$.

\blankpage