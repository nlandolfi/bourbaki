%!name:empirical_distribution
%!need:probability_distributions
%!need:datasets

\ssection{Why}

A natural distribution to associate with a dataset is to assign to each outcome a probability which reflects the number of times it appears in the dataset.

\ssection{Definition}

The \t{empirical distribution} of a dataset is the distribution which associates to each outcome a probability which is the proportion of its appearance in the dataset.
The proporitions are nonnegative and sum to one, so the function so described is indeed a probability distribution.

\ssubsection{Notation}

Let $A$ be a non-empty set.
Let $(a^1, \dots, a^n)$ be a data set in $A$.
Let $q: A \to \R$ be defined by
$$
  q(a) = \frac{1}{n} \card{\Set*{k \in \set{1, \dots, n}}{a^k = a}}
$$
Then $q$ is the empirical distribution of $(a^1, \dots, a^n)$.
In other words, to give the probability of outcome $a$, we count the number of times it appeared in the dataset of size $n$, and then divide by $n$.
