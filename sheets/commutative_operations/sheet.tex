
\section*{Why}

We introduce language for the case in which an operation does not depend on the order in which it operates.

%%Does  depend on order?
%%What of two distinct operations?

\section*{Definition}

An operation \t{commutes} if the result of two elements is the same regardless of their order.

%%An operation \ct{associates}{associates}
%%if given any three elements in order it
%%doesn't matter whether we first operate
%%on the first two and then with the result
%%of the first two the third, or the second
%%two and with the result of the second two
%%the first.
%%A first operation over a set
%%\ct{distributes}{distributes}
%%over a second operation over the same set
%%if the result of applying the first
%%operation to an element and a result of
%%the second operation is the same as
%%applying the second operation to the results
%%of the first operation with the arguments
%%of the second operation.

\subsection*{Notation}

Let $A$ be a non-empty set and let $+: A \times  A \to A$ be an operation.
If $+$ commutes, then
\[
a + b = b + a
\]
for all $a, b \in A$.

%%If $+$ associates, then
%%$$
%%  (a + b) + c = (a + b) + c
%%$$
%%for all $a, b, c \in A$.
