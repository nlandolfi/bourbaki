
%!name:linear_system_row_reductions
%!need:linear_equation_solutions

\section*{Why}

We want to solve linear equations.
Our approach is to \say{eliminate} variables from equations in our system.
Once we reach an equation in one variable, we will back-substitute to solve.

\section*{Two-variable example}

Suppose we want to find $x_1, x_2 \in \R $ to satisfy
    \[
\begin{aligned}
3x_1 + 2x_2 &= 10, \text{ and} \\
6x_1 + 5x_2 &= 20. \\
\end{aligned}
    \]
We can list the coefficients in a two-dimensional array $A = (3, 2; 6, 5)$ and $b = (10,20)$.
We can eliminate $x_1$ from the second equation by subtracting twice the first equation from the second.
In doing so we obtain the system of equations
    \[
\begin{aligned}
3x_1 + 2x_2 &= 10 \text{ and } \\
x_2 &= 0.
\end{aligned}
    \]
The key insight is that this system has the \textit{same solution set}.
We call the process of moving between these two systems a \t{row reduction}.

\section*{Four-variable example}

What if instead we have four unknowns?
Suppose
    \[
A = \barray{2 & 1 & 1 & 0 \\ 4 & 3 & 3 & 1 \\ 8 & 7 & 9 & 5 \\ 6 & 7 & 9 & 8} \text{ and } b = \barray{ 1 \\ 2 \\ 3 \\ 4}.
    \]
We might first eliminate $x_1$ (the variable associated with the first column of coefficients) from the remaining three equations to obtain the linear system $S_1 = (A^1, b^1)$ in which
    \[
A^1 = \barray{
2 & 1 & 1 & 0 \\
0 & 1 & 1 & 1 \\
0 & 3 & 5 & 5 \\
0 & 4 & 6 & 8 \\
} \text{ and } b^1 = \barray{
1 \\
0 \\
-1 \\
1 \\
}
    \]
The trick is that, since $A_{22}' \neq 0$, we can take the same route to eliminate $x_2$, to obtain the system $S_2 = (A^2, b^2)$ in which
    \[
A_2 = \barray{
2 & 1 & 1 & 0 \\
0 & 1 & 1 & 1 \\
0 & 0 & 2 & 2 \\
0 & 0 & 2 & 4 \\
} \text{ and } b^2 = \barray{
1 \\
0 \\
-1 \\
1 \\
}
    \]
Likewise for $x_3$, we obtain $S_3 = (A^3, b^3)$ in which
      \[
A^3 = \barray{
2 & 1 & 1 & 0 \\
0 & 1 & 1 & 1 \\
0 & 0 & 2 & 2 \\
0 & 0 & 0 & 2 \\
} \text{ and } b^3 = \barray{
1 \\
0 \\
-1 \\
3 \\
}.
      \]
Here, as in the two-variable case, the key insight is that all these systems have the same solution set and the last one, $(A^3, b^3)$, is easy to solve.
We solve it by \t{back substitution}.
First, since $2x_4 = 3$, we find $x_4 = 3/2$.
Second, since $2x_3 + 2x_4 = -1$, we find $x_3 = -2$.
Similarly we find $x_2 = 1/2$ and $x_3 = 5/4$.

\section*{Definition}

Let $S = (A \in \R ^{m \times n}, b \in \R ^{n})$ be a linear system.
The lower \t{row reduction} of $S$ for index $i$ with $A_{ii} \neq 0$ (or the $i$-row reduction) is the linear system $\tilde{A}_{st} = A_{st} - (A_{sj}/A_{ij})A_{it}$ if $i < s \leq m$ and $A_{st}$ otherwise.
We say that the system $(A, b)$ is \t{ordinarily reducible}.

Let $a^k, \tilde{a}^k \in \R ^{n}$ denote the $k$th row of $A$ and $\tilde{A}$, respectively.
Then if $k \neq i$, $\tilde{a}^k = a^k - \alpha _k a^i$ where $\alpha _k = A_{kj}/A_{ij}$.
In other words, a row $k$ of the matrix $\tilde{A}$ is obtained by subtracting a multiple of the $i$th row of matrix $A$ from row $k$ of matrix $A$.
We are \say{reducing} the rows of $A$.

\begin{proposition}

\label{propostion:linear_system_reductions:solution_equivalence}Let $(A \in \R ^{m \times n}, b \in \R ^{n})$ be a linear system which row reduces to $(C, d)$.
Then $x \in \R ^{n}$ is a solution of $(A, b)$ if and only if it is a solution of $(C, d)$.\footnote{Future editions will include an account.}\end{proposition}
First we reduce by subtracting twice row 1 from row 2, four times row 1 from row 3, and three times row 1 from row 4.
    \[
S_1 = \parens{\barray{
2 & 1 & 1 & 0 \\
0 & 1 & 1 & 1 \\
0 & 3 & 5 & 5 \\
0 & 4 & 6 & 8 \\
}, \barray{
1 \\
0 \\
-1 \\
1 \\
}}.
    \]
We then subtract three times row 2 from row 3 and four times row 2 from row 4 to obtain
    \[
S_2 = \left(\barray{
2 & 1 & 1 & 0 \\
0 & 1 & 1 & 1 \\
0 & 0 & 2 & 2 \\
0 & 0 & 2 & 4 \\
}, \barray{
1 \\
0 \\
-1 \\
1 \\
}\right).
    \]
Finally, we subtract two times row 3 from row 4 to obtain $S_4$, which we can write as
    \[
\begin{aligned}
2x_1 + x_2 + x_3 &= 1,& \\
x_2 + x_3 + x_4 &= 0,& \\
2x_3 + 2x_4 &= -1,& \text{ and } \\
2x_4 &= 3.&
\end{aligned}
    \]
We can now back-substitute to find $x_4 = 3/2$, $x_3 = -2$, $x_2 = 1/2$ and $x_1 = 5/4$.
The above proposition says that this is the only solution of $S$, as well.
