%!name:objects

\ssection{Why}

We want to talk about things.

\ssection{Definition}

We use the word \t{object} with its usual sense in the English language.
If we can touch the object, we say that it is \t{tangible}.
% A \t{tangible} object is one that we can touch.
Otherwise, we say that the object is \t{intangible}.

\ssection{Examples}

We pick up a pebble for an example of a tangible object.
The pebble is the object.
We can hold it and we can touch it.
We can toss it back and forth.
And because we can touch it, the pebble is tangible.

We consider the color gray for an example of an intangible object.
The pebble may be gray.
Although we can touch the pebble, we can not touch the color.
At least in the usual English sense of the word \say{touch}.
The color gray is an object.
But since we can not touch it, the color gray is intangible.

The following sheets contain many more examples of such intangible objects.
In fact, they contain little else besides this.

\subsection{Names}

To discuss objects we give them \t{names}.
For example, \say{the pebble} or \say{the color gray}.

\say{Variables}

A single Latin letter regularly suffices.
To aid our memory, we tend to choose the letter mnemonically.

The use of letters to name objects is convenient, since they are short.
But we must take care when speaking of objects by their names that we know which object is referred to.


\ssubsection{Notation}

We use italics when writing the name.
We introduce a name by the word \say{let,} followed by the name in italics and then the word \say{be} followed by a description of the object the name refers to.
For example: let $a$ be an object.
Here the description is \say{an object}.

\blankpage
