%!name:objects

\ssection{Why}

We want to talk about things.

\ssection{Definition}

We use the word \t{object} with its usual sense in the English language.
If we can touch the object, we say that it is \t{tangible}.
% A \t{tangible} object is one that we can touch.
Otherwise, we say that the object is \t{intangible}.

\ssubsection{Examples}

We pick up a pebble for an example of a tangible object.
The pebble is the object.
We can hold and touch it.
And because we can touch it, the pebble is tangible.
We consider the color of the pebble as an example of an intangible object.
The color is an object too, even though we can not hold it or touch it.
And because we can not touch it, the color is intangible.
The following sheets discuss other intangible objects and little else besides.

\ssection{Assertions}

We use the word \t{assertion} in the usual sense of the English language.

\ssection{Names}

To discuss objects we give them \t{names}.
For example, \say{the pebble} or \say{the pebble's color}.
To make statements about objects we use verbs.
It will turn out that in the sequel we will only use 'is' and 'belongs.'
All things are stated to hold in the present.
For example, the pebble is gray.

\ssection{Formal Languages}

We often will say things which are true about objects with particular properties.
These words do not officially have any meaning yet for us.
We will write our statements in terms.
A term will be a placeholder of sorts.
We will say things like "let $A$ be a term".
We are interested in deducing belonging relationships a

So we will have some signs.
An \t{accent} is $'$.
A \t{letter} is an upper or lower case Latin letter with or without accent.
The lower case latin letters are

 \t{placeholder} for us will mean any upper or lower case latin letter with or without one or more $'$ marks.
We will print the placeholder in italics.
For example, $A$, $A'$, $A''$, $A'''$, $A''''$, $B$, $C$, $D$, $E$, $F$, $f$, $f'$ are each placeholders.

We will use a formal language to
Our formal language will consist of \t{terms} and \t{relations}.
A term is a letter,
To talk about o

To help our development, we use a formal language.
The language consists of a few symbols and has enough complexity to let us express English-language sentences like \say{every object in this set is in this other set} and \say{every object in this set has this property} and \say{there exists an object in this set with this property}.

\ssubsection{Terms}

\ssubsection{Symbols}

A letter

Test

This is a sentence.

\begin{construction}
  \normalfont
  Test

  \begin{tabular}{rl}
    \texttt{name} & $A$ \\
    \texttt{name} & $B$ \\
    \texttt{name} & $C$ \\
    \texttt{have} & $A \subset B$ \\
    \texttt{have} & $A \subset C$ \\
    \texttt{thus} & $ A \subset C$
  \end{tabular}
\end{construction}

% \ssubsection{Names}
%
%
% \say{Variables}
%
% A single Latin letter regularly suffices.
% To aid our memory, we tend to choose the letter mnemonically.
%
% The use of letters to name objects is convenient, since they are short.
% But we must take care when speaking of objects by their names that we know which object is referred to.
%
%
% \ssubsection{Notation}
%
% We use italics when writing the name.
% We introduce a name by the word \say{let,} followed by the name in italics and then the word \say{be} followed by a description of the object the name refers to.
% For example: let $a$ be an object.
% Here the description is \say{an object}.
%
\blankpage
