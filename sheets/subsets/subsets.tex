\sinput{../sheet.tex}
\sbasic

\sinput{../sets/macros.tex}
\sinput{../subsets/macros.tex}

\sstart

\stitle{Subsets}

\ssection{Why}

We want to speak
of sets which contain
all the elements of
other sets.

\ssection{Two Sets}

A
\ct{subset}{subset}
of a set $A$ is any set $B$ for
which each element of the set $B$
is an element of the set $A$.
In this case, we say that
$B$ is a subset of $A$.
Conversely, we say that
$A$ is a \ct{superset}{superset} of $B$.

Every set is a subset of itself.
So if the set $A$ is the set $B$,
then $A$ is a subset of $B$
and $B$ is a subset of $A$.
Conversely, if $A$ is a
subset of $B$ and $B$ is
a subset of $A$, then $A$ is $B$.
To argue that $A$ is $B$,
we argue that
membership in $A$ implies membership in $B$
and second, we argue that membership in $B$
implies membership in $A$.


The \ct{power set}{powerset}
of a set is the set of all subsets of that
set.
It includes the set itself and the empty set.
We call these two sets
\ct{improper subsets}{impropersubsets}
of the set.
We call all other sets
\ct{proper subsets}{proper subsets}.

%We distinguish the set
%containing one element from the
%element itself.
%For example, consider a set
%which contains one element: this
%element is the empty set.
%Then the empty set is an element
%of this set.
%The empty set is contained in this
%set.
%The empty set is not equal to this set.

\ssubsection{Notation}
Let $A$ and $B$ be sets.
We denote that $A$ is a subset of $B$ by $A \subset B$.
We read the notation $A \subset B$ aloud as \say{A subset B}.

If $A \subset B$ and $B \subset A$, then $A = B$.
The converse also holds.

We denote the power set of $A$ by $\powerset{A}$, read aloud as \say{two to the A.}
$A \in \powerset{A}$ and $\emptyset \in \powerset{A}$.
However, $A \subset \powerset{A}$ is false.

\ssubsection{Examples}

Let $a, b, c$ be distinct
objects. Let $A = \set{a, b ,c}$
and $B = \set{a, b}$. Then
$B \subset A$.
In other notation,
$B \in \powerset{A}$.
As always, $\emptyset \in \powerset{A}$
and $A \in \powerset{A}$ as well.
In this case, we can
list the elements (which are sets)
of the power set:
\[
  \powerset{A} = \set{
    \emptyset,
    \set{a},
    \set{b},
    \set{c},
    \set{a, b},
    \set{b, c},
    \set{a, c},
    \set{a, b, c}
  }.
\]

\strats
