%!name:subsets
%!need:set_equality

\ssection{Why}

We want to speak
of sets which contain
all the elements of
other sets.

\ssection{Two Sets}

If every element
of a first set is
an element of a
second set we say
that the first set
is a \ct{subset}{subset}
of the second set.
Conversely, we say that
second set is a
\ct{superset}{superset} of
the first set.

Every set is a subset of itself.
Similarly, every set is a
superset of itself.
Thus, if two sets are equal,
the first is a subset of the
second and the second is
a subset of the first.
Because of our definition
of set equality, the converse
is also true.
If a first set is a subset
of a second set and a second
set is a subset of a first,
then every element of each
is an element of the other
and the sets are the same.
This conclusion follows from
the axiom of extension.

The empty set is a subset of every set,
since it has no elements and so satisfies
our definition.
Consider a set.
We call the empty set and the set itself
\ct{improper subsets}{impropersubsets}
of the set.
All other subsets we call
\ct{proper subsets}{propersubsets}.


Finally, the \ct{power set}{powerset}
of a set is the set of all subsets of that
set.
It includes the set itself and the empty set.

\ssubsection{Notation}
Let $A$ and $B$ be sets.
We denote that $A$ is a subset of $B$ by $A \subset B$.
We read the notation $A \subset B$ aloud as \say{A subset B}.

We can state the axiom of extension
in this case by
\[
  A = B \Leftrightarrow (A \subset B) \land (B \subset A)
\]
The notation $A \subset B$
is a concise symbolism for
the sentence
\say{every element of $A$ is an element of
$B$.} Or for the alternative notation
$a \in A \implies a \in B$.


We denote the power set of $A$ by $\powerset{A}$, read aloud as \say{powerset of A.}
$A \in \powerset{A}$ and $\emptyset \in \powerset{A}$.
However, $A \subset \powerset{A}$ is false.

\ssubsection{Examples}

Let $a, b, c$ be distinct
objects. Let $A = \set{a, b ,c}$
and $B = \set{a, b}$. Then
$B \subset A$.
In other notation,
$B \in \powerset{A}$.
As always, $\emptyset \in \powerset{A}$
and $A \in \powerset{A}$ as well.
In this case, we can
list the elements (which are sets)
of the power set:
\[
  \powerset{A} = \set{
    \emptyset,
    \set{a},
    \set{b},
    \set{c},
    \set{a, b},
    \set{b, c},
    \set{a, c},
    \set{a, b, c}
  }.
\]
