%!name:comparisons
%!need:orders
%!need:converse_relations
%!refs:paul_halmos/naive_set_theory/section_14

\ssection{Why}

We want language and notation involving order.\footnote{In the present edition, this sheet can be thought of as an extended notation section for \sheetref{orders}{Orders}.}

\ssection{Comparisons}

A \t{comparison} is a statement (see \sheetref{statements}{Statements}) involving a partial (which may or may not be total) order.

\ssubsection{Notation}

Let $A$ be a set.
We tend to denote an arbitrary partial order on $A$ by $\preceq$.
So $(A, \preceq)$ is a partially ordered set.

As usual (see \sheetref{relations}{Relations}), we write $a \preceq b$ to mean $(a, b) \in A$.
Alternatively, we write $b \succeq a$ to mean $a \preceq b$.
In other words, $\succeq$ is the inverse relation (see \sheetref{converse_relations}{Converse Relations}) of $\preceq$.

\ssubsection{Predecessors and successors}

If $a \preceq b$ and $a \neq b$, we write $a \prec b$ and say that $a$ \t{precedes} $b$. 
In this case we call $a$ the \t{predecessor} of $b$.
Alternatively, under the same conditions, we write $b \succ a$ and we say that $b$ \t{succeeds} $a$.
In this case we call $b$ the \t{successor} of $a$.

\ssubsection{Induced partial orders} 

Of course, the object we have defined and denoted by $\prec$ is a relation on $A$.
It satisfies (i) for no elements $x$ and $y$ do $x \prec y$ and $y \prec x$ hold simultaneously and (ii) if $x \prec y$ and $y \prec z$, then $x \prec z$ (i.e., $\prec$ is transitive).
It is worthwhile to observe that if $S$ is a relation satisfying (i) and (ii), then the relation $R$ defined to mean $(a, b) \in S$ or $a = b$ is a partial order on $A$.

\ssubsection{Strict and weak relations}

This connection between $\preceq$ and $\prec$ can be generalized.
The \t{strict relation} corresponding to a relation $R$ on a set $A$ is the relation $S$ on $A$ defined by $(a, b) \in S$ if $(a, b) \in R$ and $a \neq b$.
The \t{weak relation} corresponding to a relation $S'$ on a set $A$ is the relation $R'$ defined by $(a, b) \in R'$ if $(a, b) \in S'$ or $a = b$.
For this reason, a relation is said to \t{partially order} a set if it is a partial order or if its corresponding weak relation is one.

%// is \t{less} than or \t{smaller} than $b$,


\blankpage
