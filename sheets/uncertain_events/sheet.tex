
%!name:uncertain_events
%!need:uncertain_outcomes
%!wiki:https://en.wikipedia.org/wiki/Event_(probability_theory)

\section*{Why}

We want to talk about several uncertain outcomes at once.

\section*{Definition}

An \t{event} (or \t{compound event}, \t{random event}) is a subset of outcomes.

\subsection*{Algebra of events}

For events $A, B \subset \Omega $, we interpret $A \cup B$ as the event that either $A$ \textit{or} $B$ occurs.
Similarly we interpret $A \cap  B$ as the event that \textit{both} $A$ \textit{and} $B$ occur.
We interpret $\Omega  - A$, the \textit{complement} of $A$ in $\Omega $, as the event that $A$ \textit{does not} occur.

\subsection*{Examples}

\textit{An even and odd number of pips}.
As usual, model a die roll with outcomes $\set{1, 2, 3, 4, 5, 6}$.
Model the event that the number of pips is odd with the set $\set{1, 3, 5}$.
Similarly, model the event that the number of pips is even with the set $\set{2, 4, 6}$.

\textit{Rolling doubles}.
Suppose we model rolling two dice with the outcome set $\set{1,\dots ,6}^2$.
The event of rolling ``doubles''---the two die show the same number of pips---can be modeled as the set $D$ defined by
    \[
D = \set{(1,1),(2,2),(3,3),(4,4),(5,5),(6,6)}.
    \]
We may model the event that the die turns up four as the set $\set{4,5 , 6}$.

\blankpage