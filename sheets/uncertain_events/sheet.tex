
\section*{Why}

Often when considering an uncertain outcome, we want to treat of several possibilities at once.

\section*{Definition}

An \t{event} (or \t{compound event}, \t{random event}) is a subset of outcomes.

\subsection*{Examples}

\textit{Even or odd number of pips.}
As \sheetref{uncertain_outcomes}{before}, we model rolling a die with the sample space $\set{1, 2, 3, 4, 5, 6}$.
In this case, we model the situation that the number of pips is odd with the event (i.e., set) $\set{1, 3, 5}$.
Similarly, we may model the situation that the number of pips is even with the set $\set{2, 4, 6}$.

\textit{Rolling doubles.}
Suppose we model rolling two dice with the sample space $\set{1,\dots ,6}^2$.
We may model the situation of rolling ``doubles''---the two die show the same number of pips---with the event $D$ defined by
\[
D = \set{(1,1),(2,2),(3,3),(4,4),(5,5),(6,6)}.
\]
We may model the situation that the die turns up four with the event $\set{4, 5, 6}$.

\section*{Operations with events}

It is desirable to clarify what we mean when we make the usual compound statements about an uncertain outcome.
For example, given that we have an event for modeling that the number of pips on the die is even, and one for modeling that the number of pips on the die is odd, what are the events corresponding to the number of pips being even \textit{or} odd...even \textit{and} odd.
In this case, if $O = \set{1,3,5}$ is the event of an odd number of pips and $E = \set{2,4,6}$ is the event of an even number of pips.
The event of an even \textit{or} odd number of pips is $\set{1,2,3,4,5,6}$.
The event of an even \textit{and} odd number of pips is $\varnothing$, since no number is both even and odd.
Notice that
\[
O \cup E = \set{1,2,3,4,5,6} \text{ and } O \cap  E = \varnothing
\]

In general, given events $A, B \subset \Omega $, we call the event $A \cup B$ the event that either $A$ \textit{or} $B$ occurs.
Similarly we call the event $A \cap  B$ as the event that \textit{both} $A$ \textit{and} $B$ occur.
We call $\Omega  - A$, the \textit{complement} of $A$ in $\Omega $, as the event that $A$ \textit{does not} occur.
Two event are \t{incompatibile} if $A \cap  B = \varnothing$.
We refer to $\Omega $ as the \t{sure event} and $\varnothing$ as the \t{impossible event}.
When $A \subseteq B$, we say that $A$ \textit{must} occur if $B$ occurs.

\blankpage