
\section*{Why}

That some sequences grow without bound leads us to add two elements to the set of real numbers.

\section*{Definition}

The set of \t{extended real numbers} is the union of the set of real numbers with a set containing two elements: one which we call \t{positive infinity} and the other we call \t{negative infinity}.
We call an element of the extended real numbers \t{finite} if it is not positive or negative infinity.
Otherwise, we call it \t{infinite} (the only two infinite extended real numbers are exactly the two elements positive and negative infinity).

\subsection*{Notation}

We denote positive infinity by $+\infty$ and the negative infinity by $-\infty$.
We denote the set of extended real numbers by $\Rbar$.
In other words, $\Rbar = \R  \cup \set{+\infty,-\infty}$.

\section*{Intervals}

We extend intervals in the obvious way.
We define
\[
[-\infty, \alpha ) := \Set{x \in \R }{x <\alpha } \cup \set{-\infty}
\]
\[
[-\infty, \alpha ] := \Set{x \in \R }{x \leq \alpha } \cup \set{-\infty}
\]
Similarly,
\[
(\alpha , +\infty] := \Set{x \in \R }{x > \alpha } \cup \set{+\infty}
\]
\[
[\alpha , +\infty] := \Set{x \in \R }{x \geq \alpha } \cup \set{+\infty}
\]
We to obtain useful notation if we make these open
\[
(-\infty, \alpha ) := \Set{x \in \R }{x <\alpha }
\]
\[
(-\infty, \alpha ] := \Set{x \in \R }{x \leq \alpha }
\]
\[
(\alpha , +\infty) := \Set{x \in \R }{x > \alpha }
\]
\[
(\alpha , +\infty) := \Set{x \in \R }{x \geq \alpha }
\]
Finally, we sometimes denote $\R $ by $(-\infty, \infty)$ and $\Rbar$ by $[-\infty, +\infty]$, referring to the \t{real line} and \t{extended real line} respectively.
We refer to any of the aforementioned sets as \t{extended real intervals}.
The following notations are sometime useful
\[
\Rbar_+ := [0, \infty] \quad \text{ and } \quad \Rbar_{++} := (0, \infty].
\]
  \section*{Extended order}

Naturally, we define $-\infty < a < +\infty$ for every $\alpha \in \R $.
This new extended order can be put to good use, mostly notationally.
For example, the intervals above can be rewritten by selecting $x$ from $\Rbar$.
As a particular example, we can write
\[
[-\infty, \alpha ] =\Set{x \in \Rbar}{-\infty \leq x \leq \alpha }
\]
and likewise for the other sets.

It is logically consistent to define the greatest lower bound of the empty set to be $+\infty$ and likwise the least upper bound of the empty set to be $-\infty$.
In symbols,
\[
\inf \varnothing := +\infty \quad \text{ and } \quad \sup \varnothing = -\infty
\]

  \section*{Extended arithmetic}

We extend addition to all but one ordered pair of elements of the new set.
    \begin{enumerate}
      \item The sum of any real number with a real number is defined as before.
      \item The sum of positive infinity with positive infinity is positive infinity.
The sum any real number with positive infinity is positive infinity.
In symbols,
\[
\alpha  + \infty = \infty + \alpha  = \infty \quad \text{for } -\infty < \alpha  \leq \infty
\]
      \item The sum any real number with negative infinity is negative infinity.
The sum of negative infinity with negative infinity is negative infinity.
In symbols,
\[
\alpha  - \infty = -\infty + \alpha  = -\infty \quad \text{for } -\infty \leq x < \infty
\]
    \end{enumerate}
We do not define the sum of positive infinity and negative infinity.
In other words, the combinations $\infty - \infty$ and $-\infty + \infty$ are undefined and are avoided.

Similarly, we extend multiplication
\[
\alpha \infty = \infty\alpha  = \infty, \quad \alpha (-\infty) = (-\infty)\alpha  = -\infty \quad \text{for } 0 < \alpha  \leq \infty,
\]
\[
\alpha \infty = \infty\alpha  = -\infty, \quad \alpha (-\infty) = (-\infty)\alpha  = \infty \quad \text{for } -\infty \leq < \alpha  < 0,
\]
\[
0\infty = \infty0 = 0 = 0(-\infty) = (-\infty)0, \quad -(-\infty) = \infty
\]

Under these rules, the familiar laws remain commutative, associative, and multiplication distributes over addition.
This is provided, of course, that we never form the sum $\infty - \infty$ or $-\infty + \infty$.
These can be verified by testing all possible finte combinations for the values in the laws.
Avoiding a forbidden sum requires some cautious attention, like dividing by zero, and in practice one or the other of the infinities tends to be excluded by hypothesis.

\blankpage
