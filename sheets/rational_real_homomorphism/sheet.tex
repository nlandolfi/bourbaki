
%!name:rational_real_homomorphism
%!need:real_arithmetic

\section*{Why}

Do the rational numbers correspond (in the sense \sheetref{homomorphisms}{Homomorphisms}) to elements of the reals.

\section*{Main result}

Indeed, roughly speaking the rationals correspond to elements of the reals which are bounded above by that rational.
Denote by $\tilde{\R }$ the set $\Set{q \in \R }{\exists s \in \Q , q = \Set{t \in \Q }{t < s}}$.

\begin{proposition}
The fields $(\tilde{\R }, +_{\R } \mid \tilde{\R }, \cdot _{\R } \mid \tilde{\R })$ and $(Q, +_{\Q }, \cdot _{Q})$ are homomorphic.\footnote{Indeed, more is true and will be included in future editions. There is an \textit{order perserving} field homomorphism.}\end{proposition}
\begin{proof}The function is $f: \Q  \to \R $ with $f(q) = \Set{(r \in \Q }{r < q}$\end{proof}
\blankpage