
% can prob drop topological considerations? 

\section*{Definition}

A \t{polyhedron} (or \t{real polyhedron}, or \t{convex polyhedron}) is a set $P \subset \R ^n$ for which there exists $A \in \R ^{m \times  n}$ and $b \in \R ^{m}$ satisfying
\[
P = \Set{x \in \R ^n}{Ax \leq b}.
\]
In other words, a polyhedron is an intersection of finitely many halfspaces.
If $A$ and $b$ are rational, then $P$ is called a \t{rational polyhedron}.

\begin{center}\includegraphics[width=0.50\textwidth]{./graphics/polyhedra_unbounded.pdf}\end{center}
A polyhedron $P$ is a \t{polytope} (a \t{real polytope}) if it is \textit{bounded}.
In other words, there exists $x_0 \in P$ and $M > 0$ such that
\[
P \subset B_M(x_0) = \Set{x }{\norm{x - x_0} < M}
\]
Here $B_M(x_0)$ denotes the open ball of radius $M$, as usual.

\begin{center}\includegraphics[width=0.50\textwidth]{./graphics/polytope.pdf}\end{center}
As usual, the dimension of a polyhedron $P$ is the dimension of the affine hull of $P$, which we denote by $\dim P$.
$P$ is called \t{full-dimensional} if $\dim P = n$.
An equivalent condition for $P$ to be full-dimensional is that there exist an interior point of $P$ (as a subset of $\R ^n$)

\begin{center}\includegraphics[width=0.50\textwidth]{./graphics/polytope_bounded.pdf}\end{center}
\subsection*{Terminology}

Caution: some authors have a more relaxed notion of polyhedra, which does not require the polyhedra be convex.

% Examples!! images!!see thom rothvoss undergrade discrete optimization notes 

\blankpage