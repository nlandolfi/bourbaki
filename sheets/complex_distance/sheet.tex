
\section*{Why}

The identification of $\C $ with a plane leads $\C $ to naturally inherit $\R ^2$'s notion of distance.

\section*{Definition}

The \t{absolute value} or \t{modulus} of $z = (x, y) \in \C $ is the distance of $z$ to the origin.
If $z \in \C $, then the modulus of $z$ is
\[
\sqrt{x^2 + y^2}.
\]

In other words, the modulus of $z$ is the distance (in $\R ^2$ of $z = (x,y)$ from the origin $(0,0)$.

\subsection*{Notation}

We denote the modulus of $z$ by $\Cmod{z}$.

\section*{Properties}

\begin{proposition}[Triangle Inequality]
For all $z, w \in \C $,
\[
\Cabs{z + w} \leq \Cabs{z} + \Cabs{w}.
\]
Also, for all $z \in \C $, we have $\Cabs{\Re (z)} \leq \Cabs{z}$ and $\Cabs{\Im (z)} \leq \Cabs{z}$, and for all $z, w \in \C $,\footnote{This follows from the triangle inequality. Future editions will include an account.}
\[
\Cabs{\Cabs{z} - \Cabs{w}} \leq \Cabs{z - w}.
\]
\end{proposition}

\blankpage