%!name:event_probabilities
%!need:probability_mass_functions
%!need:probability_events
%!need:set_operations

\ssection{Why}

If we have some outcomes and
a probability mass function,
we can construct a function which
assigns probabilities to events.

\ssection{Definition}

The \ct{probability}{}
of an event is the sum
of the probabilities of
the outcomes in the event.
The \ct{event probability function}{}
is the correspondence assigning
events to their probabilities.

\ssection{Notation}

Let $A$ be a set of outcomes
and $p$ a probability mass function
on $A$.
Let $B \subset A$ be an event.
Then the probability of $B$
is
\[
  \sum_{b \in B} p(b)
\]

We can denote event probability function
by $\PE: 2^A \to R$.
Notice that $\PE$ depends on the set of
outcomes $A$ and the probability mass function
$p$. Sometimes we will include this when
denoting $\PE$ and denote it by $\PE_{A, p}$.

\ssection{Properties}

\begin{prop}
Let $\PE$ be the event probability function
of a probability mass function $p$ on a set
of outcomes $A$. then
\begin{enumerate}
\item $\PE(B) \geq 0$ for all $B \subset A$
\item $\PE(A) = 1$
\item for $B, C \subset A$ with $B \intersect C = \emptyset$,
  $\PE(A \union B) = \PE(A) + \PE(B)$.
\end{enumerate}
\end{prop}


