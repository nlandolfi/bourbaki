%!name:event_probabilities
%!need:probability_distributions
%!need:set_operations

\ssection{Why}

If we have some outcomes and a distribution, we can construct a function which assigns probabilities to events.

\ssection{Definition}

The \t{probability of an event} is the sum of the probabilities of the outcomes in the event.
The \t{event probability function} is the correspondence assigning events to their probabilities.

\ssection{Notation}

Let $A$ be a set of outcomes and $p$ a distribution on $A$.
Let $B \subset A$ be an event.
Let $\PE: 2^A \to \R $ be the event probability function, which is defined by
  \[
\textstyle
\PE(B) = \sum_{b \in B} p(b).
  \]

The event probability function $\PE$ depends on the outcomes $A$ and the distribution $p$.
We sometimes indicate this dependence by writing $\PE_{A, p}$.

\ssection{Properties}

\begin{proposition}
Let $\PE$ be the event probability function of the distribution $p: A \to [0, 1]$.
  \ifhmode\unskip\fi\footnote{
Future editions will include an account.
  }
  \begin{enumerate}
  \item $\PE(B) \geq 0$ for all $B \subset A$
  \item $\PE(A) = 1$, and $\PE(\varnothing) = 0$
  \item $\PE(B \cup C) = \PE(B) + \PE(C) - \PE(B \cap C)$ for $B, C \subset A$.
In particular,
          \begin{enumerate}
    \item if $B \cap C = \varnothing$, then $\PE(B \cup C) = \PE(B) + \PE(C)$.
          \end{enumerate}
  \end{enumerate}
\end{proposition}
