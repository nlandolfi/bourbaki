%!name:event_probabilities
%!need:probability_distributions
%!need:set_operations
%%!need:partitions

\ssection{Why}

Since one and only one outcome occurs, given a distribution on outcomes, we define the probability of a set of outcomes as the sum of their probabilities.

\ssection{Definition}

Given a distribution $p: \Omega  \to \R $, the \t{probability of an event} $A \subset \Omega $ is $\sum_{a \in A} p(a)$, the sum of probabilities of its outcomes.

\ssubsection{Notation}
Define define $\mathbfsf{P} : \powerset{\Omega } \to \R $ by
  \[
\textstyle
\mathbfsf{P} (A) = \sum_{a \in A} p(a).
  \]
We call $\mathbfsf{P} $ the \t{event probability function} (or \t{probability measure}) of (or induced by) $p$.

\ssubsection{Example: die}
Define $p: \set{1, \dots, 6} \to \R $ by $p(\omega ) = \nicefrac{1}{6}$ for $\omega  = 1, \dots, 6$.
Define the event $E = \set{2, 4, 6}$.
Then
  \[
\textstyle
\mathbfsf{P} (E) = \sum_{\omega  \in E} p(\omega ) = p(2) + p(4) + p(6) = \nicefrac{1}{2}.
  \]

\ssection{Basic properties}
Notice that for all $A \subset \Omega $, (i) $\mathbfsf{P} (A) \geq 0$.
In particular, (ii) $\mathbfsf{P} (\Omega ) = 1$ (and $\mathbfsf{P} (\varnothing) = 0$).
For all $A, B \subset \Omega $, $\mathbfsf{P} (A \cup B) = \mathbfsf{P} (A) + \mathbfsf{P} (B) - \mathbfsf{P} (A \cap B)$.
In particular, if $A \cap B = \varnothing$, (iii) $\mathbfsf{P} (A \cup B) = \mathbfsf{P} (A) + \mathbfsf{P} (B)$.

Conversely, suppose $f: \powerset{\Omega } \to \R $ satifies (i), (ii), (iii).
These three conditions are sometimes called the \t{axioms of probability} (for finite sets).
Define $p: \Omega  \to \R $ by
  \[
p(\omega ) = f(\set{\omega }).
  \]
In case $f$ satisfies the axioms, $p$ is a probability distribution (nonnegative and sums to one).
For this reason we call $f$ satisfying (i)-(iii) an \t{event probability function} (or \t{probability measure}).
In the case that we think of a probability event function $\mathbfsf{P} $ as induced by a distribution $p$, we write $\mathbfsf{P} _p$.

We conclude that $p$ and $\mathbfsf{P} $ are two perspectives.
We can think of elementary events (outcomes) and define their probabilities individually in a way that they sum to one and are nonnegative.
Or we can think of the compound events, and define their probabilities in a way consistent with (i)-(iii).

\ssection{Probability by cases}
Let $\mathbfsf{P} $ be a probability event function.
Suppose $A_1, \dots, A_n$ partition $\Omega $.
Then for an $B \subset \Omega $,
  \[
\textstyle
\mathbfsf{P} (B) = \sum_{i = 1}^{n} \mathbfsf{P} (A_i \cap B).
  \]
Some authors call this the \t{law of total probability}.
