%!name:event_probabilities
%!need:probability_distributions
%!need:probability_events
%!need:set_operations

\ssection{Why}

If we have some outcomes and a probability mass function, we can construct a function which assigns probabilities to events.

\ssection{Definition}

The \t{probability of an event} is the sum of the probabilities of the outcomes in the event.
The \t{event probability function} is the correspondence assigning events to their probabilities.

\ssection{Notation}

Let $A$ be a set of outcomes and $p$ a probability mass function on $A$.
Let $B \subset A$ be an event.
Let $\PE: 2^A \to \R$ be the event probability function, which is defined by
\[
  \PE(B) = \sum_{b \in B} p(b).
\]

The event probability function $\PE$ depends on the set of outcomes $A$ and the probability mass function $p$.
Sometimes we use notation indicating this dependence and will denote the event probability function by $\PE_{A, p}$.

\ssection{Properties}

\begin{prop}
Let $\PE$ be the event probability function
of a probability mass function $p$ on a set
of outcomes $A$. then
\begin{enumerate}
\item $\PE(B) \geq 0$ for all $B \subset A$
\item $\PE(A) = 1$
\item
  $\PE(B \union C) = \PE(B) + \PE(C)$
  for $B, C \subset A$ and $B \intersect C = \emptyset$
\end{enumerate}
\end{prop}


