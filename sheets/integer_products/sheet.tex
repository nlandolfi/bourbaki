%!name:integer_products
%!need:integer_numbers
%!need:natural_products
\ssection{Why}

We want sums to follow those of natural numbers.
  \ifhmode\unskip\fi\footnote{
Future editions will modify this.
  }

\ssection{Definition}

Consider $\eqc{(a, b)}, \eqc{(b, c)} \in \Z $.
We define \t{integer product} of $\eqc{(a, b)}$ with $\eqc{(b, c)}$ as $\eqc{(ac + bd, ad + bc)}$.
  \ifhmode\unskip\fi\footnote{
One needs to show that this is well-defined. The account will appear in future editions.
  }

\ssubsection{Notation}

We denote the product of $\eqc{(a, b)}$ and $\eqc{(c, d)}$ by $\eqc{(a, b)} \cdot  \eqc{(b, c)}$
So if $x, y \in \Z $ then the sum of $x$ and $y$ is $x\cdot y$.
As with natural products, we often drop the $\cdot $ and write $xy$ for $x\cdot y$.

\blankpage
