\sinput{../sheet.tex}
\sbasic

\sinput{../sets/macros.tex}
\sinput{../set_operations/macros.tex}
\sinput{../sequences/macros.tex}
\sinput{../extended_real_numbers/macros.tex}
\sinput{../signed_measures/macros.tex}

\sstart

\stitle{Signed Measures}

\ssection{Why}

Can we view the set of
measures as a vector space?

Not quite:
the difference of two measures
may take negative values
on some set.
This functional
will be countably
additive, however, and
so behaves similar to a measure.

\ssection{Definition}

An extended-real-valued
function on a
sigma algebra is
\ct{countably additive}{countablyadditive}
if the result of the function applied to
the union of a disjoint countable family of
distinguished sets is the limit of the partial
sums of the results of the function applied
to each of the sets individually.
The limit of the partial sums must
exist irregardless of the summand order.

A
\ct{signed measure}{signedmeasure}
is an extended-real-valued
function on a
sigma algebra that is
(1) zero on the empty set and
(2) countably additive.
We call the result of the function
applied to a set in the sigma
algebra the
\ct{signed measure}{signedmeasure}
(or when no ambiguity arises, the
\ct{measure}{})
of the set.

When speaking of a measure,
which is non-negative,
in contrast to a signed measure,
we will call the former a
\ct{positive measure}{}.

\ssubsection{Notation}

Let
$(X, \mathcal{A})$
be a measurable space
and let
$\mu: \mathcal{A} \to \eri$.
Then $\mu$ is a signed measure if
\begin{enumerate}
  \item $\mu(\emptyset) = 0$ and
  \item
  $\mu(\union_{i} A_i) =
    \lim_{n \to \infty}
      \sum_{k = 1}^{n} \mu(A_k)$
  for all disjoint $\seq{A}$.
\end{enumerate}

\begin{prop}
A signed measure never takes
both positive infinity and
negative infinity.
\begin{proof}
Let $(X,\SA)$ be a measurable space.
Let $\mu: \SA \to \eri$ be a signed measure.
First, suppose $\mu(X)$ is finite,
Then by
Proposition~\ref{prop:finitesignedmeasures}
$\mu$ is finite for each $A \in \SA$.

Suppose $\mu(X) = \infty$.
Let $A \in \SA$.
As before,
$\mu(X) = \mu(A) + \mu(X - A)$.
Since $\mu(X) = +\infty$, then
both of $\mu(A)$ and $\mu(X-A)$
must be either finite or $+\infty$.
Argue similarly for $\mu(X) = -\infty$.
\end{proof}
\end{prop}

\strats
