
\section*{Why}

Can we view the set of measures as a vector space?

Not quite: the difference of two measures may take negative values on some set.
This functional will be countably additive, however, and so behaves similar to a measure.

\section*{Definition}

Suppose $\mathcal{F} $ is a sigma algebra.
Then $f: \mathcal{F}  \to \Rbar$ is
\t{countably additive}
if the result of the function applied to
the union of a disjoint countable family of
distinguished sets is the limit of the partial
sums of the results of the function applied
to each of the sets individually.
The limit of the partial sums must
exist irregardless of the summand order.

A
\t{signed measure}
is an extended-real-valued
function on a
sigma algebra that is
(1) zero on the empty set and
(2) countably additive.
We call the result of the function
applied to a set in the sigma
algebra the
\t{signed measure}
(or when no ambiguity arises, the
\t{measure})
of the set.

When speaking of a measure, which is non-negative, in contrast to a signed measure, we will call the former a \t{nonnegative measure} (or \t{positive measure}).

\subsection*{Basic properties}

\begin{proposition}
A signed measure never takes both positive infinity and negative infinity.
\end{proposition}

\begin{proof}Let $(X, \mathcal{A} )$ be a measurable space.
Let $\mu : \mathcal{A}  \to \eri$ be a signed measure.
First, suppose $\mu (X)$ is finite,
Then by
Proposition~\ref{prop:finitesignedmeasures}
$\mu $ is finite for each $A \in \mathcal{A} $.
Suppose $\mu (X) = \infty$.
Let $A \in \mathcal{A} $.
As before,
$\mu (X) = \mu (A) + \mu (X - A)$.
Since $\mu (X) = +\infty$, then
both of $\mu (A)$ and $\mu (X-A)$
must be either finite or $+\infty$.
Argue similarly for $\mu (X) = -\infty$.
\end{proof}
%macros.tex
%%%%% MACROS %%%%%%%%%%%%%%%%%%%%%%%%%%%%%%%%%%%%%%%%%%%%%%%
%%%%%%%%%%%%%%%%%%%%%%%%%%%%%%%%%%%%%%%%%%%%%%%%%%%%%%%%%%%%
