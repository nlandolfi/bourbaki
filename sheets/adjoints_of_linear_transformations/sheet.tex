
\section*{Definition}

Suppose $T \in \mathcal{L} (V, W)$.
In other words, $T$ is a linear map from a vector space $V$ to a vector space $W$ where $V$ and $W$ are over the same field of scalars.

An \t{adjoint} of $T$ is a function $S: W \to V$ satisfying
\[
\ip{Tv, w} = \ip{v, Sw} \quad \text{for every } v \in V \text{ and every } w \in W
\]
It is not hard to see that there always exists an adjoint, and that this adjoint is unique.
Thus, we speak of \t{the adjoint} of $T$.

\subsection*{Notation}

We denote \textit{the} adjoint of $T$ by $T^*$.
This notation is meant to remind of complex conjugation, for reasons which will become apparent shortly.

\subsection*{Examples}

\textit{Space to the plane.}
Define $T: \R ^3 \to \R ^2$ by
\[
T(x_1, x_2, x_3) = (x_2 + 3x_3, 2x_1)
\]
We claim that the adjoint of $T$ is $T^*: \R ^2 \to \R ^3$ defined by
\[
T^*(y_1, y_2) = (2y_2, y_1, 3y_1)
\]
% TODO: proove; it's on axler 204 


\subsection*{Properties}

\begin{proposition}[Adjoint is Linear]
Suppose $T \in \mathcal{L} (V, W)$.
The adjoint of $T$ is linear.
\end{proposition}

\begin{proposition}[Adjoint properties]
Suppose $V$ and $W$ are finite dimensional inner product spaces over a field $\F $, which is $\R $ or $\C $.
Suppose $S, T \in \mathcal{L} (V, W)$.
Then
  \begin{enumerate}
    \item $(S + T)^* = S^* + T^*$
    \item $(\lambda T)^* = \Cconj{\lambda }T^*$ for all $\lambda \in \F $
    \item $ (T^*)^* = T$
    \item $ I^* = I$
  \end{enumerate}
\end{proposition}

\begin{proposition}
Suppose $V$, $W$, $U$ are finite dimensional inner product spaces over $\R $ or $\C $.
For all $T \in \mathcal{L} (V, W)$ and $S \in \mathcal{L} (W, U)$,
\[
(ST)^* = T^*S^*
\]
\end{proposition}

\blankpage