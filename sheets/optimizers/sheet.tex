%!name:optimizers
%!need:total_orders

\ssection{Why}

If we have a correspondence between the elements of some set and a chain, then we are often interested in the objects which correspond to minimal or maximal elements of the chain.

\ssection{Definition}

Consider a non-empty set and a chain, with a function associating to each element of a set an element of the chain (a chain is a set with a total order).

A \t{minimizer} of the function is an element of its domain whose result under the function is a minimal element of the function's range.
In other words, the result of the function on that element is less than the result of the function on any other element in its domain.
Similarly, a \t{maximizer} of the function is an element of its domain whose result under the function is a maximal element of the function's range.
An optimizer of a function is a minimizer or maximizer, depending on the context.

\ssubsection{Notation}

Let $A$ be a non-empty set and $(C, \leq)$ a chain.
Let $f: A \to C$.
An element $a \in A$ is a minimizer of $f$ if $f(a) \leq f(b)$ for all $b \in A$.
Similarly, an element $a \in A$ is a maximizer of $f$ if $f(a) \geq f(b)$ for all $b \in A$.

We denote the set of minimizers by $f$ by $\argmin f$ and the set of maximizers of $f$ by $\argmax f$. In other notation,
$$
  \argmin f = \Set*{a \in A}{\forall b \in A, f(a) \leq f(b) }
$$
and
$$
  \argmin f = \Set*{a \in A}{\forall b \in A, f(a) \geq f(b) }.
$$
