%!name:inner_product_norms
%!need:norms
%!need:real_square_roots

\ssection{Why}

An inner product gives rise to a norm naturally.

\ssection{Definition}

\begin{prop}
  Let $(V, \F)$ be a vector space.
  Let $f: V \times V \to \F$ be an inner product such that $f(x, x) \in \R$.
Let $g: V \to \R$ such that
\[
  g(x) = \sqrt{f(x, x)}.
\]
Then $g$ is a norm.
\end{prop}

The \t{norm} of a vector in an inner product space is the square root of the inner product of the vector with itself.

\ssubsection{Notation}
