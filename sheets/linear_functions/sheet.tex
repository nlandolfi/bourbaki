%!name:linear_functions
%!need:real_summation
%!need:n-dimensional_space

\ssection{Why}
  \ifhmode\unskip\fi\footnote{
Future editions will include.
  }

\ssection{Definition}
A function $f: \R^n \to \R^m$ is \t{linear} if
  \begin{enumerate}
  \item $f(x + y) = f(x) + f(y)$ for all $x, y \in \R^n$ and
  \item $f(\alpha x) = \alpha f(x)$ for all $x \in \R^n$ and $\alpha \in \R$.
  \end{enumerate}
There are simple consequences to these conditions.
For example, $f(0) =0$.
For reasons which may be clarified in future editions, these conditions are sometimes described as \t{superposition}.

\blankpage
