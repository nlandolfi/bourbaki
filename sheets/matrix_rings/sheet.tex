
\section*{Why}

Matrices with elements in a ring form a ring.

\section*{Definition}

Suppose $(R, +, \cdot )$ is a ring.
Given $A, B \in R^{n \times  n}$, define the binary operation $\bar{+}: R^{n \times  n} \times  R^{n \times  n} \to R^{n \times  }$ by
\[
\left[A \; \bar{+} \; B\right]_{ij} = A_{ij} + B_{ij}
\]
and define the binary operation $\bar{\cdot }: R^{n \times  n} \times  R^{n \times  n} \to R^{n \times  n}$ by
\[
\left[A \; \bar{\cdot } \; B\right]_{ij} = \sum_{k = 1}^{n} A_{ik}B_{kj}
\]
Both of these definitions are similar to the case with real matrices.
With these operations so defined, the object $(R^{n \times  n}, \bar{+}, \bar{\cdot })$ is a ring.
In other words, the set of $n \times  n$ matrices whose elements are in some ring $R$ is itself a ring, with the usual operations of addition and multiplication of matrices.

The additive identity of the ring is the matrix $0 \in R^{n \times  n}$ for which $0_{ij} = 0 \in R$.
The multiplicative identity the matrix $I$ for which $I_{ii} = 1 \in R$ for $i = 1, \dots , n$ and $I_{ij} = 0 \in R$ for $i \neq j = 1, \dots , n$.
As seen with real-valued matrices, multiplication on $R^{n \times  n}$ need not be commutative even if $R$ is.

\begin{exercise}
Show that $R^{n \times  n}$ is not a division ring when $n > 1$.
\end{exercise}

\blankpage