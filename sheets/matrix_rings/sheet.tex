%!name:matrix_rings
%!need:real_matrix-matrix_products
%!need:matrices
%!need:rings

\ssection{Why}
Matrices with elements in a ring form a ring.

\ssection{Example}
Let $(R, +, \cdot )$ be a ring.
Define $C = A \bar{+} B$ by $C_{ij} = A_{ij} + B_{ij}$ and define $C = A \bar{\cdot } B$ by $C_{ij} = \sum_{k = 1}^{n} A_{ik}B_{kj}$, as with real matrices, for $A, B \in R^{n \times n}$.
Then $(R^{n \times  n}, \bar{+}, \bar{\cdot })$ is a ring.
In other words, the set of $n \times  n$ matrices with elements in $R$ is a ring, with the usual addition and multiplication of matrices.

The additive identity of the ring is the matrix $0 \in R^{n \times  n}$ for which $0_{ij} = 0 \in R$.
The multiplicative identity the matrix $I$ for which $I_{ii} = 1 \in R$ for $i = 1, \dots , n$ and $I_{ij} = 0 \in R$ for $i \neq j = 1, \dots , n$.
As seen with real-valued matrices, multiplication on $R^{n \times  n}$ need not be commutative even if $R$ is.

\begin{exercise}
Show that $R^{n \times  n}$ is not a division ring when $n > 1$.
\end{exercise}

\blankpage
