
\section*{Definition}

A set of vectors $\set{v_1, \dots , v_k}$ is a \t{basis} for a subspace $S \subset \R ^n$ if
\[
S = \span\set{v_1, \dots , v_k} \quad \text{ and } \quad \set{v_1, \dots , v_k} \text{ is independent. }
\]
This definition captures two competing properties.
The first is that the set is large, in the sense that any vector in $S$ can be represented as a linear combination of vectors in $\set{v_1, \dots , v_k}$.
Simultaneously, the set is small, in the sense that no vector in the set is a linear combination of the others.
In other words, there is no extra vector in the set.

Linear independence is equivalent to uniqueness of representation of the vectors representable as a linear combination of $v_1, \dots , v_k$.
In other words, $\set{v_1, \dots , v_k}$ is a basis for $S$ if each vector $x \in S$ can be uniquely expressed as
\[
x = \alpha _1 v_1 + \cdots + \alpha _k v_k.
\]

\blankpage