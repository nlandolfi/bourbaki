
\section*{Definition}

The \t{complex conjugate} (or \t{conjugate}) of a complex number $z$ is the complex number whose real part matches $z$ and whose imaginary part is the additive inverse of $z$.
The complex conjugate of a purely real number is the same purely real number.
In other words, the complex conjugate of a complex number with no imaginary part is the same complex number.

\subsection*{Notation}

We denote the complex conjugate of the complex number $z \in \C $ by $\Cconj{z}$.
Other common notation includes $\bar{z}$, read ``z bar''.
If there exists $a, b \in \R $ so that $z = (a, b)$, then $\Cconj{z} = (a, -b)$.

\subsection*{Geometric interpretation}

Taking the conjugate of a complex numbers corresponds to a reflection across the real axis in the plane.

\subsection*{Properties}

A complex number $z$ is real if and only if $z = \Cconj{z}$ and it is imaginar if and only if $z = -\Cconj{z}$.

\begin{proposition}
For $z \in \C $, we have
\[
\Re (z) = \frac{z + \Cconj{z}}{2} \quad \text{ and } \Im (z) = \frac{z - \Cconj{z}}{2i}.
\]
\end{proposition}

\blankpage