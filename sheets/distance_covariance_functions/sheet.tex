
\section*{Why}

It is common to consider random functions whose domain is time, space, or $n$-dimensional space.

\section*{Definition}

Let $(X, d)$ be a metric space. A \t{distance covariance function} $k: X \times X \to \R $ is a covariance function satisfying
\[
k(x, y) > k(x, y) \iff d(x, y) < d(x, y).
\]
In other words, the covariance decreases as the distance between the arguments decreases.

\section*{Example: squared exponential}

Let $k: X \times X \to \R $ be defined by
\[
k(x, y) = \exp(-d(x, y)).
\]
Then $k$ is a distance covariance function.
It is often called the \t{squared exponential covariance function}.

Let $\alpha , \sigma  \in \R $.
Define $k': X \times X \to \R $ by
\[
k'(x, y) = \alpha \exp(-d(x, y)/\sigma ^2)
\]
then $k'$ is still a covariance function.
In this context $\sigma $ is often referred to as the \t{characteristic length-scale} of the process.
The scalar $\alpha $ is sometimes called a ``prefactor'' that ``controls'' the ``overall variance.''

Suppose $(X, d) = (\R ^n, \norm{\cdot })$.
Then the squared exponential covariance function
\[
\alpha \exp(-\norm{x - y}/(2\sigma ^2))
\]
is sometimes called the \t{radial basis function} or \t{gaussian covariance function}.\footnote{For reasons that will be included in future editions.}
Also called an \t{exponentiated quadratic kernel}.
