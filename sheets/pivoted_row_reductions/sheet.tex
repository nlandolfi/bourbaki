
\section*{Why}

We want to modify ordinary row reduction to handle the case in which a pivot is zero by selecting another suitable pivot.

\section*{Example}

Let $A \in \R ^{5 \times 5}$.
If $A_{11} \neq 0$, we may subtract multiples of row $1$ from row $2, \dots , 5$ to eliminate variable $x_1$ from those equations.
If $A$ reduces to $C \in \R ^{5 \times 5}$ and $C_{22} \neq 0$, then step 2 moves from
\[
{\tiny
\barray{
\times & \cross & \cross & \cross & \cross \\
& C_{22} & \times & \times & \times \\
& \times & \times & \times & \times \\
& \times & \times & \times & \times \\
& \times & \times & \times & \times \\
} \text{ to }
\barray{
\times & \cross & \cross & \cross & \cross \\
& C_{22} & \times & \times & \times \\
& {\bf 0} & \boldsymbol{\times} & \boldsymbol{\times} & \boldsymbol{\times} \\
& {\bf 0} & \boldsymbol{\times} & \boldsymbol{\times} & \boldsymbol{\times} \\
& {\bf 0} & \boldsymbol{\times} & \boldsymbol{\times} & \boldsymbol{\times} \\
}.}
\]

What if $C_{22} = 0$?
In this case suppose we pick a different row.
For example, if $C_{42} \neq 0$ we can move from
\[
\tiny{
\barray{
\times & \cross & \cross & \cross & \cross \\
& \times & \times & \times & \times \\
& \times & \times & \times & \times \\
& C_{42} & \times & \times & \times \\
& \times & \times & \times & \times \\
} \text{ to }
\barray{
\times & \cross & \cross & \cross & \cross \\
& {\bf 0} & \boldsymbol{\times} & \boldsymbol{\times} & \boldsymbol{\times} \\
& {\bf 0} & \boldsymbol{\times} & \boldsymbol{\times} & \boldsymbol{\times} \\
& C_{42} & \times & \times & \times \\
& {\bf 0} & \boldsymbol{\times} & \boldsymbol{\times} & \boldsymbol{\times} \\
}.}
\]
Alternatively, we could introduce zeros in column 3 rather than column 2.
For example, if we pick the pivot $C_{43}$ we move from
\[
\tiny{
\barray{
\times & \cross & \cross & \cross & \cross \\
& \times & \times & \times & \times \\
& \times & \times & \times & \times \\
& \times & C_{43} & \times & \times \\
& \times & \times & \times & \times \\
} \text{ to }
\barray{
\times & \cross & \cross & \cross & \cross \\
& \boldsymbol{\times} & {\bf 0} & \boldsymbol{\times} & \boldsymbol{\times} \\
& \boldsymbol{\times} & {\bf 0} & \boldsymbol{\times} & \boldsymbol{\times} \\
& \times & C_{43} & \times & \times \\
& \boldsymbol{\times} & {\bf 0} & \boldsymbol{\times} & \boldsymbol{\times} \\
}.}
\]
We can choose any nonzero entry in $C_{k:m,k:m}$ as the pivot.
Suppose we pick pivot $C_{st} \neq 0$ for $k \leq s, t \leq m$.
Define $\tilde{C}$ by swapping row $s$ of $C$ with row $k$ of $C$ and column $t$ of $C$ with column $k$ of $C$.
Then $\tilde{C}_{kk} = C_{st} \neq 0$ and there exists an ordinary row reduction for $\tilde{C}$.
We call this reduction of $(\tilde{C}, \tilde{d})$ a \t{pivoted row reduction} of $C$ or the \t{$st$-reduction} of C.

If all remaining pivots are zero, then there is no viable pivot.
In this case, at least one variable is free and we do not have a unique solution.
For convenience, in this case, we still call the system an $st$-reduction of itself.


\section*{Definition}

At step $k$ of ordinary elimination, multiples of row $k$ are subtracted from rows $k+1, \dots , m$ to introduce zeros in entry $k$ of the rows.
If we denote the matrix at the beginning of that step by $X$, then row $k$ of $X$, column $k$ of $X$ and especially the pivot $X_{kk}$ play a role.
Ordinarily, we subtract from every entry in the submatrix $X_{k+1:m,k:m}$ the product of a number in row $k$ and a number in column $k$, divided by the pivot $X_{kk}$.
Generally, however, we can choose as pivot any nonzero entry of $X_{k:m,k:m}$.

An $m$-variable system $(A, b)$ is \t{pivot reducible} (or \t{reducible}) if there exists a sequence of systems $S_1, \dots , S_{m-1}$ so that $S_1$ is a reduction of $(A, b)$ and $S_{i}$ is a reduction of $S_{i-1}$ for $i = 1, \dots , m-1$.
We call $S_{m-1}$ the \t{final reduction} (or \t{reduction}) of $(A, b)$.
An immediate consequence of our definition is

\begin{proposition}
All systems are reducible.
\end{proposition}
