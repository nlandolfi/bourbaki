%!name:real_numbers
%!need:rational_numbers

\ssection{Why}

We want a set which corresponds to our notion of points on a line.\footnote{Future editions will modify and expand this justification.}

\ssection{Definition}

First, call a subset $R$ of $\Q$ a \t{rational cut} if $R \neq \emptyset$, $R \neq \Q$, for all $q \in R$, $r \leq q \implies r \in R$, and $R$ has no greatest element.
Briefly, the intuition is that the point is the set of all rationals to the left.\footnote{This brief intuition will be expanded upon in future sheets.}

The \t{set of real numbers} is the set of all rational cuts.
This set exists by an application of the principle of selection (see \sheetref{set_selection}{Sect Selection} to the power set (see \sheetref{set_powers}{Set Powers}) of $\Q$.
We call an element of the set of real numbers a \t{real number} or a \t{real}.
We call the set of real numbers the \t{set of reals} or \t{reals} for short.

\ssection{Notation}

We follow tradition and denote the set of real numbers by $\R$, likely a mnemonic for \say{real.}

\blankpage
