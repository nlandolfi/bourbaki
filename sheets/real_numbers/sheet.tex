
\section*{Why}

We want a set which corresponds to our notion of points on a line.\footnote{Future editions will modify and expand this justification.}

\section*{Rational cuts}

We call a subset $R$ of $\Q $ a \t{rational cut} if (a) $R \neq \varnothing$, (b) $R \neq \Q $, (c) for all $q \in R$, $r \leq q \implies r \in R$, and (d) $R$ has no greatest element.
Briefly, the intuition is that the point is the set of all rationals to less than (or, potentially, equal to) some particular rational number.\footnote{This brief intuition will be expanded upon in future sheets.}

\section*{Definition}

The \t{set of real numbers} is the set of all rational cuts.
This set exists by an application of the principle of selection (see \sheetref{set_selection}{Set Selection}) to the power set (see \sheetref{set_powers}{Set Powers}) of $\Q $.
We call an element of the set of real numbers a \t{real number} or a \t{real}.
We call the set of real numbers the \t{set of reals} or \t{reals} for short.
\section*{Notation}

We follow tradition and denote the set of real numbers by $\R $, likely a mnemonic for ``real.''

\subsection*{Other terminology}

Some authors call a real number a \t{quantity} or a \t{continuous quantity}.
The real numbers, then, are said to be \t{continuous}.
When contrasting (using this terminology) a finite set with the real numbers, one refers to the finite set as \t{discrete}.\footnote{Future editions may move this discussion later, to the discussion of the cardinality of the reals.}
