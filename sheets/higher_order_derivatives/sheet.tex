%!name:higher_order_derivatives
%!need:second_derivatives

\ssection{Why}

The second derivative (if it exists) is the derivative of the derivative of a function.
Can we continue in this way?

\ssection{Definition}

Let $A \subset \R$.
Let $f: A \to \R$ be twice differentiable.
We call $f$ \t{three times differentiable} (or \t{thrice differentiable}) if its second derivative is differentiable.
We call the derivative of the second derivative of $f$ the \t{third derivative} of $f$.

Generally, for $n \geq 3$, we call $f$ \t{$n+1$-times differentiable} if $f$ is $n$-times differentiable.
The \t{$n+1$th derivative} of a $n+1$-times differenetiable function is the derivative the $n$th derivative of the function.

\ssubsection{Notation}

The $n$th derivative of a function $f: A \to \R$ is sometimes denoted $f^{(n)}: A \to \R$.

\blankpage
