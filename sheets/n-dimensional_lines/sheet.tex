
\section*{Definition}

Given two distinct points $x$ and $y$ in $\R ^n$, the \t{line} through $x$ and $y$ is the set of points expressable as the sum of $x$ and $\alpha (y-x)$ where $\alpha  \in \R $.

In other words, the line through $x$ and $y$ is the set
\[
\Set{z \in \R ^n}{\exists  \alpha  \in \R , z = x + \alpha (y - x)} = \Set{x + \alpha (y-x)}{\alpha  \in \R }
\]
The second expression is notation for the first, and is called the \t{set-generator notation}.
Notice that if $z = x + \alpha (y-x)$, then
\[
z = (1 - \alpha )x + \alpha y,
\]
where $\alpha  \in \R $ and $x, y \in \R ^n$.
This highlights the obvious (and obviously desirable) property that the definition is symmetric in $x$ and $y$.

\subsection*{Visualization in the plane}

\begin{center}\includegraphics[width=0.100\textwidth]{./graphics/L(x-y).pdf}\end{center}
\subsection*{Notation}

We denote the line through $x$ and $y$ by $L(x,y)$.

\section*{Properties}

There are a few nice properties.
If $x \neq y$, then $L(x, y) = L(y, x)$.
Also, if $v$ and $w$ are in $L(x, y)$ and $v \neq w$, then the line through $v$ and $w$ is the same as the line through $x$ and $y$.
In symbols
\[
L(v, w) = L(x, y)
\]

\blankpage