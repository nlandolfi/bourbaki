%!name:real_norm
%!need:real_vectors
%!need:space_norm
%!need:absolute_value

\section*{Why}

We generalize our notion of \textit{size} to $n$-dimensional space.

\section*{Definition}

The \t{norm} (or \t{Euclidean norm}) of $x \in \R ^n$ is
  \[
\sqrt{x_1^2 + x_2^2 + \cdots + x_n^2}.
  \]

\subsection*{Notation}

We denote the norm of $x$ by $\norm{x}$.
In other words, $\norm{\cdot}: \R ^n \to \R $ is a function from vectors to real numbers.
The notation follows the notation of absolute value, the \textit{magnitude} of a real number, and the double verticals remind us that $x$ is a vector.
A warning: some authors write $\abs{x}$ for the norm of $x$ when it is understood that $x \in \R ^n$.

We understand the norm of $x$ by comparison with the distance function $d: \R ^n \times  \R ^n \to \R $.
On one hand, the norm of $x$ is $d(x, 0)$.
So $\norm{x}$ measures the length of the vector $x$ from the origin $0$.
On the other hand, $d(x, y) = \norm{x - y}$.
So $\norm{x - y}$ measures the distance between $x$ and $y$.

\subsection*{Properties}

The norm has several important properties:
  \begin{enumerate}
  \item $\norm{\alpha x} = \abs{\alpha}\norm{x}$, called \t{(absolute) homogeneity},
  \item $\norm{x + y} \leq \norm{x} + \norm{y}$, called the \t{triangle inequality},
  \item $\norm{x} \geq 0$, called \t{non-negativity}, and
  \item $\norm{x} = 0 \iff x = 0$, called \t{definiteness}.
  \end{enumerate}

\blankpage