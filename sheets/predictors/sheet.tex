%!name:predictors
%!need:lists

\section*{Why}

We discuss inferring (or learning) functions from examples.

\section*{Definitions}

A \t{predictor} $f: \mathcal{U}  \to \mathcal{V} $ is a function from $\mathcal{U} $ to $\mathcal{V} $.
An \t{inducer} is a function from finite datasets in $\mathcal{U}  \times  \mathcal{V} $ to predictors from $\mathcal{U} $ to $\mathcal{V} $.
A \t{learner} is a function family of inducers, indexed by $n$, each defined for datasets of size $n$.
We call $\mathcal{U} $ the \t{inputs}, $\mathcal{V} $ the \t{outputs}, and $f(u)$ the \t{prediction} of $f$ on $u \in \mathcal{U} $.

\section*{Predicting relations}

A \t{relation inducer} is a function from finite datasets in $\mathcal{U}  \times  \mathcal{V} $ to \textit{relations} on $\mathcal{U}  \times  \mathcal{V} $.
Since we can associate any relation $R$ between $\mathcal{U} $ and $\mathcal{V} $ with a function $f: \mathcal{U}  \times \mathcal{V}  \to \set{0, 1}$, $f(u, v) = 1$ if and only if $(u, v) \in R$, the predictor case can accomodate learning general relations, beyond functions.

\subsection*{Notation}

Let $D$ be a dataset of size $n$ in $\mathcal{U}  \times \mathcal{V} $.
Let $g: \mathcal{U}  \to \mathcal{V} $, a predictor, which makes prediction $g(u)$ on input $u \in \mathcal{U} $.
Let $G_n: (\mathcal{U}  \times  \mathcal{V} )^n \to (\mathcal{U} \times  \mathcal{V} )$ be an inducer, so that $G_n(D)$ is the predictor which the inductor associates with dataset $D$.
Then $\set{G_n: (\mathcal{U}  \times  \mathcal{V} )^n \to \powerset{(\mathcal{U}  \times  \mathcal{V} )}}_{n \in \N }$ is a learner.

\section*{Consistent and complete datasets}

Let $D = ((u_i, v_i))_{i = 1}^{n}$ be a dataset and $R \subset X \times  Y$ a relation.
$D$ is \t{consistent with $R$} if each $(u_i, v_i) \in R$.
$D$ is \t{consistent} if there exists a relation with which it is consistent.
A dataset is always consistent (take $R = \mathcal{U}  \times \mathcal{V} $).
$D$ is \t{functionally consistent} if it is consistent with a function; in this case, $x_i = x_j \Rightarrow y_i = y_j$.
$D$ is \t{functionally complete} if $\cup_i\set{x_i} = X$.
In this case, the dataset includes every element of the relation.

\subsection*{Other terminology}

Other terms for the inputs include \t{independent variables}, \t{explanatory variables}, \t{precepts}, \t{covariates}, \t{patterns}, \t{instances}, or \t{observations}.
Other terms for the outputs include \t{dependent variables}, \t{explained variables}, \t{postcepts}, \t{targets}, \t{outcomes}, \t{labels} or \t{observational outcomes}.
An input, output pair is sometimes called a \t{record pair}.

Other terms for a learner include \t{learning algorithm}, or \t{supervised learning algorithm}.
Other terms for a predictor include \t{input-output} mapping, \t{prediction rule}, \t{hypothesis}, \t{concept}, or \t{classifier}.
