%!name:lists
%!need:direct_products
%!need:set_numbers
%!need:family_unions_and_intersections
%!refs:paul_halmos/naive_set_theory/section_11
%!refs:bert_mendelson/introduction_to_topology/theory_of_sets/indexed_families_of_sets

\ssection{Why}
We want to talk about several objects in order.

\ssection{Definition}

A \t{list} (or \t{finite sequence}, \t{string}, \t{$n$-tuple}) is a family (correspondence) whose index set is $\set{1, \dots, n}$ for $n \in \N$.
The \t{length} (or \t{size}) of a list is the size of its index set, $n$.
When the codomain of the sequence is a set $A$, we say that the sequence is \t{in} $A$ or that it is a sequence \t{of} elements of $A$.

We refer to a result of the sequence a \t{terms} (\t{entries}, \t{components}, \t{elements}
  \ifhmode\unskip\fi\footnote{
We avoid this terminology because it conflicts with sets.
  }
)

\ssubsection{Notation}

Since the natural numbers are ordered, we regularly denote finite sequences from left to right between parentheses.
For example, we denote $a: \set{1, \dots, 4} \to A$ by $(a_1, a_2, a_3, a_4)$.
$a(k)$ is the $k$th term.
Following the convention with functions, we regularly usually denote $a(n)$ by $a_n$

\ssubsection{Orderings and numberings}
Let $A$ be a set with $\num{A} = n$.
A sequence $a: \set{1, \dots, n} \to A$ is an \t{ordering} of $A$ if $a$ is invertible.
In this case, we call the inverse a \t{numbering} of $A$.
An ordering associates with each number a unique object and a numbering associates with each object a unique number (the object's \t{index}).

\ssubsection{Relation to Direct Products}

A \t{natural direct product} is a product of a sequence of sets.
We denote the direct product of a sequence of sets $A_1, \dots, A_n$ by $\prod_{i = 1}^{n} A_i$.
If each $A_i$ is the same set $A$, then we denote the product $\prod_{i = 1}^{n} A_i$ by $A^n$.
The set of sequences in a set $A$ is the direct product $A^n$.

\ssection{Natural unions and intersections}
We denote the family union of the finite sequence of sets $A_1$, $\dots$, $A_n$ by $\union_{i = 1}^{n} A_i$.
Similarly, we denote the intersection by $\intersection_{i = 1}^{n} A_i$
\ssection{Slices}
An \t{index range} for a list $s$ of length $n$ is a pair $(i, j)$ for which $1 \leq i < j \leq n$.
The \t{slice} corresponding to the index range $(i,j)$ is the length $j-i$ sequence $s'$ defined by $s'_1 = s_{i}$, $s'_2 = s_{i+1}, \dots, s'_{j} = s_{i + j-1}$.
We denote the $(i,j)$-slice of $s$ by $s_{i:j}$.
If $i = 1$ we use $s_{:j}$ and if $j = n$ we use $s_{i:}$ as shorthands for the slices $s_{1:j}$ and $s{i:n}$.
