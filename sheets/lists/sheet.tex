
\section*{Why}

We want to talk about several objects in order.

\section*{Definition}

Suppose $A$ is a set.
A \t{list} (or \t{finite sequence}, \t{$n$-tuple}, \t{string}, \t{dataset}) \t{in} $A$ (or \t{of} elements \textit{from} or \textit{of} $A$) is a function
\[
a: \set{1, \dots , n} \to A.
\]
In other words, a list is a \textit{family} whose index set is $\set{1, \dots , n}$.
The \t{length} (or \t{size}) of the list is the size of its domain.
The \t{$k$th entry} (or \t{term}, \t{record}) of $A$ is the result $a_k$ of $k$; here $k \in \set{1, \dots , n}$.

\subsection*{Notation}

Since the natural numbers are \sheetref{natural_order}{naturally ordered}, we denote lists using this order, from left to right, between parentheses.
For example, we denote the function $a: \set{1, \dots , 4} \to A$ by $(a_1, a_2, a_3, a_4)$.

\subsection*{Orderings and numberings}

Let $A$ be a set with $\num{A} = n$.
A sequence $a: \set{1, \dots , n} \to A$ is an \t{ordering} of $A$ if $a$ is invertible.
In this case, we call the inverse a \t{numbering} of $A$.
An ordering associates with each number a unique object and a numbering associates with each object a unique number (the object's \t{index}).

\subsection*{Relation to Direct Products}

A \t{natural direct product} is a product of a list of sets.
We denote the direct product of a list of sets $A_1, \dots , A_n$ by $\prod_{i = 1}^{n} A_i$.
If each $A_i$ is the same set $A$, then we denote the product $\prod_{i = 1}^{n} A_i$ by $A^n$.
The direct product $A^n$ is the set of lists in $A$ .

\section*{Natural unions and intersections}

We denote the family union of the list of sets $A_1, \dots , A_n$ by $\cup_{i = 1}^{n} A_i$.
Similarly, we denote the intersection by $\cap _{i = 1}^{n} A_i$.
\section*{Slices}

An \t{index range} for a list $s$ of length $n$ is a pair $(i, j)$ for which $1 \leq i < j \leq n$.
The \t{slice} corresponding to $(i,j)$ is the length $j-i$ list $s'$ defined by $s'_1 = s_{i}$, $s'_2 = s_{i+1}, \dots, s'_{j} = s_{i + j-1}$.

We denote the $(i,j)$-slice of $s$ by $s_{i:j}$.
If $i = 1$ we use $s_{:j}$ and if $j = n$ we use $s_{i:}$ as shorthands for the slices $s_{1:j}$ and $s_{i:n}$.
