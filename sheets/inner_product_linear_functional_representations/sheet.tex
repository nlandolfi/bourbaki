
%!name:inner_product_linear_functional_representations
%!need:dual_spaces
%!need:complete_real_inner_product_spaces
%!refs:young1988an/06dualspaces

\section*{Why}

We can identify any linear functional $F: \R ^n \to \R $ with a vector $y \in \R ^n$ so that $F(x) = \ip{x,y}$.
We generalize this result to complete inner product spaces.

\section*{Motivating result}

The following is known as the \t{Riesz representation theorem} (or \t{Riesz-Fréchet representation theorem}, or \t{Riesz theorem}, or \t{Riesz-Fréchet theorem}).

\begin{proposition}
Let $((V, k), \ip{\cdot ,\cdot })$ be a complete inner product space and let $F: V \to k$ be a continuous linear functional on $V$.
There exists a unique $y \in V$ so that
\[
F(x) = \ip{x, y}
\]
for all $x \in V$. Moreover $\norm{y} = \dnorm{F}$.
\end{proposition}

Clearly $\R ^n$ is a complete inner product space, and so this this theorem says the expected.
We can identify linear functionals on $\R ^n$ with elements (vectors) in $\R ^n$.\footnote{Future editions will expand further.}

\blankpage