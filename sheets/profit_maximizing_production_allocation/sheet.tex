
\section*{Why}

Suppose a manufacturer has raw materials, and production processes by which she can convert raw materials into finished goods.
How should she \textit{allocate} the raw materials among the finished goods to maximize revenue.

\section*{Model}

We model the quantities of $m$ raw materials available to the manufacturer by $m$ real numbers, which we denote $q_1, \dots , q_m \in \R _+$.
We suppose that there is a correspondence, $f_i: \R _+ \to \R _+^m$, which models the quantities of the $m$ raw materials that will be needed for the $i$th finished good.
In other words, $f_i(x)$ is the \t{bill of materials} to produce quantity $x$ of the $i$th finished good.
We suppose finished good $i$ can be sold for a price $p_i$ per unit.

We formulate the following optimization problem.
Given a supply of raw materials $q_1, \dots , q_m$, find the quantities $x_1, \dots , x_n$ to
\[
\begin{aligned}
\text{ maximize } \quad & \textstyle \sum_{i = 1}^{n} p_i x_i \\
\text{ subject to } \quad & \textstyle \sum_{i = 1}^{n} f_i(x_i) \leq q \\
\text{ and } \quad & x \geq 0
\end{aligned}
\]
This is sometimes called an \t{allocation problem} or \t{manufacturing problem}.

\textit{Linear simplication.}
In the case that $f_i$ is modeled (or idealized) as a linear function, we obtain a \textit{linear optimization problem}.

\blankpage