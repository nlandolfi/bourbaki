
%!name:matrix_inverses
%!need:real_matrix_inverses
%!need:matrices

\section*{Why}

What is the inverse element under matrix multiplication.

\section*{Definition}

Recall that if $A \in \R ^{m \times  n}$ then $x \mapsto Ax$ is a function from $\R ^{n}$ to $\R ^{m}$.
Clearly, if $m \neq n$, then the inverse of $f$ can not exist.\footnote{Future editions will expand.}

Now suppose that $A \in \R ^{n \times  n}$.
Of course, the inverse may not exist.
Consider, for example if $A$ was the $n$ by $n$ matrix of zeros.
If there exists a matrix $B$ so that $BA = I$ we call $B$ the \t{left inverse} of $A$ and likewise if $AC = I$ we call $C$ the \t{right inverse} of $A$.
In the case that $A$ is square, the right inverse and left inverse coincide.

\begin{proposition}
Let $A, B, C \in \R ^{n \times  n}$.
Let $BA = I$ and $AC = I$.
Then $B = C$.
\begin{proof}Since $BA = AC$ we have $BBA = BAC$ so $B = C$ since $BA = I$.\end{proof}\end{proposition}
\subsection*{Notation}

Let $\F $ be a field.
Let $A \in \F ^{n \times  n}$
be invertible.
We follow the notation of
inverse elements and denote
the inverse of $A$ by
$A^{-1}$.

\blankpage