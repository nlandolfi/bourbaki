
%!name:area
%!need:geometry

\section*{Why}

We list some principles which our intuition of area in planar geometry satisfies.

\section*{Common notions}

We take two common notions; these are analagous to those we developed for length.
  \begin{enumerate}
    \item The area of the whole is the sum of the area of the parts; the \t{additivity principle}.
    \item If one whole contains another, the first's area at least as large as the second's area; the \t{containment principle}.
  \end{enumerate}

Again, the task is to make precise the use of \say{whole,} \say{parts,} and \say{contains.} We start with rectangles.

\section*{Definition}

The \t{area} of an rectangle is the sum of the lengths of its sides.

Two rectangles are \t{non-overlapping} if their intersection is a single point or empty.
The \t{area} of the union of two non-overlapping intervals is the sum of their areas.

A \t{simple} subset of the real numbers is a finite union of non-overlapping intervals.
The length of a simple subset is the sum of the lengths of its family.

A \t{countably simple} subset of the real numbers is a countable union of non-overlapping intervals.
The length of a countably simple subset is the limit of the sum of the lengths
of its family; as we have defined it, length is positive, so this series is either bounded and increasing and so converges, or is infinite, and so converges to $+\infty$.

At this point, we must confront the obvious question: are all subsets of the real numbers countably simple?
Answer: no.
So, what can we say?

A \t{cover} of a set $A$ of real numbers is a family whose union is a contains $A$. Since a cover always contains the set $A$, it's length, which we understand, must be larger (containment principles) than $A$.
So what if we declare that the length of an arbitrary set $A$ be the greatest lower bound of the lengths of all sequences of intervals covering $A$. Will this work?

\subsection*{Cuts}

If $a, b$ are real numbers and $a < b$, then we \t{cut} an interval with $a$ and $b$ as its endpoints by selecting $c$ such that $a < c$ and $c < b$.
We obtain two intervals, one with endpoints $a,c$ and one with endpoints $c, b$; we call these two the \t{cut pieces}.

Given an interval, the length of the interval is the sum of any two cut pieces, because the pieces are non-overlapping.

\section*{All sets}

\begin{proposition}
Not all subsets of real numbers are simple.

Exhibit: R is not finite.
\end{proposition}
\begin{proposition}
Not all subsets of real numbers are countably simple.

Exhibit: the rationals.
\end{proposition}
Here's the great insight: approximate a set by a countable family of intervals.
