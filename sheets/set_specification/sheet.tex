%!name:set_specification
%!need:set_inclusion

\ssection{Why}

We want to construct new sets out of old ones.
So, can we always construct subsets?

\ssection{Definition}

We will say that we can.
More specifically, if we have a set and some statement which may be true or false for the elements of that set, a set exists containing all and only the elements for which the statement is true.

Roughly speaking, the principle is like this.
We have the set which contains some objects.
Suppose we believe the set of all playing cards in the usual deck exists.
We are taking as a principle that the set of all fives exists, so does the set of all fours, as does the set of all hearts, and the set of all face cards.
Roughly, the corresponding statements are \say{the card is a five}, \say{the card is a four}, \say{the card is a heart}, and \say{the card is a face card}.
%Of course, the sets in question will not always be so simple.
%If 

\begin{principle}[Specification]
	For any statement and any set, there is a subset whose elements satisfy the statement.
\end{principle}

We call this the \t{principle of specification}. 
We call the second set (obtained from the first) the set obtained by \t{specifying} elements according to the sentence.
The principle of extension (see \sheetref{set_equality}{Set Equality}) says that this set (here denoted $A'$) is unique.
All  basic principles about sets (other than the principle of extension, see \sheetref{set_equality}{Set Equality}) assert that we can construct new sets out of old ones in reasonable ways.

% The basic principles of set theory other than the axiom of extension allow us to construct new sets out of existing ones.
% The fir
% The basis principles of set theory--- the axiom of extension, allow us to
Let $A$ denote a set.
Let $s$ denote a statement in which the symbol $x$ and $A$ appear unbound.
We assert that there is a set, denote it by $A'$ for which belonging is equivalent to the statement denoted by $s$.
In other words, regardless of the identity of $A$ and $s$, there exists $A'$ so that
\[
	(\forall x)((x \in A') \iff ((x \in A) \land s(x))).
\]
% We assert that to every set and every sentence predicated of elements of the set there exists a second set (a subset of the first) whose elements satisfy the sentence.
% It is an consequence of the axiom of extension that this set is unique.



%For example:
%\begin{account}[Example Specification]
%\name{$A$,$y$}
%\thus{set_specification:asdf:asdf}{$(\exists A')((x \in A') \iff (x \ne y)))$}{Axiom:Specification}
%\end{account}

\ssection{No universe}

Consider the following account
\begin{account}
	\namee{$A$}{$A'$}
	\have{asdf}{$(\forall x)(x \in A' \iff ((x \in A) \land x \not\in x)$}
	\thus{fdas}{$A' \notin A$}{TODO}
\end{account}

%Consider the statement $x \not\in x$.
%Let $A$ denote a set.
%And define $A'$ by specifying the elements denoted by $x$ which satisfy $x \not\in x$.
%Then $A' \not\in A$.
%Since $x \in A'$ is equivalent to $x \not\in x$
%If $A' \in A$, then $A' \not\in A$.

\ssubsection{Notation}

Denote by $A$ a set, and by $s(x)$ a statement with the unbound variable $x$.
We denote the set of those elements of $A$ which satisfy $s$ by 
\[
  \Set*{a \in A}{\;s(a)\;}.
\]
We read the symbol $\mid$ aloud as \say{such that.}
We read the whole notation aloud as \say{a in A such that...}

We call the notation
\ct{set-builder notation}{setbuildernotation}.
Set-builder notation avoids enumerating
elements.
This notation is really indispensable for
sets which have many members, too many
to reasonably write down.


%TODO(next edition) bring back example? split out universe?
%\ssection{Example}
%
%For example, let $a, b, c, d$
%be distinct objects.
%Let $A = \set{a, b, c, d}$.
%Then
%$\Set*{x \in A}{x \neq a}$
%is the set $\set{b, c, d}$
%
%Now let $B$ be an arbitrary
%set.
%The set $\Set*{b \in B}{b \neq b}$
%specifies the empty set.
%Since the statement $b \neq b$ is
%false for all objects $b$.

%!TODO: russel's paradox? p 6 of halmos
