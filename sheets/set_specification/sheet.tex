%!name:set_specification
%%!need:set_inclusion
%!need:set_equality

\ssection{Why}

We want to construct new sets out of old ones.
So, can we always construct subsets?

\ssection{Definition}

We will say that we can.
More specifically, if we have a set and some statement which may be true or false for the elements of that set, a set exists containing all and only the elements for which the statement is true.

Roughly speaking, the principle is like this.
We have a set which contains some objects.
Suppose the set of playing cards in a usual deck exists.
We are taking as a principle that the set of all fives exists, so does the set of all fours, as does the set of all hearts, and the set of all face cards.
Roughly, the corresponding statements are \say{it is a five}, \say{it is a four}, \say{it is a heart}, and \say{it is a face card}.
%Of course, the sets in question will not always be so simple.
%If

\begin{principle}[Specification]
	For any statement and any set, there is a subset whose elements satisfy the statement.
\end{principle}

We call this the \t{principle of specification}.
We call the second set (obtained from the first) the set obtained by \t{specifying} elements according to the sentence.
The principle of extension (see \sheetref{set_equality}{Set Equality}) says that this set is unique.
All  basic principles about sets (other than the principle of extension, see \sheetref{set_equality}{Set Equality}) assert that we can construct new sets out of old ones in reasonable ways.

% The basic principles of set theory other than the axiom of extension allow us to construct new sets out of existing ones.
% The fir
% The basis principles of set theory--- the axiom of extension, allow us to
% We assert that to every set and every sentence predicated of elements of the set there exists a second set (a subset of the first) whose elements satisfy the sentence.
% It is an consequence of the axiom of extension that this set is unique.

\ssubsection{Notation}

Let $A$ denote a set.
Let $s$ denote a statement in which the symbol $x$ and $A$ appear unbound.
We assert that there is a set, denote it by $B$, for which belonging is equivalent to membership in $A$ and $s$.
In other words,
\[
	(\forall x)((x \in B) \iff ((x \in A) \land s(x))).
\]
We denote $B$ by $\Set*{x \in A}{s(x)}$.
We read the symbol $\mid$ aloud as \say{such that.}
We read the whole notation aloud as \say{a in A such that...}
We call it \ct{set-builder notation}{setbuildernotation}.


%For example:
%\begin{account}[Example Specification]
%\name{$A$,$y$}
%\thus{set_specification:asdf:asdf}{$(\exists A')((x \in A') \iff (x \ne y)))$}{Axiom:Specification}
%\end{account}


%Consider the statement $x \not\in x$.
%Let $A$ denote a set.
%And define $A'$ by specifying the elements denoted by $x$ which satisfy $x \not\in x$.
%Then $A' \not\in A$.
%Since $x \in A'$ is equivalent to $x \not\in x$
%If $A' \in A$, then $A' \not\in A$.

% Set-builder notation avoids enumerating
% elements.
% This notation is really indispensable for
% sets which have many members, too many
% to reasonably write down.

\ssection{Nothing contains everything}

As an example of the principle of specification and an important consequence, consider the statement $x \not\in x$.
Using this statement and the principle of specification, we can prove that there is not set which contains every thing.

%TODO(next edition): explain this more.
\begin{proposition}
  No set contains all sets.\footnote{We might call such a set, if we admitted its existence, a \t{universe of discourse} or \t{universal set}.
  With the principle of specification, a \say{principle of a universal set} would give a contradiction (called \t{Russell's paradox}).}
\begin{proof}
  Suppose there exists a set, denote it $A$ which contains all sets.
  In other words, suppose $(\exists A)(\forall x)(x \in A)$.
  Use the principle of specification to construct $B = \Set{x \in A}{x \not\in x}$.
  So $(\forall x)(x \in B \iff (x \in A \land x \not\in x))$
  In particular, $(B \in B \iff (B \in A \land B \not\in B))$.
  So $B \not \in A$.
% \begin{blankaccount}
%   \namee{$A$}{$B := \Set{x \in A}{x \not\in x}$}
%   \have{set_specification:noU:membership}{$(B \in B) \iff ((B \in A) \land B \not\in B)$}
%   \thus{set_specification:noU:conclusion}{$B \notin A$}{\ref{set_specification:noU:membership}}
% \end{blankaccount}
\end{proof}
\end{proposition}

% In other words, nothing contains everything.


%TODO(next edition) bring back example? split out universe?
%\ssection{Example}
%
%For example, let $a, b, c, d$
%be distinct objects.
%Let $A = \set{a, b, c, d}$.
%Then
%$\Set*{x \in A}{x \neq a}$
%is the set $\set{b, c, d}$
%
%Now let $B$ be an arbitrary
%set.
%The set $\Set*{b \in B}{b \neq b}$
%specifies the empty set.
%Since the statement $b \neq b$ is
%false for all objects $b$.

%!TODO: russel's paradox? p 6 of halmos

