%!name:set_specification
%!need:set_inclusion
% %!need:sentences

\ssection{Why}

We want to construct new sets out of old ones.
So, can we always construct subsets?

\ssection{Definition}

We will say that we can.
% The basic principles of set theory other than the axiom of extension allow us to construct new sets out of existing ones.
% The fir
% The basis principles of set theory--- the axiom of extension, allow us to
Let $A$ denote a set.
Let $s$ denote a statement in which the symbol $x$ and $A$ appear unbound.

We assert that there is a set, denote it by $A'$ for which belonging is equivalent to the statement denoted by $s$.
It is a consequence of the axiom of extension that this set is unique.
% We assert that to every set and every sentence predicated of elements of the set there exists a second set (a subset of the first) whose elements satisfy the sentence.
% It is an consequence of the axiom of extension that this set is unique.
This assertion is sometimes called the \t{axiom of specification} is this assertion.
We call the second set (obtained from the first) the set obtained by \t{specifying} elements according to the sentence.

All the basic principles of set theory other than the axiom of extension assert that we can construct new sets out of old ones in reasonable ways.

For example:
\begin{account}[Example Specification]
\name{$A$,$y$}
\thus{set_specification:asdf:asdf}{$(\exists A')((x \in A') \iff (x \ne y)))$}{Axiom:Specification}
\end{account}

\ssubsection{Notation}

Let $A$ be a set.
Let $S(a)$ be a sentence.
We use the notation
\[
  \Set*{a \in A}{\;S(a)\;}
\]
to denote the subset of $A$
specified by $S$.
We read the symbol $\mid$ aloud as
\say{such that.}
We read the whole notation aloud as
\say{a in A such that...}

We call the notation
\ct{set-builder notation}{setbuildernotation}.
Set-builder notation avoids enumerating
elements.
This notation is really indispensable for
sets which have many members, too many
to reasonably write down.

\ssection{Example}

For example, let $a, b, c, d$
be distinct objects.
Let $A = \set{a, b, c, d}$.
Then
$\Set*{x \in A}{x \neq a}$
is the set $\set{b, c, d}$

Now let $B$ be an arbitrary
set.
The set $\Set*{b \in B}{b \neq b}$
specifies the empty set.
Since the statement $b \neq b$ is
false for all objects $b$.

%!TODO: russel's paradox? p 6 of halmos
