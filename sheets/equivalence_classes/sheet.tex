%!name:equivalence_classes
%!need:equivalence_relations
%!need:partitions

\ssection{Why}

Equivalence relations
partition the base
set into sets of \say{equivalent}
elements.
We can define an appropriate
equivalence relation on
a set and then work with
the \textit{set of sets}
of equivalent objects.

\ssection{Definition}

We call all elements related to a
particular element under
an equivalence relation the
\ct{equivalence class}{equivalenceclass}
of the element.
The key observation,
recorded and proven below,
is that the equivalence classes
partition the base set.
This will allow us to define
appropriate equivalence relations
on a set and then work with the
set of equivalence classes insteaded.

We call the set of equivalence classes the \ct{quotient set}{quotientset}
of the set under the relation.
An equally good name is the divided set of the
set under the relation, but this
terminology is not standard.
The language in both cases reminds us that the
relation
partitions the set into equivalence classes.

\ssubsection{Notation}
Let $A$ be a non-empty set
and $\sim$ be an equivalence
relation on $A$.
We denote the quotient set of $A$ under $\sim$ by $A/\sim$,
read aloud as \say{A quotient sim}.

\ssubsection{Results}

TODO

