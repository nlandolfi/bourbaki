
\section*{Why}

We generalize convex functions to $\R ^n$.

\section*{Definition}

Given a set $S \subset \R ^n$ and a function $f: S \to \Rbar$, the set
\[
\Set{(x, \alpha ) \in S \times  \R }{f(x) \leq \alpha } \subset \R ^{n+1}
\]
is called the \t{epigraph} of $f$ and is denoted by $\epi f$.
The epigraph is a subset of $\R ^{n+1}$.
Recall that $\Rbar = \R  \cup \set{-\infty, +\infty}$ denotes the extended real numbers.

The function $f$ is \t{convex} if $\epi f$ is a convex subset of $\R ^{n+1}$.
$f$ is \t{concave} if $-f$ is convex.
$f$ is \t{affine} if it is finite, convex and concave.

\subsection*{Visualization of epigraph}

\begin{center}\includegraphics[width=0.80\textwidth]{./graphics/cvxepi.pdf}\end{center}
\subsection*{Effective Domains}

Suppose $f: S \to \R $ is convex.
Since our definition of epigraph restricts the last coordinate to be real, no points $x \in S$ for which $f(x) = +\infty$ are ``included'' in the epigraph.
The \t{effective domain} (or just \t{domain}) of a convex function $f: S \to \Rbar$ is the linear projection of the first $n$ coordinates of $\epi f$.
In other words,
\[
\dom f = \Set{x \in S}{\exists  \alpha  \in \R , (x, \alpha ) \in \epi f} = \Set{x \in S}{f(x) < +\infty}
\]

If we define the linear projection (canonical projection) $\pi : \R ^{n +1} \to \R $ by $\pi (x,\alpha ) = x$ where $x \in \R ^{n+1}$ and $\alpha  \in \R ^n$, then
\[
\dom f = \pi (\epi f)
\]
Since $\pi $ is linear and $\epi f$ is convex by assumption, $\dom f$ is \textit{convex}.
To say this again, the effective domain of a convex function is convex.
% todo: prove earlier in sheets 

% see rockafeller 23 for more on why we have effective domains 


Notice that the restriction $f_{\mid \dom f}$ has the same epigraph as $f$.
For this reason, $f$ is convex if and only if $f_{\mid \dom f}$ is convex.
Clearly then $f$ restricted to its domain is the principal object of interest.
The usefulness of these technicalities is that any convex $f: S \to \Rbar$ can be extended to a convex function $\bar{f}: \R ^n \to \Rbar$ with the same effective domain.
Define $\bar{f}$ so that
\[
\bar{f}(x) =
\begin{cases}
f(x) & x \in S \\
+\infty & \text{otherwise}
\end{cases}
\]
Henceforth, by \t{convex function} will be understood an extended real value function, possibly taking value $+\infty$, defined on all of $\R ^n$ whose epigraph is convex.
The upshot is that in constructing a convex function from other convex functions, the effective domains are specified implicitly.
% <div data-littype='run'> Let $A$ be a convex subset of $𝗥^n$. </div>
% <div data-littype='run'> The function $f: A → 𝗥$ is ❬convex❭ if for any $a, b ∈
%    A$ and $t ∈ [0, 1]$,
%    <div data-littype='displaymath'>
%     <div data-littype='run'> f(ta + (1-t)b) ≤ tf(a) + (1-t)f(b). </div>
%    </div></div>
% <div data-littype='run'> It is ❬concave❭ if $-f$ is convex. </div>
%</div>


\subsection*{Proper convex functions}

A convex function $f$ is \t{proper} if there exists $x$ such that $f(x) < \infty$ and $f(x) > \infty$ for all $x \in \R ^n$.
Roughly speaking, $f$ is proper if its epigraph is nonempty and contains no vertical lines.
$f$ is proper if and only if $\dom f$ is nonempty and $f_{\mid \dom f}$ is finite.

The proper convex functions arise naturally by taking a function $f: C \to \R $ defined on a convex subset $C \subset \R ^n$ and extending it to have value $+\infty$ outside of $C$.
A convex function which is not proper is \t{improper}.

Of course, proper convex functions are our main object of study, but improper ones arise from time to time and it is more convenient to admit them than exclude them.

For an example of an improper convex function, define $f: \R ^n \to \R $ by
\[
f(x) = \begin{cases}
-\infty & \text{ if } \norm{x} < 1, \\
0 & \text{ if } \norm{x} = 1, \\
+\infty & \text{ if } \norm{x} > 1, \\
\end{cases}
\]


\blankpage
