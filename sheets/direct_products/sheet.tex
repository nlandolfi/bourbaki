%!name:direct_products
%!need:families
%!refs:paul_halmos/naive_set_theory/section_09
%! for the \set{1, 2, \dots, n} and \N cases
%!refs:bert_mendelson/introduction_to_topology/theory_of_sets/products_of_sets
%! for arbitrary products
%1refs:bert_mendelson/introduction_to_topology/theory_of_sets/arbitrary_products

\ssection{Why}

We can generalize the product of two sets to a product of a family of sets.

\ssection{Discussion for pairs}

Let $A$ and $B$ be sets.
There is a natural correspondence between $A \times B$ (see \sheetref{cartesian_products}{Cartesian Products}) and a particular set of families.
The particular set of families $z: \set{i, j} \to (A \union B)$ with $z_i \in A$ and $z_b \in B$.
The family $z$ corresponds with the pair $(z_i, z_i)$.
The pair $(a, b)$ corresponds to the family $z: \set{i, j} \in (A \union B)$ defined by $z(i) = a$ and $z(j) = b$.
In other words, we can think about ordered pairs as special families.
The generalization of Cartesian products to more than two sets generalizes the notion for families.

\ssection{Direct Products}

Let $X$ be a set.
Let $A: I \to X$ be a family of subsets of $X$.
The \ct{direct product}{directproduct} or \t{family Cartesian product} of $A$ is the set of all families $a: I \to X$ which satisfy $a_i \in A_i$ for every $i \in I$.
A function on a product is called a \t{function of several variables} and, in particular, a function on the product $X \times Y$ is called a \t{function of two variables.}
% TODO: need to say what this is in terms of sets.
% We call the elements of the direct product \ct{$n$-tuples}{ntuples}.
% We call the $i$th element in an $n$-tuple the $i$th coordinate.
% This language is meant to follow that used in defining ordered pairs.
%If the index set is the natural numbers, and every set in the family
%is the same set $A$, we call the elements of the direct product the
%\ct{sequences}{sequences} in $A$.
% Two coordinates in a sequence are \t{consecutive} if their natural difference is 1.

\ssubsection{Notation}

We denote the product of the family $\set{A_i}$ by
\[
  \product_{i \in I} A_i
\]
We read this notation as \say{product over i in I of A sub-i.}
% We denote an element of $\product_{i = 1}^{n} A_{i}$ by $(a_1, a_2, \dots, a_n)$ with the understanding that $a_1 \in A_1, a_2 \in A_2, \dots, a_n \in A_n$.
%
% If $A_i = A$ for $i = 1, \dots, n$, then we denote
% $\product_{i = 1}^{n} A_{i}$ by $A^n$.

\s{Projections}

The word \say{projection} is used in two senses with families.
Let $I$ be a set, and let $\set{A_i}$ be a family of sets.
Define $A = \prod_{i \in I} A_i$.
Then if $J \subset I$, there is a natural correspondence between the elements of $X$ and those of $\prod_{j \in J} A_j$.
To each element $x \in X$, we restrict $x$ to $J$ and this is an element of $\prod_{j \in J} A_j$.
The correspondence is called the \t{projection} of $X$ onto $\prod_{i \in J} X_i$
Also, the value of $x$ at $j$ is called the \t{projection of $x$ onto index $j$} or the \t{$j$-coordinate} of $x$.

\s{Repeated set}

If $\set{A_i}$ is a finite sequence of sets and $A_i = A$ for some $A$ for all $i$ then we denote the product $\prod_{i = 1}^{n} A_i$ by $A^n$.
We call an element $a = (a_1, a_2, \dots, a_n) \in A$ an \t{$n$-tuple}.
