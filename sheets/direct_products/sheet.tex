%!name:direct_products
%!need:natural_numbers

\ssection{Why}

We can profitably generalize the notion of
cartesian product to families of sets
indexed by the natural numbers.

\ssection{Direct Products}

Let $I$ be a set.
Let $A: I \to U$ be
a function and $A_i$
be a set for each $i \in I$.
We call $I$ the index set
and we call $A_i$.
Often $I$ is the $\set{1, \dots, n}$.
We will commonly declare this
situation as let $A_1, \dots, A_n$
be sets.
a set

Consider the set of the first
$n$ natural numbers.
A \ct{family}{} of sets

A family of sets is a function
from the natural numbers

Consider a

Let $U$ be a non-empty set.
Let $A: \set{1, \dots, n} \to \powerset{U}$.
So $A_1, \dots, A_n \subset U$ are
each sets.
We will commonly

Let $A_1, \dots, A_n$ be sets.
Let $f: \set{1, \dots, n} \to A$
where $n$ is a natural number
and $A$ is a set.



The \ct{direct product}{directproduct} of family indexed by a
subset of the naturals is the set whose elements are ordered sequences
of elements from each set in the family.
The ordering on the sequences comes from the natural ordering on $N$.
If the index set is finite, we call the elements of the direct product
\ct{$n$-tuples}{ntuples}.
If the index set is the natural numbers, and every set in the family
is the same set $A$, we call the elements of the direct product the
\ct{sequences}{sequences} in $A$.

\ssubsection{Notation}

For a family $\set{A_{\alpha}}_{\alpha \in I}$ of $S$ with $I = \set{1, \dots, n}$, we denote the direct product by
\[
  \product_{i = 1}^{n} A_{i}.
\]
We read this notation as \say{product over alpha in I of A sub-alpha.}
We denote an element of $\product_{i = 1}^{n} A_{i}$ by $(a_1, a_2, \dots, a_n)$ with the understanding that $a_1 \in A_1, a_2 \in A_2, \dots, a_n \in A_n$.

If $I$ is the set of natural numbers we denote the direct product by
\[
  \product_{i = 1}^{\infty} A_{i}.
\]
We denote an element of $\product_{i = 1}^{\infty} A_{i}$ by $(a_i)$ with the understanding that $a_i \in A_i$ for all $i = 1,2,3,\dots$.
If $A_i = A$ for all $i = 1, 2, 3,\dots$, then $(a_i)$ is a sequence in $A$.

\strats
