
\section*{Why}

We generalize the product of two sets to a product of a family of sets.
To do so we discuss sets of families.

\section*{Discussion for pairs}

Suppose $X$ and $Y$ are nonempty sets.
There is a natural correspondence between the product $X \times Y$ (see \sheetref{set_products}{Set Products}) and the set of families
\[
Z = \Set*{z: \set{i, j} \to (A \cup B)}{z_i \in A \text{ and } z_j \in B}
\]
where $\set{i,j}$ is any unordered pair with $i \neq j$.

The set $Z$ can be put in one-to-one correspondence with $X \times  Y$.
The family $z \in Z$ corresponds with the pair $(z_i, z_j)$.
The pair $(a, b)$ corresponds to the family $z \in Z$ defined by $z(i) = a$ and $z(j) = b$.
So, ordered pairs can be put in one-to-one correspondence with families.
The generalization of Cartesian products to more than two sets generalizes the notion for families.

\section*{Definition}

Suppose $\set{A_i}_{i \in I}$ is a family of sets.
The \t{direct product} ( or \t{Cartesian product}, \t{family Cartesian product}) of $A$ is the \textit{set} of \textit{all} functions (i.e., families) $a: I \to X$ which satisfy $a_i \in A_i$ for every $i \in I$.

A function on a product is called a \t{function of several variables} and, in particular, a function on the product $X \times  Y$ is called a \t{function of two variables}.

%% TODO: need to say what this is in terms of sets.
%% We call the elements of the direct product \ct{$n$-tuples}{ntuples}.
%% We call the $i$th element in an $n$-tuple the $i$th coordinate.
%% This language is meant to follow that used in defining ordered pairs.
%%If the index set is the natural numbers, and every set in the family
%%is the same set $A$, we call the elements of the direct product the
%%\ct{sequences}{sequences} in $A$.
%% Two coordinates in a sequence are \t{consecutive} if their natural difference is 1.

\subsection*{Notation}

We denote the product of the family $\set{A_i}_{i \in I}$ by
\[
\textstyle
\prod_{i \in I} A_i
\]
We read this notation as ``product over i in I of A sub-i.''
Other notation in use includes $\times _{i \in I} A_i$.

%% We denote an element of $\product_{i = 1}^{n} A_{i}$ by $(a_1, a_2, \dots, a_n)$ with the understanding that $a_1 \in A_1, a_2 \in A_2, \dots, a_n \in A_n$.
%%
%% If $A_i = A$ for $i = 1, \dots, n$, then we denote
%% $\product_{i = 1}^{n} A_{i}$ by $A^n$.

\section*{Projections}

The word ``projection'' is used in two senses with families.
Let $I$ be a set, and let $\set{A_i}_{i \in I}$ be a family of sets.
Define $A = \prod_{i \in I} A_i$.

First, let $J \subset I$.
There is a natural correspondence between the elements of $A$ and those of $\prod_{j \in J} A_j$.
To each element $a \in A$, we restrict $a$ to $J$ and this is restriction is an element of $\prod_{j \in J} A_j$.
The correspondence is called the \t{projection} of $A$ onto $\prod_{i \in J} A_i$.
The projection in this sense is a set of families.

Second, consider the value of a family $a \in A$ at $j$.
We call $a_j$ the \t{projection of $a$ onto index $j$} or the \t{$j$-coordinate} of $a$.
This word \t{coordinate} is meant to follow the language used in defining ordered pairs.
The projection in this sense is an element of $A_j$.
The $j$th projection is a function mapping $\prod_{i \in I} X_i$ to $X_j$.

%macros.tex
%\newcommand{\product}{\prod}
