
\section*{Why}

We want to arrange the elements of a set in an order using only the concept of sets.\footnote{This sheet needs revision.}

\section*{Discussion}

What does this mean? Well, we often arrange objects in orders.
For example, the letters of this page are arranged into words.
Take two such words: `note' and `tone'.
If letters are objects, what are words?

A first guess is that words seem like groups of letters, and sets seem like groups, and so a word is a set of letters.
So, the word `note' is the set {'n', `o', `t', `e'}, and then word `tone' is the set {`t', `o', `n', `e'}.
The rub, of course, is that these are the same set.

The trick is that a word is not just the set of letters, it is that set in some order.
Since `tone' and `note' have the same letters, they have the same set of letters.
% Since $\set{\text{‘n}}, \text{‘o}}, \text{‘t}}, \text{‘e}}}$ denotes the same set as $\set{\text{‘t}}, \text{‘o}}, \text{‘n}}, \text{‘e}}}$ and both these denote the same set as $\set{\text{‘e}}, \text{‘t}}, \text{‘o}}, \text{‘n}}}$. 

The question is whether there is a way of saying what a word is in terms of letters by using sets in such a way that the set corresponding to `tone' is distinguishable from the set corresponding to `note'.

% %TODO(next edition): single quotes represent words and letters 

The way we read English offers a hint.
When reading `tone' we scan from left to right seeing `t', then `to', then `ton' then `tone'.
Suppose that for each spot in the ordering of the letters, we consider those letters that appear at or before the spot.
In other words, we can consider the sets {`t'}, {`t', `o'}, {`t', `o', `n'}, {`t', `o', `n', `e'}.
Let us say that `tone' corresponds to the set of these sets, denoted by $\mathcal{C} $,
\[
\mathcal{C}  = \set{\set{n, o, t}, \set{n, o, t, e}, \set{t}, \set{o, t}}.
\]

Given $\mathcal{C} $, can we recover `tone' (instead of `note')?
Sure.
First, look for a set contained in all the others.
The singleton {`t'} is the only one.
So the first letter is `t'.
Next look for a set distinct from `t' which is contained in all the rest.
The pair {`o', `t'} is the only one.
Since we already have `t', the next letter is `o'.
We do the same twice more, getting `n' and `e', in that order.

There is a certain peculiarity in all these considerations.
Every time we write down a set, we write the names (see \sheetref{names}{Names}) of the elements in some order.
Indeed, whenever we speak of objects, we must say their names in some order.
But of course, no matter how we denote or speak of the set, the concept of set has no concept of ordering.

\section*{Generally}

Let $a$, $b$, $c$ and $d$ denote objects, no two of which are the same (i.e., $a \neq b$, $b \neq c$, etc.).
Suppose we want to consider the elements of the quadruple $\set{a, b, c, d}$ in the order $c$, $b$, $d$, $a$.
We include in the set all objects that that occur at or before that position.
For the order $c$, $b$, $d$, $a$ of the objects in the set $\set{a, b, c, d}$ we use $\set{c}$, $\set{c, b}$, $\set{c, b, d}$ and $\set{c, b, d, a}$.

%% Then we can recover the ordering from
%% \[
%%   \CC = \set{\set{a, b, c, d}, \set{b, c}, \set{b, c, d}, \set{c}}.
%% \]
