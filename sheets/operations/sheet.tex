%!name:operations
%!need:functions
%!refs:paul_halmos/naive_set_theory/section_04
%!refs:paul_halmos/naive_set_theory/section_08

\ssection{Why}

We want to \say{combine} elements of a set.

\ssection{Definition}

Let $A$ be a non-empty set.
An \t{operation} on $A$ is a function from ordered pairs of elements of the set to the same set.
Operations \t{combine} elements.
We \t{operate} on ordered pairs.

\ssubsection{Notation}

Let $A$ be a set and $g: A \times A \to A$.
We tend to forego the notation $g(a, b)$ and write $a\,g\,b$ instead.
We call this \t{infix notation}.

Using lower case latin
letters for elements and for operators
confuses, so we tend to use
special symbols for operations.
For example,
$+$, $-$, $\cdot$, $\circ$, and $\star$.

Let $A$ be a non-empty set
and $+: A \times A \to A$ be
an operation on $A$.
According to the above paragraph,
we tend to write
$a+b$ for the result of applying $+$
to $(a,b)$.

\ssubsection{Example}

A first example of an operation is if we consider the set $A$ as the power set of some set $X$.
Then the pair union (see \sheetref{pair_unions}{Pair Unions}) is an operation.
For if $E \in \powerset{X}$ and $F \in \powerset{X}$ then $E \union F \in \powerset{F}$ and so $\union$ can be viewed as an operation on $\powerset{X}$.

\ssection{Properties}

Recall that $\union$ has several nice properties.
For one $A \union B = B \union A$ and $(A \union B) \union C = A \union (B \union C)$.

An operation with the first property, that the ordered pair $(A, B)$ and $(B, A)$ have the same result is called \t{commutative}.
An operation with the second property, that when given three objects the order in which we operate does not matter is called \t{associative}.