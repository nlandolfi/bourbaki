%!name:operations
%!need:functions

\ssection{Why}

We want to \say{combine}
elements of a set.

\ssection{Definition}

Let $A$ be a non-empty set.
An \ct{operation}{operation} on $A$ is
a function from ordered pairs of elements
in the set to the same set.
We use operations to combine the elements.
We operate on pairs.
%An \ct{algebra}{algebra} is a set and
%an operation.
%We call the set the
%\ct{ground set}{groundset}.


\ssubsection{Notation}

Let $A$ be a set and
$g: A \times A \to A$.
We tend to forego the notation
$g(a, b)$ and
write $a\,g\,b$ instead.
We call this
\ct{infix notation}{infixnotation}.

Using lower case latin
letters for elements and for operators
confuses, so we tend to use
special symbols for operations.
For example,
$+$, $-$, $\cdot$, $\circ$, and $\star$.

Let $A$ be a non-empty set
and $+: A \times A \to A$ be
an operation on $A$.
According to the above paragraph,
we tend to write
$a+b$ for the result of applying $+$
to $(a,b)$.
