%!name:operations
%!need:pair_intersections
%!need:functions
%!need:set_symmetric_differences
%!refs:paul_halmos∕naive_set_theory∕section_04
%!refs:paul_halmos/naive_set_theory/section_06
%!refs:paul_halmos∕naive_set_theory∕section_08

\section*{Why}

We have seen several concepts that consist of associating with a pair of sets a third set.
For example, set unions and set intersections

\section*{Definition}

An \t{operation} (or \t{binary operation}, \t{law of composition}) on a set $A$ is a function from $A \times  A$ to $A$.

Roughly speaking, operations \t{combine} (or \t{compose}) elements.
We \t{operate} on ordered pairs.

\subsection*{Example: set operations}

Let $X$ be a set.
Define $g: \powerset{X} \times  \powerset{X} \to \powerset{X}$ by $g(A, B) = A \cup B$.
Then $g$, the function which associates with two sets their union (see \sheetref{pair_unions}{Pair Unions}) is an operation on $\powerset{X}$.
Likewise, define $h : \powerset{X} \times  \powerset{X} \to \powerset{X}$ by $h(A, B) = A \cap  B$.

\subsection*{Naming their properties}

$\cup$ has several nice properties.
For one $A \cup B = B \cup A$ and $(A \cup B) \cup C = A \cup (B \cup C)$.

An operation with the first property, that the ordered pair $(A, B)$ and $(B, A)$ have the same result is called \t{commutative}.
An operation with the second property, that when given three objects the order in which we operate does not matter is called \t{associative}.
$\cap $ shares these properties with $\cup$.

We call the operation of \t{forming unions} the function $(A, B) \mapsto A \cup B$.
We call the operation of \t{forming intersections} the function $(A, B) \mapsto A \cap  B$.
We call the operation of \t{forming symmetric differences} the function $(A, B) \mapsto A \symdiff B$.
Since forming unions commutes and is associative and likewise with forming intersections, forming symmetric differences also commutes.

\section*{Algebras}

Of course, any operation is defined on some set.
For this reason, we define an \t{algebra} as an ordered pair whose first element is a non-empty set and whose second element is an operation on that set.
The \t{ground set} of the algebra is the set on which the operation is defined.
