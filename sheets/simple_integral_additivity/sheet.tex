
%!name:simple_integral_additivity
%!need:simple_integrals

\section*{Why}

If we stack two rectangles, with equal base lengths but different heights, on top of each other, the additivity principle says that the area of the so-formed rectangle is the sum of the areas of the stacked rectangles.
Our definition of integral for simple functions has this property, as it ought to.

\section*{Result}

\begin{proposition}
The simple non-negative integral operator is additive.\end{proposition}
\begin{proof}Let $(X, \mathcal{A} , \mu )$be a measure space.
Let $\SimpleF_+(X)$ denote the non-negative real-valued simple functions on $X$.
Define $s: \SimpleF_+(X) \to [0, \infty]$ by $s(f) = \int f d\mu $ for $f \in \SimpleF_+{X}$.

In this notation, we want to show that$s(f+g) = s(f) + s(g)$ for all $f, g \in \SimpleF_+(X)$.
Toward this end, let $f,g \in \SimpleF_+(X)$ with the simple partitions:
    \[
\set{A_i}_{i = 1}^{m},\set{B_j}_{j = 1}^{n} \subset \mathcal{A} \quad \text{and} \quad\set{a_i}_{i = 1}^{m},\set{b_j}_{j = 1}^{n} \subset [0, \infty].
    \]

We consider the refinement of the two partitions.
TODO: this is why you don't do the unique maximalpartition business.
$\set{A_i \cap  B_j}_{i, j = 1}^{i = m, j = n}$.

%\begin{enumerate}
%  \item
% \item
%  $s(f + g) = s(f) + s(g)$
%  for all $f, g ∈ \SimpleF_+(X)$.
%\end{enumerate}
%From (a) and (b) we obtain that
%\[
%  s\left(αfdμ + βg\right) = α s \left(f\right) + \beta s \left(g\right).
%\]
%for all $α, β ∈ R$
%$f, g ∈ \SimpleF_+(X)$.

First, let$\alpha  \in (0, \infty)$.
Then $\alpha  f \in \SimpleF_+(X)$,with the simple partition$\set{A_n} \subset \mathcal{A} $ and $\set{\alpha  a_n} \subset [0, \infty]$.
  \[
\begin{aligned}
s(\alpha  f) \&= \sum_{i = 1}^{n} \alpha  a_n \mu (A_i)
\&= \alpha  \sum_{i = 1}^{n} a_n \mu (A_i)
\&= \alpha  s(f).
\end{aligned}
  \]

If $\alpha  = 0$, then $\alpha  f$ is uniformly zero; it is the non-negative simple with partition $\set{X}$ and $\set{0}$.
Regardless of the measure of $X$, this non-negative simple function is zero.
Recall that we define $0\cdot \infty = \infty\cdot 0 = 0$.\end{proof}