%!name:rectangular_functions
%!need:characteristic_functions
%!need:real_functions
%!need:rectangles

\ssection{Why}

We represent rectangles by functions.

\ssection{Definition}

A \t{rectangular function} corresponds to a characterstic function of an interval.
It represents a rectangle whose width is the interval and whose height is one.
%!TODO: next edition, fix tthis
% If the base set is the real
% numbers and the subset is
% an interval, then the
% characteristic function
% is a rectangle with height
% one and the width of the interval.
% Put in indicator random variables
%We call
%the
%\ct{characteristic function}{characteristic}
%of a subset of some set
%the
%\ct{indicator function}{indicator function}.

\ssubsection{Notation}

Let $A$ be a non-empty
set and $B \subset A$.
Recall that we denote the characteristic function of $B$ by $\chi_{B}$.

Now suppose that $A \subset \R$.
If we embed $\set{0, 1} = 2 \in \N$ in $\R$ by associating $0$ to $0_{\R}$ and $1$ to $1_{\R}$
then every characteristic function is identifiable with a function from $\R$ to $\R$.

In particular, notice that if $B$ is an interval
and
$\alpha$ is a real number
then $\alpha \chi_{B}$ is
a rectangle with
height $\alpha$.

\blankpage
