
\section*{Why}

We represent rectangles by functions.

\section*{Definition}

A \t{rectangular function} corresponds to a characterstic function of an interval.
It represents a rectangle whose width is the interval and whose height is one.
% <div data-littype='run'> ❲%!TODO: next edition, fix tthis❳ </div>
% <div data-littype='run'> ❲% If the base set is the real❳ </div>
% <div data-littype='run'> ❲% numbers and the subset is❳ </div>
% <div data-littype='run'> ❲% an interval, then the❳ </div>
% <div data-littype='run'> ❲% characteristic function❳ </div>
% <div data-littype='run'> ❲% is a rectangle with height❳ </div>
% <div data-littype='run'> ❲% one and the width of the interval.❳ </div>
% <div data-littype='run'> ❲% Put in indicator random variables❳ </div>
% <div data-littype='run'> ❲%We call❳ </div>
% <div data-littype='run'> ❲%the❳ </div>
% <div data-littype='run'> ❲%\ct{characteristic function}{characteristic}❳ </div>
% <div data-littype='run'> ❲%of a subset of some set❳ </div>
% <div data-littype='run'> ❲%the❳ </div>
% <div data-littype='run'> ❲%\ct{indicator function}{indicator function}.❳ </div>
%  


\subsection*{Notation}

Let $A$ be a non-empty set and $B \subset A$.
Recall that we denote the characteristic function of $B$ by $\chi _{B}$.

Now suppose that $A \subset \R $.
If we embed $\set{0, 1} = 2 \in \N  $ in $\R $ by associating $0$ to $0_{\R }$ and $1$ to $1_{\R }$ then every characteristic function is identified with a function from $\R $ to $\R $.

In particular, notice that if $B$ is an interval and $\alpha $ is a real number then $\alpha  \chi _{B}$ is a rectangle with height $\alpha $.

\blankpage