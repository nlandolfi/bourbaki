
\section*{Why}

We generalize the algebraic structure of addition and multiplication over the rationals.

\section*{Definition}

A \t{field} is a ring $(R, +, \cdot  )$ for which $\cdot  $ is commutative (i.e., $ab = ba$ for all $a, b \in R$) and $\cdot  $ has inverses for all elements except $0$.
In this case, we refer to \t{field addition} and \t{field multiplication}.

\subsection*{Notation}

Since our guiding example is the set of rationals $\Q  $ with addition and multiplication defined in the usual manner, and we use a bold font for $\Q  $, we tend to denote an arbitrary field by $\F  $, a mnemonic for ``field.''

\section*{Field operations}

Along with field addition and field multiplication, we call the function which takes an element of a field to its additive inverse and the function which takes an element of a field to its multiplicative inverse the \t{field operations}.
\blankpage

%macros.tex
%\newcommand{\F}{\mathbfsf{F}}
