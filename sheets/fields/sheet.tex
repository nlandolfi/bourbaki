%!name:fields
%!need:groups
%!need:rational_numbers

\ssection{Why}

We generalize the algebraic structure of addition and multiplication over the rationals.

\ssection{Definition}

A \ct{field}{ring} is two algebras over
the same ground set with:
(1) both algebras are commutative groups
(2) the operation of the second algebra
distributes over the operation of the first
algebra.

We call the operation of the first algebra
\t{field addition}.
We call the operation of the second algebra
\t{field multiplication}.

\ssubsection{Notation}

We tend to denote an arbitrary field by $\F$, a mnemonic for \say{field.}

\section{Examples}

Of course, $\Q$ with the usual addition (see \sheetref{rational_sums}{Rational Sums}) and multiplication (see \sheetref{rational_products}{Rational Products}) and the inverse elements (see \sheetref{rational_addtive_inverses}{Rational Additive Inverse}) and \sheetref{ratioanl_multiplicative_inverses}{Ratioanl Multiplicative Invereses}) is a field.

\begin{proposition}
  $\Q$ is a field.
\end{proposition}
