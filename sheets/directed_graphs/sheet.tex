%!name:directed_graphs
%!need:undirected_graphs

\ssection{Why}

We want to visualize relations.

\ssection{Definition}

An \t{directed graph} is a finite nonempty set and a set of ordered pair its elements with distinct coordinates.
We call the elements of the first set the \t{vertices} of the graph and the elements of the second set the \t{edges}.
We say an edge is \t{from} its first coordinate \t{to} its second coordinate.

We say that an edge is \t{incident} to its first and second coordinate.
We call the first coordinate a \t{parent} of the second; and we call and the second coordinate a \t{child} of the first.
The \t{child set} of a vertex is the set of its child vertices and similarly for the \t{parent set}; we refer to these sets as the \t{children} and \t{parents} of the vertex, respectively.
A directed graph is \t{complete} if every vertex is both a child and parent of every other vertex.

%The \t{skeleton} of a directed graph is the undirected graph whose vertex set is the vertex set of the directed set and whose edges consist of all pairs which appear as ordered pairs in the directed graph.
