
\section*{Why}

We want to visualize relations.

\section*{Definition}

A \t{directed graph} (or \t{digraph}, \t{graph}) is a pair $(V, E)$ in which $V$ is a nonempty set and $E$ is a subset of $V \times  V$.
In other words, $E$ is a relation on $V$.
We call the elements of $V$ \t{vertices} and the elements of $E$ \t{edges} (or \t{arcs}).

\subsection*{Example}

For example, define $V$ and $E$ by
\[
V = \{ 1,2,3,4 \}
\quad \text{ and } \quad
E = \{ (1,2),(1,3),(2,4),(3,4) \}
\]
It is worth drawing this graph.

\subsection*{Edge and vertex terminology}

Let $(v, w) \in E$.
We say that $(v, w)$ is an edge \t{from} $v$ \t{to} $w$, and that it is an \t{outgoing edge} of $v$ and an \t{incoming edge} of $w$.
We call $v$ a \t{parent} of $w$ and we call $w$ a \t{child} of $v$.
We say that the edge $(v, w)$ is \t{incident} to $v$ and $w$.

The \t{child set} of a vertex is the set of its child vertices and similarly for the \t{parent set}; we refer to these sets as the \t{children} and \t{parents} of the vertex, respectively.
The \t{indegree} of a vertex is number parents it has and the \t{outdegree} is the number of children it has.

The \t{parents}, \t{children}, and \t{neighbors} of a \textit{set} $A$ of vertices each defined to be the set of vertices which \textit{are not} in the set but \textit{are} the parents, children or neighbors of some vertex in the set defined.

A vertex is a \t{source} vertex if it only has outgoing edges (i.e., is the child of no vertex its parent set is empty) and a vertex is a \t{sink} if it only has incoming edges (i.e., is the parent of no vertex).

A directed graph is \t{complete} if every vertex is both a child and parent of every other vertex.

\subsection*{Notation}

We denote by $\pa: V \to \powerset{V}$ and $\ch: V \to \powerset{V}$ the functions associating to each vertex its set of parents and set of children, respectively.
As usual, we denote the parents of vertex $v$ by $\pa_v$ and the children by $\ch_v$.

\subsection*{Self-loops}

If $x$ is a vertex, and $(x,x)$ is an edge, we call such an edge a \t{self-loop} (or just \t{loop}).
Many authorities exclude self-loops in their definition of directed graphs, but we allow them.
To make the distinction, we call a graph with no \t{loops} \t{simple} (a \t{simple graph}).

%<div data-littype='paragraph'>
%  <div data-littype='run'> \ssection{Visualization} </div>
%</div>
%<div data-littype='paragraph'>
%  <div data-littype='run'> ❲% We visualize a graph by drawing a point for each vertex.❳ </div>
%  <div data-littype='run'> ❲% If the pair of two vertices $u$ and $v$ is an edge we draw a line segment whose endpoints are the points corresponding to the vertices.❳ </div>
%  <div data-littype='run'> </div>
%</div>
