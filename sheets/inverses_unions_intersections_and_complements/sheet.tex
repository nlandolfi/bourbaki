
%!name:inverses_unions_intersections_and_complements
%!need:family_unions_and_intersections
% needed elsewhere
%%!need:set_complements
%!refs:paul_halmos/naive_set_theory/section_10

\section*{Why}

The inverse of a function interacts nicely with family unions, family intersections and complements.

\section*{Results}

Let $f: X \to Y$.
Throughout this sheet, let $f^{-1}: \powerset{Y} \to \powerset{X}$.
And take $\set{B_i}$ to be a family of subsets of $Y$.\footnote{The proofs of the following will appear in future editions.}

\begin{proposition}
$f^{-1}(\cup_i B_i) = \cup_{i} f^{-1}(B_i)$\end{proposition}
\begin{proposition}
$f^{-1}(\cup_i B_i) = \cap_{i} f^{-1}(B_i)$\end{proposition}
\begin{proposition}
$f^{-1}(Y - B) = X - f^{-1}(B)$\end{proposition}
\section*{Properties for function image}

Notice that $f(\cup_i A_i) = \cup_i f(A_i)$ but not for interesctions.
Nor is there a similar correspondence for complements.
There are some relations, which we list below.\footnote{Accounts of these facts will appear in future editions.}
% Notice that $f(x) = f(x) \iff x = y$ means that $f$ is one-to-one.


\begin{proposition}
$f(A \cap B) = f(A) \cap f(B)$ if and only if $f$ is one-to-one.\end{proposition}
\begin{proposition}
For all $A \subset X$, $f(X - A) = Y - f(A)$ if and only if $f$ is one-to-one.\end{proposition}
\begin{proposition}
For all $A \subset X$, $Y - f(A) \subset f(X - A)$ if and only if $f$ is onto.\end{proposition}
\blankpage