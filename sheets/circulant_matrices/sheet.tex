%!name:circulant_matrices
%!need:real_matrices

\ssection{Why}\footnote{Future sheets will include. These matrices arise in practice and each has the same eigenvectors.}

\section{Definition}

Define $S \in \R^{d \times d}$ by

\[
  S = \bmat{
    0 & 0 & \dots & 0 & 1 \\
    0 & 1 & \cdots & 0 & 0 \\
    0 & 0 & \ddots & 0 & 0 \\
    0 & 0 & \cdots & 1 & 0
  }
\]
For a vector $x \in \R^d$ the \t{down shift} of $x$ is $Sx$.

Let $A \in \R^{d \times d}$ be a matrix with columsn $a_1, \dots, a_d$.
$A$ is a \t{circulant matrix} if $a_1 = Sa_d$, $a2 = Sa_2$, and $a_i = A_{i -1}$ for $i = 2, \dots, d$.

\ssubsection{Example}

For example, the matrix
\[
  \bmat{
    1 & 4 & 3 & 2 \\
    2 & 1 & 4 & 3 \\
    3 & 2 & 1 & 4 \\
    4 & 3 & 2 & 1 \\
  }
\]
is a circulant matrix.

\ssection{Characterization}

A matrix $C \in \R^{d \times d}$ is circulant if and only if there exists $c_0, \dots, c_{d-1}$ so that
\[
  C = c_0I + c_1 S + c_2S^2 + \cdots + c_{n-1}S^{n-1}.
\]

\ssection{Properties}

The sum and product of any two circulant matrix is circulant.
In other words, the circulant matrices with the usual matrix addition and multiplication form a commutative ring.

\blankpage
