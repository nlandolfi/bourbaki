
%!name:digital_integers
%!need:digital_naturals
%!need:integer_numbers

\section*{Why}

We want to associate elements of $\Z $ with bitstrings for use on digital computers.\footnote{Future editions will discuss digital computers.}

\section*{Definition}

A \t{digital integer} is a bit-string.
The set of \t{$d$-bit integers} is the set of length-$d$ bit strings $\set{0, 1}^d$.
For example, the set of 8-bit digital integers is the set $\set{0, 1}^8$.

\section*{Correspondence with $\Z $}

The bit string $x \in \set{0, 1}^d$ corresponds to the integer $\sum_{i = 1}^{d} x_i 2^i$.

\section*{Notation}

We denote the set of 8-bit (16-bit, 32-bit, 64-bit) integers by \c{int64} (\c{int8}, \c{int16}, \c{int32}).

It is easy to embed $x$ in \c{int8} by considering $x'$ in \c{int16} defined by
    \[
x' = (x_1, x_2, x_3, x_4, x_5, x_6, x_7, x_8, 0, 0, 0, 0, 0, 0, 0, 0)
    \]
In other words, we associate an 8-bit integer with a 16-bit integer.
Naturally, we associate the integers with bit strings.


\section*{Signed integers}
\footnote{Future editions will include an account of signed integers, or will place this in another sheet.}
\blankpage