
\section*{Definition}

The \t{translate} of $S \subset \R ^n$ \t{by} the vector $a \in \R ^n$ is the set
\[
\Set*{z \in \R ^n}{\exists x \in S \text{ such that } z = x + a}.
\]

\subsection*{Notation}

We often use the abbreviated notation $S + a$ for the translate of $S$ by $a$.
It is sometimes also convenient to extend set-builder notation and write
\[
S + a = \Set{x + a}{x \in M}.
\]
The right hand side is slick notation for the definition given above.

\section*{Sums and differences}

The \t{sum} (or \t{Minkowski sum}) of two sets $S, T \subset \R ^n$ is the set
\[
\Set{z \in \R ^n}{(\exists x \in S)(\exists y \in T)(z = x + y)}.
\]
Likewise, the \t{difference} (or \t{Minkowski difference}) of two sets $S, T \subset \R ^n$ is the set
\[
\Set{z \in \R ^n}{(\exists x \in S)(\exists y \in T)(z = x - y)}.
\]

\subsection*{Notation}

We denote the sum of $S$ and $T$ by $S + T$, and the difference by $S - T$.\footnote{This second notation unfortunately conflicts with our notation for set differences.
Future editions will correct.}
We often use the slick notation
\[
\Set{x + y}{x \in S, y \in T} \text{ and } \Set{x - y}{x \in S, y \in T},
\]
for these two sets.
Notice that in this notation
\[
\set{a} + B = a + B
\]

\section*{Scaled sets}

Given a set $A \subset \R ^n$ and a $\lambda  \in \R $, the set which is $A$ \t{scaled by} (or \t{scaled set}, \t{scaling}) is
\[
\Set{z \in \R ^n}{(\exists x \in A)(z = \lambda x)}
\]
We often denote this set by $\lambda A$.
As before, we often use the slick notation
\[
\lambda A = \Set{\lambda a}{a \in A}
\]
The set $(-1)A$ is denoted $-A$

\subsection*{Homothetic sets}

A set $A$ is \t{homothetic} to a set $B$ if there is $x \in \R ^n$ and $\lambda  \neq 0$ so taht
\[
A = x + \lambda B
\]
If $\lambda  > 0$, $A$ is \t{positively homothetic} to $B$.

\blankpage