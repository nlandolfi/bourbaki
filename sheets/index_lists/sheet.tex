
\section*{Why}

We want language and notation for selecting some of the entries (possibly, with reordering---i.e. permuting) from a list.

\section*{Definition}

An \t{index list} of \t{order} $n$ and \t{length} $r \leq n$ is a list of \textit{distinct} elements of $\upto{n}$.
Its \t{$i$-index} is the $i$th coordinate, where $i = 1, \dots , r$.

\subsection*{Examples}

Here are some index lists of order 5: $(1,2,3)$, $(3,2,1)$, $(4,5,1)$, $(5,4,3,2,1)$, $(3,)$.
These have lengths 3, 3, 3, 7 and 1, respectively.
The $3$-index of the first is 3, and of the second is 1.

\section*{Induced sublist}

The \t{sublist} of an length-$n$ list $x$ \t{induced} by a length-$r$ index list $\alpha $ is the length-$r$ list $y$ whose \textit{$i$th} entry is the value $x_{\alpha _i}$.
In other words,
\[
y_i = x_{\alpha _i}
\]
For example, define $x = (6, 4, 5, 3, 9)$.
The sublists associated with the example index lists above are $(6, 4, 5)$, $(5, 4, 6)$, $(3, 9, 6)$, $(9, 3, 5, 4, 6)$ and $(5,)$.

For a particular case, the third holds because
\[
(3, 9, 6) = (x_4, x_5, x_1)= (x_{\alpha _1}, x_{\alpha _2}, x_{\alpha _3})
\]

\subsection*{Notation}

We denote the induced sublist of list $x$ induced by index list $\alpha $ by $x_\alpha $.
This is a slight abuse of notation, since we have so far defined a list with a subscript symbol mean the subscript-symbol term of that list.
This ambiguity is avoided in our discussion if we keep in mind the types of the objects.

\section*{Index sets}

An \t{index set} $S \subset {1,\dots ,n}$ can be associated with an index list in a natural way.
It corresponds to the length-$\num{S}$ index list which has the elements of $S$ in their \sheetref{natural_order}{natural order}.
We denote the induced sublist by $x_S$.
