%!name:identities
%!need:names

\section*{Why}

We can give the same object two different names.

\section*{Definition}

An object \t{is} itself.
If the object denoted by one name is the same as the object denoted by a second name, then we say that the two names are \t{equal}.
The object associated with a \t{name} is the \t{identity} of the name.

Let $A$ denote an object and let $B$ denote an object.
Here we are using $A$ and $B$ as placeholders.
They are names for objects, but we do not know---or care---which objects.
We say \say{$A$ equals $B$} as a shorthand for \say{the object denoted by $A$ is the same as the object denoted by $B$.}
In other words, $A$ and $B$ are two names for the same object.

\section*{Symmetry}

Let $A$ denote an object and let $B$ denote an object.
\say{$A$ equals $B$} means the same as \say{$B$ equals $A$}.
The identity of the names is not dependent on the order in which the names are given.
We call this the \t{symmetry of identity}.
It means we can switch the spots of $A$ and $B$ and say the same thing.
In other words, there are two ways to make the statement.
  %  <div data-littype='run'> ❲% If we switch the spots of $A$ and $B$.❳ </div>
  %  <div data-littype='run'> ❲% What it means❳ </div>
  %  <div data-littype='run'> ❲% \say{the object denoted by $A$ is the same as the object denoted by $B$} and \say{the object denoted by $B$ is the same as the object denoted by $A$}.❳ </div>


\section*{Reflexivity}

Let $A$ denote an object.
Since every object is the same as itself, the object denoted by $A$ is the same as the object denoted by $A$.
We say \say{$A$ equals $A$}.
In other words, every name equals itself.
This fact is called the \t{reflexivity of identity}.
A name is equal to itself because an object is itself.


%% \ssubsection{Notation}
%%
%% Denote an object by $a$ and another object $b$.
%% We denote that the object $a$ is $b$ by $a = b$.
%% We denote that the object named $a$ and the object named $b$ refer to the same object by $a = b$.
%% We read this notation aloud as: \say{a equals b} or \say{the object denoted by a is the same as object denoted by b}.
%% We denote that the object $a$ and $b$ refer to different objects by $a \neq b$.
%% We read this aloud as \say{the object denoted by a is not the object denoted by b} or \say{a does not equal b}.
%%
%% %We may also read the notation
%% %$a = b$ aloud as \say{a
%% %equals b}.
%% Other English readings of $a = b$ include: \say{a is the same as b},
%% \say{a is equivalent to b}, \say{a refers to the same object as b.}
%%
%% \ssubsection{Properties}
%%
%% Given an object $a$, $a = a$ is true.
%% We say that equivalence is \t{reflexive}.
%% Given objects $a$ and $b$, $a = b$ implues $b = a$.
%% We say that equality is \t{symmetric}.
%% Given objects $a$, $b$, and $c$, $a = b$ and $b = c$ implies $a = c$.
%% We say that equality is \t{transitive}.
%% \blankpage
%% %!name:identity
%% %!need:objects
%%
%% \ssection{Why}
%%
%% We can give the same object two different names.
%%
%% \ssection{Definition}
%%
%% An object \ct{is}{} itself.
%% If the object that two names refer to is the same, then we say that the first name \t{equals} the second name.
%%
%% \ssubsection{Notation}
%%
%% We denote that the object named $a$ and the object named $b$ refer to the same object by $a = b$.
%% We read this notation aloud as: \say{a is b} or \say{a equals b}.
%% We denote that the object $a$ and $b$ refer to different objects by $a \neq b$.
%% We read this aloud as \say{a is not b} or \say{a does not equal b}.
%%
%% %We may also read the notation
%% %$a = b$ aloud as \say{a
%% %equals b}.
%% Other English readings of $a = b$ include: \say{a is the same as b},
%% \say{a is equivalent to b}, \say{a refers to the same object as b.}
%%
%% \ssubsection{Properties}
%%
%% Given an object $a$, $a = a$ is true.
%% We say that equivalence is \t{reflexive}.
%% Given objects $a$ and $b$, $a = b$ implies $b = a$.
%% We say that equality is \t{symmetric}.
%% Given objects $a$, $b$, and $c$, $a = b$ and $b = c$ implies $a = c$.
%% We say that equality is \t{transitive}.
%%
%% \blankpage
%% %!name:equations
%% %!need:identity
%%
%% \ssection{Why}
%%
%% \ssection{Definition}
%%
%% An \ct{equation}{equation} is
%% a statement of equality between
%% two objects.
%%
%% For example, if we have an object $a$
%% and an object $b$, then an equation
%% is
%% \[
%%   a = b.
%% \]
%%
%% Equations are useful when we do not
%% know the particular identity of an
%% object, but only that it belongs to
%% some set.
%% In this case, we call the
%% equation \ct{indeterminate}{}
