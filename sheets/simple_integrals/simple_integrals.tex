\sinput{../sheet.tex}
\sbasic

\sinput{../simple_integrals/macros.tex}

\sstart

\stitle{Simple Integrals}

\ssection{Why}


We want to define area
under a real function.
We begin with functions
whose area under the curve
is self-evident.

\ssection{Definition}

Consider a measure space.
The characteristic function
of any measurable set
is measurable.
A simple function is measurable
if and only if
each element of
its simple partition
is measurable.


The
\ct{integral}{simpleintegral}
of a measurable non-negative simple function
is the sum of the products
of the measure of each piece
with the value of the function on that piece.
For example, the integral
of a measurable characteristic function
of a subset is the
measure of that subset.

The
\ct{integral operator}{simpleintegraloperator}
is the real-valued function
which associates
each measurable non-negative simple
function with its integral.
The simple integral is non-negative,
so the integral operator is
a non-negative function.

\ssubsection{Notation}

Let $(X, \mathcal{A}, \mu)$
be a measure space.
Let
$R$ be the
set of real numbers.

%Let $A \in \mathcal{A}$.
%We denote the integral
%of $\chi_{A}$ with
%respect to measure $\mu$ by
%$\int \chi_{A} d \mu$.
%We defined:
%\[
%  \int \chi_{A} = \mu(A).
%\]

Let $f: X \to R$ be
a measurable simple function.
So there exist
$A_1, \dots, A_n \in \mathcal{A}$
and $a_1, \dots, a_n \in R$ with:
\[
  f = \sum_{i = 1}^{n} a_i \chi_{A_i}.
\]

We denote the integral of
$f$
with respect to measure $\mu$
by $\int f d\mu$.
We defined:
\[
  \int f d\mu = \sum_{i = 1}^{n} a_i \mu(A_i).
\]


\strats
