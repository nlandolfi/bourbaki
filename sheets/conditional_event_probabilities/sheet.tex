%!name:conditional_event_probabilities
%!need:event_probabilities

\ssection{Why}

Given that we know that one event has occured, we want language for what the new probabilities should be.
  \ifhmode\unskip\fi\footnote{
Future editions will improve.
  }

\ssection{Definition}
Let $\mathbfsf{P} : \powerset{\Omega } \to \R $ be a finite probability measure.
Let $A, B \subset \Omega $ and $\mathbfsf{P} (B) \neq 0$.
The \t{conditional probability} of $A$ \t{given} $B$ is fraction of the probability of $A \cap B$ over the probability of $B$.

\ssubsection{Notation}
In a slightly slippery but universally standard notation, we denote the conditional probability of $A$ given $B$ by $\mathbfsf{P} (A \mid B)$.
In other words, we define
  \[
\mathbfsf{P} (A \mid B) = \frac{\mathbfsf{P} (A \cap B)}{\mathbfsf{P} (B)},
  \]
whenever $A, B \subset \Omega $ and $\mathbfsf{P} (B) \neq 0$.

\ssubsection{Induced conditional distribution}
Conditioning on an event $B$ induces a new distribution on the set of outcomes.
For $\mathbfsf{P} _p$, define $q: \Omega  \to \R $ by
  \[
q(\omega ) = \begin{cases}
\frac{p(\omega )}{\mathbfsf{P} (B)} & \text{ if } \omega  \in B \\
0 & \text{ otherwise. } \\
\end{cases}
  \]
In this case $\mathbfsf{P} _q(A) = \mathbfsf{P} _p(A \mid B)$.
We call $q$ the \t{conditional distribution} induced by \t{conditioning on} the event $B$.

\ssection{Total probability}
Using the notation $\mathbfsf{P} (\cdot \mid \cdot)$, we can express the law of total probability for $\mathbfsf{P} (B)$, $B \subset \Omega $, as
  \[
\textstyle
\mathbfsf{P} (B) = \sum_{i = 1}^{n} \mathbfsf{P} (A_i)\mathbfsf{P} (B \mid A_i),
  \]
where $A_1, \dots, A_n$ partition $\Omega $.
