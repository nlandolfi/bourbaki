
%!name:absolute_value
%!need:interval_length

\section*{Why}

We want a notion of distance between elements of the real line.

\section*{Definition}


% We define a function which maps the real numbers onto the nonnegative real numbers.

The \t{absolute value} of a real number is the greater of itself and its additive inverse.
In other words, if $x$ is positive, then the absolute value of $x$ is $x$.
If $x$ is negative, then the absolute value of $x$ is $-x$ (a positive real number).

\subsection*{Notation}

We denote the absolute value of a real number $x \in \R $ by $\abs{x}$.
%Thus $\abs{·}: 𝗥 → 𝗥$ can be viewed as a real-valued function on the real numbers which is nonnegative.


\subsection*{Distance}

The absolute value can be interpreted as the distance between the point corresponding to the real number and the point corresponding to 0.
We can generalize this idea.
Consider $x, y \in \R $.
If $x > y$, then $x - y > 0$ and so the distance between the corresponding points is $x - y$.
If $x < y$ then $y - x > 0$, and so the distance is $y - x$.

The observation is that $\abs{-x} = \abs{x}$.
So
\[
\abs{y - x} = \abs{-(x - y)} = \abs{x-y}.
\]
So if we just care about the distance between the points corresponding to $y$ and $x$, we can consider $\abs{x - y}$, without regard for their order.
In other words, the function $(x, y) \mapsto \abs{x - y}$ is symmetric in $x$ and $y$.

\blankpage