%!name:real_length_impossible
%!need:real_numbers
%!need:subsets
%!need:set_operations

\ssection{Why}

Given a subset of the real
line, what is its length?

\ssection{Background}

Let $a, b \in R$ with $a \leq b$.
The \ct{length}{}
of the closed interval
of the real numbers
$[a, b]$ is $b - a$.
The length is non-negative.

A family
$\set{A_{\alpha}}_{\alpha \in I}$
is
\ct{disjoint}{}
if for $\alpha, \beta \in I$,
$\alpha \neq \beta$, then
$A_\alpha \intersect A_\beta = \emptyset$.
A set $A$ can be
\ct{partioned}{}
into a family
if there exists a disjoint
family whose union is $A$.
A set $A \subset R$ is
\ct{simple}{} if
it can be partitioned
into a countable family
whose members are closed intervals.
The above discussion suggests that
we should define the length of a
simple set as the sum of the lengths
of sets which parition it.

The above discussion suggests that
if we wish to define a function
$\text{length}: 2^{R} \to R \union \set{-\infty, \infty}$,
we should ask that
(1) $\text{length}(A) \geq 0$,
(2) $\text{length}([a, b]) = b-a$,
(3) for disjoint closed intervals
$\set{A_n}_{n \in N}$,
$\text{length}(A_i) = \sum_i \text{length}(A_i)$,
and
(4) for all $A \subset R$ and $a \in R$,
$\text{length}(A+x) = \text{length}(A)$.

\ssection{Converse}

Define the equivalence relation
$\sim$ on $R$ by
by $x \sim y$ if $x \sim y \in Q$

\ssubsection{Notation}

Let $A$ be a set and $\mathcal{A} \subset \powerset{A}$.
We denote the subset algebra of $A$ and $\mathcal{A}$
by $\tuple{A, \mathcal{A}}$, read aloud as \say{A, script
  A.}

\ssection{Properties}

\begin{prop}
  For any set $A$, $2^{A}$ is a sigma algebra.
\end{prop}

\begin{prop}
  The intersection of a family of sigma algebras is
  a sigma algebra.
  \label{sigma_algebra:sigmaintersection}
\end{prop}

\ssection{Generation}

\begin{prop}
  Let $A$ a set and $\mathcal{B}$ a set of subsets.
  There is a unique smallest sigma algebra
  $(A, \mathcal{A})$ with
  $\mathcal{B} \subset \mathcal{A}$.
\end{prop}

We call the unique smallest sigma algebra
containing $B$ the
\ct{generated sigma algebra}{}
of $B$.
