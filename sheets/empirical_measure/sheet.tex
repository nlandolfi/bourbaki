
\section*{Why}

There is a natural probability measure on a measurable space to associate with a dataset from the base set of that space.

\section*{Definition}

The \t{empirical measure} for a dataset in some measurable space is the measure which associates to each event the proportion of the records which are elements of that event.

\subsection*{Notation}

Let $(a^1, \dots , a^n)$ be a dataset in a measurable space $(A, \mathcal{A} )$.
Let $P: \powerset{A} \to [0, 1]$ be the probability measure that assigns to each set $B \subset A$ the number
\[
P(B) = \frac{1}{n} \num{\Set{k \in \set{1, \dots , n}}{a^k \in B}}.
\]
Then $P$ is the empirical measure.

\blankpage