
\section*{Why}

When is a linear transformation \t{onto}?
In other words, when is the range the whole space?
This question is a bit more invovled, but we will start with observing in this sheet that the range of a linear map happens to be a subspace.

\section*{Definition}

For a \textit{linear} transformation $T \in \mathcal{L} (V, W)$, we refer to $\range(T)$ as the \t{range space} (or \t{image space}) of $T$.

% this is 3.19 of axler 

\begin{proposition}
Suppose $T \in \mathcal{L} (V, W)$.
Then $\range T$ is a subspace of $W$.
\end{proposition}

\begin{proof}We verify that $\range(T)$ contains $0$ and is closed under vector addition and scalar multiplication.
Clearly $T(0) = 0$, so $0 \in \range T$.
Next, suppose $w_1, w_2 \in \range T \subset W$
So there exists $v_1, v_2 \in V$ so that
\[
Tv_1 = w_1 \text{ and } Tv_2 = w_2
\]
We conclude $w_1 + w_2 = Tv_1 + Tv_2 = T(v_1 + v_2)$.
So $w_1 + w_2 \in \range T$.
Likeiwse, if $w \in \range T$, then there exists $v \in V$ such that $w = Tv$.
So then $\lambda  w = \lambda  Tv = T(\lambda v)$, and so $\lambda w \in \range T$.\end{proof}
\subsection*{Examples}

To come.
\blankpage