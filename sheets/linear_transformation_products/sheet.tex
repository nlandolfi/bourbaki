
\section*{Why}

We can consider function composition for linear maps?

\section*{Definition}

Given vector spaces $U, V, W$ over the same field $\F $ and linear maps $T \in \mathcal{L} (U, V)$ and $S \in \mathcal{L} (V, W)$, the \t{product} of $S$ and $T$ is the linear map $R \in \mathcal{L} (U, W)$ defined by
\[
R(u) = S(T(u)) \quad \text{for all } u \in U
\]
(Prove that $R$ so defined is linear).
In other words, the product is $S \circ T$.

This definition only makes sense if $T$ maps into the domain of $S$.
We often say that the maps are \t{conforming} in this case.

\subsection*{Notation}

Often the product is denoted $ST$ (instead of $S \circ T$).

\section*{Algebraic properties}

\begin{proposition}[associativity]
Suppose $T_1, T_2, T_3$ are three linear maps so that conforming for $T_1T_2T_3$.
Then
\[
(T_1T_2)T_3 = T_1(T_2T_3)
\]
\end{proposition}

\subsection*{Not commutative}

\section*{Image of zero}

\begin{proposition}
Suppose $T$ is a linear map from $V$ to $W$.
Then $T(0) = 0$.
\end{proposition}

\blankpage