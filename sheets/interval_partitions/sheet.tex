%!name:interval_partitions
%!need:intervals

\ssection{Why}

We partition
a real interval
into interval pieces.

\ssection{Definition}

An
\ct{interval partition}{intervalpartition}
is a finite partition of a closed real interval.

An interval partition is
\ct{regular}{regularintervalpartition}
if all pieces except the largest are
closed on the left and open
on the right and the largest
is closed.

Any regular interval partition with
$n-1$ elements
can be represented by
$n+1$ real numbers: the
endpoints of each interval.
We call these
the
\ct{cut points}{intervalpartitioncutpoints}
of the interval partition.

\ssubsection{Notation}

Let $R$ denote the set of
real numbers.
Let $[a, b]$ a closed
interval in $R$ with
endpoints $a, b \in R$.

Consider a regular partition.
of $[a,b]$ with $n-1$ pieces.
We can identify its
cut points:
\[
  a = a_1 < a_2 < \dots a_{n-1} < a_n = b.
\]
The pieces of the partition
are:
\[
[a_1, a_2),
[a_2, a_3),
\dots,
[a_{n-2}, a_{n-1})
[a_{n-1}, a_n].
\]
