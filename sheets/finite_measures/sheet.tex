%!name:finite_measures
%!need:measures

\ssection{Why}

We want finite measures.\footnote{This why is likely to be expanded on in future editions. It may be that we need to reference probability theory. It may suffice to note that the Lebesgue measure on $[0, 1]$ is finite, and so such measures exist.}

\ssection{Definition}

A measurable set is \t{measure-finite} (or when clear that we do not mean finite in the sense of \sheetref{finite_sets}{Finite sets}), a set is \t{finite}) if its measure is a real number (as opposed to the only alternative, $+\infty$).
The measure space itself is \t{measure-finite} if the base set is finite.

A measurable set is \t{sigma-finite} if there exists a sequence of finite measurable sets whose union is the set.\footnote{The justification of this seems distinct from the previous paragraph, and therefore may need its own sheet. The reason for these seems to me to be that the Lebesgue measure is sigma-finite.}
The measure space itself is \t{sigma-finite} if the base set is sigma finite.

\ssubsection{Notation}

Let $(A, \CA, \mu)$ be a measure space.
It is finite if $\mu(A) < +\infty$.

\begin{expl}
Let $(A, \mathcal{A})$ be a measurable space.

The counting measure on $(A, \mathcal{A})$ is
finite if and only if the base set is finite.
It is sigma finite if and only if the base
set is a union of a sequence of finite sets.

If $\mathcal{A} = 2^A$, then the counting
measure is sigma finite if and only if
$A$ is countable.
\end{expl}

\begin{expl}
A point mass measure is finite.
\end{expl}

\begin{expl}
Let $R$ be the set of real numbers.
The Lebesgue measure on
$(R, \mathcal{B}(R))$ is sigma finite.
\end{expl}
