%!name:function_properties
%!need:functions
%!need:set_specification

\ssection{Why}

TODO

\ssection{Definition}

Let $f: A \to B$.
The \definition{image}{image} of a set $C \subset A$
is the set $\Set{f(c) \in B}{c \in C}$.
The \definition{range}{range} of $f$ is the image
of the domain.
The \definition{inverse image}{inverseimage} of a set
$D \subset B$ is the set $\Set{a \in A}{f(a) \in B}$.

The range need not equal the codomain; though it,
like every other image, is a subset of the codomain.
The function \rt{maps}{functionmaps} to domain
\ct{on}{functionon} to the codomain if the
range and codomain are equal;
in this case we call the function \definition{onto}{onto}.
This language suggests that every element of
the codomain is used by $f$.
It means that for each element $b$ of the codomain,
we can find an element $a$ of the domain so that
$f(a) = b$.

An element of the codomain may be the result
of several elements of the domain.
This overlapping, using an element of the
codomain more than once, is a regular occurrence.
If a function is a unique correspondence in that
every domain element has a different result,
we call it \ct{one-to-one}{one-to-one}.
This language is meant to suggest that each
element of the domain corresponds to one and
exactly one element of the codomain, and vice versa.

\ssubsection{Notation}

Let $f: A \to B$.
We denote the image of $C \subset A$ by $f(C)$, read aloud as \say{f of C.}
This notation is overloaded: for $c \in C$, $f(c) \in A$, whereas $f(C) \subset A$.
Read aloud, the two are indistinguishable, so we must be careful to specify whether we mean an element $c$ or a set $C$.
The property that $f$ is onto can be written succintly as $f(A) = B$.
We denote the inverse image of $D \subset B$ by $f^{-1}(D)$, read aloud as \say{f inverse D.}
