%!name:real_plane
%!need:real_order
%!need:lists
%!need:geometry

\section*{Why}

We are constantly thinking of the elements of $\R ^2$ as points of a plane.
  \ifhmode\unskip\fi\footnote{
Future editions will modify this sheet.
  }

\section*{Discussion}

We commonly associate elements of $\R ^2$ with points on a plane. (see \sheetref{geometry}{Geometry}).

\begin{principle}[Line Sets]
Given a plane, there exists a set of its (infinite) lines.
\end{principle}

\begin{principle}[Real Plane Correspondence]
Let $L$ be the set of lines of a plane.
Then $\cup L$ is the set of points of the plane.
There exists a one-to-one correspondence mapping elements of $\cup L$ onto elements of $\R ^2$.
\end{principle}
For this reason, we sometimes call elements of $\R ^2$ \t{points}.
We call the point associated with $(0, 0)$ the \t{origin}.
We call the element of $\R ^2$ which corresponds to a point the \t{coordinates} of the point.

\section*{Visualization}

To visualize the correspondence we draw two perpendicular lines.
We then associate a point of the line with $(0, 0) \in \R ^2$.
We can label it so.
We then pick a unit length.
And proceed as usual.
  \ifhmode\unskip\fi\footnote{
Future editions will expand this.
  }

\begin{figure}[h]
\centering
\vspace{0.5cm}

\includegraphics[width=0.90\textwidth]{./graphics/real_plane.pdf}
\caption{The real plane}
\label{real_plane:figure:real_plane}
\end{figure}

Given that we have identified a plane with $\R ^2$ in this way, we call $(x, y) \in \R ^2$ the \t{coordinates} of the point it corresponds to.
Many authors refer to this identification as a \t{Cartesian coordinate system} (or \t{Rectangular coordinate system}, \t{$x$-$y$ coordinate system}).
