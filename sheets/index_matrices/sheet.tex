
%!name:index_matrices
%!need:index_lists
%!need:matrices
%!need:permutation_matrices
%!need:matrix_transpose

\section*{Why}

The (surprising fact) is that the operation of going from an index list to its induced sublist is \textit{linear}, if the elements of the list are over a \sheetref{fields}{field}, and thus may be viewed as a \sheetref{vectors}{vector space}.

\section*{Definition}

The \t{index matrix} associated with the index sequence $\alpha $ of length $r$ and order $n$ (recall, $r \leq n$) is the $r \times  n$ matrix whose \textit{$i,j$th} entry is 1 if the sequence's \textit{$i$th} coordinate is $j$, and $0$ otherwise.

\subsection*{Examples}

Here are some order-5 index lists: $(1,2,3)$, $(3,2,1)$, $(4,5,1)$, $(5,4,3,2,1)$, $(3,)$.

The matrices corresponding to these examples are:
    \[
\bmat{
1 & 0 & 0 & 0 & 0 \\
0 & 1 & 0 & 0 & 0 \\
0 & 0 & 1 & 0 & 0 \\
},
\bmat{
0 & 0 & 1 & 0 & 0 \\
0 & 1 & 0 & 0 & 0 \\
1 & 0 & 0 & 0 & 0 \\
},
    \]
for the first two examples, and
    \[
\bmat{
0 & 0 & 0 & 1 & 0 \\
0 & 0 & 0 & 0 & 1 \\
1 & 0 & 0 & 0 & 0 \\
},
\bmat{
0 & 0 & 0 & 0 & 1 \\
0 & 0 & 0 & 1 & 0 \\
0 & 0 & 1 & 0 & 0 \\
0 & 1 & 0 & 0 & 0 \\
1 & 0 & 0 & 0 & 0 \\
},
\bmat{
0 & 0 & 1 & 0 & 0 \\
}
    \]
for the last three.

In this case, we refer to the induced sublist as an induced \t{subvector}.
The value of index matrices is that they give induced subvectors via the usual and familiar operation of matrix multiplication.
The \t{subvector} of an $n$-vector associated with a length-$r$ index sequence is the product of the sequence's $r \times  n$ corresponding index matrix with the $n$-dimensional vector.

For example, define $x = \bmat{6 & 4 & 5 & 3 & 9}^\top $.
Then the subvector of $x$ associated with the index sequence $(3, 2, 1)$ is the vector $\bmat{3 & 9 & 6}^\top  \in \R ^3$, because
    \[
\bmat{3 & 9 & 6 } =
\bmat{
0 & 0 & 0 & 1 & 0 \\
0 & 0 & 0 & 0 & 1 \\
1 & 0 & 0 & 0 & 0 \\
}\bmat{6 \\ 4 \\ 5 \\ 3 \\ 9}
    \]

If $r = n$ then the index matrix is a \sheetref{permutation_matrices}{permutation matrix}.

\subsection*{Notation}

Let $r \leq n$ be natural numbers.
Let $\alpha : \upto{r} \to \upto{n}$ be an index sequence.
Denote the index matrix associated with $\alpha $ by $P_\alpha $.
This matrix $P_\alpha $ is an element of $\N  ^{r \times  n}$ and is defined by
    \[
(P_a)_{ij} = \begin{cases}
1 & j = \alpha (i) \\
0 & \text{otherwise}
\end{cases}
    \]

Let $A$ be a nonempty set and let $x \in A^n$.
then the subvector of $x$ associated with $P_\alpha $ (and so with $\alpha $) is
    \[
P_\alpha  x = \pmat{x_{\alpha (1)}, \dots , x_{\alpha (r)}}
    \]
We denote the product $P_\alpha x$ by $x_{\alpha }$.
%— TODO: call the below: principal submatrix --

We denote the product $P_\alpha  X P_\alpha ^\top $ by $X_{\alpha \alpha }$.
