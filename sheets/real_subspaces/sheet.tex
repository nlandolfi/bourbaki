
\section*{Definition}

A nonempty set $S \subset \R ^n$ is called a \t{subspace} (or \t{linear subspace}, \t{vector subspace}) if
    \begin{enumerate}
      \item $x + y \in S$ for all $x, y \in S$, and
      \item $\alpha x \in S$ for all $\alpha  \in \R $, $x \in S$.
    \end{enumerate}
We say that that $S$ is (1) \t{closed under vector addition} and (2) \t{closed under scalar multiplication}.

\section*{Examples}

The set $S_1 = \R ^n$ is a subspace. In other words, the entire set is a subspace of itself.
The set $S_2 = \set{0}$, consisting of a single point, the origin, is a subspace.
$S_1$ is the biggest subspace.
In other words, if $S'$ is another subspace of $\R ^n$, then $S' \subset S_1$.
If $S$ is a subspace, it is nonempty, so there is $x \in S$, and it is closed under scalar multiplication, so $0\cdot x = 0 \in S$.
In other words, every subspace contains the origin.
So $S_2$ is the smallest subspace, in the sense that if $S'$ is another subspace $S_2 \subset S'$.

The span (see \sheetref{real\_vectors\_span}{Real Vectors Span}) of a set of vectors $v_1, \dots , v_k$ is a subspace.
For two subspaces $S, T \subset \R ^n$, their sum
\[
S + T = \Set{x + y}{x \in S, y \in T}
\]
is a subspace.

\subsection*{Geometric intuition}

Roughly speaking, a subspace $S$ is a flat set which passes through the origin.
In $\R ^2$, the subspaces are the lines.
In $\R ^3$, the lines and the planes.

\blankpage