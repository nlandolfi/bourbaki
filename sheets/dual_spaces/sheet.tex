
%!name:dual_spaces
%!need:continuous_linear_functionals
%!need:complete_normed_spaces

\section*{Why}

Take a vector space, and consider the set of continuous linear functionals on that space.
Given a suitable norm, this space is a complete normed space.

\section*{Defining result}

\begin{proposition}
Let $(V, \norm{\cdot })$ be a normed space.
The set $\dual{V}$ of all continuous linear functionals on $V$ is a complete normed space with respect to pointwise algebraic operations and norm $\dnorm{\cdot }: V \to \R $ defined by
%don't use normal { .. | .. } notation because we
%        have |F(x)| before the | 

  \[
\dnorm{F} = \underset{x \in V, \;\norm{x} \leq 1}{\sup} \abs{F(x)}.
  \]\end{proposition}
\begin{proof}We argue (1) $\dual{V}$ is a vector space, (2) $\dnorm{\cdot }$ is a norm, and (3) $(V, \dnorm{\cdot })$ is complete.\footnote{Future editions will include an account.}\end{proof}
We call $(\dual{V}, \dnorm{\cdot })$ the \t{dual space} (or \t{Banach dual of $V$}).
Notice that $(\dual{V}, \dnorm{\cdot })$ is complete regardless of whether the original normed space $(V,\norm{\cdot })$ is complete.

\subsection*{Basic dual norm property}

Notice that the dual norm satisfies a familiar property.

\begin{proposition}
For any vector $x$ in a normed space $(V, \norm{\cdot })$ and any continuous linear functional $F$ on $E$,
  \[
\abs{F(x)} \leq \dnorm{F}\norm{x}.
  \]\end{proposition}
\begin{proof}If $x = 0$, then $\norm{x} = 0$ and $F(x) = 0$ ($F$ is linear). Otherwise, $x/\norm{x}$ is a unit vector and so
  \[
\dnorm{F} \geq \abs{F(x/\norm{x})} = \frac{\abs{F(x)}}{\norm{x}}.
  \]
where the inequality is from the definition of $\dnorm{\cdot }$ (as a supremum) and the equality follows from the linearity of $F$.\end{proof}