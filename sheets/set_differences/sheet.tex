
\section*{Why}

We consider elements of one set which are not contained in another set.

\section*{Definition}

Let $A$ and $B$ denote sets.
The \t{difference} between $A$ and $B$ is the set $\Set{x \in A}{x \not\in B}$.
In other words, the difference between $A$ and $B$ is the set of all points of $A$ which do not belong to $B$.

It is not necessary that $B \subset A$; the difference is called \t{proper} if $A \supset B$.
This terminology is from that of \sheetref{set_inclusion}{proper subsets}.

\subsection*{Notation}

We denote the difference between $A$ and $B$ by $A \setminus B$.
Other notations used include $-$ or $\sim$.\footnote{The first will conflict with convenient notation for the difference of two sets of vectors.
The second will conflict with convenient notation for equivalence relations .}
% see Halmos Measure Theory Chap 1 for notation - 

% also Halmos Naive Set Theory Section 5 - 

% also Mendelson Introduction to Topology  Ch 1 Section 3 - 

% see Branko Grunbaum Convex Polytopes Chap 1 for notation ~ 


\section*{Properties}

The following are straightforward.\footnote{Accounts will appear in future editions.}

\begin{proposition}
$A \setminus \varnothing = A$
\end{proposition}

\begin{proposition}
$A \setminus A = \varnothing$
\end{proposition}

\blankpage
%macros.tex
%\renewcommand{\setminus}{-}
