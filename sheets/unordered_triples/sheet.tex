%!name:unordered_triples
%!need:pair_unions

\s{Why}

$\set{a} \union \set{b} = \set{a, b}$

\s{Definition}

Let $a$, $b$ and $c$ denote objects.
From the associativity of the pair union
Notice
By the associativty pair unions (see \sheetref{pair_unions}{Pair Unions}) we have that
\[
  (\set{a} \union \set{b}) \union \set(b) = \set{a} \union (\set{b} \union \set{c}).
\]
So we will drop the parantheses, and write $\set{a} \union \set{b} \union \set{b}$.
We call such a set the \t{unordered triple} of $a$, $b$ and $c$.

\ss{Notation}

Such sets are so commonlplace that we denote the unordered triple of $a$, $b$ and $c$ by $\set{a, b, c}$.

\s{Extensions}

Let $d$ denote an object.
It is also the case that we can drop the parantheses from
\[
  (((\set{a} \union \set{b}) \union \set{c}) \set{d})).
\]
We can therefore write $\set{a} \union \set{b} \union \set{c} \union \set{d}$ without ambiguity.
We call this set the \t{unordered quadruple} denote this set by $\set{a, b, c, d}$.

In a similar way we speak of \t{unordered pentuples}, \t{unordered sextuples}, \t{unordered septuples} and so on.
If we have several objects named, we denote the set containing these objects be writing their names in between braces $\{$ and $\}$.
