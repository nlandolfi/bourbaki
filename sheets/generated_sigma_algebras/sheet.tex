%!name:generated_sigma_algebras
%!need:sigma_algebras

\ssection{Why}

A simple way to obtain a sigma algebra, is to start with some sets, and then to add all the sets needed to make the starting set closed under the various operations.

\ssection{Definition}

The \t{generated sigma algebra} for a set of subsets is the smallest sigma algbera containing the set of subsets.
We must prove the existence and uniqueness of this sigma algebra.

\begin{proposition}
\label{intersectsigmaalgebras}
The intersection of a non-empty set of sigma algebras over the same set is a sigma algebra.
\begin{proof}
Given a family of sigma algebras $\set{(A, \mathcal{A} _\alpha }_{\alpha  \in I}$ over some set, define $\mathcal{A}  = \cap_{\alpha  \in I} \mathcal{A} _\alpha $.
  \begin{enumerate}
  \item For all $\alpha  \in I$, $A \in \mathcal{A} _{\alpha }$, thus $A \in \mathcal{A} $; condition (a).
  \item For all $B \in \mathcal{A} $, for all $\alpha  \in I$, $B \in \mathcal{A} _{\alpha }$.
Thus, for all $\alpha  \in I$, $C_{A}(B) \in \mathcal{A} _{\alpha }$.
And so $C_{A}(B) \in \mathcal{A} $; condition (b).
  \item For all sequences $\set{B_n} \subset \mathcal{A} $, $\set{B_n} \subset \mathcal{A} _{\alpha }$ for all $\alpha $.
Thus $\cup_{n} B_n \in \mathcal{A} _{\alpha }$ for all $\alpha $ and so $\cup_{n} B_n \in \mathcal{A} $; condition (c).
  \end{enumerate}
\end{proof}
\end{proposition}

On the other hand, the union of a set of sigma algebras can fail to be a sigma algebra.

\begin{proposition}
If $A$ is a set and $\mathcal{A}  \subset 2^A$, then there is a unique a smallest sigma algebra containing $\mathcal{A} $.

\begin{proof}
We know of one sigma algebra containing $\mathcal{A} $: the power set of $A$.
Thus, the set of sigma algebras containing $\mathcal{A} $ is not empty.
Proposition~\ref{intersectsigmaalgebras} implies the intersection of all such sigma algebras (containing $\mathcal{A} $) is a sigma algebra.
The intersection contains $\mathcal{A} $, and is contained in all other sigma algebras with this property, so is a smallest sigma algebra containing $\mathcal{A} $.
If $\mathcal{B} , \mathcal{C} $ were two smallest sigma algebras, then $\mathcal{B}  \subset \mathcal{C} $ and $\mathcal{C}  \subset \mathcal{B} $, but then $\mathcal{B}  = \mathcal{C} $; thus the smallest sigma algebra is unique.
\end{proof}
\end{proposition}

\ssection{Notation}

Let $A$ be a set and $\mathcal{A}  \subset \powerset{A}$.
We denote the sigma algebra generated by $\mathcal{A} $ by $\sigma (\mathcal{A} )$.
