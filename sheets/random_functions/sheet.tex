%!name:random_functions
%!need:random_vectors

\ssection{Why}

We can generalize random variables and random vectors.

\ssection{Definition}

A \t{random function} (or \t{random process}, \t{stochastic process}\footnote{The word \say{process} is most frequently used when the index set is associated with time.} or \t{random field}\footnote{The word \say{field} is most frequently used when the index set is associated with space.}) is an outcome variable whose codomain is a set of functions.

\ssubsection{Notation}

Let $(\Omega, \CA, \PM)$ be a probability space.
Let $A$ and $B$ be sets and let $x: \Omega \to (A \to B)$
Then $x$ is a random function.
For each outcome $\omega \in \Omega$, $x_{\omega}: A \to B$ is a function from $A$ to $B$.

\ssection{A family of random variables}

We can associate to $x$ a family of random variables indexed by $A$.
Define $y: A \to (\Omega \to B)$, defined by
\[
  y(a)(\omega) = x(\omega)(a).
\]
The index set of the family is the domain of the set of functions and the codomain of each random variable is the codomain of the set of functions.
Conversely, any family of random variables can be associated with a random function.
In this way, a random function corresponds to a family of random variables, and any family of random variables corresponds to a random function.
For this reason, some authors define a random function (random process) as a family of random variables.


\blankpage
