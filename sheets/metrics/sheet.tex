
\section*{Why}

We want to talk about a set with a prescribed quantitative degree of closeness (or distance) between its elements.

\section*{Definition}

The correspondences which serve as a degree of closeness, or measure of distance, must satisfy our previously developed (see \sheetref{distance}{Distance}) notion of distance.

A function on ordered pairs which does not depend on the order of the elements so considered is \t{symmetric}.
A function into the real numbers which takes only nonnegative values is \t{nonnegative}.
A repeated pair is an ordered pair of the same element twice.
A function which satisfies a triangle inequality for any three elements is \t{triangularly transitive}.

A \t{metric} (or \t{distance function}) is a function on ordered pairs of elements of a set which is symmetric, non-negative, zero only on repeated pairs, and triangularly transitive.
A \t{metric space} is an ordered pair whose first coordinate is a nonempty set and whose second coordinate is a metric.

In a metric space, we say that one pair of objects is \t{closer} together if the metric of the first pair is smaller than the metric value of the second pair.

Notice that a set can be made into different metric spaces by using different metrics.

\subsection*{Notation}

Let $A$ be a set.
We commonly denote a metric by the letter $d$, as a mnemonic for ``distance.''
Let $d: A \times  A \to \R $.
Then $d$ is a metric if:
    \begin{enumerate}
      \item it is non-negative, which we tend to denote by
\[
d(a, b) \geq 0 \quad \forall a,b \in A.
\]
      \item it is $0$ only on repeated pairs, which we tend to denote by
\[
d(a, b) = 0 \iff a = b, \quad \forall a,b \in A.
\]
      \item it is symmetric, which we tend to denote by:
\[
d(a, b) = d(b, a), \quad \forall a,b \in A.
\]
      \item it is triangularly transitive, which we tend to denote by
\[
d(a, b) \leq d(a, c) + d(c, b), \quad \forall a,b,c \in A.
\]
    \end{enumerate}
As usual, we denote the metric space of $A$ with $d$ by $(A, d)$.
Another common choice of letter for a metric is $\rho $.

\section*{Examples}

$\R $ with the absolute value distance is a metric space.
As is $\R ^2$ and $\R ^3$ with the Euclidean distance.
$\R ^n$ with Euclidean metric is an example of a metric space for which the objects ($n$-dimensional tuples of real numbers) are impossible to visualize.

%macros.tex
%%%%% MACROS %%%%%%%%%%%%%%%%%%%%%%%%%%%%%%%%%%%%%%%%%%%%%%%
%%%%%%%%%%%%%%%%%%%%%%%%%%%%%%%%%%%%%%%%%%%%%%%%%%%%%%%%%%%%
