%!name:linear_predictors
%!need:inductors
%!need:linear_transformations
%!need:matrix_transpose

\ssection{Why}

Here's a simple idea.
If the set of postcepts is a vector space, use a predictor that is a linear transformation.\footnote{Future editions will expand on this why.}

\ssection{Definition}

A \t{linear predictor} is a predictor which is linear in the precepts.
Such a model is simple to implement and interpretable, at the cost of flexibility.

\ssection{$\R^d$ Example}

A linear function $f: \R^d \to \R$ over the vector space $(\R^d, \R)$ has a set of parameters $w \in \R^d$ so that \[
  f(x) = \sum_{i} w_ix_i = \transpose{w}x.
\]
The parameters of a linear predictor on $\R^d$ are often called \t{weights}.

\blankpage
