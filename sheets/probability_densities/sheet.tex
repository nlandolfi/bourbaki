%!name:probability_densities
%!need:probability_distributions

\ssection{Why}

We want to talk about
probability over
infinite sets.
We will use our intuition from
probability distributions.

\ssection{Definition}

The principal difficulty is
assigning nonzero numbers to
infinitely many elements of
a set.

Consider the set of natural
numbers. suppo
Suppose the set
that we can not assign nonzero
probability to each element of
the set. There are infinitely
many, and so
We can not assign nonzero
probability to each individual
element of the set, since there
are infinitely many it would
never normalize.
Instead, we assign probability
to subsets of the set.
In other words, we associate
a measure with the set.
The measure is normalized, meaning
that the measure of the whole set,
the event corresponding
to any outcome occuring is 1.


A \ct{probability density}{} or
\ct{probability density function}{}
is a real-valued function
from a set of outcomes which is
non-negative and normalized.
A real-valued function on a
an infinite set is  set
is \ct{normalized}{} if the sum over the
its results is 1.
