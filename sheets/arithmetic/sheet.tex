%!name:arithmetic
%!need:natural_numbers
%!need:operations

\ssection{Why}

Counting one by one is slow so we define
an algebra on the naturals.

\ssection{Sums and Addition}

Let $m$ and $n$ be two natural numbers.
If we apply the successor function to $m$ $n$
times we obtain a number.
If we apply the successor function to $n$ $m$
times we obtain a number.
Indeed, we obtain the same number in both cases.
We call this number the \definition{sum}{sum}
of $m$ and $n$.
We say we \definition{add}{add} $m$ to $n$,
or vice versa.
We call this correspondence, between
$(m, n)$ and the sum, \definition{addition}{addition}.

\ssubsection{Notation}

We denote the function addition by $+$
and so denote the sum of the naturals
$m$ and $n$ by $m + n$.

\ssection{Products and Multiplication}

Let $m$ and $n$ naturals.
If we add $n$ copies of $m$ we obtain a number.
If we add $m$ copies of $n$ we obtain a number.
Indeed, we obtain the same number in both cases.
We call this number the \definition{product}{product} of $m$ and $n$.
We say we \definition{multiply}{multiply} $m$ to $n$, or vice versa.
We call this symmetric operation mapping $(m, n)$ to their product
\definition{multiplication}{multiplication}.

\ssubsection{Notation}

We denote the operation of multiplication by
$\cdot$ and so denote the product of the naturals
$m$ and $n$ by $m \cdot n$.
