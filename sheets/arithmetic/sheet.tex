%!name:arithmetic
%!need:operations
%!need:natural_exponents
%!refs:paul_halmos/naive_set_theory/section_13


\ssection{Why}

We name the operations which produce natural sums, products and powers.

\ssection{Definition}

Consider the set of natural numbers.
The we can define three functions corresponding to sums, products and powers which are operations (see \sheetref{operations}{Operations}) on this set.

We call \t{addition} the function $+: \omega \times \omega \to \omega$, which maps two natural numbers $m$ and $n$ to their sum $m + n$.
We call \t{multiplication} the function $\cdot: \omega \times \omega \to \omega$, which maps two natural numbers $m$ and $n$ to their sum $m \cdot n$.
We call \t{exponentiation} the function $(m, n) \mapsto m^n$.

In other words, we can think of sums, produces, and powers as obtainable by applying a function to pairs of natural numbers.
This function gives another natural numbers
We call these three operations the operations of \t{arithmetic}.

\blankpage