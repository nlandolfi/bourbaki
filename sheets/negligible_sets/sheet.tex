
\section*{Why}

Some reasonable statements do not hold \textit{for all} elements of the base set of a measurable space.
Even so, these statements often hold \textit{broadly}, in the sense that the measure of the set on which they fail is zero.
This idea allows one to handle statements which fail on a set of measure zero as though they hold everywhere.
This approach is useful in discussions of convergence and integration.

\section*{Definition}

For this sheet, suppose $(X, \mathcal{A} , \mu )$ is a \sheetref{measures}{measure space}.
Call a set $N \subset X$ \t{negligible} if there exists a measurable set $A \in \mathcal{A} $ with $\mu (A) = 0$ and $N \subset A$.
In english, there is a measurable set containing $N$ that has measure zero.

The qualification ``if there exists a measurable set...'' enables one to speak of \textit{nonmeasurable} negligible sets.
Negligible sets need not be measurable.

Given the measure $\mu $, a \sheetref{set_specification}{statement}$s$ holds \t{almost everywhere} with respect to $\mu $ if the set of elements on which the statement fails is negligible.
In symbols, there exists $A \in \mathcal{A} $ with $\mu (A) = 0$ and
\[
\Set{x \in X}{\neg s(x)} \subset A
\]

A statement which holds ``everywhere'' holds ``almost everywhere'' also.
With this in mind, we call the almost everywhere sense ``weaker'' than the everywhere sense.

\subsection*{Notation}

We abbreviate almost everywhere as ``a.e.,'' read ``almost everywhere''.
We say that a statement ``holds a.e.''
If the measure $\mu $ is not clear from context, we say that the property holds almost everywhere $[\mu ]$ or $\mu $-a.e., read ``mu almost everywhere.''

\subsection*{Function comparisons}

Let $f, g: X \to \R $, not necessarily measurable.
Then $f = g$ almost everywhere if the set of points at which the functions disagree is
$\mu $-negligible.
Similarly, $f \geq g$ almost everywhere if the set of points where $f$ is less than $g$ is $\mu $-negligible.

If $f$ and $g$ were both in fact $\mathcal{A} $-measurable, then the sets
\[
\Set*{x \in X}{f(x) \neq g(x)}
\text{ and }
\Set*{x \in X}{f(x) < g(x)}
\]
would be measurable also.
But in general, these sets need not be measurable, and so this is one simple example to justify our including nonmeasurable sets in the definition of neglible sets, as mentioned above.

\subsection*{Function limits}

Let $\seqt{f}: X \to R$ for each natural number $n$ and let $f: X \to R$ be a function.
The sequence $\seq{f}$ \t{converges to $f$ almost everywhere} if
\[
\Set*{x \in X}{\lim_{n} \seqt{f}(x) \text{ does not exist, or } f(x) \neq \lim_{n} \seqt{f}(x)}
\]
is $\mu $-negligible.
In this case, we write ``$f = \lim_n \seqt{f}$ a.e.''
