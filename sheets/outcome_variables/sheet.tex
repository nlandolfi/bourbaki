
\section*{Why}

We want to talk about particular attributes of an outcome, instead of the details of the outcomes themselves.
These may be useful to specify events.

\section*{Definition}

Given a sample space $\Omega $, an \t{outcome variable} (or \t{random variable}) is any function on $\Omega $.
In this context, the range of the function is called the set of \t{values} of the random variable.

\subsection*{Notation}

Standard convention denotes outcome random variables by capitals $X, Y, Z$ and elements of their codomain by corresponding lower-case, $x, y, z$.
Thus, if $X: \Omega  \to V$ is a random variable then the lower case $x$ is often reserved for an element of $V$.
The event $\Set{\omega  \in \Omega }{X(\omega ) = x}$, where $x \in V$, is often abbreviated $\set{X = x}$.
The probability of this event is often abbreviated $P(X = x)$.

Similarly, for a subset $A \subset V$, the event $\Set{\omega \in \Omega }{X(\omega ) \in A}$ is often abbreviated $\set{X \in A}$ and its probability abbreviated $P(X \in A)$.\footnote{Occasionally, in the present edition of these sheets, we use the notation $P[X = a]$.}
If $X: \Omega  \to V_1$ and $Y: \Omega  \to V_2$, and $A \subset V_1$ and $B \subset V_2$, then the event $\set{X \in A} \cap  \set{Y \in B}$ is often written $\set{X \in A, Y \in B}$.

\section*{Examples}

\textit{Sum of two dice.}
Suppose we model rolling two dice.
We are interested in the sum of the pips shown facing up.
Suppose we take as the set of outcome $\set{1, \dots , 12}$, whose elements correspond to the sum.
We interpret $\Set{x \in \Omega }{x \geq 10}$ as the event that the sum of the two dice is greater than or equal to 10.

Alternatively, we may take the usual set of outcomes $\set{1, \dots , 6}^2$ and define an outcome variable $s: \set{1, \dots , 6}^2 \to \set{1, \dots , 12}$ by
\[
s(d_1, d_2) = d_1 + d_2.
\]
We interpret this natural-number-valued outcome variable $s$ as sum of the two dice.
The event that the sum of the two dice is greater than or equal to to 10 corresponds to the set $\Set{(d_1, d_2) \in \set{1, \dots , 6}}{s(d_1, d_2) \geq 10}$.

% math151 notes 

As a third alternative, again take the usual set of outcomes $\Omega  = \set{1, \dots , 6}^2$.
Define the outcome variable $X: \Omega  \to \N  $ to be the number of pips showing on the first die, $Y: \Omega  \to \N  $ to be the number of pips showing on the second die, and define $Z: \Omega  \to \N  $ by
\[
Z(\omega ) = X(\omega ) + Y(\omega ) \quad \text{for all } \omega  \in \Omega
\]
A standard notation for this relation between $Z$ and $X,Y$ by $Z = X + Y$.
For example, if $\omega  = (2,5)$, $X(\omega ) = 2$, $Y(\omega ) = 5$, and $Z(\omega ) = 7$.
The event $\set{Z = 4}$ is the set $\set{(2,2), (1,3), (3,1)}$
If we take the usual distribution $p$ on $\Omega $ with $p(\omega ) = 1/36$ for every $\omega $, the the probability of this event is
\[
P(Z = 4) = p(2,2) + p(1,3) + p(3, 1) = 1/36+1/36+1/36 = 1/12
\]

The preceding three paragraphs highlight that there are several ways of denoting the same situation.

\textit{Tossing a fair coin $n$ times.}
As before, we model $n$ tosses of a fair coin with the sample space $\Omega  = \set{0,1}^n$.
Define $X_i: \Omega  \to \set{0,1}$ by $X_i(\omega ) = \omega _i$.
We interpret $\Set{\omega  \in \Omega }{X_i(\omega ) = 1}$ as the event that toss $i$ turns up heads, and likewise $\Set{\omega \in \Omega }{X_i(\omega ) = 0}$ as the event that toss $i$ turns up tails, for $i = 1, \dots , n$.
We can define the function $X: \Omega  \to {0,1}^n$ by $X(\omega ) = \omega $ or by saying $X = (X_1, \dots , X_n)$.
Suppose we define $H: \Omega  \to \N  $ by
\[
H(\omega ) = \sum_{i= 1}^{n} X_i(\omega )
\]
Or, alternatively, $H = \sum_{i = 1}^{n} X_i$.
In this case, we interpret $H$ as the number of heads observed in the $n$ tosses.

% math 151 

\blankpage