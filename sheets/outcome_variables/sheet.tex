
\section*{Why}

We want to talk about particular attributes of an outcome, instead of the details of the outcomes themselves.
These may be useful to specify events.

An \t{outcome variable} (or \t{random variable}) is a function from $\Omega $ to $V$, where $V$ is a set.
In this context, $V$ is called the set of \t{values} of the random variable.

\section*{Example: two dice}

We want to talk about the sum of the pips shown facing up after rolling two dice.
We may take as our set of outcome $\set{1, \dots , 12}$, whose elements correspond to the sum.
We interpret $\Set{x \in \Omega }{x \geq 10}$ as the event that the sum of the two dice is greater than or equal to 10.

Alternatively, we may take the outcomes $\set{1, \dots , 6}^2$ and define an outcome variable $s: \set{1, \dots , 6}^2 \to \set{1, \dots , 12}$ by
\[
s((d_1, d_2)) = d_1 + d_2.
\]
We interpret this natural-number-valued outcome variable $s$ as sum of the two dice.
The event that the sum of the two dice is greater than or equal to to 10 corresponds to the set $\Set{(d_1, d_2) \in \set{1, \dots , 6}}{s((d_1, d_2)) \geq 10}$.

\blankpage