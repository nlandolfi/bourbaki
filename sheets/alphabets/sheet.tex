
%!name:alphabets
%!need:lists

\section*{Why}

We return to our discussion of symbols and scripts, to make precise these concepts in the language of sets and lists.

\section*{Definition}

An \t{alphabet} is a finite set.
For example, let $A$ be the set
\[
\set{a, b, c, d, e, f,g,h,i,j,k,l,m,n,o,p,q,r,s,t,u,v,w, x, y,z},
\]
where $a$ denotes the latin lower case letter ``a'', $b$ denotes the latin lower case letter ``b'', and so on.
In other words, $A$ is the set of lowercase latin letters.
It is an alphabet.
By analogy with this familiar case, we frequently refer to the elements of an alphabet as \t{letters} or \t{symbols}.

A \t{word} is a list of letters in an alphabet, and a \t{phrase} is a list of words.
For example, $(c,a,t,s)$ is a word in $\mathcal{A} $ (meant to correspond to the word ``cats'') and
\[
((c,a,t,s), (a,n,d), (d,o,g,s))
\]
is a phrase in $\mathcal{A} $ (meant to correspond to the phrase ``cats and dogs'').

\section*{Strings}

Let $A$ be an alphabet.
In this case (in which $A$ is a finite set), we refer to the lists of $A$ as \t{strings}.
The string whose length is zero is the empty set.

\subsection*{Notation}

We denote the set of all lists (strings) in $A$ by $\str(A)$.
We read $\str(A)$ aloud as ``the strings in $A$.''
