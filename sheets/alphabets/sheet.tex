%!name:alphabets
%!need:lists

\ssection{Why}

We return to our discussion of symbols and scripts, to make precise these concepts in the language of sets and sequences.

\ssection{Definition}

An \t{alphabet} is a finite set.
For example, let $A$ be the set
  \[
\set{a, b, c, d, e, f,g,h,i,j,k,l,m,n,o,p,q,r,s,t,u,v,w, x, y,z},
  \]
where $a$ denotes the latin lower case letter \say{a}, $b$ denotes the latin lower case letter \say{b}, and so on.
In other words, $A$ is the set of lowercase latin letters.
It is an alphabet.
By analogy with this familiar case, we frequently refer to the elements of an alphabet as \say{letters} or \say{symbols.}
A \t{word} is a finite sequence of letters in an alphabet, and a \t{phrase} is a finite sequence of words.
For example, $(c,a,t,s)$ is a word in $\CA$ (mean to correspond to the word \say{cats}) and
    \[
((c,a,t,s), (a,n,d), (d,o,g,s))
    \]
is a phrase in $\CA$ (meant to correspond to the phrase \say{cats and dogs}).
\ssection{Strings}

Let $A$ be an alphabet.
In this case (in which $A$ is a finite set), we refer to the finite sequences of $A$ as \t{strings}.
The length zero string is $\emptyset$.

\ssection{Notation}

We denote the set of all finite sequences (strings) in $A$ by $\strings(A)$.
We read $\strings(A)$ aloud as \say{the strings in $A$.}



\blankpage
