%!name:linear_transformations
%!need:transformations
%!need:linear_combinations

\ssection{Why}

Lots of things are (approximately) linear.\footnote{Future editions will expand on this why. In particular, the intuition of proportionality.}

\ssection{Definition}

An \t{linear transformation} is a transformation which maps  a linear combination of two vectors to a linear combination of the results of those two vectors, with the same coeeficients.
A transformation is always \say{linear} with respect to some field.
The field is implicit, somewhat, in the definition but always present.

\ssubsection{Terminology}

As in \sheetref{transformations}{Transformations}, we prefer in these sheets to refer to a linear \say{transformation} in order to emphasize that the function is onand to a vector space.
We note here that in spite of this convention, it is, of course, natural in common language to discuss linear \say{functions}.
Many authorities call linear functions \t{operators}.\footnote{Though this is commonly reserved for the case in which the vector space on which the transformation is defined is infinite-dimensional. In other words, the case in which we are dealing with functions of functions.}

\ssubsection{Notation}

Let $(V_1, F)$
and $(V_2, F)$ be two
vector spaces over the
same field.
Let $f: V_1 \to V_2$.
$f$ is linear means
\[
  f(au + bv) = af(u) + bf(v)
\]
for all $a, b \in F$ and $u, v \in V_1$.
