
\section*{Why}

Many functions of interest are additive and homogenous.

\section*{Definition}

A transformation is \t{linear} (a \t{linear transformation}, \t{linear map}) if the result of a linear combination of the two vectors is the linear combination of the results of the vectors (using the same coefficients).
The transformation is linear \t{with respect to} the field of the two vector spaces.

We use the term transformation (see \sheetref{transformations}{Transformations}) to emphasize and remind that the function is defined on a \textit{vector} space.
Of course, $\R $ is a vector space and so a function $f: \R \to \R $ may be linear.
The linear maps from $\R $ to $\R $ are the the \t{linear functions} (see \sheetref{real_linear_functions}{Real Linear Functions}).

% this is wrong; see new sheet on operators
%<div data-littype='paragraph'>
% <div data-littype='run'> Often authors will use the word ❬operator❭ for linear
%    functions. </div>
% <div data-littype='run'> It seems, generally, that this term is commonly reserved for
%    the case in which the vector space discussed is a function
%    space (or, at least, infinite dimensional). </div>
%</div>

\subsection*{Notation}

Let $(V, \F )$ and $(W, \F )$ be two vector spaces over the same field.
Suppose $T: V \to W$.
$T$ is linear means
\[
T(\alpha u + \beta v) = \beta T(u) + \alpha T(v) \quad \text{for all } \alpha , \beta  \in F \text{ and } u, v \in V.
\]

As usual, the condition that $T$ is linear condition is equivalent to the two conditions:
    \begin{enumerate}
      \item $T(u + v) = f(u) + T(v)$ for all $u, v \in V$, and
      \item $f(\lambda u) = \lambda f(u)$ for all $\lambda  \in \F $ and $u \in V$.
    \end{enumerate}
If $T$ satisfies (1), we call $T$ \t{additive} (has the property of \t{additivity}).
If $T$ satisfies (2) we call $T$ \t{homogeneous} (has the property of \t{homogeneity}).

For linear maps, it is common to denote $T(v)$ by $Tv$; notice that we have dropped the usual parentheses.

We denote the set of all linear maps by $\mathcal{L} (V, W)$.
It is understood when using this notation that $V$ and $W$ are vector spaces with respect to the same field $\F $.

\subsection*{Examples}

Throughout, we consider vector spaces $V$ and $W$ over some fixed field $\F $.

\textit{Constant zero map}.
The map $T \in \mathcal{L} (V, W)$ defined by
\[
T(v) = 0 \in W \quad \text{for all } v \in V
\]
is called the \t{zero map} (or \t{zero transformation}).
It is common to overload the symbol 0 so that $0 \in \mathcal{L} (V, W)$ denotes the zero map.
In other words, the map $0$ is defined by
\[
0v = 0
\]
Some care is required to interpret this equation.
The 0 on the left hand side refers to a function, from $V$ to $W$.
The 0 on the right hand side is the additive identity in $W$.
Usually context disambiguates the overloaded notation.

\textit{The identity map}.
The map $T \in \mathcal{L} (V, V)$ defined by
\[
Tv = v \quad \text{for all } v \in V
\]
is called the \t{identity map} (or \t{identity transformation})
It is common to denote this map by $I$.

\textit{Differentiation of polynomials}
Suppose $P$ is the set of all polynomials with coefficients in $\R $.
(Some authors denote this set by $\mathcal{P} (\R )$.
Recall that every $p \in \mathcal{P} (\R )$ is differentiable and $p' \in \mathcal{P} (\R )$.
The map $T \in \mathcal{L} (\mathcal{P} (\R ), \mathcal{P} (\R ))$ defined by
\[
Tp = p'
\]
is linear.
To see this, recall $(f + g)' = f' + g'$ and $(\lambda f)' = \lambda f'$ whenever $f, g$ are differentiable and $\lambda  \in \R $ (see \sheetref{derivatives_of_sums}{Derivative of Sums}and \sheetref{derivatives_of_scalar_multiples}{Derivatives of Scalar Multiples}) .

\textit{Integration of polynomials}
As in the previous paragraph, $\mathcal{P} (\R )$ denotes the vector space of polynomials with coefficients in $\R $.
The map $T \in \mathcal{L} (\mathcal{P} (\R ), \R )$ defined by
\[
Tp = \int _{[0,1]} p
\]
is linear
To see this, recall that $\int (f + g) = \int f + \int g$ and $\int \lambda f = \lambda \int f$ whenever $f, g$ are differentiable and $\lambda  \in \R $ (see \sheetref{real_integral_additivity}{Real Integral Additivity}and \sheetref{real_integral_homogeneity}{Real Integral Homogeneity}.

\textit{Multipliciation by a quadratic}.
As in the previous paragraph, $\mathcal{P} (\R )$ denotes the vector space of polynomials with coefficients in $\R $.
The map $T \in \mathcal{L} (\mathcal{P} (\R ), \mathcal{P} (\R ))$ defined by
\[
(Tp)(x) = x^2p(x) \quad \text{for all } x \in \R , p \in \mathcal{P} (\R )
\]
is linear.
(Prove this).

\textit{Sequence backward shift}.
Denote the space of infinite sequences in a field $\F $ by $\F ^\N  $ as usual.
Define $T \in \mathcal{L} (\F ^\N  , \F ^\N  )$ by
\[
T(x_1, x_2, x_3, \dots ) = (x_2, x_3, \dots )
\]
$T$ is called the \t{backward shift operator}.

\textit{From real space the the real plane}.
Define $T \in \mathcal{L} (\R ^3, \R ^2)$ by
\[
T(x, y, z) = (2x - y + 3z, 7x+5y - 6z)
\]

\textit{From $\F ^n$ to $\F ^m$}.
Generalizing the previous example, suppose $m$ and $n$ are natuarl numbers, and let $A_{i, j} \in \F $ for $i = 1, \dots , m$ and $j = 1, \dots , m$.
Define $T \in \mathcal{L} (\F ^3, \F ^2)$ by
\[
T(x_1, \cdots, x_n) = (A_{1,1}x + \cdots + A_{1,n} x_n, \dots , A_{m,1}x_1 + \cdots + A_{m,n}x_n)
\]
(It happens that every linear map from $\F ^n$ to $\F ^m$ has this form.)

\textit{A counterexample: $\cos$}\footnote{Need to add a sheet for trigonometric functions.}
Notice $\cos(x + y) = \cos(x) + \cos(y)$.
True, $\cos$ is not homogenous. that $\cos2x = 2\cos(x)$ and
But this does not hold for all reals: $\cos \lambda x \neq \lambda \cos(x)$.

\blankpage