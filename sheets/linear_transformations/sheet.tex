
\section*{Why}

Lots of things are (approximately) linear.\footnote{Future editions will expand on this why. In particular, the intuition of proportionality.}

\section*{Definition}

A transformation is \t{linear} if the result of a linear combination of the two vectors is the linear combination of the results of the vectors (using the same coefficients).
The transformation is linear \t{with respect to} the field of the two vector spaces.

We use the term transformation ( \sheetref{transformations}{Transformations}) for emphasis and reminder that the function is defined on a vector space.
Of course, $\R $ is a vector space and so a function $f: \R \to \R $ may be linear.
It is, therefore, common to speak of \t{linear functions}.

Often authors will use the word \t{operator} for linear functions.
It seems, generally, that this term is commonly reserved for the case in which the vector space discussed is a function space (or, at least, infinite dimensional).

\subsection*{Notation}

Let $(V_1, F)$ and $(V_2, F)$ be two vector spaces over the same field.
Let $f: V_1 \to V_2$. $f$ is linear means
\[
f(au + bv) = af(u) + bf(v)
\]
for all $a, b \in F$ and $u, v \in V_1$.

\blankpage