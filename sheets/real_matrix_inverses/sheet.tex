%!name:real_matrix_inverses
%!need:real_matrix-matrix_products
%!need:inverse_elements

\section*{Why}

Let $A \in \R ^{m \times n}$ and define $f: \R ^n \to \R ^m$ by $f(x) = Ax$.
Then $f$ is a linear function from $\R ^{n}$ to $\R ^{m}$.
Conversely, suppose $g: \R ^n \to \R ^m$ is a linear function.
Then there exists a matrix $B \in \R ^{m \times n}$ so that $g(z) = Bz$.
Does this function have an inverse?

\section*{Derivation}

If $A \in \R ^{m \times n}$, with $m \neq n$, then the inverse of $f$ can not exist.
For a \textit{square} matrix $A \in \R ^{n \times n}$, $B \in \R ^{n \times n}$ is a \t{left inverse} if $BA = I$.
In other words, $B$ is a left inverse element of $A$ in the algebra of matrices with the operation of multiplication.
$C \in \R ^{n \times n}$ is a \t{right inverse} if $AC = I$.

\section*{Definition}

We call a square matrix $A$ \t{invertible} if there is $B \in \R ^{n \times n}$ so that $BA = I$.

Now suppose that $A \in \R ^{n \times n}$.
Of course, the inverse may not exist.
Consider, for example if $A$ was the $n$ by $n$ matrix of zeros.
If there exists a matrix $B$ so that $BA = I$ we call $B$ the \t{left inverse} of $A$ and likewise if $AC = I$ we call $C$ the \t{right inverse} of $A$.
In the case that $A$ is square, the right inverse and left inverse coincide.

\begin{proposition}
Let $A, B, C \in \R ^{n \times n}$.
Let $BA = I$ and $AC = I$.
Then $B = C$.
\begin{proof}
Since $BA = AC$ we have $BBA = BAC$ so $B = C$ since $BA = I$.
\end{proof}
\end{proposition}

\blankpage