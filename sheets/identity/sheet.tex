%!name:identity
%!need:objects

\ssection{Why}

We can give the same object two different names.

\ssection{Definition}

An object \ct{is}{} itself.
If the object that two names refer to is the same, then we say that the first name \t{equals} the second name.

\ssubsection{Notation}

We denote that the object named $a$ and the object named $b$ refer to the same object by $a = b$.
We read this notation aloud as: \say{a is b} or \say{a equals b}.
We denote that the object $a$ and $b$ refer to different objects by $a \neq b$.
We read this aloud as \say{a is not b} or \say{a does not equal b}.

%We may also read the notation
%$a = b$ aloud as \say{a
%equals b}.
Other English readings of $a = b$ include: \say{a is the same as b},
\say{a is equivalent to b}, \say{a refers to the same object as b.}

\ssubsection{Properties}

Given an object $a$, $a = a$ is true.
We say that equivalence is \t{reflexive}.
Given objects $a$ and $b$, $a = b$ implues $b = a$.
We say that equality is \t{symmetric}.
Given objects $a$, $b$, and $c$, $a = b$ and $b = c$ implies $a = c$.
We say that equality is \t{transitive}.

\blankpage
