
\section*{Why}

We name the image measure of a collection of real-valued random variables.

\section*{Definition}

The \t{joint law} of a sequence of $n$ real-valued random variables is the image measure of the tuple-valued function whose components are the individual random variables.

\subsection*{Notation}

Let $(X, \mathcal{A} , \mu )$ be a probability space and $(Y, \mathcal{B} )$ be a measurable space.
Let $f_1, \dots , f_n: X \to Y$ be random variables.
Define $f: X \to Y^n$ by $(f(x))_i = f_i(x)$.
The joint law is the image measure of $f$.

We denote the joint law of $\set{f_i}$ by $\rvlaw{\mu }{f_1, \dots , f_n}: \mathcal{A}  \to \nneri$.
We defined it by
\[
\rvlaw{\mu }{f_1, \dots , f_n}(A)
= \mu (\Set*{x \in X}{f(x) \in A}).
\]
for all $A$ in the product sigma algebra on $Y^n$.

\blankpage
%macros.tex
%%%%% MACROS %%%%%%%%%%%%%%%%%%%%%%%%%%%%%%%%%%%%%%%%%%%%%%%
%%%%%%%%%%%%%%%%%%%%%%%%%%%%%%%%%%%%%%%%%%%%%%%%%%%%%%%%%%%%
