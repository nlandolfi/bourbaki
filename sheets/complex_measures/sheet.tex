
%!name:complex_measures
%!need:measures

\section*{Why}

We allow measures to take complex values.\footnote{Future editions will expand.}

\section*{Definition}

A complex-valued
function on a
sigma algebra is
\t{countably additive}
if the result of the function applied to
the union of a disjoint countable family of
distinguished sets is the limit of the partial
sums of the results of the function applied
to each of the sets individually.
The limit of the partial sums must
exist irregardless of the summand order.

A \t{complex measure} is an complex-valued function on a sigma algebra that is
(1) zero on the empty set and
(2) countably additive.
We call the result of the function applied to a set in the sigma algebra the \t{complex measure} (or when no ambiguity arises, the \t{measure}) of the set.

Since the codomain of
a complex measures is
the complex numbers,
the sum corresponding
to every countable union
must be absolutely convergent (?).\footnote{Future editions will define.}

\subsection*{Notation}

We denote complex measures by $\mu $ a mnemonic for ``measure.''
Let $C$ denote the set of complex numbers.
Let $(X, \mathcal{A} )$ be a measurable space and let $\mu : \mathcal{A}  \to C$.
Then $\mu $ is a complex measure if
    \begin{enumerate}
      \item $\mu (\varnothing) = 0$ and
      \item $\mu (\cup_{i} A_i) = \lim_{n \to \infty} \sum_{k = 1}^{n} \mu (A_k)$
for all disjoint families $\seq{A}$.
    \end{enumerate}
