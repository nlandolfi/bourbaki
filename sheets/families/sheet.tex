%!name:families
%!need:functions
%!need:set_specification
%!need:set_inclusion

\ssection{Why}

It is useful to
have some language and
notation for talking about
a set of sets.

\ssection{Definition}

A \ct{family} of sets
is a set of sets.
Experience shows that it is
useful to have these associated
with the elements of a well-known
second set.

An \ct{indexed family of sets}{} is a function
from one set to the power set of a second set.
We call the first set the \ct{index set}{}.
We call the second set the \ct{base set}{}.
We call the range of the family
a \ct{family}{} of sets.

%If the index set is a finite set,
%we call the family a
%\ct{finite family}{finitefamily}.
%If the index set a countable set,
%we call the family a
%\ct{countable family}{countablefamily}.
%If the index set is an uncountable set,
%we call the family a
%\ct{uncountable family}{uncountablefamily}.

%If the codomain is a set of sets, we
%call the family a \ct{family of sets}{familyofsets}.
%We often use a subset of the natural numbers
%as the index set.
%In this case, and for other indexed sets with
%orders, we call the family an \ct{ordered family}.

\ssubsection{Notation}

Let $A$ and $I$ be be a non-empty sets.
We use $I$ as a mnemonic for \say{index}
set.
Let $a: I \to \powerset{A}$ be a family.
For $i \in I$, we follow the function notation
and denote the result of applying $a$ to $i$ by
$a_{i}$.

We denote the range of the family by
family of $a_{\alpha}$ indexed with $I$
by $\set{a_{\alpha}}_{\alpha \in I}$, which is short-hand
for \rt{set-builder notation}{setbuildernotation}.
We read this notation \say{a sub-alpha, alpha in I.}
%If for each $\alpha \in I$ we have $A_{\alpha} \subset S$, we'd write $\set{A_{\alpha}}{\alpha \in I}$.

