
%!name:families
%!need:functions
%!refs:paul_halmos/naive_set_theory/section_09
%!refs:bert_mendelson/introduction_to_topology/theory_of_sets/indexed_families_of_sets

\section*{Why}

We often use functions to keep track of several objects by the objects of some well-known set with which they correspond.
In this case, we use specific language and notation.

\section*{Definition}

Let $I$ and $X$ denote sets.
A \t{family} is a function from $I$ to $X$.
We call an element of $I$ an \t{index} and we call $I$ the \t{index set}.
Of course, the letter $I$ was picked here to be a mnemonic for \say{index}.
We call the range of the family the \t{indexed set} and we call the value of the family at an index $i$ a \t{term} of the family at $i$ or the \t{$i$th term} of the family.

Experience shows that it is useful to discuss sets using indices, especially when discussing a set of sets.
If the values of the family are sets, we speak of a \t{family of sets}.
Indeed, we often speak of a \t{family of} whatever object the values of the function are.
So for instance, a family of subsets of $X$ is understood to be a function from some index set into $\powerset{X}$.

%If the codomain is a set of sets, we
%call the family a \ct{family of sets}{familyofsets}.
%We often use a subset of the natural numbers
%as the index set.
%In this case, and for other indexed sets with
%orders, we call the family an \ct{ordered family}.

\subsection*{Notation}

Let $x: I \to X$ be a family.
We denote the $i$th term of $x$ by $x_i$.
We sometimes denote the family by $\set{x_i}_{i \in I}$.

\blankpage