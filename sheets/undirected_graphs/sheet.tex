%!name:undirected_graphs
%!need:finite_sets

\section{Why}

We want to visualize symmetric relations.

\section{Definition}

An \t{undirected graph} is a finite nonempty set and a set of pairs of its elements.
We call the elements of the first set the \t{vertices} of the graph and the elements of the second set the \t{edges}.

Two vertices are \t{adjacent} if their pair is an edge in the set.
We say that the corresponding edge is \t{incident} to those vertices.
The \t{adjacency set} of a vertex is the set of vertices adjacent to it.
The \t{degree} of a vertex is the number of vertices adjacent to it; in other words, the size of its adjacency set.
A graph is \t{complete} if each pair of two distinct vertices is adjacent.
A \t{clique} is a maximal complete subgraph.

\ssubsection{Other Terminology}

Some authors define a clique as any set of vertices whose corresponding subgraph is complete; we prefer the term \t{complete subgraph} here.
Some authors call the adjacency set the \t{neighborhood} of the vertex and call the union of the adjacency set of a vertex with the singleton of that same vertex the \t{closed neighborhood} of the vertex.

\ssubsection{Notation}

Let $V$ be a non-empty set and let $E$ be a subset of $\Set*{\set{u, v}}{u, v \in V}$.
As usual, we denote the ordered pair consisting of $V$ and $E$ by $(V, E)$.
We say \say{Let $G = (V, E)$} be a graph, implying the relevant properties of $V$ and $E$.

\ssection{Example}

\begin{figure}
  \centering
  \includegraphics[width=0.5\textwidth]{graphics_included/undirected_graph}
  \caption{Undirected graph.}
\end{figure}

Let $V = \set{a, b, c, d, e}$
and
$$
  E = \set*{\set{a, b}, \set{a, c}, \set{a, e}, \set{b, d}, \set{b, e}, \set{c, d}, \set{c, e}, \set{d,e}}.
$$
