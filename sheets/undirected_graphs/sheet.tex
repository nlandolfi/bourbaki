
\section*{Why}

We want to visualize symmetric relations.

\section*{Definition}

An \t{undirected graph} (or \t{graph}) is a pair $(V, E)$ in which $V$ is a finite nonempty set and $E$ is a subset of unordered pairs of elements in $V$.
We call the elements of $V$ the \t{vertices} of the graph and the elements of $E$ the \t{edges}.
We call $(V, E)$ an undirected graph \t{on} $V$.

Two vertices are \t{adjacent} (or \t{neighboring}) if their pair is in the edge set.
We say that the corresponding edge is \t{incident} to those vertices.
The \t{adjacency set} of a vertex is the set of vertices adjacent to it.
The \t{degree} of a vertex is the number of vertices adjacent to it; in other words, the size of its adjacency set.
A graph is \t{complete} if each pair of two distinct vertices is adjacent.

The \t{complement} of $(V, E)$ is the graph $(V, F)$ where $F$ is the complement of $E$ in the set of pairs from $V$.

\subsection*{Other terminology}

Some authors call the adjacency set the \t{neighborhood} of the vertex.
They call the union of the adjacency set of the vertex $v \in V$ with the singleton $\set{v}$ the \t{closed neighborhood} of $v$.

When $\set{x,y} \in E$, some authors say that the edge ``joins'' the vertices, and call $x$ and $y$ the \t{end vertices} of the edge.

Some authors call two \textit{edges} \t{adjacent} if they have exactly one common endvertex.

\subsection*{Notation}

Let $V$ be a nonempty set.
Let $E \subset \Set*{\set{v, w}}{v, w \in V}$.
Then the pair $(V, E)$ is an undirected graph.
We regularly say ``Let $G = (V, E)$'' be a graph, in which the relevant properties of $V$ and $E$ are implicit.

The notation $\set{v, w} \in E$ for an edge between vertices $v, w \in V$ reminds us that the edges are unordered pairs of distinct vertices.
We denote the adjacency set of $v$ by $\adj(v)$ and the degree of $v$ by $\deg(v)$.

Some authors will denote the vertex set of a graph they are denoting by $G$ by $V(G)$ and the edges set by $E(G)$.

\section*{Example}

\begin{center}\includegraphics[width=0.50\textwidth]{./graphics/undirected_graph.pdf}\end{center}
%<div data-littype='paragraph'>
% <div data-littype='run'> \begin{figure} </div>
% <div data-littype='run'> \centering </div>
%  <img src='./graphics/undirected_graph.pdf' width='30%'/>
% <div data-littype='run'> \caption{Undirected graph.} </div>
% <div data-littype='run'> \end{figure} </div>
%</div>

Suppose $V = \set{a, b, c, d, e}$.
Define
\[
E = \set*{\set{a, b}, \set{a, c}, \set{a, e}, \set{b, d}, \set{b, e}, \set{c, d}, \set{c, e}, \set{d,e}}.
\]
In visualizations of graphs, the vertices are frequently represented as circles or rectangles in the plane and edges are shown as lines connecting the vertices.

%<div data-littype='paragraph'>
%  <div data-littype='run'> ❲%If we have a relation between two sets, the❳ </div>
%  <div data-littype='run'> ❲%standard corresponding graph is that obtained❳ </div>
%  <div data-littype='run'> ❲%by considering the relation as defined on the❳ </div>
%  <div data-littype='run'> ❲%union of the two sets. In this case we say the❳ </div>
%  <div data-littype='run'> ❲%graph is \ct{bipartite}{bipartite”.❳ </div>
%  <div data-littype='run'> </div>
%</div>
