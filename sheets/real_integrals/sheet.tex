%!name:real_integrals
%!need:nonnegative_integrals

\section*{Why}

We define the area under an extended real function.

\section*{Definition}

The \t{positive part} of an extended-real-valued function is the function mapping each element to the maximum of the function's result and zero.
The \t{negative part} of an extended-real-valued function is the function mapping each element to the maximum of the additive inverse of function's result and zero.

We decompose an extended-real-valued function as the difference of its positive part and its negative part.
Both the positive and negative parts are non-negative extended-real-valued functions.

Consider a measure space.
An \t{integrable} function is a measurable extended-real-valued function for which the non-negative integral of the posititve part and the non-negative integral of the negative part of the function are finite.
The \t{integral} of an integrable function is the difference of the non-negative integral of the posititive part and and the non-negative integral of the negative part.

If one but not both of the parts of the function are finite, we say that the integral \t{exists} and again define it as before.
In this way we avoid arithmetic between two infinities.

\subsection*{Notation}

Let $A$ a non-empty set.
Let $g: A \to [-\infty, \infty]$.
We denote the positive part
of $g$ by $g^+$ and the negative
part of $g$ by $g^-$:
    \[
g^+(x) = \max\set{g(x), 0} \quad \text{ and } \quad g^-(x) = \max\set{-g(x), 0}.
    \]
Moreover, we decompose $g$ as $g = g^+ - g^-$.
We observed that $g^+(x) \geq 0$ and $g^-(x) \geq 0$ for all $x \in X$.

Let $(X, \mathcal{A} , \mu )$ be a measure space.
Let $f: X \to [-\infty, +\infty]$ measurable and one of $\int f^+ d \mu $ or $\int  f^- d \mu $ is finite (if both are finite, $f$ is integrable).

We denote the integral of $f$ with respect to the measure $\mu $ by $\int  f d \mu $.
We defined:
    \[
\int  f d\mu  = \left(\int  f^+ d\mu \right) - \left(\int  f^- d\mu \right).
    \]
