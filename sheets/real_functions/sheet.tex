%!name:real_functions
%!need:intervals

\ssection{Why}

We name functions whose domain is the real numbers.

\ssection{Definition}

A \t{real function} is a real-valued function.
The domain is often an interval of real numbers, but may be any non-empty set.

\ssubsection{Notation}

Let $A$ be a set.
Let $f: A \to \R$.
$f$ is a real function.
If $A = \R$, then $f \in \R \to \R$.
To speak of functions defined on intervals, let $a, b \in \R$.
Let $g: [a, b] \to \R$.
Then $g$ is a real function defined on a closed interval.
Let $h: (a, b) \to \R$.
Then $h$ is a real function defined on an open interval.

We regularly declare the interval and the function at once.
For example,
\say{let $f: [a, b] \to \R$} is understood to mean \say{let $a$ and $b$ be real numbers with $a < b$, let $[a, b]$ be the closed interval with them as endpoints, and let $f$ bea  real-valued function whose domain is this interval}.
We read teh notation $f: [a, b] \to \R$ aloud as \say{$f$ from closed $a$ $b$ to $\R$.}
We use $f: (a, b) \to \R$ similarly (read aloud \say{$f$ from open $a$ $b$ to $\R$}).

\ssection{Examples}

\begin{expl}
Let $c \in \R$.
Let $f: \R \to \R$ be such that
$f(x) = c$ for every $x \in \R$.
$f$ is a real function.
\end{expl}

\begin{expl}
\item Let $f: \R \to \R$ with $f(x) = 2x^2 + 1$ for all $x \in R$.
$f$ is a real function.
\end{expl}

\begin{expl}
  Let $f: \R \to \R$ with
  \[
    f(x) = \begin{cases}
      1 & \text{ if } x \in \Q \\
      0 & \text{ otherwise. }
    \end{cases}
  \]
$f$ is a real function.
\end{expl}

