
%!name:real_functions
%!need:intervals

\section*{Why}

We name functions whose codomain is the real numbers.

\section*{Definition}

A \t{real function} is a real-valued function.
The domain is often an interval of real numbers, but may be any non-empty set.

\subsection*{Notation}

Given any set $A$, $f: A \to \R $ is a real function.
If $A = \R $, then $f \in \R  \to \R $.

We often speak of functions defined on intervals.
Given $a, b \in \R $, then $g: [a, b] \to \R $ is a real function defined on a closed interval.
The function $h: (a, b) \to \R $ is a real function defined on an open interval.

We regularly declare the interval and the function at once.
For example, \say{let $f: [a, b] \to \R $} is understood to mean \say{let $a$ and $b$ be real numbers with $a < b$, let $[a, b]$ be the closed interval with them as endpoints, and let $f$ bea real-valued function whose domain is this interval}.
We read the notation $f: [a, b] \to \R $ aloud as \say{$f$ from closed $a$ $b$ to $\R $.}
We use $f: (a, b) \to \R $ similarly (read aloud \say{$f$ from open $a$ $b$ to $\R $}).

\section*{Examples}

\begin{example}
Given $c \in \R $, define $f: \R  \to \R $ by
  \[
f(x) = c \quad \text{for all } x \in \R
  \]\end{example}
\begin{example}
Define $f: \R  \to \R $ by
  \[
f(x) = 2x^2 + 1 \quad \text{for all } x \in \R
  \]\end{example}
\begin{example}
Define $f: \R  \to \R $ by
  \[
f(x) = \begin{cases}
1 & \text{ if } x \in \Q  \\
0 & \text{ otherwise. }
\end{cases}
  \]\end{example}