
%!name:matrix_scalar_product
%!need:matrix_space
%!need:matrix_trace
%!need:real_inner_product_norms

\section*{Why}

We have seen that the matrices are a vector space.
Are they an inner product space?

\section*{Definition}

The \t{matrix scalar product} of $A \in \R ^{n \times k}$ and $B \in \R ^{n \times k}$ is the following product
  \[
\sum_{i = 1}^{n} \sum_{j = 1}^{k} a_{ij}b_{ij}.
  \]
Using the matrix trace, we can denote this as $\tr A^\top  B$.
Some authors call this the \t{Euclidean matrix scalar product}, \t{matrix inner product} or \t{Frobenius inner product}.

\begin{proposition}
The matrix scalar product is an inner product.\end{proposition}

For example, symmetry of the product is a consequence of the fact that a square matrix and its tranpose have identical traces.
Commutativity of the trace yields $\tr \tranpose{A} B = \tr B\tranpose{A}$, where the LHS is the scalar product of $\tranpose{B}$ and $\transpose{A}$.
In other words, transposition \say{preserves} the matrix scalar product.

With this inner product, $\R ^{n \times k}$ is a Euclidean vector space (see \sheetref{real_inner_products}{Inner procuts}) of dimension $nk$.
For the case of $k = 1$, we recover a model\footnote{Future editions will define this term.}
for the usual space $\R ^n$.

\subsection*{Notation}

We commonly denote the matrix inner product by $\langle A, B \rangle$.

\subsection*{Induced norm}

The matrix inner product induces a norm in the usual way.
This norm is sometimes called the \t{matrix-vector norm} (or \t{Frobenius norm}) and is often denoted for a matrix $A \in \R ^{m \times  n}$ by $\norm{A}_F$.
