%!name:matrix_scalar_product
%!need:matrix_space
%!need:matrix_trace

\ssection{Why}

We have seen that the matrices are a vector space.
Are they an inner product space?

\ssection{Definition}

The \t{matrix scalar product} of $A \in \R^{n \times k}$ and $B \in \R^{n \times k}$ is the following product
\[
  \sum_{i = 1}^{n} \sum_{j = 1}^{k} a_{ij}b_{ij}.
\]
Using the matrix trace, we can denote this as $\tr \tranpose{A} B$.
Some authors call this the \t{Euclidean matrix scalar product}.

\begin{proposition}
The matrix scalar product is an inner product.\footnote{Future editions will provide an account.}
\end{proposition}

For example, symmetry of the product is a consequence of the fact that a square matrix and its tranpose have identical traces.
Commutativity of the trace yields $\tr \tranpose{A} B = \tr B\tranpose{A}$, where the LHS is the scalar product of $\tranpose{B}$ and $\transpose{A}$.
In other words, transposition \say{preserves} the matrix scalar product.

With this inner product, $\R^{n \times k}$ is a Euclidean vector space (see \sheetref{inner_products}{Inner Products}) of dimension $nk$.
For the case of $k = 1$, we recover a model\footnote{Future editions will define} for the usual space $\R^n$.

\ssubsection{Notation}

We commonly denote the matrix inner product by $\langle A, B \rangle$.


\blankpage
