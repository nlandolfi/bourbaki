%!name:equations
%!need:functions

\ssection{Why}

We want to relate objects which may involve many symbols to name.
We name a statement which involves an identity.\footnote{Future editions will improve upon this description.}

\ssection{Definition}

An \t{equation} is any statement (see \sheetref{statements}{Statements}) relating two terms by the relation of identity (see \sheetref{identities}{Identities}).
Some authors also call an equation an \t{equality}.
The symbol ``$=$'' is called the (or an) \t{equals sign} or \t{equals symbol}.

\ssection{Variables}

It is regularly the case that we are interested in equations relating all objects of one set to another.
For example:
Let $X$ and $Y$ be sets and let $f: X \to Y$ and $g: X \to Y$.
We may write the logical assertion $(\forall x)(f(x) = g(x))$.
In this case it is understood that $f$ and $g$ are \t{free} names and $x$ is a \t{bound} name (see \sheetref{quantified_statements}{Quantified Statements}).

We will regularly, however, refer to the equation $f(x) = g(x)$ without the quantifier $\forall x$.
In this case, $x$ appears free, but is not.
In other words, in the statement $f(x) = g(x)$, depending on context, $x$ is an implicitly bound name.
This usage is in slight offense to normal English usage.
The name is \textit{bound} (see \sheetref{quantified_statements}{Quantified Statements}) because the use of the particular symbol $x$ is irrelevant.
We may have just as well used the symbol $y$.
In this sense, the name is bound to the quantifier $\forall x$.
In another sense, though, the choice of name is \say{free}, and the name is meant as a placeholder (see \sheetref{names}{Names}).

For these reasons, we introduce terminology for this common case.
The symbol $x$ we call a \t{variable}.
It is a placeholder name, which is bound in the quantified statement.
But the particular choice of name is irrelevant.

\blankpage
