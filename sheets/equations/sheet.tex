
%!name:equations
%!need:functions

\section*{Why}

We name a statement which involves an identity.\footnote{Future editions will modify this statement and sheet.}

\section*{Definition}

An \t{equation} is statement (see \sheetref{statements}{Statements}) relating two terms by the relation of identity (see \sheetref{identities}{Identities}).
Some authors also call an equation an \t{equality}.
The symbol ``$=$'' is called the (or an) \t{equals sign} or \t{equals symbol}.

\section*{Variables}

Let $X$ and $Y$ be sets and let $f,g: X \to Y$.
In the statement
\[
(\forall x)(f(x) = g(x)),
\]
$f$ and $g$ are \t{free} names and $x$ is a \t{bound} name (see \sheetref{quantified_statements}{Quantified Statements}).

For convenience, we often refer to the equation $f(x) = g(x)$ without the quantifier $\forall x$.
In this case, $x$ appears free, but is not.
In this context, the statement $f(x) = g(x)$ has $x$ implicitly bound.
There are two senses here, though.
The first is that $x$ is bound because it is ``subordinate'' to the quantifier $\forall$.
The particular symbol is irrelevant; the symbol $y$ works just as well.
In a second sense, though, the name is ``free,'' as it is a placeholder and the choice of the symbol $x$ does not matter.

We use different terminology for this common case.
In discussing $f(x) = g(x)$, we call the placeholder name $x$ a \t{variable} and we call the names $f$ and $g$ \t{constants}.
The language is meant to convey $f$ and $g$ are \t{fixed} in the present discussion, as indicated by the usual language ``Let $f$ and $g$ ...''.

\section*{Solutions}

We are often interested in finding objects in some set to satisfy an equation.
For example, we are interested in finding an object $\xi  \in X$ to satisfy $f(\xi ) = g(\xi )$.
In this setting we call the variable $\xi $ in the equation an \t{unknown}.

We call an element $\xi  \in X$ a \t{solution} of the equation if $f(\xi ) = g(\xi )$.
We call the set
\[
\Set*{\xi  \in X}{f(\xi ) = g(\xi )}
\]
the \t{solution set}.
If the solution set is non-empty, we say that a solution \t{exists}.
If the solution set is a singleton, we say that the solution is \t{unique}.

We are often interested in solutions which satisfy several equations at once.
For example, we have the equations $f_1(x) = g_1(x)$ and $h(x) = i(x)$ and so on.
We want $x$ to satisfy these.
Here it is \t{set of equations}, \t{simultaneous equations}, or a \t{system of equations}.

\section*{Finding solutions}

We often talk about \t{finding} or \t{searching} for solutions or \t{solving equations}.
We say: ``We want to \textit{find} $x \in X$ to satisfy $f(x) = g(x)$.''
In addition to $f(x) = g(x)$, we may include other statements about $x$.
The language is meant to convey that we are searching for an object which we will name, as a variable, $x$, and we want this object to satisfy the statements.
