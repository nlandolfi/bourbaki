%!name:multivariate_normals
%!need:normal_densities
%!need:multivariate_real_densities
%!need:positive_definite_matrices
%!need:matrix_determinants
%!need:matrix_inverses
%!need:matrix_transpose

\ssection{Why}

We generalize the normal density
to $d$-dimensional space.

\ssection{Definition}

Let $f: \R^d \to \R$ be a density such that $$f(x) = \mgaussiandensity{x}{\mu}{\Sigma}$$ where $\mu \in \R^d$, $\Sigma \in \SymMat{d}$, and $\Sigma \succ 0$.
We call $f$ a \t{multivariate normal density}.
A multivariate normal density with $d = 1$ is a normal density, so we refer to multivariate normal densities as \t{normal densities} without ambiguity.
We frequently use the word normal as a substantive, and refer to \t{normals} when we mean multivariate normal densities.
Some people call a multivariate normal distribution a \t{multivariate gaussian distribution} and speak of \t{gaussians} instead of normals.

We call $\mu$ the \t{mean} and $\Sigma$ the \t{covariance matrix}.
We call $\Sigma^{-1}$ the \t{precision matrix}.

\blankpage
