%!name:multivariate_normals
%!need:normal_densities
%!need:multivariate_real_densities
%!need:positive_definite_matrices
%!need:matrix_determinants
%!need:real_matrix_inverses
%!need:matrix_transpose

\section*{Why}

We generalize the normal density to $d$-dimensional space.

\section*{Definition}

Let $f: \R ^d \to \R $ be a density such that $$f(x) = \mgaussiandensity{x}{\mu }{\Sigma }$$ where $\mu  \in \R ^d$, $\Sigma  \in \mathbfsf{S} ^d$, and $\Sigma  \succ 0$.
We call $f$ a \t{multivariate normal density}.
A multivariate normal density with $d = 1$ is a normal density, so we refer to multivariate normal densities as \t{normal densities} without ambiguity.
We frequently use the word normal as a substantive, and refer to \t{normals} when we mean multivariate normal densities.
Many people call a multivariate normal distribution a \t{multivariate gaussian distribution} and speak of \t{gaussians} instead of normals.
  \ifhmode\unskip\fi\footnote{
We avoid this usage in accordance with the project's policy on historical names.
  }

We call $\mu $ the \t{mean} and $\Sigma $ the \t{covariance matrix}.
We call $\Sigma ^{-1}$ the \t{precision matrix}.

\ssection{Maximum}

The maximum of a normal density is its mean, $\mu  \in \R ^d$.

\blankpage

