%!name:joint_probability_matrices
%!need:real_matrix_rank
%!need:dependent_events

\ssection{Why}
We can characterize the dependence of two events in terms of the rank of a particular matrix.

\ssection{Definition}
Given a probability measure $\mathbfsf{P} : \powerset{\Omega } \to \R $ on the finite set $\Omega $ and two events $A, B \subset \Omega $, the \t{joint probability matrix} of $A$ and $B$ is the matrix
  \[
M = \bmat{
\mathbfsf{P} (A \cap B) & \mathbfsf{P} (A \cap \relcomplement{B}{\Omega }) \\
\mathbfsf{P} (\relcomplement{A}{\Omega } \cap B) & \mathbfsf{P} (\relcomplement{A}{\Omega } \cap \relcomplement{B}{\Omega }) \\
}.
  \]
Since $B$ and $\relcomplement{B}{\Omega }$ partition $\Omega $, use the the total law of probability to express is $\mathbfsf{P} (A) = \mathbfsf{P} (A \cap B) + \mathbfsf{P} (A \cap \relcomplement{B}{\Omega })$.
Likewise, $\mathbfsf{P} (\relcomplement{A}{\Omega }) = \mathbfsf{P} (\relcomplement{A}{\Omega } \cap B) + \mathbfsf{P} (\relcomplement{A}{\Omega } \cap \relcomplement{B}{\Omega })$.
For these reasons, the sum of the entries of the first row of $M$ is $\mathbfsf{P} (A)$ and the sum of the entries of the second row of $M$ is $\mathbfsf{P} (\relcomplement{A}{\Omega })$.
The events $A$ and $B$ are independent if and only if $\rank(M) = 1$

\blankpage
