
%!name:entire_functions
%!need:complex_analytic_functions
%!refs:yellow/IX/4

\section*{Definition}

An \t{entire function} is a complex function $f: \C  \to \C $ which is analytic for all $z \in \C $.

\blankpage
\sbasic
%%%% MACROS %%%%%%%%%%%%%%%%%%%%%%%%%%%%%%%%%%%%%%%%%%%%%%%

\newcommand{\PM}{\mathbf{P}}

%%%%%%%%%%%%%%%%%%%%%%%%%%%%%%%%%%%%%%%%%%%%%%%%%%%%%%%%%%%

%%%% MACROS %%%%%%%%%%%%%%%%%%%%%%%%%%%%%%%%%%%%%%%%%%%%%%%

\newcommand{\PM}{\mathbf{P}}

%%%%%%%%%%%%%%%%%%%%%%%%%%%%%%%%%%%%%%%%%%%%%%%%%%%%%%%%%%%

%%%% MACROS %%%%%%%%%%%%%%%%%%%%%%%%%%%%%%%%%%%%%%%%%%%%%%%

\newcommand{\PM}{\mathbf{P}}

%%%%%%%%%%%%%%%%%%%%%%%%%%%%%%%%%%%%%%%%%%%%%%%%%%%%%%%%%%%

%%%% MACROS %%%%%%%%%%%%%%%%%%%%%%%%%%%%%%%%%%%%%%%%%%%%%%%

% use \set{stuff} for { stuff }
% use \set* for autosizing delimiters
\DeclarePairedDelimiter{\set}{\{}{\}}

% use \Set{a}{b} for {a | b}
% use \Set* for autosizing delimiters
\DeclarePairedDelimiterX{\Set}[2]{\{}{\}}{#1 \nonscript\;\delimsize\vert\nonscript\; #2}

% use \powerset{A} for power set of A
\newcommand{\powerset}[1]{2^{#1}}

\renewcommand{\emptyset}{\varnothing}

\newcommand{\SA}{\mathcal{A}}
\newcommand{\SB}{\mathcal{B}}
\newcommand{\SC}{\mathcal{C}}
\newcommand{\SD}{\mathcal{D}}
\newcommand{\SE}{\mathcal{E}}
\newcommand{\SF}{\mathcal{F}}
\newcommand{\SG}{\mathcal{G}}
\newcommand{\SH}{\mathcal{H}}
\newcommand{\SI}{\mathcal{I}}
\newcommand{\SJ}{\mathcal{J}}
\newcommand{\SK}{\mathcal{K}}
\newcommand{\SL}{\mathcal{L}}

%%%%%%%%%%%%%%%%%%%%%%%%%%%%%%%%%%%%%%%%%%%%%%%%%%%%%%%%%%%

%%%% MACROS %%%%%%%%%%%%%%%%%%%%%%%%%%%%%%%%%%%%%%%%%%%%%%%

\newcommand{\PM}{\mathbf{P}}

%%%%%%%%%%%%%%%%%%%%%%%%%%%%%%%%%%%%%%%%%%%%%%%%%%%%%%%%%%%

%%%% MACROS %%%%%%%%%%%%%%%%%%%%%%%%%%%%%%%%%%%%%%%%%%%%%%%

\newcommand{\PM}{\mathbf{P}}

%%%%%%%%%%%%%%%%%%%%%%%%%%%%%%%%%%%%%%%%%%%%%%%%%%%%%%%%%%%

%%%% MACROS %%%%%%%%%%%%%%%%%%%%%%%%%%%%%%%%%%%%%%%%%%%%%%%

\newcommand{\PM}{\mathbf{P}}

%%%%%%%%%%%%%%%%%%%%%%%%%%%%%%%%%%%%%%%%%%%%%%%%%%%%%%%%%%%

%%%% MACROS %%%%%%%%%%%%%%%%%%%%%%%%%%%%%%%%%%%%%%%%%%%%%%%

\newcommand{\PM}{\mathbf{P}}

%%%%%%%%%%%%%%%%%%%%%%%%%%%%%%%%%%%%%%%%%%%%%%%%%%%%%%%%%%%

%%%% MACROS %%%%%%%%%%%%%%%%%%%%%%%%%%%%%%%%%%%%%%%%%%%%%%%

\newcommand{\PM}{\mathbf{P}}

%%%%%%%%%%%%%%%%%%%%%%%%%%%%%%%%%%%%%%%%%%%%%%%%%%%%%%%%%%%

%%%% MACROS %%%%%%%%%%%%%%%%%%%%%%%%%%%%%%%%%%%%%%%%%%%%%%%

\newcommand{\PM}{\mathbf{P}}

%%%%%%%%%%%%%%%%%%%%%%%%%%%%%%%%%%%%%%%%%%%%%%%%%%%%%%%%%%%

%%%% MACROS %%%%%%%%%%%%%%%%%%%%%%%%%%%%%%%%%%%%%%%%%%%%%%%

\newcommand{\PM}{\mathbf{P}}

%%%%%%%%%%%%%%%%%%%%%%%%%%%%%%%%%%%%%%%%%%%%%%%%%%%%%%%%%%%

\sstart
\stitle{Natural Numbers}

\ssection{Why}

We want to define the natural numbers.
TODO: better why
%We want to count, forever.

\ssection{Definition}

The \t{successor} of a set is the union of the set with the singleton whose element is the set.
This definition holds for any set, but is of interest only for the sets which will be defined in this sheet.

These sets are the following (and their successors):
\t{One} is the successor of the empty set.
\t{Two} is the successor of one.
\t{Three} is the successor of two.
\t{Four} is the successor of three.
And so on; using the English language in the usual manner.

Can this be carried on and on?
We will say yes.
We will say that there exists a set which contains one and contains the successor of each of its elements.
So, this set contains one.
Since it contains one, it contains two.
Since it contains two, it contains three.
And so on.
We call this assertion the \t{axiom of infinity}.

A set is a \t{successor set} if it contains one and if it contains the successor of each of its elements.
In these words, the axiom of infinity asserts the existence of a successor set.
We want this set to be unique.
So we have a successor set.
By the axiom of specification, the intersection of all the successor sets included in this first successor set exists.
Moreover, this intersection is a successor set.
Even more, this intersection is unique.
For this, take a second successor set.
Its intersection with the first successor set is contained in the first successor set.
Thus, this intersection of two sets is one of the successor sets contained in the first set, and so, is contained in the intersection of all such sets.
So then, that first intersection is contained in second intersection of two sets, which is, of course, contained in the second successor set.
In other words, we start with a successor set.
Use it to construct a succesor set contained in it, in such a way that every other successor set also contains this successor set so constructed.
The axiom of extension guarantees that this intersection, which is a successor set contained in every other successor set, is unique.

A \t{natural number} or \t{number} or \t{natural} is an element of this minimal successor set.
The \t{set of natural numbers} or \t{natural numbers} or \t{naturals} or \t{numbers} is the minimal successor set.
% We are saying, in the language of sets, that the essence
% of counting is starting with one and adding one repeatedly.

% \begin{prop}
% The intersection of every nonempty family of successor sets is a successor set.
% \end{prop}
%
% So the intersection of all the successor sets is a successor set.
% Since the intersection of any two sets is contained in the two sets, the intersection of the family of successor sets is contained in every other succesor set.
%
%
% The
% successor function
% is the correspondence between elements
% of the natural numbers and their successors.
% Its domain and codomain is the set of natural numbers.
% It is a one-to-one correspondence.
% It is not onto, however, since the element one
% has no successor.

\ssubsection{Notation}

Let $x$ be a set.
We denote the successor of $x$ by $\ssuc{x}$.
We defined it by
$$
  \ssuc{x} := x \union \set{x}
$$

We denote one by $1$.
We denote two by $2$.
We denote three by $3$.
We denote four by $4$.

We denote the set of natural numbers by $\N$, a mnemonic for natural.
We often denote elements of $\N$ by $n$, a mnemonic for number, or $m$, a letter close to $n$.
\strats
