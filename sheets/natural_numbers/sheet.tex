
% 
%Notes on starting at 1; Spivak does in Calculus

\section*{Why}

What are numbers?
We want to count, forever.
Does a set exist which contains zero, and one, and two, and three, and all the rest?

\section*{Definition}

In \sheetref{successor_sets}{Successor Sets}, we said ``and we continue as usual using the English language...'' in our definition of zero, and one and two and three.
Can this really be carried on and on?
We will say yes.
We will say that there exists a set which contains zero and contains the successor of each of its elements.

\begin{principle}[Natural Numbers]
A set which contains 0 and contains the successor of each of its elements exists.
\end{principle}

This principle is sometimes called the \t{principle of infinity} (or \t{axiom of infinity}).

We want this set to be unique.
The principle says one successor set exists, but not that it is unique.
To see that it is unique, notice that the intersection of a nonempty family of successor sets is a successor set.\footnote{This account will be expanded in future editions.}
Consider the intersection of the family of all successor sets.
The intersection is nonempty by the principle of infinity (see \sheetref{intersection_of_empty_set}{Intersection of Empty Set}for this subtlety).
The principle of extension guarantees that this intersection, which is a successor set contained in every other successor set, is unique.
We summarize:

\begin{proposition}[Minimal Successor Set]

\label{natural_numbers:proposition:omega}There exists a unique smallest successor set.
\end{proposition}

The \t{set of natural numbers} is the minimal successor set.
A \t{natural number} (or \t{number}, \t{natural}) is an element of this minimal successor set.

\subsection*{Notation}


% todo fix 
%    Proposition~\ref{natural_numbers:proposition:omega} both refs and the underscore in  natural_numbers
%    which gets escaped in tex
%    

We denote the unique smallest successor set by $\omega $.\footnote{We use this notation to follow many authorities on the subject, and to meet the exigencies of time in producing this first edition.
Future editions are likely to rework the treatment.}
We denote the set of natural numbers without 0 by $\N  $, a mnemonic for natural.
In other words $\N   = \omega  - \set{0}$.
We often denote elements of $\omega $ or $\N  $ by $n$, a mnemonic for number, or $m$, the letter before $m$ in the conventional ordering of the \sheetref{letters}{Latin alphabet}(see \sheetref{letters}{Letters}).

We denote the natural numbers up to $n$ by $\upto{n}$.
Recall that $n$ \textit{is a set}.
In other words, we have defined $n$ so that $n - \set{0} = \upto{n}$.

%% We are saying, in the language of sets, that the essence
%% of counting is starting with one and adding one repeatedly.
%% \begin{proposition}
%% The intersection of every nonempty family of successor sets is a successor set.
%% \end{proposition}
%%
%% So the intersection of all the successor sets is a successor set.
%% Since the intersection of any two sets is contained in the two sets, the intersection of the family of successor sets is contained in every other succesor set.
%%
%%
%% The
%% successor function
%% is the correspondence between elements
%% of the natural numbers and their successors.
%% Its domain and codomain is the set of natural numbers.
%% It is a one-to-one correspondence.
%% It is not onto, however, since the element one
%% has no successor.
%%%!name:zero
%%%!need:equation_solutions
%%%!need:natural_numbers
%%
%% \ssection{Why}
%%
%% If I am holding
%% three pebbles, and I have three
%% in my left hand, how many
%% might I have in my right hand?
%% None, of course!
%%
%% In the notation we have developed
%% we find solutions of
%% \[
%%   3 + n = 3,
%% \]
%% where $n$ is a natural number.
%% Unfortunately, any natural number
%% added to three is a number different
%% than three.
%% So there is no natural number
%% such that this equation holds.
%%
%% How can we expand our algebra
%% so that we can express this
%% new, but common, situation in
%% our language?
%%
%% \ssection{Definition}
%%
%% We consider a superset
%% of the natural numbers with
%% one additional element.
%% We call this new elemenet
%% \t{zero}
%% and we call this new set
%% the
%% \t{natural numbers with zero}.
%%
%% \ssubsection{Extending Arithmetic}
%%
%% We extend the algebra on
%% the natural numbers to an
%% algebra on the natural numbers
%% with zero.
%%
%% We define an extension of addition,
%% also called
%% \t{addition}.
