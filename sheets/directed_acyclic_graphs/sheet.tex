
\section*{Why}

If a directed graph has no cycles, then it has a nice property.\footnote{Future editions will expand this vague introduction.}

\section*{Definition}

Directed and acyclic graphs (or \t{directed acyclic graphs}, \t{DAGs}) have partial ordering property on vertices.
We call a vertex $s$ an \t{ancestor} of a vertex $u$ if there is a directed path from $s$ to $u$; c.f. .

\section*{Partial Order}

We call a vertex $s$ and \t{ancestor} of a vertex $t$ if there is a directed path from $s$ to $t$.
The relation $R$ defined by $(s, t) \in R$ if $s$ is an ancestor of $t$ is a partial order.

Clearly, every subgraph induced on a directed acyclic graph is a directed acyclic graph.

\begin{proposition}
Let $(V, E)$ be a directed acyclic graph. Then there exists a vertex $v \in V$ which is a source and a vertex $w \in V$ which is a sink.
\begin{proof}There exists a directed path of maximum length. It must start at a source and end at a sink.\footnote{Future editions will expand.}\end{proof}
\end{proposition}

\subsection*{Topological numbering}

We can choose a total ordering that is consistent with the partial order of ancestry.

A \t{topological numbering} (or \t{topological sort}, \t{topological ordering}) of a directed graph $(V, E)$ is a numbering $\sigma : \set{1, \dots , \num{V}} \to V$ satisfying
\[
(v, w) \in E \implies \inv{\sigma }(v) < \inv{\sigma }(w).
\]\footnote{Future editions will further explain this concept.}

\begin{proposition}
There exists a topological sort for every acyclic graph.
\end{proposition}

\begin{proof}Let $(V, F)$ be a directed acyclic graph.
There exists a source vertex, $v_1$.
Set $\sigma (1) = v_1$.
Take the subgraph induced by $V - \set{v_1}$.
It is directed acyclic, and so has a source vertex, $v_2$.
Set $\sigma (2) = v_2$.
Continue in this way.\footnote{Future editions will clarify and expand.}\end{proof}
\blankpage