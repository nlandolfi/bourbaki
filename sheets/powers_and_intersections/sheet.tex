
%!name:powers_and_intersections
%!need:set_powers
%!need:set_intersections
%!refs:paul_halmos/naive_set_theory/section_05

\section*{Why}

How does the power set relate to an intersection?

\subsection*{Notation preliminaries}

First, if we have a set of sets---denote it $\mathcal{C} $---and all members are subsets of a fixed set---denote it $E$---then the set of sets is a subset of $\powerset{E}$.
In this case, we can write
\[
\bigcap \Set{X \in \powerset{E}}{x \in \mathcal{C} }
\]
Which is a sort of justification for the notation
\[
\bigcap_{X \in \mathcal{C} } X.
\]

\section*{Basic properties}

Here are some basic interactions between the powerset and intersections.\footnote{Future editions will expand on these propositions and provide accounts of them.}

\begin{proposition}
$\powerset{A} \cap  \powerset{F} = \powerset{(A \cap  F)}$
\end{proposition}

\begin{proposition}
$\bigcap_{X \in \mathcal{A} } \powerset{A} = \powerset{(\bigcap_{X \in \mathcal{A} } A)}$
\end{proposition}

\begin{proposition}
$\bigcap_{X \in \powerset{E}} X = \emptyset$
\end{proposition}

\blankpage