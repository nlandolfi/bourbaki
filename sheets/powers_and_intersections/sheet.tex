%!name:powers_and_intersections
%!need:set_powers
%!need:set_intersections
%!refs:paul_halmos/naive_set_theory/section_05

\ssection{Why}

How does the power set relate to an intersection?

\ssubsection{Notation Preliminaries}

First, if we have a set of sets---denote it $\CC$---and all members are subsets of a fixed set---denote it $E$---then the set of sets is a subset of $\powerset{E}$.
In this case, we can write
\[
  \bigcap \Set{X \in \powerset{E}}{x \in \CC}
\]
Which is a sort of justification for the notation
\[
  \bigcap_{X \in \CC} X.
\]

\ssection{Basic Properties}

Here are some basic interactions between the powerset and intersections.\footnote{Future editions will expand on these propositions and provide accounts of them.}

\begin{proposition}
  $\powerset{A} \cap \powerset{F} = \powerset{\parens{A \cap F}}$
\end{proposition}

\begin{proposition}
  $\bigcap_{X \in \CA} \powerset{A} = \powerset{\parens{\bigcap_{X \in \CA} A}}$
\end{proposition}

\begin{proposition}
  $\bigcap_{X \in \powerset{E}} X = \emptyset$
\end{proposition}


\blankpage
