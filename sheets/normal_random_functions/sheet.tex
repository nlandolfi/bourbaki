%!name:normal_random_functions
%!need:random_functions
%!need:multivariate_normals

\ssection{Why}

We want to have random functions with simple marginal distributions.

\ssection{Definition}

A \t{normal random function} (or \t{normal process} or \t{gaussian process})\footnote{The choice of \say{normal} is a result of the Bourbaki project's convention to eschew historical names. Though here, as in \sheetref{multivariate_normals}{Multivariate Normals} the language of the project is nonstandard.} is a real-valued random function whose family of random variables has the property that the image of any finite set of indices is a normal random vector.


\ssubsection{Notation}

Let $(\Omega, \CA, \PM)$ be a probability space and $A$ a set.
Let $x: \Omega \to (A \to \R)$ be a random function with family $y: A \to (\Omega \to \R)$.

Then $x$ is a normal random function if there exists $m: I \to \R$ and positive definite $k: I \times I \to \R$ with the property that if $J \subset I$, $\abs{J} = d$, then $x_J \sim \mathcal{N}(m(J), k(J \times J))$.
In other words, $x_J: \Omega \to \R^d$ is a Gaussian random vector.
We call $m$ the \t{mean function} and $k$ the \t{covariance function}.\footnote{Future editions will change this description.}

\ssubsection{Random function interpretation}

Many authorities discuss a normal random function as \say{putting a prior} on a \say{space} (see, for example, \sheetref{real_function_space}{Real Function Space}) of functions.
One samples functions by drawing an outcome $\omega \in \Omega$, and then defining the sample $f: I \to \R$ by $f(i) = x(i)(\omega)$.

\ssubsection{Multivariate normal special case}

If the index set is finite, and can be ordered, then the normal random function is a multivariate normal random vector.

%\blankpage
