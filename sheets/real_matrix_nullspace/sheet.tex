
%!name:real_matrix_nullspace
%!need:real_matrices
%!need:real_vector_angles
%!need:orthogonal_real_subspaces

\section*{Definition}

The \t{nullspace} (or \t{kernel}) of a matrix $A \in \R ^{m \times  n}$ is the set
\[
\Set{x \in \R ^n}{Ax = 0}.
\]
It is the set of vectors mapped to zero by $A$.
Equivalently, it is the set of vectors orthogonal to the rows of $A$.

\subsection*{Notation}

We denote the nullspace of $A \in \R ^{m \times  n}$ by $\null(A) \subset \R ^{n}$.
Some authors denote the nullspace of $A$ by $\mathcal{N}(A)$.

\subsection*{A subspace}

The nullspace of a matrix is a subspace (this justifies the terminology null\textit{space}!).
There are a few routes to see this.

The first is direct.
If $w, z \in \null(A)$, then $Aw = 0$ and $Az = 0$.
So then $A(w + z) = Aw + Az = 0$.
So $\null(A)$ is closed under vector addition.
Also $A(\alpha w) = \alpha (Aw) = 0$ for all $\alpha  \in \R $.
[In particular $A0 = 0$, so $0 \in \null(A)$; i.e., $\null(A)$ contains the origin.]
So $\null(A)$ is closed under scalar multiplication.

The second is by thinking about orthogonal complements.
Second, we can view the $\null(A)$ as the set of vectors orthgonal to all the rows of $A$.
In other words, $\null(A) = \set{\tilde{a}_1, \dots , \tilde{a}_m}^\perp $.
The orthogonal complement of any set is a subspace (see \sheetref{orthogonal\_real\_subspaces}{Orthogonal Real Subspaces}).

\section*{Ambiguity in solutions}

Suppose we have a solution to the system of linear equation with data $(A, y)$.
In other words, we have a vector $x \in \R ^n$ so that $y = Ax$.
If we have a vector $z \in \null(A)$, then $x + z$ is also a solution to the system $(A, y)$, since
\[
A(x + z) = Ax + Az = Ax + 0 = y
\]
Conversely, suppose there were another solution $\tilde{x} \in \R ^{n}$ to the system $(A, y)$.
Then $y = Ax = A\tilde{x}$, so
\[
0 = y - y = Ax - A\tilde{x} = A(x - \tilde{x}).
\]
Consequently, $(x - \tilde{x}) \in \null(A)$, and so $\tilde{x}$ is the solution $x$ plus some vector in the null space of $A$.
Consequently we are interested in whether $A$ has vectors in its nullspace.

\subsection*{Zero nullspace}

The origin $0$ is always in the nullspace of $A$.
However, this vector does not mean that we can find different solutions, since $x + 0 = x$ for all $x \in \R ^n$.
If, on the other hand, there is a nonzero vector $z \in \null(A)$, then $x + z \neq x$, and $x+z$ is a solution for $(A, y)$.
We think about $A$ as a function from $\R ^n$ to $\R ^m$.
In the case that there is a nonzero element in the nullspace, $A$ maps different vectors to the same vector.
Here, $x$ and $x + z$ both map to $y$.
In this case, the function is \textit{not invertible}, because it is not one-to-one.
If, however, zero is the only element of the null space, the function is one-to-one.
So call $A$ \t{one-to-one} if $\null(A) = 0$.

\subsection*{Equivalent statements}

A matrix $A \in \R ^{m \times  n}$ is \t{one-to-one} if the linear function $f: \R ^n \to \R ^m$ defined by $f(x) = Ax$ is one-to-one.
In this case, if there exists $x \in \R ^n$ so that $y = Ax$, then there is only one such $x$.
Different elements in $\R ^n$ map to different elements in $\R ^m$.

\blankpage