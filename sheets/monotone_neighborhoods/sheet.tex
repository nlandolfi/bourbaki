
\section*{Why}
\footnote{Future editions will include. This sheet is needed, for example, in discussing perfect elimination orderings.}
\section*{Definition}

The \t{higher adjacency set} or \t{higher neighborhood} of a vertex $v$ in an ordered undirected graph is all vertices in the neighborhood of $v$ whose index is greater the $v$.
Similarly, the \t{lower adjacency set} or \t{lower neighborhood} of $v$ is all vertices in the neighborhood of $v$ whose index is less the $v$.
We call these \t{monotone neighborhoods}.

The \t{higher degree} of a vertex is the size of the higher adjacency set and the \t{lower degree} of a vertex is the size of its lower adjacency set.

The \t{closed monotone neighborhoods} are the \t{closed higher adjacency set}, the higher adjacency set of $v$ union with the singleton $\set{v}$ and the \t{closed lower adjacency set}, the lower adjacency set of $v$ union with the singleton $\set{v}$.

\subsection*{Notation}

We denote the higher neighborhood of $v$ by $\adjh(v)$ and the lower neighborhood by $\adjl(v)$.

\section*{Visualization}

%% \begin{figure}
%%   \centering
%%   \begin{subfigure}{0.4\textwidth}
%%   \includegraphics[width=1\textwidth]{graphics_included/ordered_undirected_graph}
%%   \caption{Ordered undirected graph.}
%%   \end{subfigure}
%%   \begin{subfigure}{0.4\textwidth}
%% \[
%% \barray{
%%   a     &   & & & \\
%%   \bullet & c & & & \\
%%    & \bullet & d & & \\
%%    \bullet & & \bullet & b & \\
%%    \bullet & \bullet & \bullet & \bullet & e
%% }
%% \]
%%     \caption{Tabular representation.}
%%   \end{subfigure}
%% \end{figure}

\begin{center}\includegraphics[width=0.50\textwidth]{./graphics/orderedundirectedgraph.pdf}\end{center}
%    <div data-littype='paragraph'>
%     <div data-littype='run'> \begin{figure} </div>
%     <div data-littype='run'> \centering </div>
%     <div data-littype='run'> ∈cludegraphics[width=0.3\textwidth]{graphics_included/ordered_undirected_graph} </div>
%     <div data-littype='run'> \caption{Ordered undirected graph.} </div>
%     <div data-littype='run'> \end{figure} </div>
%    </div>
%    


To help think about the monotone neighborhoods of the graph we visualize ordered graphs as triangular arrays with vertices along the diagonal and a bullet in row $i$ and column $j$ of the array if $i > j$ and the vertices $\sigma (i)$ and $\sigma (j)$ are adjacent.

An example is shown below for the ordered undirected graph in the figure (to understand this visualization, see \sheetref{ordered_undirected_graphs}{Ordered Undirected Graphs}) we use the
%% Let $(V, E)$ be an undirected graph with $V = \set{a,b,c,d,e}$ and
%%   $E = \set*{\set{a, b}, \set{a, c}, \set{a, e}, \set{b, d}, \set{b, e}, \set{c, d}, \set{c, e}, \set{d,e}}$.
%% Let $\sigma: \set{1, \dots, 5} \to V$ be an ordering with
%% \[
%%   \sigma(1) = a \quad  \sigma(2) = c \quad \sigma(3) = d \quad \sigma(4) = b \quad \sigma(e) = 5.
%% \]
%% An example is show below.

\[
\barray{
a & & & & \\
\bullet & c & & & \\
& \bullet & d & & \\
\bullet & & \bullet & b & \\
\bullet & \bullet & \bullet & \bullet & e
}
\]

In this array representation the higher and lower neighborhoods are easily identified.
The indices of the elements of $\adjh(v)$ are the column indices of the entries in row $\inv{\sigma }(v)$ of the array.
For example, $\inv{\sigma }(d) = 3$, and the only bullet entry in row three is $c$ so $\adjn(d) = \set{c}$.
Likewise, $\adjn(c) = \set{a}$.
And so on.
Similalry, the indices of $\adjp(v)$ are the row indices of the entries in column $\inv{\sigma }(v)$.
For example, $\inv{\sigma }(d)$ is $3$, and there are indices 4 and 5 corresponding to $b$ and $e$ so $\adjp(d) = \set{b, e}$.
Likewise, $\adjp(c) = \set{d, e}$.

For this reason, we use the notation $\col(v)$ and $\row(v)$ for the closed upper and lower neighborhoods.
So $\col(v) = \adjp(v) \union \set{v}$ and $\row(v) = \adjn(v) \union \set{v}$.

\blankpage
%macros.tex
%\newcommand{\row}{\mathword{row}}
%\newcommand{\col}{\mathword{col}}
