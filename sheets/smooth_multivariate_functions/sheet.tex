
\section*{Why}

What is the natural generalization of a smooth function to functions defined on sets of $\R ^k$.\footnote{Future editions will expand.}

\section*{Definition}

Let $U \subset \R ^d$ be an open set (see \sheetref{real_open_sets}{Real Open Sets}).
A function $f: U \to \R $ is \t{smooth} if all its partial derivatives exists and are continuous.

More generally, let $X \subset \R ^d$.
A function $f: X \to R$ is \t{smooth} if there exists an open set $U \subset \R ^d$ and a smooth $F: U \to \R $ so that $F(x) = f(x)$ for all $x \in U \cap  X$.

\subsection*{Example}

The identity map is smooth.
In other words, let $f: \R ^d \to \R $ be so that $X \subset \R ^d$.
Then $f: X \to \R $ s

\section*{Properties}

\begin{proposition}
The composition of two smooth functions is smooth.
\end{proposition}

\blankpage