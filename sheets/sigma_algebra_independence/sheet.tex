%!name:sigma_algebra_independence
%!need:sigma_algebras
%!need:dependent_events

\ssection{Why}
  \ifhmode\unskip\fi\footnote{
Future editions will include this.
  }

\ssection{Definition}

A \t{sub-$\sigma $-algebra} (\t{sub-sigma-algebra}) is a subset of a sigma algebra which is itself a sigma algebra.

A collection of sub sigma algebras is \t{independent} if the measure
of every set which is an intersection of distinguished sets from the sub sigma algebras is the product of the individual measures of the sets.
An arbitrary (not finite) collection of sigma algebras is \t{independent}
if any finite sub-collection is independent.

\ssubsection{Notation}

Let $(X, \SA, \mu )$ be a probability space.
Let $Y_1, \dots, Y_n$ be sub-sigma-algebras of $X$.
Then $\set{Y_1, \dots, Y_n}$ are independent if for all $A_1 \in Y_1, A_2 \in A_2, \dots, A_n \in Y_n$,
  \[
\mu (\cap_{i = 1}^{n} A_i) = \prod_{i = 1}^{n} \mu (A_i).
  \]
In this case we write $Y_1 \perp Y_2 \perp \cdots \perp Y_n$.

\blankpage
