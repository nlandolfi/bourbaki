
%!name:entire_functions
%!need:complex_analytic_functions
%!refs:yellow/IX/4

\section*{Definition}

An \t{entire function} is a complex function $f: \C  \to \C $ which is analytic for all $z \in \C $.

\blankpage
\sbasic

\sinput{../sets/macros.tex}
\sinput{../set-builder_notation/macros.tex}

\sstart

\stitle{Set-Builder Notation}

\ssection{Why}

We specify a
subset of a known
set via a property.

\ssection{Definition}


Let $A$ be a nonempty set.
We use the notation
\[
  \Set*{a \in A}{\text{------}\;}
\]
to indicate a subset of $A$ that
satisfies some property specified
after the $\mid$.
We read the symbol $\mid$ aloud as
\say{such that.}
We read the whole notation aloud as
\say{a in A such that...}

We call the notation
\ct{set-builder notation}{setbuildernotation}.
Set-builder notation avoids enumerating
elements.

\ssection{Example}

For example, let $L$
be the set of Latin letters
and $V$ the set of Latin
vowels.
A first notation for $V$ is
$\set{\text{a},\text{e},\text{i},\text{o},\text{u}}$.
A second notation for $V$ is
$\Set{l \in L}{l \text{ is a vowel}}$.
We may prefer the second, in cases
when it saves time.
This notation is really indispensable for
sets which have many members, too many
to reasonably write down.

%We develop herein a language
%for specifying things by either
%listing them explicitly or
%by listing their defining properties.

\strats
