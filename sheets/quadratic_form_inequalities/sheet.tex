
%!name:quadratic_form_inequalities
%!need:symmetric_real_matrix_eigenvalues
%!need:quadratic_forms

\section*{Why}

How big can a quadratic form be?
How small?

\section*{Result}

\begin{proposition}
Suppose $A \in \mathbfsf{S} ^n$ has real eigenvalues $\lambda _1 \geq \lambda _2 \geq \cdots \geq \lambda _n$.
Then
\[
\lambda _n x^\top x \leq x^\top  A x \leq \lambda _1 x^\top x,
\]
for all $x \in \R ^n$.
\end{proposition}

\begin{proof}Since $A$ is symmetric, there exists an orthogonal matrix $Q \in \R ^{n \times  n}$ with $A = Q\Lambda  Q^\top $.
Express
\[
\begin{aligned}
x^\top  A x = x^\top  Q \Lambda   Q^\top  x &= (Q^\top x)^\top  \Lambda   (Q^\top  x) \\
&= \sum_{i = 1}^{n} \lambda _i (q_i^\top  x)^2 \\
&= \lambda _1 \sum_{i = 1}^n (q_i^\top x) = \lambda _1 \norm{Q^\top  x}^2 = \lambda _1 \norm{x}^2.
\end{aligned}
\]
Similarly,
\[
\begin{aligned}
x^\top  A x &= \sum_{i = 1}^{n} \lambda _i (q_i^\top  x)^2 \\
&\geq \lambda _n \sum_{i = 1}^n (q_i^\top x) = \lambda _n \norm{Q^\top  x}^2 = \lambda _n \norm{x}^2.
\end{aligned}
\]\end{proof}
\subsection*{Notation}

For this reason, it is common to order the eigenvalues of $A \in \mathbfsf{S} ^n$ by magnitude with $\lambda _1 \geq \lambda _2 \geq \cdots \geq \lambda _n$.
$\lambda _1$ is sometimes denoted $\lambda _{\max}$ and $\lambda _n$ is sometimes denoted $\lambda _{\min}$.

\section*{Optimization implication}

If $z = \alpha  x$, then $z^\top  A z = \alpha ^2 x^\top  A x$.
Consider finding $x \in \R ^n$ to maximize
\[
\begin{aligned}
\text{ maximize } & \quad x^\top  A x \\
\text{ subject to } & \quad \norm{x} = 1.
\end{aligned}
\]
Since the objective is $x^\top  A x \leq \lambda _1$ for all $x \in \R ^n$ with $\norm{x} = 1$, a solution of this problem is the eigenvector $q_1 \in \R ^n$ corresponding to $\lambda _1$.
In other words, these inequalities are tight.