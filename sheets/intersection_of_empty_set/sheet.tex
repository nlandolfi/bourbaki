
%!name:intersection_of_empty_set
%!need:set_intersections
%!need:pair_unions

\section*{Why}

We only define set intersections for nonempty sets of sets.
Why?

\section*{Discussion}

Which objects are specified by the sentence $(\forall x \in \varnothing)(x \in X)$?
Well, since no objects fail to satisfy the statement,\footnote{Future editions will offer an account of this.}
the sentence specifies all objects.
So in other words, the condition we used to define set intersections ( \sheetref{set_intersections}{Set Intersections}) specifies the \say{set of everything}.
In order to maintain other more desirable set principles like selection, we have said that such a set does not exist (see \sheetref{set_specification}{Set Specification}).

If, however, all sets under consideration are subsets of one paticular set---denote it $E$---then we can define intersections as follows.
Let $\mathcal{C} $ be a possibly nonempty collection of sets
    \[
\bigcap \mathcal{C}  = \Set{X \in E}{(\forall X \in \mathcal{C} )(x \in X)}.
    \]
This definition agrees with that given in \sheetref{set_intersections}{Set Intersections}.
In particular, it is the intersection of the set $\mathcal{C} \cup \set{E}$

\section*{Another definition}

This begs the following question.
Why not define intersections by selecting from the union.
Let $\mathcal{A} $ be a possibly nonempty set of sets.
Then define:
    \[
\bigcap \mathcal{A}  = \Set{x \in \bigcup \mathcal{A} }{(\forall A \in \mathcal{A} )(x \in A)}.
    \]
If $\mathcal{A} $ is empty, so is $\bigcup \mathcal{A} $ and then there are no elements in the set to select from so $\bigcap \mathcal{A} $ is empty.
This does not agree with the previous definitions for the empty set, but does for all other sets of sets.
For these reasons, the intersection of the empty set is delicate.\footnote{Future editions will expand on the preference for the former definition.}


