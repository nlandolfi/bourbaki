%!name:intersection_of_empty_set
%!need:set_intersections
%!need:pair_unions

\s{Why}

We only define set intersections for nonempty sets of sets.
Why?

\s{Discussion}

Which objects are specified by the sentence $(\forall x \in \emptyset)(x \in X)$?
Well, since no objects fail to satisfy the statement,\footnote{Future editions will offer an account of this.} the sentence specifies all objects.
So in other words, the condition we used to define set intersections (\sheetref{set_intersections}{Set Intersections}) specifies the \say{set of everything}.
In order to maintain other more desirable set principles like selection, we have said that such a set does not exist (see \sheetref{set_specification}{Set Specification}).

If, however, all sets under consideration are subsets of one paticular set---denote it $E$---then we can define intersections as follows.
Let $\CC$ be a possibly nonempty collection of sets
\[
  \bigcap \CC = \set{X \in E}{(\forall X \in \CC)(x \in X)}.
\]
This definition agrees with that given in \sheetref{set_intersections}{Set Intersections}.
In particular, it is the intersection of the set $\CC \union \set{E}$


\s{Another definition}

This begs the following question.
Why not define intersections by selecting from the union.
Let $\CA$ be a possibly nonempty set of sets.
Then define:
\[
  \bigcap \CA = \Set{x \in \bigcup \CA}{(\forall A \in \CA)(x \in A)}.
\]
If $\CA$ is empty, so is $\bigcup \CA$ and then there are no elements in the set to select from so $\bigcap\CA$ is empty.
This does not agree with the previous definitions for the empty set, but does for all other sets of sets.

For these reasons, the intersection of the empty set is a delicate thing.\footnote{Future editions are likely to expand on, and perhaps mention why we prefer the former definition.}
