
\section*{Why}

We want to solve linear equations.

\section*{Example}

Suppose we want to find $x_1, x_2 \in \R $ to satisfy the system of linear equations
\[
\begin{aligned}
3x_1 + 2x_2 &= 10, \text{ and} \\
6x_1 + 5x_2 &= 20, \\
\end{aligned}
\]
with constants $((3, 2), 10)$ and $((6, 4), 20)$.

We can associate to the first equation an \t{equation for} $x_1$ \t{in terms of} $x_2$.
We call this \t{solving the first equation for $x_1$}.
\[
3x_1 + 2x_2 = 10 \iff 3x_1 = 10 - 2x_2 \iff x_1 = (1/3)(10 - 2x_2).
\]
Define $f_1: \R  \to \R $ as $f_1(y) = (1/3)(10 - 2y)$.
Then $3x_1 + 2x_2 = 10$ if and only if $x_1 = f_1(x_2)$.
We have written $x_1$ \t{as a function} of $x_2$ and obtained a new equation.
The equation is not linear, however, as $f_1$ is not linear.

Using the equation for $x_1$ in terms of $x_2$, we can substitute this equation into our second linear equation.
The two linear equations hold if and only if
\[
6f(x_2) + 5x_2 = 20 \iff 20 - 4x_2 + 5x_2 = 20 \iff x_2 = 0.
\]
So the equations are satisifed if and only if $x_2$ is $0$.
If $x_2 = 0$,then $3x_1 = 10$ and $6x_1 = 20$.
Both of these are equivalent to $x_1 = 10/3$.
So we have that $x_1$ must be $10/3$ and $x_2$ must be 0.

Clearly this is \textit{a solution}.
Is it the only one?\footnote{Future editions will expand.}

\blankpage