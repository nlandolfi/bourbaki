\sinput{../sheet.tex}
\sbasic

\sinput{../absolute_value/macros.tex}
\sinput{../norms/macros.tex}

\sstart

\stitle{Norms}

\ssection{Why}

We want to measure
the size of an element
in a vector space.

\ssection{Definition}

A
\ct{norm}{norm}
is a real-valued
functional that is
(a) non-negative,
(b) definite,
(c) absolutely homogeneous,
(d) and satisifies a
triangle inequality.
The triangle inequality property
requires that the norm applied
to the sum of any two vectors
is less than the sum of the norms.

\ssubsection{Examples}

\begin{expl}
  The absolute value
  function is a norm
  on the vector space
  of real numbers.
\end{expl}

\begin{expl}
  The Euclidean distance
  is a norm on the various
  real spaces.
\end{expl}

\ssubsection{Notation}

Let $(X, F)$ be a vector
space where $F$ is the
field of real numbers
or the field of
complex numbers.
Let $R$ denote
the set of real numbers.
Let $f: X \to R$.
The functional $f$ is a norm
if
\begin{enumerate}
  \item $f(v) \geq 0$ for all $x \in V$
  \item $f(v) = 0$ if and only if $x = 0 \in X$.
  \item $f(\alpha x) = \abs{\alpha}f(x)$ for all $\alpha \in F$, $x \in X$
  \item $f(x + y) \leq f(x) + f(y)$ for all $x, y \in X$.
\end{enumerate}

In this case, for $x \in X$,
we denote $f(x)$ by $\norm{x}$,
read aloud \say{norm x}.
The notation follows the notation
of absolute value as a norm.
When we wish to distinguish
the norm from the absolute
value function, we may write
$\normm{x}$. In some cases,
we go further, and for a norm
indexed by some parameter $\alpha$
or set $A$ we write $\normm{x}_\alpha$
or $\normm{x}_A$.

\strats
