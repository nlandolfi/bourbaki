
%!name:entire_functions
%!need:complex_analytic_functions
%!refs:yellow/IX/4

\section*{Definition}

An \t{entire function} is a complex function $f: \C  \to \C $ which is analytic for all $z \in \C $.

\blankpage
\sbasic
%%%% MACROS %%%%%%%%%%%%%%%%%%%%%%%%%%%%%%%%%%%%%%%%%%%%%%%

\newcommand{\PM}{\mathbf{P}}

%%%%%%%%%%%%%%%%%%%%%%%%%%%%%%%%%%%%%%%%%%%%%%%%%%%%%%%%%%%

%%%% MACROS %%%%%%%%%%%%%%%%%%%%%%%%%%%%%%%%%%%%%%%%%%%%%%%

\newcommand{\PM}{\mathbf{P}}

%%%%%%%%%%%%%%%%%%%%%%%%%%%%%%%%%%%%%%%%%%%%%%%%%%%%%%%%%%%

%%%% MACROS %%%%%%%%%%%%%%%%%%%%%%%%%%%%%%%%%%%%%%%%%%%%%%%

\newcommand{\PM}{\mathbf{P}}

%%%%%%%%%%%%%%%%%%%%%%%%%%%%%%%%%%%%%%%%%%%%%%%%%%%%%%%%%%%

%%%% MACROS %%%%%%%%%%%%%%%%%%%%%%%%%%%%%%%%%%%%%%%%%%%%%%%

\newcommand{\PM}{\mathbf{P}}

%%%%%%%%%%%%%%%%%%%%%%%%%%%%%%%%%%%%%%%%%%%%%%%%%%%%%%%%%%%

%%%% MACROS %%%%%%%%%%%%%%%%%%%%%%%%%%%%%%%%%%%%%%%%%%%%%%%

% use \set{stuff} for { stuff }
% use \set* for autosizing delimiters
\DeclarePairedDelimiter{\set}{\{}{\}}

% use \Set{a}{b} for {a | b}
% use \Set* for autosizing delimiters
\DeclarePairedDelimiterX{\Set}[2]{\{}{\}}{#1 \nonscript\;\delimsize\vert\nonscript\; #2}

% use \powerset{A} for power set of A
\newcommand{\powerset}[1]{2^{#1}}

\renewcommand{\emptyset}{\varnothing}

\newcommand{\SA}{\mathcal{A}}
\newcommand{\SB}{\mathcal{B}}
\newcommand{\SC}{\mathcal{C}}
\newcommand{\SD}{\mathcal{D}}
\newcommand{\SE}{\mathcal{E}}
\newcommand{\SF}{\mathcal{F}}
\newcommand{\SG}{\mathcal{G}}
\newcommand{\SH}{\mathcal{H}}
\newcommand{\SI}{\mathcal{I}}
\newcommand{\SJ}{\mathcal{J}}
\newcommand{\SK}{\mathcal{K}}
\newcommand{\SL}{\mathcal{L}}

%%%%%%%%%%%%%%%%%%%%%%%%%%%%%%%%%%%%%%%%%%%%%%%%%%%%%%%%%%%

%%%% MACROS %%%%%%%%%%%%%%%%%%%%%%%%%%%%%%%%%%%%%%%%%%%%%%%

\newcommand{\PM}{\mathbf{P}}

%%%%%%%%%%%%%%%%%%%%%%%%%%%%%%%%%%%%%%%%%%%%%%%%%%%%%%%%%%%

%%%% MACROS %%%%%%%%%%%%%%%%%%%%%%%%%%%%%%%%%%%%%%%%%%%%%%%

\newcommand{\PM}{\mathbf{P}}

%%%%%%%%%%%%%%%%%%%%%%%%%%%%%%%%%%%%%%%%%%%%%%%%%%%%%%%%%%%

\sstart
\stitle{Natural Order}

\ssection{Why}

We count in order.

\ssection{Definition}

Let $\preceq$ be a relation
on $N$ where $a \preceq b$,
$a, b \in N$ if we can obtain
$b$ by applying the successor
function to $a$ finitely many
times.
The relation $\preceq$ is a
total order, so $(N, \preceq)$
is a lattice.


We define the set of \ct{natural numbers}{naturalnumbers}
implicitly.
There is an element of the set which we call
\ct{one}{naturalone}.
Then we say that for each element $n$ of the set,
there is a unique corresponding element called the
\ct{successor}{naturalsuccessor} of $n$ which is
also in the set.
The \ct{successor function}{naturalsuccessorfunction}
is the implicitly defined a function from the set
into itself associating elements with their successors.
We call the elements \ct{numbers}{numbers} and the refer
to the set itself as the \ct{naturals}{naturals}.

To recap, we start by knowing that \rt{one}{naturalone} is
in the set, and the successor of one is in the set.
We call the \rt{successor}{naturalsuccessor} of
\rt{one}{naturalone} \ct{two}{naturaltwo}.
We call the \rt{successor}{naturalsuccessor} of
\rt{two}{naturaltwo} \ct{three}{naturalthree}.
And so on using the English language in the usual manner.
We are saying, in the language of sets, that the essence
of counting is starting with one and adding one repeatedly.

\ssubsection{Notation}

We denote the set of natural numbers by $N$, a mnemonic for natural.
We often denote elements of $N$ by $n$, a mnemonic for number, or $m$, a letter close to $n$.
We denote the element called one by $1$.

\ssection{Induction}

We assert two additonal self-evident and indispensable properties of these natural numbers.
First, one is the successor of no other element.
Second, if we have a subset of the naturals containing one with the property that it contains successors of its elements, then that set is equal to the natural numbers.
We call this second property the \textbf{principle of mathematical induction.}

These two properties, along with the existence
and uniqueness of successors are together called
\ct{Peano's axioms}{peanosaxioms} for the natural numbers.
When in familiar company, we freely assume Peano's axioms.

\ssection{Notation}

As an exercise in the notation assumed so far, we can write Peano's axioms: $N$ is a set along with a function $s: N \to N$ such that
\begin{enumerate}
  \item $s(n)$ is the successor of $n$ for all $n \in N$.
  \item s is one-to-one; $s(n) = s(m) \implies m = n$  for all $m, n \in N$.
  \item There does not exist $n \in N$ such that $s(n) = 1$.
  \item If $T \subset N$, $1 \in T$, and $s(n) \in T$ for all $n \in T$, then $T = N$.
\end{enumerate}

\ssection{Order}

Let $\preceq$ be a relation on $N$ where $a \preceq b$, $a, b \in N$ if we can obtain $b$ by applying the successor function to $a$ finitely many times.
It happens that $\preceq$ is a total order, so $(N, \preceq)$ is a lattice.
\strats
