
\section*{Why}

We are regularly referring to a few common growth classes.

\section*{Definitions}

Let $c \in \R $. Then we name the following growth classes
\begin{center}
\vspace{0.3cm}
\begin{tabular}{ll}
growth class & name\\
$O(1)$ & \t{constant growth class}\\
$O(\log(x))$ & \t{logarithmic growth class}\\
$O(\log(x)^c)$ & \t{polylogarithmic growth class}\\
$O(x)$ & \t{linear growth class}\\
$O(x^2)$ & \t{quadratic growth class}\\
$O(x^c)$ & \t{polynomial growth class}\\
$O(c^x)$ & \t{exponential growth class}\\
\end{tabular}
\vspace{0.1cm}
\end{center}

We have written these in order:
\[
O(1) \subset O(\log(x)) \subset O((\log(x))^c) \subset \cdots \subset O(x^c) \subset O(c^x).
\]

A function that grows faster (is in the upper growth class) of a power of $x$ is called \t{superpolynomial}.
One that grows slower than $c^n$ for some $c \in \R $ is called \t{subexponential}.
The class $O(\log(x^c))) = O(\log(x))$ since $\log(x^c) = c\log x$.
Similarly, for all $c_1, c_2 > 0$, $O(\log_{c_1}(x)) = O(\log_{c_2}(x))$.

This list is useful because of the following
\begin{proposition}
Let $f, g: \R  \to \R $ and defined $h: \R  \to \R $ by $h = f + g$. If $O(f) \subset O(g)$, then $h \in O(g)$.
\end{proposition}


In other words, if a function $h$ is the sum of $f$ and $g$ and $g$ is growing faster, then $g$ (the one growing faster) determines the order of $h$.
\blankpage