
%!name:real_egoprox_sequences
%!need:real_limits
%!need:absolute_value
%!refs:elias_stein∕complex_analysis∕01_preliminaries_to_complex_analysis

\section*{Why}

In the case that it is not possible to easily identify (or guess) the limit of a sequence, we are naturally interested in a simple condition on the sequence which is equivalent to convergence.

\section*{Definition}

A sequence $(x_n)_{n \in \N  }$ in $\R $ is said to be \t{egopox} (or \t{Cauchy} or a \t{Cauchy sequence}) if for every $\epsilon  > 0$, there exists $N \in \N  $ so that for all $m, n > N$, $\abs{x_m - x_n} < \epsilon $.
We call this property of the sequence \t{(eventual) egoproximity}.

\subsection*{Notation}

We sometimes denote this property as
    \[
\abs{x_n - x_m} \to 0 \quad \text{ as } \quad m, n \to \infty.
    \]

\subsection*{Examples}

For example, consider $\lim_{N \to \infty} \sum_{n = 1}^{N} 1/n^3$.

\subsection*{Sufficiency in $\R $}

Clearly a convergent sequence is egoprox.\footnote{Future editions may elaborate here.}
What of the converse?
Recall that we think of egoprox sequences as \say{bunching up.}
For the reals, if a sequence is bunching up, then our intuition is that it should be converging.
In other words, an egoprox real sequence always converges.
The egoprox condition is sufficient.
Bunching up is sufficient.

\begin{proposition}
If $(x_n)_{n \in \N  }$ is egoprox, then there exists $x_0 \in \R $ so that $\lim_{n \to \infty} x_n = x_0$.\end{proposition}
In other words, in $\R $ egoproximity is equivalent to convergence.
The above is sometimes called the \t{Bolzano-Weierstrass theorem}.

\blankpage