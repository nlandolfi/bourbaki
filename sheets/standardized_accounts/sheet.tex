%!name:standardized_accounts
%!need:accounts

\s{Why}

We want to do our best to have only one way to write accounts.\footnote{This sheet has to do with using a standard (perhaps formal) language through the project. We have not done so for the first edition. We have included this sheet to indicate how this might be done, and some typesetting ideas for future ideas.}

\s{Definition}

A \t{standard account}\footnote{This sheet will be expanded in future editions.} lists all names, then lists all premisses, then lists all conclusions.

\s{Example}

Consider the account.

\begin{account}[First Example]
\name{$a$}
\name{$b$}
\have{standardized_accounts:firstexample:equality}{$a = b$}
\name{$c$}
\have{standardized_accounts:firstexample:secondequality}{$c = b$}
\thus{standardized_accounts:firstexample:thirdequality}{$a = c$}{\ref{standardized_accounts:firstexample:equality},\ref{standardized_accounts:firstexample:secondequality}}
\end{account}

\clearpage
\begin{account}[Standardized First Example]
\name{$a$}
\name{$b$}
\have{standardized_accounts:stdfirstexample:equality}{$a = b$}
\name{$c$}
\have{standardized_accounts:stdfirstexample:secondequality}{$c = b$}
\thus{standardized_accounts:stdfirstexample:thirdequality}{$a = c$}{\ref{standardized_accounts:stdfirstexample:equality},\ref{standardized_accounts:stdfirstexample:secondequality}}
\end{account}

We can abbreviate the names:

\begin{account}[Abbreviated First Example]
\nameee{$a$}{$b$}{$c$}
\have{standardized_accounts:abrfirstexample:equality}{$a = b$}
\have{standardized_accounts:abrfirstexample:secondequality}{$c = b$}
\thus{standardized_accounts:abrfirstexample:thirdequality}{$a = c$}{\ref{standardized_accounts:abrfirstexample:equality},\ref{standardized_accounts:abrfirstexample:secondequality},IdentityAxioms:1}
\end{account}
