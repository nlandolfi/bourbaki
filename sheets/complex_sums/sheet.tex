
%!name:complex_sums
%!need:complex_numbers

\section*{Why}

We want to extend addition to $\C $.

Let $z_1, z_2 \in \C $ with $z_1 = (x_1, y_1)$ and $z_2 = (x_2, y_2)$.
The \t{complex sum} of $z_1$ and $z_2$ is the complex number $(x_1+x_2,y_1+y_2)$.

\subsection*{Notation}

For $z_1, z_2 \in \C $, we denote the complex sum of $z_1$ and $z_2$ by $z_1 + z_2$.
The notation is justified because the complex sum of two purely real complex numbers corresponds to the purely real complex numbers whose real part is the real sum of the real parts of the first two numbers.

Recall that we denote $z_1 = x_1 + iy_1$ and $z_2 = x_2 + iy_2$.
For example, we can express the definition of addition as
\[
z_1 + z_2 = (x_1 + x_2) + i(y_1 + y_2).
\]

\section*{Properties}

\begin{proposition}[Commutativity]
For all $z_1, z_2 \in \C $, we have $z_1 + z_2 = z_2 + z_1$.
\end{proposition}

\begin{proposition}[Associativity]
For all $z_1, z_2, z_3 \in \C $, we have and $z_1 + (z_2 + z_3) = (z_1 + z_2) + z_3$.
\end{proposition}

\section*{Complex addition}

We call the operation that associates a pair of complex numbers with their sum \t{complex addition}.
The operation is symmetric (commutative).

\section*{Additive identity and inverse}

Notice that the complex number $(0, 0)$ is the additive identity.
It is unique,\footnote{Future editions will include an account}
and so we call it the \t{complex multiplicative identity.}
Likewise, notice that additive inverse for $(x, y)$ is $(-x, -y)$.

\blankpage