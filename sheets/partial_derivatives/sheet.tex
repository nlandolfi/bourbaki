%!name:partial_derivatives
%!need:differentiable_functions
%!need:n-dimensional_space

\ssection{Why}

We want to talk about how a function
of multiple real-valued arguments
changes with respect to changes
in its arguments.

\ssection{Definition}

Consider a real-valued function on
$d$-dimensional space.
For $i = 1, \dots, d$,
Fix a point $x$.
consider the limit of a sequence
of quotients of the difference
of the result of that function
at a point
  the
consider the limit of a sequence of
quotients of the value changed at
component
The \t{partial derivative} of the
function with respect to the $i$th
the function which maps $d$-dimensional
vectors of real numbers to the limit
of a seq
of all of
the quotient between the point
to
argument is the limit of the rate
with a
The partial derivative of a


Let $f: \R^d \to \R$
For $i = 1,\dots,d$,
define
Let $g_i: \R^d \to \R$
by
\[
  g_i(x) = \lim_{h \to 0} \frac{f(x + he_i) - f(x)}{h}
\]
for each $x$

\ssubsection{Notation}


\ssection{Gradient}

The \t{gradient} of a multivariate function is the vector-valued function whose $i$th component is the the partial derivative of the function with respect to its $i$th argument.

\ssubsection{Notation}

Let $f: \R^n \to \R$.
The gradient of $f$ is frequently denoted $\nabla f$.
It is understood that $(\nabla f) \in \R^d \to \R^d$.
An alternative notation is to use that similar for single derivatives and to denote the gradient (sometimes called derivative) of $f$ by $f'$ (assuming it exists).
It is important to here note that although when $g: \R \to \R$, $g' \in (\R \to \R)$, (and so is another function from and to reals) when $f: \R^d \to \R$, $f' \in \R^d \to \R^d$, and so is a vector-valued (not a real-valued) function.

There is (unfortunately) much notation for the individual partial derivatives; most of which we shall not (fortunately) have occasion to use in these sheets.
One popular usage is the use of the $\partial$ symbol, read aloud as \say{partial.}
For example, if $f: \R^2 \to \R$ is a function of two arguments, each being referred to as $x$ and $y$, then $\partial_x f$ denotes the partial derivative of $f$ with respect to $x$ and $\partial_y f$ denotes the partial derivative of $f$ with respect to $y$.
It is understood that $(\partial_x f) \in \R^d \to \R$. and likewise for $\partial_y f$.
Another popular usage is $\nicefrac{\partial f}{\partial x}$ for $\partial_x f$ and $\nicefrac{\partial f}{\partial y}$ for $\partial_y f$.
We will almost exclusively prefer the gradient notation.
