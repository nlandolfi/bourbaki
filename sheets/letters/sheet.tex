
%!name:letters

\section*{Why}

We want to communicate and remember.

\section*{Discussion}

A \t{language} is a conventional correspondence of sounds to affections of mind.
We deliberately leave the definition of \t{affections} vague.
A \t{spoken word} is a succession of sounds.
By using these sounds, our mind can communicate with other minds.

A \t{symbol} is a written mark.
A \t{script} is a collection of symbols called \t{letters}.
In \t{phonetic} languages the letters correspond to sounds and rules for composing these letters into successions called written words.
This succession of letters corresponds to a succession of sounds and so a written word corresponds to a spoken word.
By making marks, we communicate with other minds---including our own---in the future.

To write this sheet, we use Latin letters arranged into written words which are meant to denote the spoken words of the English language.
The written words on this page are several letters one after the other.
For example, the word \say{word} is composed of the letters \say{w}, \say{o}, \say{r}, \say{d}.

These endeavors are at once obvious and remarkable.
They are obvious by their prevalence, and remarkable by their success.
We do not long forget the difficulty in communicating affections of the mind, however, and this leads us to be very particular about how we communicate throughout these sheets.

\section*{Latin letters}

We will start by officially introducing the letters of the Latin language.
These come in two kinds, or cases.
The \t{lower case latin letters}.
\begin{center}
\vspace{0.3cm}
\begin{tabular}{ccccccccc}
a & b & c & d & e & f & g & h & i\\
j & k & l & m & n & o & p & q & r\\
s & t & u & v & w & x & y & z & \\
\end{tabular}
\vspace{0.1cm}

%    \begin{tabular}{ccccccccc}
%   a & b & c & d & e & f & g & h & i \\
%   j & k & l & m & n & o & p & q & r \\
%   s & t & u & v & w & x & y & z & \\
%   \end{tabular}
%   
\end{center}
And the \t{upper case latin letters}.
\begin{center}
\vspace{0.3cm}
\begin{tabular}{ccccccccc}
A & B & C & D & E & F & G & H & I\\
J & K & L & M & N & O & P & Q & R\\
S & T & U & V & W & X & Y & Z & \\
\end{tabular}
\vspace{0.1cm}

%\begin{tabular}{ccccccccc}
%   A & B & C & D & E & F & G & H & I \\
%   J & K & L & M & N & O & P & Q & R \\
%   S & T & U & V & W & X & Y & Z & \\
%\end{tabular}
\end{center}
So, A is the upper case of a, and a the lower case of A.
Similarly with b and B, with c and C, and all the rest.

\section*{Arabic numerals}

We also use the \t{Arabic numerals}.
\begin{center}
\vspace{0.3cm}
\begin{tabular}{cccccccccc}
0 & 1 & 2 & 3 & 4 & 5 & 6 & 7 & 8 & 9\\
\end{tabular}
\vspace{0.1cm}

%   \begin{tabular}{cccccccccc}
%   0 & 1 & 2 & 3 & 4 & 5 & 6 & 7 & 8 & 9 \\
%   \end{tabular}
\end{center}

\section*{Other symbols}

We also use the following symbols.
\begin{center}
\vspace{0.3cm}
\begin{tabular}{cccccccccccccccc}
$'$ & $($ & $)$ & $\{$ & $\}$ & $\lor$ & $\land$ & $\neg$ & $\forall$ & $\exists $ & $\Rightarrow$ & $\iff$ & $=$ & $\in$ & $\to$ & $\sim$\\
\end{tabular}
\vspace{0.1cm}

%     \begin{tabular}{cccccccccccccccc}
%   $’$ & $($ & $)$ & $\{$ & $\}$ & $\lor$ & $\land$ &
%    $\neg$ & $∀$ & $∃$ & $⇒$ & $\iff$ & $=$ & $∈$ & $→$ &
%    $∼$
%   \end{tabular}
%   
\end{center}
