%!name:letters

\ssection{Why}

We want to communicate and remember.

\ssection{Discussion}

A \t{language} is a conventional correspondence of sounds to affections of mind.
We deliberately leave the definition of \t{affections} vague.
A \t{spoken word} is a succession of sounds.
By using these sounds, our mind can communicate with other minds.

A \t{script} is a collection of written marks or symbols called \t{letters}.
In \t{phonetic} languages, the letters correspond to sounds.
A \t{written word} is a succession of letters.
This succession of letters corresponds to a succession of sounds and so a written word corresponds to a spoken word.
By making marks, we communicate with other minds---including our own---in the future.

To write this sheet, we use Latin letters arranged into \t{written words} which are meant to denote the \t{spoken words} of the English language.
The written words on this page are several letters one after the other.
For example, the word \say{word} is composed of the letters \say{w}, \say{o}, \say{r}, \say{d}.

These endeavors are at once obvious and remarkable.
They are obvious by their prevalence, and remarkable by their success.
We do not long forget the difficulty in communicating affections of the mind, however, and this leads us to be very particular about how we communicate throughout these sheets.

\ssection{Latin letters}

We will start by officially introducing the letters of the Latin language.
These come in two kinds, or cases.
We call these the \t{lower case latin letters}.
\begin{center}
\begin{tabular}{ccccccccc}
  a & b & c & d & e & f & g & h & i \\
  j & k & l & m & n & o & p & q & r \\
  s & t & u & v & w & x & y & z &   \\
\end{tabular}
\end{center}
And we call these the \t{upper case latin letters}.
\begin{center}
\begin{tabular}{ccccccccc}
  A & B & C & D & E & F & G & H & I \\
  J & K & L & M & N & O & P & Q & R \\
  S & T & U & V & W & X & Y & Z &   \\
\end{tabular}
\end{center}
So, A is the upper case of a, and a the lower case of A.
Similarly with b and B, with c and C, and all the rest.

\ssection{Arabic numerals}

We will also use the following symbols.
We call these the \t{Arabic numerals}.
\begin{center}
\begin{tabular}{cccccccccc}
  0 & 1 & 2 & 3 & 4 & 5 & 6 & 7 & 8 & 9\\
\end{tabular}
\end{center}

\ssection{Other symbols}

We will also use the following symbols
We call thes the \t{logical symbols}.
\begin{center}
\begin{tabular}{ccccccccccc}
  $($ & $)$ & $\lor$ & $\land$ & $\neg$ & $\forall$ & $\exists$ & $\implies$ & $\iff$ & $=$ & $\in$
\end{tabular}
\end{center}



% When we write a latin letter on its own, and mean the letter itself as opposed to a word, we write it in italics as we did in the preceding two sentences.


% \begin{algorithmic}[1]
%   \texttt{1-3} & \texttt{name} & $x, y, z$ & \\
%   \texttt{4-5} & \texttt{have} & $x \subset y, \quad y \subset z$ \\
%   \texttt{6} & \texttt{thus} & $x \subset z & \texttt{by} & Prop. 2 with \texttt{4-5}. \\
% \end{algorithmic}


% \begin{construction}
% {\normalfont
% \begin{tabular}{r|rlcc}
% %
%   \refstepcounter{here}
%   \thehere
%   \texttt{1-3} & \texttt{name} & $x, y, z$ & \\
%   \texttt{1-3} & \texttt{name} & $x, y, z$ & \\
%   \refstepcounter{here}
%   \thehere
%   \texttt{1-3} & \texttt{func} & $f: x \to y$ & \\
%   \texttt{2} & \texttt{name} & $y$ \\
%   \texttt{3} & \texttt{name} & $z$ \\
%   \texttt{4-5} & \texttt{have} & $x \subset y, \quad y \subset z$ \\
%   \texttt{5} & \texttt{have} & $y \subset z$ \\
%   \texttt{6} & \texttt{thus} & $x \subset z & \texttt{by} & Prop. 2 with \texttt{4-5}. \\
%   \texttt{6} & \texttt{thus} & $\int_{A} f dx$ & \texttt{by} & Construction 2.\\
% \end{tabular}
% }
% \end{construction}
