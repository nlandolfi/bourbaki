%!name:convex_sets
%!need:set_operations
%!need:intervals
%!need:affine_sets

\ssection{Why}\footnote{Future editions will clarify.}

\ssection{Definition}

The \t{closed line segment between} two points in $n$-dimensional space is the set of points which can be expressed as the sum of the first point and a scalar multiple of the difference between the second point and the first; where the scalar is in the interval $[0, 1]$.
Thus, the closed line segment between two points is a subset of the line though the two points.
The \t{open line segment} between $x$ and $y$ is the closed line segment with the points $x$ and $y$.

A \t{convex set} contains every closed line segement between any two points.
Every affine set is convex.
Thus, convex sets are more general.

\ssubsection{Notation}

Let $x$ and $y$ in $\R^n$. We can express the closed line segment between $x$ and $y$ as
$$
  \Set*{x + a(y - x)}{0 \leq a \leq 1, x, y \in \R^n}.
$$
Notice that $x + a(y - x) = (1-a)x + ay$.


\begin{prop}
  Every affine set is convex.
\end{prop}

\begin{prop}
  The intersection of a family of convex sets is convex.
\end{prop}

\begin{prop}
  The translate of a convex set is convex.
  The scalar multiple of a convex set si convex.
\end{prop}

% A \t{convex combination} of two distinct real numbers
% is an element of the closed interval they
% delimit.
% A
% \ct{convex set}{convexset}
% of real numbers
% contains each convex combination
% of any two of its elements.
%
% The \ct{length}{}
% of a convex
% combination
% is the real
% number in $[0,1]$
% which is
% the ratio of
% the combination
% less the lower
% endpoint
% to the
% upper endpoint
% less the lower
% endpoint.
%
% \ssubsection{Notation}
%
% Denote the real numbers
% by $R$.
% Let $A \subset R$ be convex.
% Then
% for all $a, b \in A$,
% $[a, b] \subset A$.
%
% Suppose $A$ contains
% at least two element.
% Let $a, b \in A$
% with $a < b$.
% If $c$ is the combination
% of $a$ and $b$, then
% the length of $c$
% is $(c-a)/(b-a)$.
%
% If $A$ is convex,
% then for each
% $a, b \in A$,
% and $\theta \in [0, 1]$,
% \[
%   \theta a + (1-\theta) b \in A.
% \]
%
% \ssection{Examples}
%
% \begin{expl}
%   The real numbers
%   are a convex set.
% \end{expl}
%
% \begin{expl}
%   Real intervals are convex.
% \end{expl}
%
% \begin{expl}
%   Let $a,b$ be
%   non-equal real numbers.
%   The set $\set{a,b}$ is
%   not convex.
% \end{expl}
%
% \begin{expl}
%   The empty set is convex
% \end{expl}
%
% \begin{expl}
%   Let $a$ be a real number.
%   The set $\set{a}$ is convex.
% \end{expl}
%
% \begin{expl}
%   Let $[a, b]$
%   and $[c,d]$ be
%   two disjoint real
%   intervals.
%   The set
%   $[a,b]\union[c,d]$
%   is not convex.
% \end{expl}
