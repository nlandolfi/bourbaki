
\section*{Definition}

The \t{skeleton} of the directed graph $(V, E)$ is the undirected graph $(V, F)$ where
\[
F = \Set*{\set{v, w} \subset V}{(v, w) \in E \text{ or } (w, v) \in E}.
\]
In other words, the skeleton is an undirected graph whose vertex set is $V$ and whose edges are all (unordered) pairs which appear as an ordered pair in the directed graph.

In the case that $(V, E)$ is a directed graph and $E$ is a symmetric relation, the skeleton of $(V, E)$ is a natural undirected graph to associate with $(V, E)$.
An \t{orientation} of an undirected graph $G$ is a directed graph whose skeleton is $G$.

An \t{oriented graph} is a directed graph without self-loops satisfying the property for any two vertices $x$ and $y$, either $(x,y)$ or $(y,x)$ is an edge, but not both.
An oriented graph can be obtained from an undirected graph by selecting an ``orientation'' of the undirected edges.

\blankpage