%!name:arrays
%!need:sequences

\ssection{Why}

We name sequences of sequences, and other such generalizations.

\ssection{Definition}

Let $s$ be a sequence of natural numbers: $s = (n_1, \dots, n_d)$.
An \t{array} of \t{size} (or \t{shape}) $s$ is a function whose domain is the set
\[
  I = \Set{(m_1, \dots, m_d)}{ 1 \leq m_1 \leq n_1, \dots, 1 \leq m_d \leq n_d}.
\]
We call the set $I$ the set of \t{indices} of the array.
We call the codomain of the function the set of \t{values} of the array.
If $A$ is the set of values, we say that the array is \t{in} A.
We call the length of $s$ (here denoted $d$) the \t{dimension} of the array.

\ssubsection{Case $d = 1$}

If the shape of the array has length one, then the array is no different from a sequence.
In this case, the terminology for arrays coincides with that for sequences.

\ssubsection{Case $d = 2$}

If the shape of the array has length two, then the array can be thought of as a table with $n_1$ rows and $n_2$ columns.\footnote{Compare with \sheetref{matrices}{Matrices}}
We say that the array is two-dimensional.

\blankpage
