%!name:convergence_in_measure
%!need:real_limits
%!need:absolute_value

\ssection{Why}

convergence in measure
form of DCT?

\ssection{Definition}

A sequence of real-valued
measurable functions
\ct{converges in measure}{}
to a real-valued measurable
limit function if for every positive
number the measures of the set
where the function deviates
from the limit function by
more than the positive
number converges to zero.

\ssubsection{Notation}

Let $(X, \SA, \mu)$ be
a measure space.
Let $\seq{f}$ a sequence
of real-valued measurable
functions on $X$. Let
$f$ be a measurable real-valued
function on $X$.
If $f_n$ converges in measure
to $f$ we write:
$f_n \goesto f$ in measure,
read aloud as
\say{f n goes to f in measure.}

Suppose $f_n \goesto f$ in measure.
Then for every $\epsilon > 0$,
\[
\lim_{n \to \infty} \mu(\Set*{x \in X}{\abs{f_n(x) - f(x)} > \epsilon}) = 0.
\]
