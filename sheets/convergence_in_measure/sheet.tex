
\section*{Why}

We want a form of the dominated convergence theorem in terms of convergence in measure.\footnote{Future editions will expand.}

\section*{Definition}

A sequence of real-valued measurable functions \t{converges in measure} to a real-valued measurable limit function if for every positive number the measures of the set where the function deviates from the limit function by more than the positive number converges to zero.

\subsection*{Notation}

Suppose $\seq{f}$ is a sequence of real-valued measurable funtions on a measure space $(X, \mathcal{A} , \mu )$ and $f: X \to \R $ measurable.
If $f_n$ converges in measure to $f$ we write: $f_n \goesto f$ in measure, read aloud as ``$f_n$ goes to $f$ in measure.''
This notation is an abbreviation of the following relation
\[
\lim_{n \to \infty} \mu (\Set*{x \in X}{\abs{f_n(x) - f(x)} > \epsilon }) = 0 \quad \text{ for all } \epsilon  > 0
\]

\blankpage
%macros.tex
%%%%% MACROS %%%%%%%%%%%%%%%%%%%%%%%%%%%%%%%%%%%%%%%%%%%%%%%
%%%%%%%%%%%%%%%%%%%%%%%%%%%%%%%%%%%%%%%%%%%%%%%%%%%%%%%%%%%%
