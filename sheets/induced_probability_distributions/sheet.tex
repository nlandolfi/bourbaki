
\section*{Why}

We can interpret the codomain of a random variable as a new sample space, since the underlying probability distribution induces a new probability distribution.

\section*{Definition}

Let $p: \Omega  \to \R $ be a probability distribution and $x: \Omega  \to V$ an outcome variable. Define $q: V \to \R $ by
\[
q(a) = \mathbfsf{P} [x = a].
\]
Since events $x^{-1}(a)$ for $a \in V$ partition $\Omega $, $\sum_{a \in A} q(a) = 1$.
We call $q$ the \t{induced distribution} (or \t{induced probability mass function}) of the random variable $x$.
Thus we can think of $V$ as a set of outcomes, which we call the outcomes \t{induced} by $x$.

\subsection*{Notation}

It is common to denote it by $p_{x}$.

If $x: \Omega  \to V$ is a random variable and $f: V \to U$, then if we define $y: \Omega  \to V$ so that $y \equiv f(x)$, $y$ is a random variable with induced distribution $p_{y}: \Omega  \to \R $ satisfying
\[
\textstyle
p_{y}(b) = \sum_{a \in V \mid  y(a) = b} p_x(a).
\]

As a matter of practical computation, we can evaluate probabilities having to do with the outcome variable $x$ using $p_x$ instead of $p$.
For example with $x$ as in the example above, $\mathbfsf{P} (x = 4 \text{ or } x = 5) = p_x(4) + p_x(5)$, rather than $\sum_{\omega  \in \Omega  \mid  x(\omega ) = 4 \text{ or } x(\omega ) = 5} p(\omega )$.

\blankpage