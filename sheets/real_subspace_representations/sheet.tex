%!name:real_subspace_representations
%!need:real_matrices_with_orthonormal_columns

\section*{Why}

How should we represent a subspace computationally?

\section*{Definition}

Given a subspace $S \in \R ^n$, since it is finite dimensional, there exists a finite basis for the space.
This basis can be made orthonormal.
Therefore every subspace $S$ has an orthonormal basis $q_1, \dots , q_k$, where $k$ is the dimension of the subspace.
We can stack these as a matrix. Define $Q \in \R ^{n \times k}$ by
  \[
Q = \bmat{q_1 & q_2 & \cdots & q_k}.
  \]
For every $x \in S$, there exists unique coefficients $z \in \R ^k$ so that
  \[
x = Qz.
  \]
Therefore we have a one-to-one correspondence between vectors $x \in S$ and their coordinates $z \in \R ^k$.

\blankpage