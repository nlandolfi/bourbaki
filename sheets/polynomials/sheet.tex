%!name:polynomials
%!need:rings
%!need:sequences

\ssection{Why}

\footnote{Future editions will include, and most likely will build on quadratics and an appeal to the simplicity of the \say{natural} algebraic operations.}

\ssection{Definition}

Let $(A, +, \cdot)$ be a ring.


A \t{polynomial} in $A$ of \t{degree} $d$ is a function $p: A \to A$ for which there exists a finite sequence $c = (c_0, c_1, \dots, c_{d-1}, c_d) \in A^{d+1}$ satisfying
\[
	p(a) = c_0 + c_1a^1 + c_2a^2 + \dots + c_da^d,
\]
for all $a \in A$.
We call the sequence $c$ the \t{polynomial coefficients}, and call the $c_i$ the \t{coefficients} of $p$.
We call $d+1$ the \t{order} of the polynomial.

Clearly, to every polynomial in $A$ of degree $d$ there corresponds a sequence in $A$ of length $d+1$, and vice versa.
For this reason, we can identify polynomials by their coefficients.

\ssubsection{Examples}

The function $f: A \to A$ is a polynomial of degree 0 and order 1 if there exists $c_0$ so that
\[
  f(a) = c_0
\]
for all $a \in A$.

The function $g: A \to A$ is a polynomial of degree 1 and order 2 if there exists $c_0$ and $c_1$ so that
\[
  g(a) = c_0 + c_1a
\]
The function $h: A \to A$ is a polynomial of degree 2 and order 3 if there exists $c_0$ and $c_1$ so that
\[
  h(a) = c_0 + c_1a + c_2a^2.
\]
In other words, a second degree polynomial is a quadratic.

The function $p: A \to A$ is a \t{polynomial} of degree $d$ and order $d+1$ if there exists a $d+1$ length sequence $(c_0, c_1, \dots, c_d)$ in $A$ so that
\[
  p(a) = c_0 + c_1a + \cdots + c_da^d.
\]

% \blankpage
