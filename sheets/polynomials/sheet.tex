%!name:polynomials
%!need:rings

\ssection{Why}

\footnote{Future editions will include, and most likely will build on quadratics.}

\ssection{Definition}

Let $(A, +, \cdot)$ be an ring.
A \t{polynomial} in $A$ of \t{degree} $d$ is a finite sequence of length $d+1$.
We call the elements of the sequence the \t{coefficients} of the polynomial.

Let $c = (c_0, c_1, \dots, c_{d-1}, c_d)$ be a polynomial of degree $d$.
The \t{polynomial function} or \t{function of the polynomial} $c$ is the function $f: A \to A$ defined by
\[
  f(a) = c_0 + c_1a^1 + c_2a^2 + \dots + c_da^d.
\]
In accordance with this terminology, we often call  function $f: A \to A$ a polynomial if there exists a polynomial $c$ so that $f$ is the polyonomial function of $c$.

The function $f: A \to A$ is a polynomial of degree 0 and order 1 if there exists $c_0$ so that
\[
  f(a) = c_0
\]
for all $a \in A$.

The function $g: A \to A$ is a polynomial of degree 1 and order 2 if there exists $c_0$ and $c_1$ so that
\[
  g(a) = c_0 + c_1a
\]
The function $h: A \to A$ is a polynomial of degree 2 and order 3 if there exists $c_0$ and $c_1$ so that
\[
  h(a) = c_0 + c_1a + c_2a^2.
\]
In other words, a second degree polynomial is a quadratic.

The function $p: A \to A$ is a \t{polynomial} of degree $d$ and order $d+1$ if there exists a $d+1$ length sequence $(c_0, c_1, \dots, c_d)$ in $A$ so that
\[
  p(a) = c_0 + c_1a + \cdots + c_da^d.
\]

% \blankpage
