
%!name:integer_numbers
%!need:natural_numbers
%!need:equivalence_relations

\section*{Why}

We want to subtract numbers.\footnote{Future editions will change this why. In particular, by referencing \sheetref{inverse_elements}{Inverse Elements}and the lack thereof in $\omega $.}

\section*{Definition}

Consider the set $\omega  \times \omega $.
This set is the set of ordered pairs of $\omega $.
In other words, the ordered pairs of natural numbers.

We call two such pairs $(a, b)$ and $(c, d)$ \t{equivalent} $a + d = b + c$.
Briefly, the intuition is that $(a, b)$ represents $a$ less $b$, or in the usual notation \say{$a - b$}.\footnote{This account will be expanded in future editions.}
So this equivalence relation says these two are the same if $a - b = c - d$.
Rearranging gives $a + d = b + c$.

\begin{proposition}
Integer equivalence is an equivalence relation.\footnote{The proof is straightforward. It will be included in future editions.}\end{proposition}
We define the \t{set of integer numbers} to be the set of equivalence classes (see \sheetref{equivalence_relations}{Equivalence Relations}) under integer equivalence on $\omega  \times \omega $.
We call an element of the set of integer numbers an \t{integer number} or an \t{integer}.
We call the set of integer numbers the \t{set of integers} or \t{integers} for short.

\subsection*{Notation}

We denote the set of integers by $\Z $.
If we denote integer equivalence by $\sim$ then $\Z  = (\omega \times \omega )/\mathord{\sim}$.
