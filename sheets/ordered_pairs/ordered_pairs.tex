
%!name:entire_functions
%!need:complex_analytic_functions
%!refs:yellow/IX/4

\section*{Definition}

An \t{entire function} is a complex function $f: \C  \to \C $ which is analytic for all $z \in \C $.

\blankpage
\sbasic
%%%% MACROS %%%%%%%%%%%%%%%%%%%%%%%%%%%%%%%%%%%%%%%%%%%%%%%

\newcommand{\PM}{\mathbf{P}}

%%%%%%%%%%%%%%%%%%%%%%%%%%%%%%%%%%%%%%%%%%%%%%%%%%%%%%%%%%%

%%%% MACROS %%%%%%%%%%%%%%%%%%%%%%%%%%%%%%%%%%%%%%%%%%%%%%%

% use \set{stuff} for { stuff }
% use \set* for autosizing delimiters
\DeclarePairedDelimiter{\set}{\{}{\}}

% use \Set{a}{b} for {a | b}
% use \Set* for autosizing delimiters
\DeclarePairedDelimiterX{\Set}[2]{\{}{\}}{#1 \nonscript\;\delimsize\vert\nonscript\; #2}

% use \powerset{A} for power set of A
\newcommand{\powerset}[1]{2^{#1}}

\renewcommand{\emptyset}{\varnothing}

\newcommand{\SA}{\mathcal{A}}
\newcommand{\SB}{\mathcal{B}}
\newcommand{\SC}{\mathcal{C}}
\newcommand{\SD}{\mathcal{D}}
\newcommand{\SE}{\mathcal{E}}
\newcommand{\SF}{\mathcal{F}}
\newcommand{\SG}{\mathcal{G}}
\newcommand{\SH}{\mathcal{H}}
\newcommand{\SI}{\mathcal{I}}
\newcommand{\SJ}{\mathcal{J}}
\newcommand{\SK}{\mathcal{K}}
\newcommand{\SL}{\mathcal{L}}

%%%%%%%%%%%%%%%%%%%%%%%%%%%%%%%%%%%%%%%%%%%%%%%%%%%%%%%%%%%

%%%% MACROS %%%%%%%%%%%%%%%%%%%%%%%%%%%%%%%%%%%%%%%%%%%%%%%

\newcommand{\PM}{\mathbf{P}}

%%%%%%%%%%%%%%%%%%%%%%%%%%%%%%%%%%%%%%%%%%%%%%%%%%%%%%%%%%%

%%%% MACROS %%%%%%%%%%%%%%%%%%%%%%%%%%%%%%%%%%%%%%%%%%%%%%%

\newcommand{\PM}{\mathbf{P}}

%%%%%%%%%%%%%%%%%%%%%%%%%%%%%%%%%%%%%%%%%%%%%%%%%%%%%%%%%%%

%%%% MACROS %%%%%%%%%%%%%%%%%%%%%%%%%%%%%%%%%%%%%%%%%%%%%%%

\newcommand{\PM}{\mathbf{P}}

%%%%%%%%%%%%%%%%%%%%%%%%%%%%%%%%%%%%%%%%%%%%%%%%%%%%%%%%%%%

%%%% MACROS %%%%%%%%%%%%%%%%%%%%%%%%%%%%%%%%%%%%%%%%%%%%%%%

\newcommand{\PM}{\mathbf{P}}

%%%%%%%%%%%%%%%%%%%%%%%%%%%%%%%%%%%%%%%%%%%%%%%%%%%%%%%%%%%

%%%% MACROS %%%%%%%%%%%%%%%%%%%%%%%%%%%%%%%%%%%%%%%%%%%%%%%

\newcommand{\PM}{\mathbf{P}}

%%%%%%%%%%%%%%%%%%%%%%%%%%%%%%%%%%%%%%%%%%%%%%%%%%%%%%%%%%%

%%%% MACROS %%%%%%%%%%%%%%%%%%%%%%%%%%%%%%%%%%%%%%%%%%%%%%%

\newcommand{\PM}{\mathbf{P}}

%%%%%%%%%%%%%%%%%%%%%%%%%%%%%%%%%%%%%%%%%%%%%%%%%%%%%%%%%%%

%%%% MACROS %%%%%%%%%%%%%%%%%%%%%%%%%%%%%%%%%%%%%%%%%%%%%%%

\newcommand{\PM}{\mathbf{P}}

%%%%%%%%%%%%%%%%%%%%%%%%%%%%%%%%%%%%%%%%%%%%%%%%%%%%%%%%%%%

%%%% MACROS %%%%%%%%%%%%%%%%%%%%%%%%%%%%%%%%%%%%%%%%%%%%%%%

\newcommand{\PM}{\mathbf{P}}

%%%%%%%%%%%%%%%%%%%%%%%%%%%%%%%%%%%%%%%%%%%%%%%%%%%%%%%%%%%

\sstart
\stitle{Ordered Pairs}

\ssection{Why}

We speak of an ordered pair of objects: one selected from a first set and one selected from a second set.

\ssection{Definition}

Let $A$ and $B$ be nonempty \rt{sets}{set}.
Let $a \in A$ and $b \in B$.
The \ct{ordered pair}{pair} of $a$ and $b$ is the set
$\set{\set{a}, \set{a, b}}$.
The \t{first coordinate} of $\set{\set{a}, \set{a, b}}$ is $a$ and the \t{second coordinate} is $b$.

The \t{product} of $A$ and $B$ is the set of all ordered pairs.
This set is also called the \t{cartesian product}.
If $A \neq B$, the ordering causes the product of $A$ and $B$ to differ from the product of
$B$ with $A$.
If $A = B$, however, the symmetry holds.

\ssubsection{Notation}

We denote the ordered pair $\set{\set{a},\set{a,b}}$ by $(a, b)$.
We denote the product of $A$ with $B$ by $A \cross B$, read aloud as \say{A cross B.}
In this notation, if $A \neq B$, then $A \cross B \neq B \cross A$.

\ssection{Taste}

Notice that $a \not\in (a, b)$ and similarly $b \not\in (a, b)$.
These facts led us to use the terms first and second \say{coordinate} above rather than element.
Neither $a$ nor $b$ is an element of the ordered pair $\op{a, b}$.
On the other hand, it is true that $\set{a} \in \op{a, b}$ and $\set{a, b} \in \op{a, b}$.
These facts are odd.
Should they bother us?

We chose to define ordered pairs in terms of sets so that we could reuse notions about a particular type of object (sets) that we had already developed.
We chose what we may call conceptual simplicty (reusing notions from sets) over defining a new type of object (the ordered pair) with its own primitive properties.
Taking the former path, rather than the latter is a matter of taste, really, and not a logical consequence of the nature of things.

The argument for our taste is as follows.
We already know about sets, so let's use them, and let's forget cases like $\set{a, b} \in \op{a, b}$ (called by some authors \say{pathologies}).
It does not bother us that our construction admits many true (but irrelevant) statements.
Such is the case in life.
Plus, suppose we did choose to make the object
$\op{a, b}$ primitive.
Sure, we would avoid oddities like $\set{a} \in \op{a, b}$.
And we might even get statements like $a \in \op{a, b}$ to be true.
But to do so we would have to define the meaning of $\in$ for the case in which the right hand object is an \say{ordered pair}.
Our current route avoids introducing any new concepts, and simply names a construction in our current concepts.

\ssection{Equality}

\begin{prop}
$(a, b) = (c, d)$ if and only if $a = b$ and $c = d$.
  \begin{proof}
    TODO
  \end{proof}
\end{prop}
\strats
