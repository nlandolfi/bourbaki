
\section*{Why}

We want to order two objects.

\section*{Definition}

Let $a$ and $b$ denote objects.
The \t{ordered pair} of $a$ and $b$ is the set $\set{\set{a}, \set{a, b}}$.
The \t{first coordinate} of $\set{\set{a}, \set{a, b}}$ is the object denoted by $a$ and the \t{second coordinate} is the object denoted by $b$.

\subsection*{Notation}

We denote the ordered pair $\set{\set{a},\set{a,b}}$ by $(a, b)$.

\section*{Equality}

Our intuition of two objects in order dictates that if we have the same objects in the same order then we have the same ordered pair.
Conversely, if we have two identical ordered pairs, they must consist of the same objects in the same location.
In other words, two ordered pairs should be equal if and only if they consist of the same objects in the same order.
Our definition agrees with this intuition.
Indeed,
\begin{proposition}
$(((a, b) = (x, y)) \iff (a = x \land b = y))$\footnote{The proof of this proposition will be found in future editions.
%%the proof can be found in Halmos p23 right
%          after the definition section 6 

}
\end{proposition}


\blankpage
%macros.tex
%%%%% MACROS %%%%%%%%%%%%%%%%%%%%%%%%%%%%%%%%%%%%%%%%%%%%%%%
%\newcommand{\cross}{\times}
%\newcommand{\op}[1]{\left(#1\right)}
%\newcommand{\tuple}[1]{\left(#1\right)}
%\newcommand{\tu}[1]{\left(#1\right)}
%%%%%%%%%%%%%%%%%%%%%%%%%%%%%%%%%%%%%%%%%%%%%%%%%%%%%%%%%%%%
