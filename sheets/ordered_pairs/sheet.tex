%!name:ordered_pairs
%!need:unordered_triples
%!refs:paul_halmos/naive_set_theory/section_06

\ssection{Why}

We want to order two objects.

\ssection{Definition}

Let $a$ and $b$ denote objects.
The \ct{ordered pair}{pair} of $a$ and $b$ is the set $\set{\set{a}, \set{a, b}}$.
The \t{first coordinate} of $\set{\set{a}, \set{a, b}}$ is the object denoted by $a$ and the \t{second coordinate} is the object denoted by $b$.


\ssubsection{Notation}

We denote the ordered pair $\set{\set{a},\set{a,b}}$ by $(a, b)$.

\ssection{Pathologies}

Notice that $a \not\in (a, b)$ and similarly $b \not\in (a, b)$.
These facts led us to use the terms first and second \say{coordinate} above rather than element.
Neither $a$ nor $b$ is an element of the ordered pair $\op{a, b}$.
On the other hand, it is true that $\set{a} \in \op{a, b}$ and $\set{a, b} \in \op{a, b}$.
These facts are odd.
Should they bother us?

We chose to define ordered pairs in terms of sets so that we could reuse notions about a particular type of object (sets) that we had already developed.
We chose what we may call conceptual simplicty (reusing notions from sets) over defining a new type of object (the ordered pair) with its own primitive properties.
Taking the former path, rather than the latter is a matter of taste, really, and not a logical consequence of the nature of things.

The argument for our taste is as follows.
We already know about sets, so let's use them, and let's forget cases like $\set{a, b} \in \op{a, b}$ (called by some authors \say{pathologies}).
It does not bother us that our construction admits many true (but irrelevant) statements.
Such is the case in life.

Suppose we did choose to make the object $\op{a, b}$ primitive.
Sure, we would avoid oddities like $\set{a} \in \op{a, b}$.
And we might even get statements like $a \in \op{a, b}$ to be true.
But to do so we would have to define the meaning of $\in$ for the case in which the right hand object is an \say{ordered pair}.
Our current route avoids introducing any new concepts, and simply names a construction in our current concepts.
