%!name:vector_space_bases
%!need:linearly_dependent_vectors
%!need:span

\ssection{Why}

TODO

\ssection{Definition}

A \t{basis} \t{for} a vector space is a set of linearly independent vectors whose span is the set of vectors of the space.
For any vector in the space, there exists a linear combination of the basis vectors whose result is that vector.
In this case, it is common to say that any vector in the space \say{can be written as a linear combination of the basis vectors.}

Since the basis is a linearly independent set, the linear combination of basis vectors is unique.
We consider the

If we have a basis of $n$ vectors for $(V, \F)$ then each vector $v \in V$ can be written uniquely as a linear combination of the vectors in the basis.
If we take the vector in the field which is these coefficients, then this is an isomorphism with the vector space $(\F^n, \F)$
We call this the \t{coordinate vector}.

\ssection{Characterizations}

\begin{prop}
  A set of vectors is a basis if and only if no proper superset of it is linearly independent.
\end{prop}

\begin{prop}
  A set of vectors that spans the space is a basis if and only if no proper subset of it spans the space.
\end{prop}

