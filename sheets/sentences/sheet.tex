%!name:sentences
%!need:sets

\ssection{Why}

We want to say things about objects and sets of objects.

\ssection{Discussion}

An \t{assertion} is a sequence of symbols which is assumed to be true.

Let $a$ be an object.
Let $A$ be a set.
A \t{membership assertion} is $a \in A$.
Notice that $\in$ is not symmetric.
$a \in A$ does not assert the same meaning as $A \in a$.

Let $b$ be an object.
An \t{identity assertion}
is $a = b$.
Notice that $a = b$
asserts the same as $b = a$.

A \t{primitive sentence} is a belonging assertion or an equality assertion.
The symbolism used includes three pieces: the names of the two objects and the symbols $\in$ or $=$.

A \t{logical form} is one of several structures:

\begin{enumerate}

  \item

    and

  \item

    or (in the sense of \say{--- or --- or both})

  \item

    not

  \item

    implies (in the sense of \say{if ---, then ---}

  \item

    if and only if

  \item

    for some

  \item

    for all
\end{enumerate}

This list is redundant.

A \t{sentence} primitive sentence or a logical form with a primitive sentence or a logical form with sentences.
