%!name:sentences
%!need:sets

\ssection{Why}

We want to talk about objects and sets of objects.

\ssection{Symbols and Words}

On this page are the Latin letters forming words of the English language.
The letters of the English language are a, b, c, d, e, f, g, h, i, j, k, l, m, n, o, p, q, r, s, t, u , v, w, x, y, z.
A word is several letters next to each other.
The words are many, and listing them is no task for this sheet. 
An example will suffice.
The word \say{example} consists of the letters e, x, a, m, p, l, e.

It is an old truth that the words are visual marks corresponding to sounds, and that the sounds themselves are auditory marks corresponding to thoughts.
It is an old debate whether thought needs language, but that will not concern us.
If $x$ is an object.
$\in$$\epsilon$

\fbox{x}

\begin{center}
\fbox{
\begin{tabular}{|r|l|}
	\hline
	{\tt 1} & {\tt name A Set} \\
	{\tt 2} & {\tt name B Set} \\
	{\tt 3} & {\tt name C Set} \\
	{\tt 4} & {\tt have A subset B} \\
	{\tt 5} & {\tt have B subset C} \\
	{\tt 6} & {\tt thus A subset C} \\
	{\ttfamily
	\hline
\end{tabular}
}
\end{center}


$$
\begin{aligned}
	&\mathword{let} (x:\mathword{obj}, A: \mathword{set}) :: \\
	&x \in A
\end{aligned}
$$
$$
\begin{aligned}
	&\mathword{let} (A: \mathword{set}, B:\mathword{set}, C:\mathword{set}) :: \\
	& (A \subset B \mathword{and} B \subset C) \implies (A \subset C)
\end{aligned}
$$
\[
	[A: \mathword{set}, p: \mathword{Dist}(A)] :: (\forall a \in A, p(a) > 0 \land \sum_{a} p(a) = 1) \implies p(a) \subset [0, 1]/
\]


Let $x$ be an object and $A$ be a set.
Then if $x$ is an element of $A$ we write $x \in A$.

\texttt{$\in$}

\begin{tabular}{cl}
  & \texttt{a isa object} \\
  & \texttt{A isa set} \\
  & \texttt{a bel A}

\end{tabular}


We ask for a standard way of denoting statements about objects and sets.
We will often have a set, and another set, and another set, and we we will very quickly want to give these sets names.
Sometimes I am referencing a particular set, and sometimes I am referenc

 a standard way of denoting statements about objects and sets of objects.
It will be succinct, easily read, and easily verifiable.


Similarly we will use a formal language to succinctly denote mathematics statements about objects and sets of objects.



A \textit{symbol} is 

\ssection{Discussion}

An \t{assertion} is a sequence of symbols which is assumed to be true.

Let $a$ be an object.
Let $A$ be a set.
A \t{membership assertion} is $a \in A$.
Notice that $\in$ is not symmetric.
$a \in A$ does not assert the same meaning as $A \in a$.

Let $b$ be an object.
An \t{identity assertion}
is $a = b$.
Notice that $a = b$
asserts the same as $b = a$.

A \t{primitive sentence} is a belonging assertion or an equality assertion.
The symbolism used includes three pieces: the names of the two objects and the symbols $\in$ or $=$.

A \t{logical form} is one of several structures:

\begin{enumerate}

  \item

    and

  \item

    or (in the sense of \say{--- or --- or both})

  \item

    not

  \item

    implies (in the sense of \say{if ---, then ---}

  \item

    if and only if

  \item

    for some

  \item

    for all
\end{enumerate}

This list is redundant.

A \t{sentence} primitive sentence or a logical form with a primitive sentence or a logical form with sentences.
