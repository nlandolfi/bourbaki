%!name:empirical_law
%!need:cardinality
%!need:subsets

\ssection{Why}

Suppose we have
collected data.

\ssection{Definition}

Let $A$ be a non-empty set.
Let $n$ be a natural number.
A \ct{data set}{dataset}
of size $n$ for $A$
is a function from
$\set{1, \dots, n}$
into $A$.
It may be that $a_i = a_j$
for some $i \neq j$.

To each data set we
associate an
\ct{empirical law}{empiricallaw}
which is a probability measure $P$
on the measurable space
$(A, \powerset{A})$
that assigns to each set $B \subset A$
the number
\[
  P(B) = \frac{\card{\Set{i \in [n]}{a_i \in B}}}{n},
\]
