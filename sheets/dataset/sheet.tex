%!name:dataset
%!need:direct_products

\ssection{Why}

We want language and notation for collecting elements
from a set.

\ssection{Definition}

A \ct{dataset}{} is a finite
sequence of elements
from a set. We call the
elements of a dataset
\ct{records}{}.

We use the single word
dataset since \say{data set}
carries the connotation
of being a set of data.
The terminology dataset is
standard.
Unfortunately, however, datasets
are not sets.
Datasets (in theory and practice) contain repeat
elements.
More linguistically, \say{data} is a Latin
plural and so is analogous to saying
something like
\say{records set,} instead of
something like \say{record set.}
For these reasons we
prefer the single word dataset.

\subsection{Notation}

Let $A$ be a non-empty set.
We usually denote sequences
by $(a_1, \dots, a_n)$,
 but for datasets we will
 use a superscript,
$(a^1, \dots, a^n)$.
Often $A$ is itself
a set of sequences,
and we will want to refer
to the $i$th component of
$a^1$ by $a^1_i$.
