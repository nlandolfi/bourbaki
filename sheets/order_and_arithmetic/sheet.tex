%!name:order_and_arithmetic
%!need:natural_order
%!need:natural_products

\ssection{Why}

How does arithmetic preserve order?

\ssection{Results}

The following are standard useful results.\footnote{The accounts of which will appear in future editions.}

\begin{proposition}
	If $m < n$, then $m + k < n + k$ for all $k$.
\end{proposition}

\begin{proposition}
	If $m < n$ and $k \neq 0$, then $m \cdot k < n \cdot k$.
\end{proposition}

\begin{proposition}[Least Element]
	If $E$ is a nonempty set of natural numbers, there exists $k \in E$ such that $k \leq m$ for all $m \in E$.
\end{proposition}

\begin{proposition}[Greatest Element]
	If $E$ is a nonempty set of natural numbers, there exists $k \in E$ such that $m \leq k$ for all $m \in E$.
\end{proposition}

\blankpage