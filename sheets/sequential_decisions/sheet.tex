%!name:sequential_decisions
%!need:decisions
%!need:sequences
%!need:families

\ssection{Why}

We want to discuss making several decisions at different stages.

\ssection{Definition}

Let $A$ be a set of actions and $O$ a set of outcomes.
Let $\set{A'_{o,a}}$ be a family of action sets, $\set{O'_{o,a}}$ be a family of outcome sets and $\set{\preceq_{o,a}}$ be a family of preferences indexed by $O \times A$.

We call the sequence $(A, O, \set{A'_{a,o}}, \set{O'_{a,o}}, \set{\preceq_{a,o}})$ a \t{two-stage decision problem}.
The intuition is that we will select an action $a \in A$ prior to observing an outcome $o \in O$. 
We will then face the simple decision problem $(A'_{s,a}, O'_{s,a}, \preceq_{s,a})$.

A \t{preference} for the two stage decision problem is a total order on the set
\[
	X = \Set*{(a, o, a', o')}{a \in A, o \in O, a' \in A_{a, o}', o' \in O_{a, o}'}.
\]


Let $\CA^{1}, \dots, \CA^{n-1}$ be a finite sequence of families of actions and $\CO^1, \dots, \CO^{n}$ be a finite sequence of families of outcomes with $\CO^{1}$ and $\CO^{indexed by $(A \times O)$
\[
	
\].
In general, a \t{$n$-stage decision problem} is a sequence $(A, O, 



\blankpage
