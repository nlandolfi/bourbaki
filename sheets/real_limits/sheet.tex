%!name:real_limits
%!need:real_sequences
%!need:absolute_value

\ssection{Why}

We want to talk about sequences of real numbers which, as we go further and further in the sequence, get closer and closer to some fixed real number.

\ssection{Definition}

We need to talk about \say{where} a sequence is \say{approaching.}

A \t{limit} of a sequence $(x_n)_{n \in \N}$ of real numbers is a real number $x_0$ with the property that, for any interval centered at $x_0$, we can find a final part of the sequence wholly contained in that interval; no matter how small the interval.
By making the interval small, we capture to intuition that the sequence is \say{close} to its limit.

In other words, you propose a limit for a sequence.
To test this proposal, I specify some small positive real number.
Then we look for a final part of the sequence which is wholly contained in the interval whose width is twice the small positive number.
If we can always find the final part, no matter how small the positive number I specified, then the proposed limit is true.

\ssubsection{Existence}

Some sequences have no
limits.
Consider the sequence
which alternates between
the $+1$
and $-1$.
To show that the limit
does not exist, we
argue indirectly.
We take any real
number and test it
with the interval length
one.
No matter which
real number we have
selected,
$+1$ and
$-1$ are a
distance two apart,
and so can not
both be contained
in an interval
of width one.

In the case that there exists a limit for a sequence, we say that the sequence \t{converges to} its limit (or we say that it \t{converges}, or call it \t{convergent} or \t{converging}).

\ssubsection{Uniqueness}

If a sequence has a limit,
it has only one limit.
So, from here on, we will speak
of \t{the limit} of
the sequence.

To see this uniqueness,
suppose that two
real numbers satisfy the
limiting property.
We now argue indirectly:
suppose also that they are
not equal.
Denote the distance between
them by $x$.
Then ask for final parts
in intervals of width $x/2$
for both limits.

\ssubsection{Approximation}

We use limits to speak about
the terminating behavior of
infinite processes.
We think about the sequence
as approximating the limit.
The sequence may never
actually take the value
of its limit, so the
limit need not be in the
set of terms of the sequence.
But the idea, of course, is that the set of terms, especially those \say{far out} in the sequence, are close to the limit.

The definition, moreover,
ensures that the sequence
will get arbitrarily close.
We can operationalize this
property, by taking the first
element of that final part
after which all elements are
close to the limit.
This element of the sequence approximates the limit value well.

\ssubsection{Notation}

Let $\seq{a}$ be a sequence of real numbers.
Let $a$ be a real number.
We denote that $a$ is the limit of $\seq{a}$ by
\[
  a = \lim_{n \to \infty} a_n.
\]
The abbreviation \say{lim} is from teh Latin word \textit{limes}, which means boundary or limit.

We read this statement aloud as
\say{a is the limit of a sub n.}
The above statement asserts two
facts: (1) the sequence
$\seq{a}$ has a limit and (2)
the limit is the real number $a$.
We sometimes abbreviate
the by writing
$a = \lim_{n} a_n$.

