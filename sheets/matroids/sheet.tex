
\section*{Why}

We generalize the notion of linear dependence.

\section*{Definition}

A \t{matroid} is a finite subset algebra satisfying
    \begin{enumerate}
      \item Any subset of a distinguished set is distinguished.
      \item For two distinguished subsets of nonequal cardinality, there is an element of the base set in the complement of the smaller set in the bigger set whose singleton union with the smaller set is a distinguished set.

    \end{enumerate}
We call the distinguished subsets \t{independent subsets} and we call the undistinguished subsets the \t{dependent subsets}.

%%The first property captures the intuition that if
%%a set of vectors was linearly independent, and we dropped
%%one of the vectors, the remaining subset would still be
%%independent.
%%The second property captures the intuition that if we had
%%a set of $m$ linearly independent vectors and $n > m$ linearly
%%independent vectors, we can find a vector to add to the first
%%set and have it remain linearly independent.

\subsection*{Notation}

We follow the notation of subset algebras, but use $M$ for the base set, a mnemonic for matroid, and $\mathcal{I} $ for the distinguished sets, a menomic for independent.

Let $(M, \mathcal{I} )$ a matroid.
We denote the properties by
    \begin{enumerate}
      \item $A \in \mathcal{I}  \land B \subset A \implies B \in \mathcal{I} $.
      \item $A, B \in \mathcal{I}  \land \num{A} < \num{B} \implies \exists  x \in M: (A \cup \set{x}) \in \mathcal{I} $
    \end{enumerate}

\blankpage