
\section*{Definition}

Let $f: D \to R$ be a multivariate real-valued function where $D \subset \R ^d$.
The \t{graph} of $f$ is the set in $\R ^{d + 1}$ defined by
\[
\Set{(x, f(x)) \in \R ^d \times  \R }{x \in D}
\]
Of course, in these sheets, the graph is the \textit{same object} as the function $f$ (see discussion in \sheetref{functions}{Functions}).
The \t{epigraph} of $f$ is the set in $\R ^{d + 1}$ defined by
\[
\Set{(x, \alpha ) \in D \times  \R }{f(x) \leq \alpha }.
\]
The prefix ``epi'' is Greek, meaning ``upon'' or ``above''.
It is merited (see the visualization below) by the fact that $f \neq \epi f$.

\subsection*{Visualization of epigraph}

\begin{center}\includegraphics[width=0.80\textwidth]{./graphics/epi.pdf}\end{center}
\subsection*{Notation}

We denote the epigraph of a function $f$ by $\epi f$.

\section*{Extension to extended real numbers}

We can extend this concept in the natural way to extended real value function $f: D \to \Rbar$.
\[
\epi f \Set{(x, \alpha ) \in D \times  \R }{f(x) \leq \alpha }
\]
Caution, in this case $f \not\subset \epi f$.

% TODO: possibly rework, define convex functions in terms of sets; nearly done - NCL 2/22/23 

% historically, Archimedes had convex curves first, see Toeplitz genetic approach

\blankpage