%!name:inner_products
%!need:vectors

\ssection{Why}\footnote{Future editions will include complete and rework this sheet.}

\ssection{Definition}

Two vectors in an inner product space are \t{orthogonal} if their inner product is zero.
An \t{orthogonal family of vectors} in an inner product space is a family of vectors for which distinct family members are orthogonal.

A vector is \t{normalized} if its inner product with itself is one.

\ssubsection{Examples}

$\R^n$ with the usual inner product is an inner product space.
Some authors call any finite-dimensional inner product space over the real numbers is a \t{Euclidean vector space}.

\blankpage
