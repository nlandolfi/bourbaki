%!name:inner_products
%!need:vectors
%!need:real_inner_product

\ssection{Why}
  \ifhmode\unskip\fi\footnote{
Future editions will complete and rework this sheet.
  }

\ssection{Definition}

Let $(X, \R)$ be a vector space.
A function $f: X \times X \to \R$ is an \t{inner product} on the vector space $(X, \R)$ if
  \begin{enumerate}
  \item $f(x,x) \geq 0$, $= 0 \iff x = 0$,
  \item $f(x+y,z) = f(x,z) + f(y, z)$,
  \item $f(x,y) = f(y,x)$, and
  \item $f(\alpha x,y) = \alpha f(x,y)$.
  \end{enumerate}
An \t{inner product space} is an ordered pair: a real vector space and an inner product.
  \ifhmode\unskip\fi\footnote{
Future editions will discuss complex inner products.
  }

\ssubsection{Examples}

$\R^n$ with the usual inner product is an inner product space.
Some authors call any finite-dimensional inner product space over the real numbers is a \t{Euclidean vector space}.

\ssubsection{Examples}

If $f: X \times X \to \R$ is an inner product we regularly denote $f(x, x)$ by $\ip{x,x}$.

\ssubsection{Orthogonality}

Two vectors in an inner product space are \t{orthogonal} if their inner product is zero.
An \t{orthogonal family of vectors} in an inner product space is a family of vectors for which distinct family members are orthogonal.

A vector is \t{normalized} if its inner product with itself is one.

\blankpage
