
\section*{Why}

We abstract the notion of inner product to an arbitrary vector space.

\section*{Definition}

Suppose $\F $ is a field which is either $\R $ or $\C $. Let $(V, \F )$ be a vector space.
Then a function $f: V \times  V \to \F $ is an \t{inner product} on $V$ if
    \begin{enumerate}
      \item $f(x, x) \geq 0$, $f(x, x) = 0 \Leftrightarrow x = 0$;
      \item $f(x, y) = \overline{f(y, x)}$
      \item $f(ax + by, z) = a(x, z) + b(y, z)$
    \end{enumerate}
A \t{inner product space} (or \t{pre-Hilbert space}) is a tuple ($V, f)$ where $V$ is an inner product space over $\F $ and $f: V^2 \to \F $ is an inner product.

\subsection*{Notation}

Suppose $V$ is a vector space over the field $\F $.
We regularly denote an arbitrary inner product for $V$ by $\ip{\cdot ,\cdot }: V^2 \to \F $.
So we would denote the inner product of the vector $x$ with the vector $y$ by $\ip{x, y}$.

\blankpage