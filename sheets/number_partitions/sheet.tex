
\section*{Why}

How many ways are there to split $n$ objects into nonoverlapping groups, when the objects are indistinguishable?

\section*{Definition}

Suppose $n$ is a nonzero natural number.
A \t{partition} of $n$ is a \textit{nonincreasing} list of \textit{nonzero} natural numbers whose sum is $n$.
The requirement that the list of numbers be nonincreasing makes the representation unique.
The terms of the list are called the \t{parts} of the partition.
The number $n$ is sometimes called the \t{weight} of the partition.
The number of times a particular number appears in the list is called the \t{multiplicity} of that part.

\subsection*{Examples}

What are the partitions of the number 5?
\[
\begin{aligned}
5
&= 5 \\
&= 4 + 1 \\
&= 3 + 2 \\
&= 3 + 1 + 1 \\
&= 2 + 2 + 1 \\
&= 2 + 1 + 1 + 1 \\
&= 1 + 1 + 1 + 1 + 1
\end{aligned}
\]
These seven identities correspond to the seven partitions of 5, namely $(5,)$, $(4, 1)$, $(3, 2)$, $(3,1,1)$, $(2,2,1)$, $(2,1,1,1)$, $(1,1,1,1,1)$.
The multiplicity of 1 in these partitions is 0, 1, 0, 2, 1, 3, 5, respectively.

\subsection*{Notation}

Suppose $\lambda $ is a list in $\N  $ of length $r \geq 1$.
Then $\lambda  = (\lambda _1, \dots , \lambda _r)$ is a \textit{partition} of $n \in \N  $ if
\[
\lambda _1 + \lambda _2 + \cdots + \lambda _r = n
\]
and $\lambda _1 \geq \lambda _2 \geq \cdots \geq \lambda _r$.
The terms of $\lambda $ are the \textit{parts} of the partition, so $\lambda _i$ is the $i$th part, where $i = 1, \dots , r$.
Some authors denote the \textit{weight} of $\lambda $ by $\num{\lambda }$.

\section*{Partition function}

How many partitions are there of the number $n$?
We denote \textit{this number} by $p(n).$
From the examples above, $p(5) = 7$.

\blankpage