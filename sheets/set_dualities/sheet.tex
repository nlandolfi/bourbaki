%!name:set_dualities
%!need:set_complements
%!need:pair_unions
%!need:pair_intersections
%!refs:paul_halmos/naive_set_theory/section_05

\ss{Why}

How does taking complements relate to forming unions and intersections.

\ss{Complements of unions or intersections}

Let $E$ denote a set.
Let $A$ and $B$ denote sets and $A, B \subset E$.
All complements are taken with respect to $E$.
The following are known as \t{DeMorgan's Laws}.\footnote{Proofs will appear in a future edition.}

\begin{proposition}
  $\complement{A \union B} = \complement{A} \intersect \complement{B}$
\end{proposition}

\begin{proposition}
  $\complement{A \intersect B} = \complement{A} \union \complement{B}$
\end{proposition}

\ss{Principle of duality}

As a result of DeMorgan's Laws\footnote{A future edition will change the name to remove the reference to DeMorgan in accordance with the project's policy.} and basic facts about complements (see \sheetref{set_complements}{Set Complements}) theorems having to do with sets come in pairs.
In other words, given an inclusion or identity relation involving complements, unions and intersections of some set (above $E$) if we replace all sets by their complemnets, swap unions and intersections, and flip all inclusions we obtain another result.
This is called the \t{principle of duality for sets}.

\blankpage
