%!name:set_dualities
%!need:set_complements
%!need:pair_unions
%!need:pair_intersections
%!refs:paul_halmos∕naive_set_theory∕section_05
%!refs:bert_mendelson∕introduction_to_topology∕theory_of_sets∕set_operations

\section*{Why}

How does taking complements relate to forming unions and intersections.

\subsection*{Complements of unions or intersections}

Let $E$ denote a set.
Let $A$ and $B$ denote sets and $A, B \subset E$.
All complements are taken with respect to $E$.
The following are known as \t{DeMorgan's Laws}.
  \ifhmode\unskip\fi\footnote{
Proofs will appear in a future edition.
  }

\begin{proposition}$\complement{A \union B} = \complement{A} \intersect \complement{B}$\end{proposition}
\begin{proposition}$\complement{A \intersect B} = \complement{A} \union \complement{B}$\end{proposition}
\subsection*{Principle of duality}

As a result of DeMorgan's Laws
  \ifhmode\unskip\fi\footnote{
Future editions will change the name to remove the reference to DeMorgan in accordance with the project's policy on naming.
  }
and basic facts about complements (see \sheetref{set\_complements}{Set Complements}) theorems about sets often come in pairs.
In other words, given an inclusion or identity relation involving complements, unions and intersections of some set (above $E$) if we replace all sets by their complemnets, swap unions and intersections, and flip all inclusions we obtain another, true, result.
The correspondence is called the \t{principle of duality for sets}.

\blankpage