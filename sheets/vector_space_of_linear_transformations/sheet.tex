
\section*{Why}

Can we think of linear maps as vectors?

\section*{Definitions}

Suppose $V$ and $W$ are some vector spaces over a field $\F $.
Denote the linear maps from $V$ to $W$ by $\mathcal{L} (V, W)$ as usual.

\textit{Addition.}
Given $S, T \in \mathcal{L} (V, W)$ the \t{sum} of $S$ and $T$ is the linear map $R \in \mathcal{L} (V, W)$ defined by
\[
Rv = Sv + Tv \quad \text{for all } v \in V
\]

\textit{Scalar multiplication.}
Given $S \in \mathcal{L} (V, W)$ the \t{(scalar) product} of $\lambda $ and $T$ is the linear map $Q \in \mathcal{L} (V, W)$ defined by
\[
Qv = \lambda Tv \quad \text{for all } v \in V
\]

\begin{proposition}
Suppose $V$ and $W$ are two vector spaces over the same field $\F $.
Then $\mathcal{L} (V, W)$ is a vector space over the field $\F $ with respect to the operations of addition and scalar multiplication just defined.
\end{proposition}

The additive identity of the vector space $\mathcal{L} (V, W)$ is the zero map $0 \in \mathcal{L} (V, w)$.

\subsection*{Notation}

Given $S, T \in \mathcal{L} (V, W)$ the \t{sum} of $S$ and $T$ and $\lambda  \in \F $, we denote the sum of $S$ and $T$ by $S + T$.
Hence,
\[
(S + T)(v) = Sv + Tv \quad \text{for all } v \in V
\]
We denote the product of $\lambda $ and $T$ by $\lambda T$.
Hence,
\[
(\lambda T)(v) = \lambda (Tv) \quad \text{for all } v \in V
\]

\blankpage