
%!name:outcome_variable_events
%!need:event_probabilities

\section*{Why}

For each value of a random variable's codomain, the set of outcomes corresponding to that value is the inverse image of the random variable.
We can speak of the probability that a random variable takes a value then, by assigning it the probability of the set of outcomes corresponding to that value.

\section*{Definition}

Let $p: \Omega  \to \R $ be a probability distribution with corresponding probability measure $\mathbfsf{P} : \powerset{\Omega } \to \R $.
Suppose $x: \Omega  \to V$ is an outcome variable.
The \t{probability $x = a$}, for $a \in \Omega $, is
\[
\textstyle
\mathbfsf{P} (\Set{\omega  \in \Omega }{x(\omega ) = a}).
\]
From the definition of $\mathbfsf{P} $, we express the above as
\[
\textstyle
\sum_{\omega  \in \Omega  \mid  x(\omega ) = a} p(\omega ).
\]
We refer to the probability of the \textit{event} that $x = a$.

\subsection*{Notation}

We denote the probability that $x = a$ by $\mathbfsf{P} [x = a]$.
Our square brackets deviate from the slightly slippery but universally standard notation $\mathbfsf{P} (x = a)$.
We prefer the square brackets, since $x=a$ is not itself an argument to $\mathbfsf{P} $, but shorthand for the set $\Set{\omega  \in \Omega }{x(\omega ) = a}$.

There are many similar notations.
For example, $\mathbfsf{P} [x \in C]$ means $\mathbfsf{P} (\Set{x \in \Omega }{x(\omega ) \in C})$.
In particular, if $x: \Omega  \to \R $, $\mathbfsf{P} [x \geq a]$ means $\mathbfsf{P} (\Set{\omega  \in \Omega }{x(\omega ) \geq a})$.
Since the \textit{event} that $x = a$ is the inverse image of $\set{a}$ under $x$, we also use the notations $\mathbfsf{P} (x^{-1}(a))$ and $\mathbfsf{P} (x^{-1}(C))$.

\subsection*{Example: sum of two dice}

Define $\Omega  = \set{1, \dots , 6}^2$ and define $p: \Omega \to \R $ with $p(\omega ) = 1/36$ for each $\omega  \in \Omega $.
Define $x: \Omega  \to \N  $ by $x(\omega _1, \omega _2) = \omega _1 + \omega _2$.
Then
\[
\mathbfsf{P} [x = 4] = p((2, 2)) + p(1, 3) + p(3, 1) = 1/12.
\]
