%!name:matrix_eigenvectors
%!need:matrix_vector_products
%!need:norms

\ssection{Why}

TODO

\ssection{Definition}

If the result of multiplying a vector by a real square matrix is the same as scaling the vector by a real number then we call the vector a \t{scaled vector}.
If a vector is a scaled vector, then the result of scaling it with any real number is a scaled vector.
% ($Ax = \alpha x \implies A \beta x = \beta Ax = \beta \alpha x$).

The scaled version of a scaled vector is acted upon by the matrix in a similar manner as the original scaled vector.
So we want to characterize this notion by picking one representative member from this set of vectors which are scalar multiples of each other (TODO: line?), and a reasonable choice is the vector which has norm one.

An \t{eigenvector} is a scaled vector whose norm is one.
An \t{eigenvalue} corresponding to an eigenvector is the real number by which the eigenvector is scaled by the matrix.
So eigenvectors and eigenvalues come in pairs.
