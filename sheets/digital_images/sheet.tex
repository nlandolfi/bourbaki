
%!name:digital_images
%!need:bit_strings
%!need:arrays

\section*{Why}

We want to represent images.

\section*{Definitions}

A \t{binary image} is a two-dimensional array in $\set{0, 1}$.
We call the first coordinate of the shape the \t{height} and the second coordinate the \t{width} of the image.
A \t{grayscale image} is a two-dimensional array in $\set{0, \dots , n}$, in which $n$ represents full saturation and all integers less than $n$ represent increasing saturation.

An \t{RGB digital color} is a length three sequence of bit strings.
The terms of the color are known as the channels.
So, for example, the first term of the sequence is the \t{first channel}, the second term is the \t{second channel}, and the third term is the \t{third channel}.

An \t{RGB digital image} is a two-dimensional array in the set of RGB digital colors.
The \t{image channels} are the two-dimensional array whose values are those of the corresponding channel in the original image.
So, for example, the \t{first channel image} is the grayscale image whose values are the first channel of the original digital image.

We refer to any of these (binary images, grayscale images, RGB digital images) as \t{digital images}, or often, as \t{images}.

\blankpage