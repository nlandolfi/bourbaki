
%!name:entire_functions
%!need:complex_analytic_functions
%!refs:yellow/IX/4

\section*{Definition}

An \t{entire function} is a complex function $f: \C  \to \C $ which is analytic for all $z \in \C $.

\blankpage
\sbasic

\sinput{../sets/macros.tex}
%%%% MACROS %%%%%%%%%%%%%%%%%%%%%%%%%%%%%%%%%%%%%%%%%%%%%%%

\newcommand{\PM}{\mathbf{P}}

%%%%%%%%%%%%%%%%%%%%%%%%%%%%%%%%%%%%%%%%%%%%%%%%%%%%%%%%%%%


\sstart

\stitle{Set Notation}

\ssection{Why}

We want to write down
statements about objects
and sets.

\ssection{Notation}

To aid in discussing and
denoting objects, let
us tend to give them
short names.
A single Latin
letter regularly suffices:
for example,
$a$, $b$ or $c$.
Let us denote that
the object $a$ and
the object $b$ are
the same object
by $a = b$,
read aloud as
\say{a is b.}

For sets,
let us tend to use
upper case Latin letters:
for example,
$A$, $B$, and $C$.
To aid our memory,
let us tend to use the lower
case form of the letter for
an element of the set.
For example,
if $A$ is a set,
we tend to denote by
$a$ an element of $A$.
Likewise, if $B$ is a set,
we tend to denote
by $b$ an element of $B$.


Let us denote that
an object $a$
is an element of a set $A$
by $a \in A$.
We read the notation
$a \in A$ aloud as \say{a in A.}
The $\in$ is a stylized
lower case Greek letter: $\epsilon$.
It is
read aloud \say{ehp-sih-lawn} and
is a mnemonic for \say{element of}.
We write $a \not\in A$, read aloud
as \say{a not in A,} if $a$ is not
an element of $A$.

If we have named
the elements of a set,
and can list them,
let us do so between braces.
For example,
let $a$, $b$, and $c$
be three distinct objects.
Denote by $\set{a, b, c}$
the set containing theses
three objects and only these
three objects.
We can further compress notation,
and denote this set of
three objects by $A$:
so, $A = \set{a, b, c}$.
Then $a \in A$,
$b \in A$, and $c \in A$.
Moreover, if $d$
is an object and
$d \in A$, then $d = a$
or $d = b$ or $d = c$.

If the elements of a set are
so well-known that we can
avoid ambiguityi, then we can
describe the set in English.
To aid our memory,
let us tend to name such sets
mnemonically.
For example,
let $L$ be the set of Latin letters.

Often to be more precise, we should
explicitly deal with objects which
satisfy several conditions.
If the elements of a set satisfy
some common condition, then we use
the braces and include the condition.
For example, let $V$ be the set of
Latin vowels.
We can denote $V$ by
$\Set{l \in L}{l \text{ is a vowel}}$.
We read the symbol $\mid$ aloud as
\say{such that.}
We read the whole notation aloud as
\say{l in L such that l is a vowel.}
We call the notation
\ct{set-builder notation}{setbuildernotation}.
Set-builder notation is indispensable for
sets defined implicitly by some condition.
Here we could have alternatively denoted
$V$ by
$\set{\say{a},\say{e},\say{i},\say{o},\say{u}}$.
We prefer the former, slightly more concise notation.

%We develop herein a language
%for specifying things by either
%listing them explicitly or
%by listing their defining properties.

\strats
