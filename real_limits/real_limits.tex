
%!name:entire_functions
%!need:complex_analytic_functions
%!refs:yellow/IX/4

\section*{Definition}

An \t{entire function} is a complex function $f: \C  \to \C $ which is analytic for all $z \in \C $.

\blankpage
\sbasic

\sinput{../sets/macros.tex}
\sinput{../absolute_value/macros.tex}
\sinput{../sequences/macros.tex}
%%%% MACROS %%%%%%%%%%%%%%%%%%%%%%%%%%%%%%%%%%%%%%%%%%%%%%%

\newcommand{\PM}{\mathbf{P}}

%%%%%%%%%%%%%%%%%%%%%%%%%%%%%%%%%%%%%%%%%%%%%%%%%%%%%%%%%%%


%%%% MACROS %%%%%%%%%%%%%%%%%%%%%%%%%%%%%%%%%%%%%%%%%%%%%%%

\newcommand{\PM}{\mathbf{P}}

%%%%%%%%%%%%%%%%%%%%%%%%%%%%%%%%%%%%%%%%%%%%%%%%%%%%%%%%%%%


\sstart

\stitle{Real Limits}

\ssection{Why}

We want to speak of an
infinite process which,
although never arrives,
does terminate.

\ssection{Definition}

A
\ct{limit}{alimit}
of a sequence of real
numbers is a real number
for which we can always
find a final part of the
sequence wholly contained
in an interval around the limit,
no matter how small the interval.

You propose a limit
for a sequence.
To test this proposal,
I specify some small
positive real number.
Then we look for a
final part wholly contained
in the interval of that
width.
If we can always find the
final part, no matter
how small the positive
number I specified,
then the proposed limit
is true.

\ssubsection{Existence}

Some sequences have no
limits.
Consider the sequence
which alternates between
the $+1$
and $-1$.
To show that the limit
does not exist, we
argue indirectly.
We take any real
number and test it
with the interval length
one.
No matter which
real number we have
selected,
$+1$ and
$-1$ are a
distance two apart,
and so can not
both be contained
in an interval
of width one.


\ssubsection{Uniqueness}

If a sequence has a limit,
it has only one limit.
So, from here on, we will speak
of \ct{the limit}{limit} of
the sequence.

To see this uniqueness,
suppose that two
real numbers satisfy the
limiting property.
We now argue indirectly:
suppose also that they are
not equal.
Denote the distance between
them by $x$.
Then ask for final parts
in intervals of width $x/2$
for both limits.

\ssubsection{Approximation}

We use limits to speak about
the terminating behavior of
infinite processes.
We think about the sequence
as approximating the limit.
The sequence may never
actually take the value
of its limit, so the
limit need be in the
set of terms of the sequence,
but it does get close.

The definition, moreover,
ensures that the sequence
will get arbitrarily close.
We can operationalize this
property, by taking the first
element of that final part
after which all elements are
close to the limit.
This element is an element
of the sequence approximates
the limit value well.

\ssubsection{Notation}

Let $\seq{a}$
be a sequence of
real numbers.
Let $a$ be a real number.
We denote that $a$ is the limit of
$\seq{a}$ by
\[
  a = \lim_{n \to \infty} a_n.
\]

We read this statement aloud as
\say{a is the limit of a sub n.}
The above statement asserts two
facts: (1) the sequence
$\seq{a}$ has a limit and (2)
the limit is the real number $a$.
We sometimes abbreviate
the by writing
$a = \lim_{n} a_n$.

\strats
