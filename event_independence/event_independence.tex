
%!name:entire_functions
%!need:complex_analytic_functions
%!refs:yellow/IX/4

\section*{Definition}

An \t{entire function} is a complex function $f: \C  \to \C $ which is analytic for all $z \in \C $.

\blankpage
\sbasic

\sinput{../sets/macros.tex}
\sinput{../set_operations/macros.tex}
%%%% MACROS %%%%%%%%%%%%%%%%%%%%%%%%%%%%%%%%%%%%%%%%%%%%%%%

\newcommand{\PM}{\mathbf{P}}

%%%%%%%%%%%%%%%%%%%%%%%%%%%%%%%%%%%%%%%%%%%%%%%%%%%%%%%%%%%


\sstart

\stitle{Event Independence}

\ssection{Why}

TODO

\ssection{Definition}

The sigma algebra
\ct{generated by an event}{}
is the sigma algebra consisting
of the empty set, the event,
the complement (in the base set)
of the event,
and the base set.

A family of events
events are
\ct{independent}{eventindependence}
if the sigma algebras generated by
the events are independent.

\ssubsection{Notation}

Let $(X, \SA, \mu)$ be a probability
space.
Let $A \in \SA$ be an event.
The sigma algebra generated by
$A$ is $\set*{\emptyset, A, X - A, X}$.
We denote it by $\sigma(A)$.

Let $B \in \SA$.
If $A$ is independent of $B$ we
write $A \perp B$.

\ssection{Equivalent Condition}

\begin{prop}
Two events are independent
if and only if the measure
of their intersection is
the product of their measures.
\begin{proof}
Let $(X, \SA, \mu)$
be a probability space.
Let $A, B \in \SA$.

$(\Rightarrow)$ If $A \perp B$, then
by definition $A \in \sigma(A)$
and $B \in \sigma(B)$ and so:
\[
  \mu(A \intersect B) = \mu(A)\mu(B).
\]

$(\Leftarrow)$ Conversely, let $a \in \sigma(A)$
and $b \in \sigma(B)$.
If $a = \emptyset$ or $b = \emptyset$
then $a \intersect b = \emptyset$. So
\[
  \mu(a \intersect b) = \mu(\emptyset) = \mu(a)\mu(b),
\]
since one of the two measures on the right hand side
is zero.
On the other hand, if $a = X$, then
$a \intersect b = b$ and so
\[
  \mu(a \intersect b) = \mu(b) = \mu(a)\mu(b),
\]
since $\mu(a) = \mu(X) = 1$.
Likewise if $b = X$.

So it remains to verify $\mu(a\intersect b) = \mu(a)\mu(b)$
for the cases $a \in \set{A,X -A}$ and $b \in \set{B, X-B}$.
%\[
%  \begin{aligned}
%    \mu(A \intersect B) &= \mu(A)\mu(B) \\
%    \mu(A \intersect (X - B)) &= \mu(A)\mu(X - B) \\
%    \mu((X - A) \intersect B) &= \mu(X - A)\mu(B) \\
%    \mu((X - A) \intersect (X- B)) &= \mu(X - A)\mu(X - B) \\
%  \end{aligned}
%\]
If $a = A$, and $b = B$, then the identity
follows by hypothesis.
Next, observe that
$A \intersect (X - B) = A - (A \intersect B)$
and $(A \intersect B) \subset A$ so $\mu(X) < \infty$
allows us to deduce:
\[
  \begin{aligned}
    \mu(A \intersect (X - B)) &= \mu(A - (A \intersect B)) \\
    &= \mu(A) - \mu(A \intersect B) \\
    &= \mu(A)(1 - \mu(B)) \\
    &= \mu(A)\mu(X - A).
  \end{aligned}
\]
Similar for $X-A$ and $B$.
Finally, recall that
$\mu(A\union B) = \mu(A) + \mu(B) - \mu(A\intersect B)$.
So then,
\[
  \begin{aligned}
    \mu((X - A) \intersect (X - B)) &= 1 - \mu(A \union B) \\
    &= 1 - \mu(A) - \mu(B) + \mu(A \intersect B) \\
    &= 1 - \mu(A) - \mu(B) + \mu(A)\mu(B) \\
    &= (1 - \mu(A))(1 - \mu(B)) \\
    &= \mu(X - A)\mu(X - B).
  \end{aligned}
\]
\end{proof}
\end{prop}

\strats
