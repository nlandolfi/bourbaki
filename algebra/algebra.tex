
%!name:entire_functions
%!need:complex_analytic_functions
%!refs:yellow/IX/4

\section*{Definition}

An \t{entire function} is a complex function $f: \C  \to \C $ which is analytic for all $z \in \C $.

\blankpage
\sbasic

\sinput{../sets/macros.tex}
\sinput{../cartesian_product/macros.tex}
\sinput{../relations/macros.tex}

%%%% MACROS %%%%%%%%%%%%%%%%%%%%%%%%%%%%%%%%%%%%%%%%%%%%%%%

\newcommand{\PM}{\mathbf{P}}

%%%%%%%%%%%%%%%%%%%%%%%%%%%%%%%%%%%%%%%%%%%%%%%%%%%%%%%%%%%



\sstart
\stitle{Algebra}

\ssection{Why}

We want to combine set elements to get
other set elements.

\ssection{Basics}


An \ct{operation}{operation} on a set is
a function from ordered pairs of elements
in the set to the same set.
We use operations to combine the elements.
We operate on pairs.
An \ct{algebra}{algebra} is a set and
an operation.
We call the set the
\ct{ground set}{groundset}.


\ssubsection{Notation}

Let $A$ a set and $g: A \times A \to A$.
We commonly forego the notation $g(a, b)$ and instead write $a\,g\,b$.
We call this \ct{infix notation}{infixnotation}.

Using lower case latin letters for every the elements and for the operation is confusing, but we often have special symbols for particular operations.
For example,
$+$, $-$, $\cdot$, $\circ$, and $\star$.

If we had a set $A$ and an operation $+: A \times A \to A$, we would write $a+b$ for the result of applying $+$ to $(a,b)$.
In denoting the algebra, we would say let $(A, +)$ be an algebra.

\ssection{Operation Properties}

An operation \ct{commutes}{commutes} if
the result of two elements is the same
regardless of their order; we call the
operation \ct{commutative}.

An operation \ct{associates}{associates}
if given any three elements in order it
doesn't matter whether we first operate
on the first two and then with the result
of the first two the third, or the second
two and with the result of the second two
the first.

A first operation over a set
\ct{distributes}{distributes}
over a second operation over the same set
if the result of applying the first
operation to an element and a result of
the second operation is the same as
applying the second operation to the results
of the first operation with the arguments
of the second operation.

\ssubsection{Notation}

Let $(A, +)$ an algebra.

We denote that $+$ commutes by asserting
$$
  a + b = b + a
$$
for all $a, b \in A$.
We denote that $+$ associates by asserting
$$
  (a + b) + c = (a + b) + c
$$
for all $a, b, c \in A$.
Let $(A, \cdot)$ a second algebra over the
same set.
We denote that $\cdot$ distributes over $+$
by
$$
  a \cdot (b + c) =
  (a \cdot b) + (a \cdot c)
$$
for all $a, b, c \in A$.

\ssection{Identity Elements}

We call $e \in A$ an \ct{identity element}{identityelement}
if (1) $e + a = e$ and (2) $a + e = e$ for all $a \in A$.
If only (1) holds, we call $e$ a
\ct{left identity}{leftidentityelement}.
If only (2) holds, we call $e$ a
\ct{right identity}{rightidentityelement}.

\ssection{Inverse Elements}

We call $b \in A$ an \ct{inverse element}{inverseelement}
of $a \in A$ if (1) $b + a = e$ and (2) $a + b = e$.
If only (1) holds, we call $e$ a
\ct{left inverse}{leftinverseelement}.
If only (2) holds, we call $e$ a
\ct{right inverse}{rightinverseelement}.

\strats
