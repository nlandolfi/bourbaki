
%!name:entire_functions
%!need:complex_analytic_functions
%!refs:yellow/IX/4

\section*{Definition}

An \t{entire function} is a complex function $f: \C  \to \C $ which is analytic for all $z \in \C $.

\blankpage
\sbasic

\sinput{../sets/macros.tex}
\sinput{../cartesian_product/macros.tex}
\sinput{../relations/macros.tex}

%%%% MACROS %%%%%%%%%%%%%%%%%%%%%%%%%%%%%%%%%%%%%%%%%%%%%%%

\newcommand{\PM}{\mathbf{P}}

%%%%%%%%%%%%%%%%%%%%%%%%%%%%%%%%%%%%%%%%%%%%%%%%%%%%%%%%%%%



\sstart
\stitle{Algebra}

\ssection{Why}

We want to combine set elements to get other set elements.

\ssection{Basics}


Let $A$ be a non-empty set.
An \ct{operation}{operation} on $A$ is a function $g: A \times A \to A$.
Operations map ordered pairs of elements of a set to elements of the same set.
An \ct{algebra}{algebra} is a set and an operation.
We call the set the \ct{ground set}{groundset}.


\ssubsection{Notation}

Let $A$ a set and $g: A \times A \to A$.
We commonly forego the notation $g(a, b)$ and instead write $a\,g\,b$.
We call this style \ct{infix notation}{infixnotation}.

Using lower case latin letters for every the elements and for the operation is confusing, but we often have special symbols for particular operations.
Examples of such symbols include $+$, $-$, $\cdot$, $\circ$, and $\star$.

If we had a set $A$ and an operation $+: A \times A \to A$, we would write $a+b$ for the result of applying $+$ to $(a,b)$.
In denoting the algebra, we would say let $(A, +)$ be an algebra.

\ssection{Operation Properties}

Let $(A, +)$ be an algebra.
$+$ is \ct{commutative}{commutative} if $a + b = b + a$ for all $a, b \in A$.
In this case we say that $+$ \ct{commutes}{commutes}.
$+$ is \ct{associative}{associative} if $(a + b) + c = a + (b + c)$ for all $a, b, c, \in A$.


\strats
