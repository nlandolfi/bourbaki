
%!name:entire_functions
%!need:complex_analytic_functions
%!refs:yellow/IX/4

\section*{Definition}

An \t{entire function} is a complex function $f: \C  \to \C $ which is analytic for all $z \in \C $.

\blankpage
\sbasic

\sinput{../sets/macros.tex}
\sinput{../ordered_pairs/macros.tex}
\sinput{../set_operations/macros.tex}
\sinput{../cardinality/macros.tex}
%%%% MACROS %%%%%%%%%%%%%%%%%%%%%%%%%%%%%%%%%%%%%%%%%%%%%%%

\newcommand{\PM}{\mathbf{P}}

%%%%%%%%%%%%%%%%%%%%%%%%%%%%%%%%%%%%%%%%%%%%%%%%%%%%%%%%%%%


\sstart

\stitle{Subset Algebra}

\ssection{Why}

We speak of a subset space
with set-algebraic
properties.

\ssection{Definition}

A \ct{subset algebra}{subsetalgebra}
is a subset space for which
(1) the base set is distinguished
(2) the complement of a distinguished
set is distinguished
(3) the union of two distinguished sets
is distinguished.

We say that the set of distinguished sets
is an \ct{algebra}{} on the the base set.
The properties listed below justify this
language by showing that the standard set
operations applied to distinguished sets
result in distinguished sets.

\ssubsection{Notation}

The notation follows that of a subset space.
Let $(A, \mathcal{A})$ be a subset algebra.
We also say \say{let $\mathcal{A}$ be
an algebra on $A$.}
Moreover, since the largest element of the
algebra is the base set, we can say without
ambiguity: \say{let $\mathcal{A}$ be an algebra.}

\ssection{Properties}

\begin{prop}
  For any subset algebra, $\emptyset$ is distinguished.
\end{prop}

\begin{prop}
  For any subset algebra,
  for any distinguished sets,
  (a) the intersection is distinguished and
  (b) their symmetric difference is distinguished.
  So, if one contains the other, the complement
  of the smaller in the larger is distinguished.
\end{prop}

\begin{prop}
  For any subset algebra,
  for any finite family of distinguished sets,
  (a) the finite family union and
  (b) the finite family intersection
  are both distinguished.
\end{prop}

\ssection{Examples}

\begin{expl}
  For any set $A$, $(A, 2^{A})$ is a subset algebra.
\end{expl}

\begin{expl}
  For any set $A$, $(A, \set{A, \emptyset})$ is a subset algebra.
\end{expl}

\begin{expl}
  For any infinite set $A$,
  let $\mathcal{A}$ be the set
  $$
  \Set*{
    B \subset A
  }{
    \card{B} < \aleph_0 \lor
    \card{C_{A}(B)} < \aleph_0
  }.
  $$
  $\mathcal{A}$ is an algebra;
  the
  \ct{finite/co-finite algebra}{}.
\end{expl}

\begin{expl}
  For any infinite set $A$,
  let $\mathcal{A}$ be the set
  $$
  \Set*{
    B \subset A
  }{
    \card{B} \leq \aleph_0 \lor
    \card{C_{A}(B)} \leq \aleph_0
  }.
  $$
  $\mathcal{A}$ is an algebra;
  the
  \ct{countable/co-countable algebra}{}.
\end{expl}

\begin{expl}
  For any infinite set $A$,
  let $\mathcal{A}$ be the set
  $$
  \Set*{
    B \subset A
  }{
    \card{B} \leq \aleph_0
  }.
  $$
  $\mathcal{A}$ is not an algebra.
\end{expl}

\strats
