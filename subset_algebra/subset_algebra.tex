
%!name:entire_functions
%!need:complex_analytic_functions
%!refs:yellow/IX/4

\section*{Definition}

An \t{entire function} is a complex function $f: \C  \to \C $ which is analytic for all $z \in \C $.

\blankpage
\sbasic

\sinput{../sets/macros.tex}
\sinput{../cartesian_product/macros.tex}
%%%% MACROS %%%%%%%%%%%%%%%%%%%%%%%%%%%%%%%%%%%%%%%%%%%%%%%

\newcommand{\PM}{\mathbf{P}}

%%%%%%%%%%%%%%%%%%%%%%%%%%%%%%%%%%%%%%%%%%%%%%%%%%%%%%%%%%%


\sstart

\stitle{Subset Algebra}

\ssection{Why}

We often talk about a \rt{set}{set} and
a \rt{set}{set} of its \rt{subsets}{subset}
satisfying properties.

\ssection{Definition}

A \ct{subset algbera}{subsetalgebra} is a
\rt{tuple}{tuple} of \rt{sets}{set}: the second
is a \rt{set}{set} of \rt{subsets}{subset} of the first.

We call the first \rt{set}{set} the
\ct{base set}{baseset}.
If the \rt{base set}{baseset} is finite,
we call the subset algebra a
\ct{finite subset algebra}{finitesubsetalgebra}.
We call an \rt{element of}{element} the second
\rt{set}{set} a \ct{distinguished subset}{distinguishedsubset}.
A \rt{subset}{subset} is an
\ct{undistingished subset}{undistinguishedsubset}
if it is not \rt{distinguished}{distinguishedsubset}.

Useful \rt{subset algebras}{subsetalgebra} are those for
which the \rt{distinguished subsets}{distinguishedsubset}
satisfy some \rt{set-algebraic}{setalgebraicoperations}
properties.
For one example, the distinguished sets may be closed under set union or
set intersection.
As a second example, the distinguished sets may include the base
set.
As a third example, the distinguished sets may be closed under
complements or under subsets.

\ssubsection{Notation}

Let $A$ a set and $\mathcal{A} \subset \powerset{A}$.
We denote the subset algebra of $A$ and $\mathcal{A}$
by $\tuple{A, \mathcal{A}}$, read aloud as \say{A, script
  A.}

\strats
