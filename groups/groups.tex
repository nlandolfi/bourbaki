
%!name:entire_functions
%!need:complex_analytic_functions
%!refs:yellow/IX/4

\section*{Definition}

An \t{entire function} is a complex function $f: \C  \to \C $ which is analytic for all $z \in \C $.

\blankpage
\sbasic

%%%% MACROS %%%%%%%%%%%%%%%%%%%%%%%%%%%%%%%%%%%%%%%%%%%%%%%

\newcommand{\PM}{\mathbf{P}}

%%%%%%%%%%%%%%%%%%%%%%%%%%%%%%%%%%%%%%%%%%%%%%%%%%%%%%%%%%%


\sstart

\stitle{Groups}

\ssection{Why}

We generalize the algebraic structure of addition over the integers.

\ssection{Definition}

Let $(A, +)$ be an algebra.

We call $e \in A$ an \ct{identity element}{identityelement}
if (1) $e + a = e$ and (2) $a + e = e$ for all $a \in A$.
If only (1) holds, we call $e$ a
\ct{left identity}{leftidentityelement}.
If only (2) holds, we call $e$ a
\ct{right identity}{rightidentityelement}.

We call $b \in A$ an \ct{inverse element}{inverseelement}
of $a \in A$ if (1) $b + a = e$ and (2) $a + b = e$.
If only (1) holds, we call $e$ a
\ct{left inverse}{leftinverseelement}.
If only (2) holds, we call $e$ a
\ct{right inverse}{rightinverseelement}.

A \ct{group}{group} is an algebra $(A, +)$ where
$+$ is associative,
there exists an \rt{identity element}{identityelement}
in $A$, and there exists an \rt{inverse}{inverseelement} for
each \rt{element of}{element} $A$.
A \ct{commutative group}{commutativegroup} is a group
$(A, +)$ where $+$ commutes.
A commutative group is also called an
\ct{Abelian group}{abeliangroup}.

\ssubsection{Notation}

\boxed{TODO}

\strats
