
%!name:entire_functions
%!need:complex_analytic_functions
%!refs:yellow/IX/4

\section*{Definition}

An \t{entire function} is a complex function $f: \C  \to \C $ which is analytic for all $z \in \C $.

\blankpage
\sbasic

%%%% MACROS %%%%%%%%%%%%%%%%%%%%%%%%%%%%%%%%%%%%%%%%%%%%%%%

\newcommand{\PM}{\mathbf{P}}

%%%%%%%%%%%%%%%%%%%%%%%%%%%%%%%%%%%%%%%%%%%%%%%%%%%%%%%%%%%


\sstart

\stitle{Groups}

\ssection{Why}

We generalize the algebraic structure of addition over the integers.

\ssection{Definition}

Let $(A, +)$ be an algebra.

We call $e \in A$ an \definition{identity} if
(1) $e + a = e$ and (2) $a + e = e$ for all $a \in A$.
If only (1) holds, we call $e$ a \definition{left identity}.
If only (2) holds, we call $e$ a \definition{right identity}.

We call $b \in A$ an \definition{inverse} of $a \in A$
if (1) $b + a = e$ and (2) $a + b = e$.
If only (1) holds, we call $e$ a \definition{left inverse}.
If only (2) holds, we call $e$ a \definition{right inverse}.

A \definition{group} is an algebra $(A, +)$ where
$+$ is associative,
there exists an identity element in $A$, and
there exists an inverse for each element of $A$.
A \definition{commutative group} is a group $(A, +)$ where $+$ commutes.
A commutative group is also called an \definition{Abelian group}.

\ssubsection{Notation}

\boxed{TODO}

\strats
