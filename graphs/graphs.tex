
%!name:entire_functions
%!need:complex_analytic_functions
%!refs:yellow/IX/4

\section*{Definition}

An \t{entire function} is a complex function $f: \C  \to \C $ which is analytic for all $z \in \C $.

\blankpage
\sbasic

%%%% MACROS %%%%%%%%%%%%%%%%%%%%%%%%%%%%%%%%%%%%%%%%%%%%%%%

\newcommand{\PM}{\mathbf{P}}

%%%%%%%%%%%%%%%%%%%%%%%%%%%%%%%%%%%%%%%%%%%%%%%%%%%%%%%%%%%


\sstart

\stitle{Graphs}

\ssection{Why}

We want to visualize relations.

\ssection{Definition}

A \ct{graph}{graph} is a set and a relation
on the set.
The graph is \ct{undirected}{undirectedgraph}
if the relation is symmetric; otherwise the
graph is \ct{directed}{directedgraph}.

A \ct{vertex}{vertex} of the graph is an element
of the set.
The set is called the \ct{vertex set}{vertexset}.
An \ct{edge}{edge} of the graph is an element of
the relation.
The relation is called the \ct{edge set}{edgeset}.

%If we have a relation between two sets, the
%standard corresponding graph is that obtained
%by considering the relation as defined on the
%union of the two sets. In this case we say the
%graph is \ct{bipartite}{bipartite}.

\ssubsection{Notation}

We denote the vertex set by $V$, a mnemonic for
vertex.
We denote the edge set by $E$, a mnemonic for
edge.
We denote a graph by $(V, E)$.
If the vertex set is assumed we can
unambiguously refer to the graph by $E$.

\ssubsection{Visualization}

We visualize the graph by drawing a
point for each vertex.
If two vertices $u$ and $v$ are in relation,
we draw a line from the point corresponding
to $u$ to the point corresponding to $v$ with
an arrow at the point corresponding to $v$.
If the graph is undirected, we omit arrows.

\ssection{Paths}

A path in a relation is a sequence of
elements in which consecutive elements
are related.
A path \ct{cycles}{cycles} if an element
appears more than once.
A path is \ct{finite}{pathfinite} if the
sequence is finite.
A finite path is a \ct{loop}{loop} if it
cycles once.

\strats
