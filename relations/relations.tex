
%!name:entire_functions
%!need:complex_analytic_functions
%!refs:yellow/IX/4

\section*{Definition}

An \t{entire function} is a complex function $f: \C  \to \C $ which is analytic for all $z \in \C $.

\blankpage
\sbasic

\sinput{../sets/macros.tex}
\sinput{../cartesian_product/macros.tex}

%%%% MACROS %%%%%%%%%%%%%%%%%%%%%%%%%%%%%%%%%%%%%%%%%%%%%%%

\newcommand{\PM}{\mathbf{P}}

%%%%%%%%%%%%%%%%%%%%%%%%%%%%%%%%%%%%%%%%%%%%%%%%%%%%%%%%%%%


\sstart

\stitle{Relations}

\ssection{Why}
We want to relate elements of two sets.


\ssection{Definition}

A \definition{relation}{relation} between two non-empty \term{sets}{sets} $A$ and $B$ is a subset of $A \cross B$.
So then, naturally, a relation on a single set $C$ is a subset of $C \cross C$.

\ssubsection{Notation}
As relations are sets, our de facto protocol is to denote them by upper case capital letters, for example, the letter $R$.
Let $R$ a relation on $A$ and $B$.
If $(a, b) \in R$, we often write $a R b$, read aloud as \say{a in relation $R$ to b.}

In many cases, though, we forego the set notation and use particular symbols.
Often the symbols we use are meant to be suggestive of the relation.
Some examples include $\sim$, $=$, $<$, $\leq$, $\prec$, and $\preceq$.

\ssection{Properties}

Let $R$ a relation on a non-empty set $A$.
$R$ is \definition{reflexive}{reflexive} if
$(a, a) \in R$ for all $a \in A$.
$R$ is \definition{transitive}{transitive} if
$(a, b) \in R \land (b, c) \in R \implies (a, c) \in R$
for all $a, b, c \in A$.
$R$ is \definition{symmetric}{symmetric} if
$(a, b) \in R \implies (b, a) \in R$
for all $a, b \in A$.
$R$ is \definition{anti-symmetric}{anti-symmetric} if
$(a, b) \in R \implies (b, a) \not\in R$ for all $a, b \in A$.

\strats
