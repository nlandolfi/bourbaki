
%!name:entire_functions
%!need:complex_analytic_functions
%!refs:yellow/IX/4

\section*{Definition}

An \t{entire function} is a complex function $f: \C  \to \C $ which is analytic for all $z \in \C $.

\blankpage
\sbasic

\sinput{../sets/macros.tex}
\sinput{../ordered_pairs/macros.tex}

%%%% MACROS %%%%%%%%%%%%%%%%%%%%%%%%%%%%%%%%%%%%%%%%%%%%%%%

\newcommand{\PM}{\mathbf{P}}

%%%%%%%%%%%%%%%%%%%%%%%%%%%%%%%%%%%%%%%%%%%%%%%%%%%%%%%%%%%


\sstart

\stitle{Relations}

\ssection{Why}
We want to relate elements
of two sets.


\ssection{Definition}

A
\definition{relation}{relation}
between two non-empty
\term{sets}{sets}
$A$ and $B$ is a subset of
$A \cross B$.
A relation on a single set
$C$ is a subset of $C \cross C$.

\ssubsection{Notation}
We denote relations with upper
case capital latin letters because
they are sets.
Let $R$ be a relation on $A$ and $B$.
We denote that $(a, b) \in R$ by
$a R b$, read aloud as
\say{a in relation $R$ to b.}

Often, instead of latin letters we use
other symbols; these symbols suggest the
nature of the relation.
For example,
$\sim$, $=$, $<$,
$\leq$, $\prec$, and $\preceq$.

\ssection{Properties}

Let $R$ be a relation on
a non-empty set $A$.
$R$ is \definition{reflexive}{reflexive} if
$(a, a) \in R$ for all $a \in A$.
$R$ is \definition{transitive}{transitive} if
$(a, b) \in R \land (b, c) \in R \implies (a, c) \in R$
for all $a, b, c \in A$.
$R$ is \definition{symmetric}{symmetric} if
$(a, b) \in R \implies (b, a) \in R$
for all $a, b \in A$.
$R$ is \definition{anti-symmetric}{anti-symmetric} if
$(a, b) \in R \implies (b, a) \not\in R$ for all $a, b \in A$.

\strats
