\documentclass[12pt]{extarticle}
%\usepackage[margin=1.75in]{geometry}
\usepackage[a5paper,margin=.72in]{geometry}
\usepackage{graphicx}
\renewcommand{\baselinestretch}{1.25}
\setlength{\parskip}{0.5em}

\usepackage{titlesec}
%\titleformat*{\section}{\large\bfseries\sffamily}
%\titleformat*{\subsection}{\normalsize\bfseries\sffamily}
%\titleformat*{\subsubsection}{\large\bfseries}
%\titleformat*{\paragraph}{\large\bfseries}
%\titleformat*{\subparagraph}{\large\bfseries}

\titleformat*{\section}{\bfseries\sffamily}
\titleformat*{\subsection}{\small\bfseries\sffamily}
\titleformat*{\subsubsection}{\large\bfseries}
\titleformat*{\paragraph}{\large\bfseries}
\titleformat*{\subparagraph}{\large\bfseries}
\titlespacing*{\section}{0pt}{0.3cm}{0.1cm}
\titlespacing*{\subsection}{0pt}{0.3cm}{0.1cm}

%\renewcommand{\title}[1]{
%\begin{center}
%  \includegraphics[width=0.05\textwidth]{../../trademark}
%  \\
%  \vspace{0.20cm}
%  {\large \textsf{ #1 }}
%\end{center}
%}

\newcommand{\bpage}{
\vspace*{\fill}
\begin{center}
\includegraphics[width=1cm]{../../trademark}
\end{center}
\vspace{\fill}
}

\newcommand{\gpage}{
\begin{center}
\vspace*{\fill}
\includegraphics{graph}
\vspace{\fill}
\end{center}
}


\newcommand*{\vcenteredhbox}[1]{\begin{tabular}{@{}c@{}}#1\end{tabular}}
\renewcommand{\title}[1]{
\begin{center}
\hspace*{-1cm}
\vcenteredhbox{\vspace*{-0.1cm} \includegraphics[height=0.5cm]{../../trademark}\hspace*{0.1cm}}
\vcenteredhbox{\large \textsf{#1}}
\end{center}
}

\usepackage{hyperref}
\hypersetup{
  colorlinks,
  citecolor=black,
  filecolor=black,
  linkcolor=black,
  urlcolor=black
}


%\usepackage{ccfonts}% http://ctan.org/pkg/{ccfonts}
%\usepackage[cm]{sfmath}
\usepackage[T1]{fontenc}
\DeclareMathAlphabet{\mathbfsf}{\encodingdefault}{\sfdefault}{bx}{n}
\def\mathword#1{\mathop{\textup{#1}}}

\usepackage{caption}
\usepackage{subcaption}

%%%% MACROS %%%%%%%%%%%%%%%%%%%%%%%%%%%%%%%%%%%%%%%%%%%%%%%

\newcommand{\PM}{\mathbf{P}}

%%%%%%%%%%%%%%%%%%%%%%%%%%%%%%%%%%%%%%%%%%%%%%%%%%%%%%%%%%%


%%%% MACROS %%%%%%%%%%%%%%%%%%%%%%%%%%%%%%%%%%%%%%%%%%%%%%%

% use \set{stuff} for { stuff }
% use \set* for autosizing delimiters
\DeclarePairedDelimiter{\set}{\{}{\}}

% use \Set{a}{b} for {a | b}
% use \Set* for autosizing delimiters
\DeclarePairedDelimiterX{\Set}[2]{\{}{\}}{#1 \nonscript\;\delimsize\vert\nonscript\; #2}

% use \powerset{A} for power set of A
\newcommand{\powerset}[1]{2^{#1}}

\renewcommand{\emptyset}{\varnothing}

\newcommand{\SA}{\mathcal{A}}
\newcommand{\SB}{\mathcal{B}}
\newcommand{\SC}{\mathcal{C}}
\newcommand{\SD}{\mathcal{D}}
\newcommand{\SE}{\mathcal{E}}
\newcommand{\SF}{\mathcal{F}}
\newcommand{\SG}{\mathcal{G}}
\newcommand{\SH}{\mathcal{H}}
\newcommand{\SI}{\mathcal{I}}
\newcommand{\SJ}{\mathcal{J}}
\newcommand{\SK}{\mathcal{K}}
\newcommand{\SL}{\mathcal{L}}

%%%%%%%%%%%%%%%%%%%%%%%%%%%%%%%%%%%%%%%%%%%%%%%%%%%%%%%%%%%


%%%% MACROS %%%%%%%%%%%%%%%%%%%%%%%%%%%%%%%%%%%%%%%%%%%%%%%

\newcommand{\PM}{\mathbf{P}}

%%%%%%%%%%%%%%%%%%%%%%%%%%%%%%%%%%%%%%%%%%%%%%%%%%%%%%%%%%%


\begin{document}
\title{Relations}

\section{Why}
We want a precise notion for how the elements of one set relate to elements of another set, or how elements of a set relate to other elements of the same set.

\section{Definition}

A \definition{relation} between two non-empty \term{sets}{sets} $A$ and $B$ is a subset of $A \cross B$.
So then, naturally, a relation on a single set $C$ is a subset of $C \cross C$.

\subsection{Notation}
As relations are sets, our de facto protocol is to denote them by upper case capital letters, for example, the letter $R$.
Let $R$ a relation on $A$ and $B$.
If $(a, b) \in R$, we often write $a R b$, read aloud as \say{a in relation $R$ to b.}

In many cases, though, we eschew the set notation and use particular symbols.
Often the symbols we use are meant to be suggestive of the relation.
Some examples include $\sim$, $=$, $<$, $\leq$, and $\prec$.

\section{Equivalence Relations}

Here we survey a special relation on a set.
Let $R$ a relation on the non-empty set $A$.
If $aRa$, then we call $R$ \definition{reflexive}.
If $aRb$ if and only if $bRa$ then we call $R$ \definition{symmetric}.
If $aRb$ and $bRc$ together imply $aRc$, then we call $R$ \definition{transitive}.
If $R$ is reflexive, symmetric, and transitive we call it an \definition{equivalence relation}.

For an element $a \in A$, we call the set of elements in relation $R$ to $a$ the \definition{equivalence class} of $a$.
The key observation, recorded and proven below, is that the equivalence classes partition the set $A$.
A frequent technique is to define an appropriate equivalence relation on a large set $A$ and then to work with the set of equivalence classes of $A$.

We call the set of equivalence classes the \definition{quotient set} of $A$ under $R$.
An equally good name is the divided set of $A$ under $R$, but this terminology is not standard.
The language in both cases reminds us that $\sim$ partitions the set $A$ into equivalence classes.

\subsection{Notation}
If $R$ is an equivalence relation on a set $A$, we use the symbol $\sim$.
When alone, $\sim$ is read aloud as \say{sim,} but we still read $a \sim b$ aloud as \say{a equivalent to b.}
We denote the quotient set of  $A$ under $\sim$ by $A/\sim$, read aloud as \say{A quotient sim}.

\subsection{Results}

\section{Order Relations}

Here we survey a two other special relation on a set.
Let $R$ a relation on the non-empty set $A$.
We call $R$ \definition{anti-symmetric} if for two nonequal elements $a, b \in A$, $(a, b) \in R \implies (b, a) \not\in R$.
If $R$ is reflexive, transitive, and anti-symmetric then we call $R$ a \definition{partial order} on $A$.

A \definition{partially ordered set} is a set together with a partial order.
The language partial is meant to suggest that two elements need not be comparable.j
For example, suppose $R$ is $\Set{(a,a)}{a \in A}$; we may justifiably call this no order at all and call $A$ totally unordered, but it is a partial order by our definition.

Often we want all elements of the set $A$ to be comparable.
We call $R$ \definition{connexive} if for all $a, b \in A$, $(a, b) \in R$ or $(b, a) \in R$.
If $R$ is a partial order and connexive, we call it a \definition{total order}.


A \definition{totally ordered set} is a set together with a total order.
The language is a faithful guide: we can compare any two elements.
Still, we prefer one word to three, and so we will use the shorter term \definition{chain} for a totally ordered set; other terms include \definition{simply ordered set} and \definition{linearly ordered set}.

\subsection{Notation}
We denote total and partial orders on a set $A$ by $\preceq$.
We read $\preceq$ aloud as \say{precedes or equal to} and so read $a\preceq b$ aloud as \say{a precedes or is equal to b.}
If $a \preceq b$ but $a \neq b$, we write $a \prec b$, read aloud as \say{a precedes b.}

\end{document}
